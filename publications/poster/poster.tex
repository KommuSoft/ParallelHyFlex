\documentclass[a1paper,landscape,showframe]{baposter}


\usepackage{amssymb,amsmath,slantsc,algorithm2e,bbm,tikz,parcolumns,multicol,booktabs}
\usepackage[scaled]{helvet}

\usepackage{enumitem}
\setitemize{noitemsep,topsep=0pt,parsep=0pt,partopsep=0pt}

% \makeatletter
% \newcommand*{\compress}{\@minipagetrue}
% \makeatother
% \newenvironment{ilist}{\setdefaultleftmargin{1em}{1em}{}{}{}{}\compress\begin{itemize}\setlength{\itemsep}{0pt}\setlength{\topsep}{0pt}\setlength{\partopsep}{0pt}\setlength{\parsep}{0pt}\setlength{\parskip}{0pt}}{\end{itemize}}

\usetikzlibrary{shapes,fit,calc}

% \setlength{\abovedisplayskip}{0cm}
% \setlength{\belowdisplayskip}{0.0cm}
% \setlength{\abovedisplayshortskip}{0pt}
% \setlength{\belowdisplayshortskip}{0pt}

\newcommand{\deflab}[1]{\label{def:#1}}
\newcommand{\defref}[1]{Definitie \ref{def:#1}}
\newcommand{\imglab}[1]{\label{fig:#1}}
\newcommand{\imgref}[1]{Afbeelding \ref{fig:#1}}
\newcommand{\tbllab}[1]{\label{tbl:#1}}
\newcommand{\tblref}[1]{Tabel \ref{tbl:#1}}
\newcommand{\alglab}[1]{\label{alg:#1}}
\newcommand{\algref}[1]{Algoritme \ref{alg:#1}}
\newcommand{\eqnlab}[1]{\label{eqn:#1}}
\newcommand{\eqnref}[1]{Vergelijking (\ref{eqn:#1})}
\newcommand{\thelab}[1]{\label{the:#1}}
\newcommand{\theref}[1]{Theorema \ref{the:#1}}

\newtheorem{definition}{Definitie}
\newtheorem{theorem}{Theorema}

\newcommand{\true}{\ensuremath{\mbox{\textbf{true}}}}
\newcommand{\fals}{\ensuremath{\mbox{\textbf{false}}}}

\newcommand{\brak}[1]{\ensuremath{\left(#1\right)}}
\newcommand{\fbrk}[1]{\ensuremath{\left[#1\right]}}
\newcommand{\tupl}[1]{\ensuremath{\left\langle #1\right\rangle}}
\newcommand{\accl}[1]{\ensuremath{\left\{ #1\right\}}}
\newcommand{\abs}[1]{\ensuremath{\left| #1\right|}}
\newcommand{\dabs}[1]{\ensuremath{\left\| #1\right\|}}

\newcommand{\fun}[2]{\ensuremath{#1\brak{#2}}}
\newcommand{\funf}[2]{\ensuremath{#1\fbrk{#2}}}
\newcommand{\funm}[2]{\fun{\mbox{#1}}{#2}}
\newcommand{\funsig}[3]{#1:#2\rightarrow #3}
\newcommand{\funsigimp}[5]{#1:#2\rightarrow #3:#4\mapsto #5}

\newcommand{\bigoh}[1]{\fun{\mathcal{O}}{#1}}

\newcommand{\xstar}{\ensuremath{x^{\star}}}
\newcommand{\xdot}{\ensuremath{x^{\circ}}}
\newcommand{\calXop}{\calX^{\star}}

\newcommand{\argmin}{\ensuremath{\mbox{argmin}}}
\newcommand{\mean}[2][]{\funf{\EEE_{#1}}{#2}}
\newcommand{\Prob}[1]{\funf{\Pr}{#1}}

\newcommand{\comp}[1]{\textsc{\mbox{#1}}}
\newcommand{\prbm}[1]{\textsc{#1}}
\newcommand{\algo}[1]{\textsc{#1}}

\newcommand{\AAA}{\mathbb{A}}
\newcommand{\BBB}{\mathbb{B}}
\newcommand{\CCC}{\mathbb{C}}
\newcommand{\DDD}{\mathbb{D}}
\newcommand{\EEE}{\mathbb{E}}
\newcommand{\FFF}{\mathbb{F}}
\newcommand{\GGG}{\mathbb{G}}
\newcommand{\HHH}{\mathbb{H}}
\newcommand{\III}{\mathbb{I}}
\newcommand{\JJJ}{\mathbb{J}}
\newcommand{\KKK}{\mathbb{K}}
\newcommand{\LLL}{\mathbb{L}}
\newcommand{\MMM}{\mathbb{M}}
\newcommand{\NNN}{\mathbb{N}}
\newcommand{\OOO}{\mathbb{O}}
\newcommand{\PPP}{\mathbb{P}}
\newcommand{\QQQ}{\mathbb{Q}}
\newcommand{\RRR}{\mathbb{R}}
\newcommand{\SSS}{\mathbb{S}}
\newcommand{\TTT}{\mathbb{T}}
\newcommand{\UUU}{\mathbb{U}}
\newcommand{\VVV}{\mathbb{V}}
\newcommand{\WWW}{\mathbb{W}}
\newcommand{\XXX}{\mathbb{X}}
\newcommand{\YYY}{\mathbb{Y}}
\newcommand{\ZZZ}{\mathbb{Z}}

\newcommand{\zeromatrix}[1][]{\boldsymbol{0}_{#1}}
\newcommand{\identitymatrix}[1][]{\boldsymbol{I}_{#1}}
\newcommand{\onematrix}[1][]{\boldsymbol{1}_{#1}}

\newcommand{\calA}{\mathcal{A}}
\newcommand{\calB}{\mathcal{B}}
\newcommand{\calC}{\mathcal{C}}
\newcommand{\calD}{\mathcal{D}}
\newcommand{\calE}{\mathcal{E}}
\newcommand{\calF}{\mathcal{F}}
\newcommand{\calG}{\mathcal{G}}
\newcommand{\calH}{\mathcal{H}}
\newcommand{\calI}{\mathcal{I}}
\newcommand{\calJ}{\mathcal{J}}
\newcommand{\calK}{\mathcal{K}}
\newcommand{\calL}{\mathcal{L}}
\newcommand{\calM}{\mathcal{M}}
\newcommand{\calN}{\mathcal{N}}
\newcommand{\calO}{\mathcal{O}}
\newcommand{\calP}{\mathcal{P}}
\newcommand{\calQ}{\mathcal{Q}}
\newcommand{\calR}{\mathcal{R}}
\newcommand{\calS}{\mathcal{S}}
\newcommand{\calT}{\mathcal{T}}
\newcommand{\calU}{\mathcal{U}}
\newcommand{\calV}{\mathcal{V}}
\newcommand{\calW}{\mathcal{W}}
\newcommand{\calX}{\mathcal{X}}
\newcommand{\calY}{\mathcal{Y}}
\newcommand{\calZ}{\mathcal{Z}}

\newcommand{\BoolSet}{\BBB}
\newcommand{\NatSet}{\NNN}
\newcommand{\RealSet}{\RRR}
\newcommand{\OpProblem}{\Pi}
\newcommand{\ConfigSet}{\calX}
\newcommand{\ConfigValSet}{\ConfigSet'}
\newcommand{\ConfigOpSet}{\ConfigSet^{\star}}
\newcommand{\sol}{x}
\newcommand{\bestSol}{\sol^{\star}}
\newcommand{\goodSol}{\sol^{\circ}}
\newcommand{\SolSet}{\calS}
\newcommand{\PopSet}{\calP}
\newcommand{\hcfun}{c}
\newcommand{\evalfun}{f}
\newcommand{\evalfuna}{f'}
\newcommand{\hittime}{\theta}
\newcommand{\phittime}{\Theta}
\newcommand{\neighbr}{\calN}
\newcommand{\nvar}{n}
\newcommand{\VarDom}{A}
\newcommand{\PopChain}{\mathfrak{P}}


\usepackage{epstopdf,amsmath,amssymb,algorithm2e}
\usepackage[dutch]{babel}
\definecolor{kulblue}{cmyk}{0.65,0.00,0.00,0.35}
\definecolor{corprgb}{rgb}{0.11,0.56,0.69}
%\definecolor{lightblue}{rgb}{0.145,0.6666,1}

%\newcommand{\fun}[2]{#1\left(#2\right)}
%\newcommand{\true}{\mbox{\bf true}}

\begin{document}
\background{
	\begin{tikzpicture}[remember picture,overlay]%
	\draw (current page.north west)+(-1.0em,+1.0em) node[anchor=north west]
	{\includegraphics[height=7.0em]{sedescrop.pdf}};
	\end{tikzpicture}
}
\pagecolor{kulblue!70}
\begin{poster}{grid,columns=3,background=user,bgColorOne=kulblue!50,eyecatcher,
borderColor=kulblue,
headerColorOne=kulblue,
headerColorTwo=kulblue,
headerFontColor=white,
boxColorOne=kulblue!30,
boxColorTwo=kulblue!30,
% Format of textbox
textborder=rectangle,
% Format of text header
headerborder=closed,
headerheight=0.1\textheight,
textfont=\small\fontfamily{phv}\selectfont,
headershape=roundedright,
headershade=plain,
headerfont=\Large\bf\fontfamily{phv}\selectfont,%\Large\bf, %Sans Serif
textfont={\setlength{\parindent}{0cm}},
boxshade=plain,
linewidth=2pt
}{}{\textbf{\fontfamily{phv}\selectfont Parallelle Hyperheuristieken}}{Willem Van Onsem; \emph{promotor:} prof. B. Demoen}{\includegraphics[height=3.0em]{KULEUVEN_CMYK_LOGO.eps}}%\includegraphics[height=4.0em]{DTAI-logo.png}

\headerbox{Optimalisatieprobleem}{name=problem,column=0,row=0}{
Gegeven een tuple $\left\langle X,c:X\rightarrow\mathbb{B},f:X\rightarrow\mathbb{R}\right\rangle$, zoek een element $x\in X$ zodat $\fun{c}{x}=\true$ en $\fun{f}{x}$ minimaal is.\\
Algemeen: \comp{NP-hard}, enkele gevallen in \comp{P}
}



\headerbox{Metaheuristieken}{name=metaheuristiek,column=1,row=0}{
Heuristiek met behulp van incrementele Monte-Carlo simulatie.\\
\begin{algorithm}[H]
\SetAlgoLined
$\PopSet_1\leftarrow\funm{init}{\xi}$\;
$b_1\leftarrow\displaystyle\argmin_{\sol\in\PopSet_1}{\fun{\evalfun}{\sol}}$\;
\Repeat{het stopcriterium is bereikt}{
 $\PopSet_{t+1}\leftarrow\funm{nextPopulation}{\PopSet_t,t,\xi}$\;
 $b_{t+1}\leftarrow\displaystyle\argmin_{\sol\in\PopSet_{t+1}\cup\accl{b_t}}{\fun{\evalfun}{\sol}}$\;
 $t\leftarrow t+1$\;
}
\KwRet{$b_t$}
\end{algorithm}
\setlength{\abovedisplayskip}{0pt}
\setlength{\belowdisplayskip}{0pt}
\setlength{\abovedisplayshortskip}{0pt}
\setlength{\belowdisplayshortskip}{0pt} 
Verwachte uitvoertijd met $p$ processoren:
\begin{equation*}
\mean{\hittime_p}\approx\sigma^p/\lambda^p\cdot\brak{1-\lambda^p}
\end{equation*}
met $0\leq\lambda< 1,0<\sigma\sim\mbox{probleem en methode}$\\
Minimale te realiseren speed-up:
\begin{equation*}
\mbox{Speed-up}\geq\brak{p+1}/2
\end{equation*}
}

\headerbox{Hyperheuristieken (\emph{HyFlex})}{name=hyperheuristiek,column=0,below=problem}{
Probleemonafhankelijk optimaliseren: gegeven een set transitiefuncties $h_i:X^{k_i}\rightarrow X$, voer een sequentie van $h_i$'s uit zodat de resulterende populatie optimaal is.\\
Software-bibliotheek uit 2010: \emph{HyFlex}.\\
Competitie uit 2011: \emph{CHeSC2011}\\
winnaar: M. M\i{}s\i{}r (\emph{AdapHH})\\%, \emph{CODeS research group}
\emph{AdapHH} basis voor parallelle hyperheuristiek
}

\headerbox{Afdwingbare beperkingen}{name=afdwingbareConstraints,column=2,row=0}{
Een \emph{afdwingbare beperking} is een 3-tuple: $EC=\left\langle c:X\rightarrow\BBB,c^+:X\rightarrow X, c^-:X\rightarrow X \right\rangle$. Waarbij $c$ kan bepalen of een configuratie $x$ aan de beperking voldoet. $c^+$ een oplossing desbetreffend minimaal kan aanpassen, en $c^-$ een oplossing $x$ kan \emph{tweaken} zodat niet meer aan de beperking voldaan is.\\\\
Een hypothese-generator (hypogen) is een functie $g_i:X^{l_i}\rightarrow EC$ die gegeven een configuratie een afdwingbare beperking kan genereren.\\\\
Een beperkingsruimte is een verzameling van alle mogelijke afdwingbare beperkingen die gegenereerd kunnen worden
}

\headerbox{\emph{ParHyFlex}: een Parallel systeem voor Hyperheuristieken}{name=parhyflex,column=0,span=2,below=metaheuristiek}{
\begin{multicols}{2}
\begin{tikzpicture}[scale=\sc]
\begin{scope}[yshift=-0.2 cm]
  \node[draw,rectangle,minimum width=\sc*6.5 cm,fill=white, minimum height=\sc*3cm] (PD) at (0.75,-1.4) {};
  \coordinate (PDL) at ($(PD.north west)+(-0.5,0)$);
  \draw[<->] (PDL |- PD.south west) to node[above,midway,sloped]{\small{Afhankelijk}} (PDL);
  \begin{scope}[xshift=1.5 cm]
    \node[draw,rectangle,minimum width=\sc*1cm, minimum height=\sc*0.75cm] (LL1) at (-3.35,-0.625) {};
    \fill (-3.65,-0.55) circle (0.05cm) node[anchor=west,xshift=0.125 cm] {\small{$h_0$}};
    \fill (-3.35,-0.85) circle (0.05cm) node[anchor=west,xshift=0.125 cm] {\small{$h_1$}};
    \fill (-3.25,-0.65) circle (0.05cm) node[anchor=west,xshift=0.125 cm] {\small{$h_2$}};
    \draw (LL1.south) node[anchor=north,yshift=-0.1 cm] {\small{Mutatie}};
  \end{scope}
  \begin{scope}[xshift=2.75 cm]
    \node[draw,rectangle,minimum width=\sc*1cm, minimum height=\sc*0.75cm] (LL2) at (-3.35,-0.625) {};
    \fill (-3.65,-0.55) circle (0.05cm) node[anchor=west,xshift=0.125 cm] {\small{$h_3$}};
    \fill (-3.35,-0.85) circle (0.05cm) node[anchor=west,xshift=0.125 cm] {\small{$h_4$}};
    \fill (-3.25,-0.65) circle (0.05cm) node[anchor=west,xshift=0.125 cm] {\small{$h_5$}};
    \fill (-3.25,-0.35) circle (0.05cm) node[anchor=west,xshift=0.125 cm] {\small{$h_6$}};
    \draw (LL2.south) node[anchor=north,yshift=-0.1 cm] {\small{LS}};
  \end{scope}
  \begin{scope}[xshift=4 cm]
    \node[draw,rectangle,minimum width=\sc*1cm, minimum height=\sc*0.75cm] (LL3) at (-3.35,-0.625) {};
    \fill (-3.65,-0.55) circle (0.05cm) node[anchor=west,xshift=0.125 cm] {\small{$h_7$}};
    \fill (-3.35,-0.85) circle (0.05cm) node[anchor=west,xshift=0.125 cm] {\small{$h_8$}};
    \draw (LL3.south) node[anchor=north,yshift=-0.1 cm] {\small{RR}};
  \end{scope}
  \begin{scope}[xshift=5.25 cm]
    \node[draw,rectangle,minimum width=\sc*1cm, minimum height=\sc*0.75cm] (LL4) at (-3.35,-0.625) {};
    \fill (-3.65,-0.55) circle (0.05cm) node[anchor=west,xshift=0.125 cm] {\small{$h_9$}};
    \draw (LL4.south) node[anchor=north,yshift=-0.1 cm] {\small{Kruising}};
  \end{scope}
  \begin{scope}[xshift=6.5 cm]
    \node[draw,rectangle,minimum width=\sc*1cm, minimum height=\sc*0.75cm] (OBJ) at (-3.35,-0.625) {};
    \fill (-3.75,-0.45) circle (0.05cm) node[anchor=west,xshift=0.125 cm] {\small{$O_0^*$}};
    \fill (-3.35,-0.85) circle (0.05cm) node[anchor=west,xshift=0.125 cm] {\small{$O_1$}};
    \fill (-3.25,-0.65) circle (0.05cm) node[anchor=west,xshift=0.125 cm] {\small{$O_2$}};
    \draw (OBJ.south) node[anchor=north,yshift=-0.1 cm] {\small{Objectief$^{\star}$}};
  \end{scope}
  \begin{scope}[xshift=1.5 cm,yshift=-1.5 cm]
    \node[draw,rectangle,minimum width=\sc*1cm, minimum height=\sc*0.75cm] (DIS) at (-3.35,-0.625) {};
    \fill (-3.75,-0.45) circle (0.05cm) node[anchor=west,xshift=0.125 cm] {\small{$\delta_0$}};
    \fill (-3.35,-0.85) circle (0.05cm) node[anchor=west,xshift=0.125 cm] {\small{$\delta_1$}};
    \fill (-3.25,-0.65) circle (0.05cm) node[anchor=west,xshift=0.125 cm] {\small{$=$}};
    \draw (DIS.south) node[anchor=north,yshift=-0.1 cm] {\small{Afstand$^{\star}$}};
  \end{scope}
  \begin{scope}[xshift=2.75 cm,yshift=-1.5 cm]
    \node[draw,rectangle,minimum width=\sc*1cm, minimum height=\sc*0.75cm] (HYP) at (-3.35,-0.625) {};
    \fill (-3.75,-0.45) circle (0.05cm) node[anchor=west,xshift=0.125 cm] {\small{$g_0$}};
    \fill (-3.35,-0.85) circle (0.05cm) node[anchor=west,xshift=0.125 cm] {\small{$g_1$}};
    \draw (HYP.south) node[anchor=north,yshift=-0.1 cm] {\small{Hypogen$^{\star}$}};
  \end{scope}
  \fill (0.45,-2.25) circle (0.05cm) node[anchor=west,xshift=0.125 cm] {\small{Initialisatie functie $I$}};
  \node[draw,rectangle,inner sep=1pt,minimum width=\sc*2 cm, minimum height=\sc*0.5 cm] at (1.35,-1.75) {\small{Beperkingsruimte$^{\star}$ $\HypSet$}};
\end{scope}

\begin{scope}[xshift=-0.5 cm]
  \node[draw,rectangle,fill=white,minimum width=\sc*4 cm, minimum height=\sc*2cm] (PI) at (0,1.1) {};
  \coordinate (PIL) at ($(PI.north west)+(-0.5,0)$);
  \draw[<->] (PIL |- PI.south west) to node[above,midway,sloped]{\small{Onafhankelijk}} (PIL);
  \drawmem{MEM}{7}{3}
  \node[draw,rectangle,fill=gray!20,minimum width=\sc*3.5 cm, minimum height=\sc*0.5cm] (EXC) at (0,1.5) {Uitvoermechanisme};
\end{scope}
\node[draw,rectangle,fill=orange!20,minimum width=\sc*2.25 cm, minimum height=\sc*2cm] (PH) at (2.875,1.1) {};
\draw (PH.north) node[anchor=south] {\small{\emph{\textbf{ParHyFlex}}}$^{\star}$};
\node[draw,fill=white,rectangle,minimum width=\sc*2 cm, minimum height=\sc*0.5 cm] at (2.875,1.1) {\small{Ervaring$^{\star}$}};
\node[draw,fill=white,rectangle,minimum width=\sc*2 cm, minimum height=\sc*0.5 cm] at (2.875,1.7) {\small{Uitwisseling$^{\star}$}};
\node[draw,fill=white,rectangle,minimum width=\sc*2 cm, minimum height=\sc*0.5 cm] at (2.875,0.5) {\small{Zoekruimte$^{\star}$}};
\end{tikzpicture}
\setlength{\columnsep}{0.7em}
\setlength{\columnseprule}{0mm}
\setlength{\itemsep}{0pt}%
\setlength{\topsep}{0pt} 
\setlength{\partopsep}{0pt}
\setlength{\parsep}{0pt}
\setlength{\parskip}{0pt}
\vfill\columnbreak
\emph{Intensification-Diversification} cycle:
\begin{itemize}
 \item \emph{uitwisselen} van configuraties
 \item maar \emph{beperken} van zoekruimtes (per core)
\end{itemize}
\emph{Configuraties} over netwerk sturen en \emph{aanpassen} volgens de lokale \emph{zoekruimte}.\\
Importeren van goede eigenschappen via \emph{kruising heuristieken}.\\
Eigenschappen van de \textcolor{red!50!black}{\emph{Zoekruimtes}}:
\begin{itemize}
 \item \emph{Ervaring} opdoen via \emph{hypogen}
 \item sets van \emph{positieve} en \emph{negatieve afdwingbare beperkingen}
 \item \emph{amnesie}: elke hypothese kunnen vergeten
 \item \emph{onderhandelen}: ervaring uitwisselen en concentratie vermijden
\end{itemize}
\end{multicols}}

\headerbox{Resultaten}{name=resultaat,column=2,below=metaheuristiek}{
Optimaliseren van \prbm{MAX-3SAT} met $10'000$ variabelen en $42'000$ clauses. \emph{ParHyFlex} zoekt 25 seconden naar een oplossing. Voor elke processorconfiguratie werden de experimenten 100 keer herhaald.\\
\begin{center}
\small{\begin{tabular}{crrr}
\toprule
$p$&Correct&Tijd (s)&top17\% (s)\\\midrule
1&17\%&9.800&9.800\\
2&38\%&13.017&6.089\\
3&43\%&16.192&8.551\\
4&43\%&20.761&11.153\\
\bottomrule
\end{tabular}}
\end{center}
}

\end{poster}
\end{document}