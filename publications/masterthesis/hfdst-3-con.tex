\section{Besluit van dit hoofdstuk}

\importtikz[1.4]{parhyflexstructure}{parhyflexstructure}{Structuur van \emph{ParHyFlex}.}
Op \imgref{parhyflexstructure} geven we schematisch de structuur van \emph{ParHyFlex} weer.	De componenten die gemarkeerd worden met een ster ($\star$), zijn component die niet aanwezig zijn in \emph{HyFlex}.

\paragraph{}
Een deel van de geheugencellen is gemarkeerd met een schuine streep. Deze geheugencellen stellen vreemd geheugen voor waarvan er lokaal een kopie wordt bijgehouden. De geheugencellen kunnen uitgelezen worden, maar er kan geen oplossing naar geschreven worden.

\paragraph{}
\importtikz[1.4]{parhyflexwerking}{parhyflexwerking}{Schematische voorstelling van de kern van \emph{ParHyFlex}.}
In \imgref{parhyflexwerking} beschrijven we kort het proces die een berekende of ontvangen oplossing doormaakt. Deze oplossing -- op de figuur $s_1^{(0)}$ -- wordt eerst aangepast door de zoekruimte: alle positieve hypotheses en \'e\'en negatieve hypothese worden toegepast op de oplossing en wordt aangepast tot $s_1^{(E)}$ die binnen de zoekruimte valt. De fitness-waarde wordt berekend en de evaluaties van de reeds aanwezige hypotheses in de ervaring-set worden aangepast (de data wordt voor elke hypothese opgenomen in \'e\'en van de twee normale verdelingen). Verder wordt met behulp van \'e\'en van de hypothesegeneratoren  een hypothese gegenereerd die met een bepaalde kans opgenomen wordt in de ervaring-set. De oplossing wordt vervolgens in het geheugen opgenomen en eventueel doorgestuurd naar andere processoren.

\paragraph{}
Op geregelde tijdstippen treed er amnesie op in de \emph{ervaring-set}: een deel van de hypothese worden uit de set verwijdert. Dit gebeurt op basis van de twee normale verdelingen per hypothese. Op die manier kan men zich ontdoen van foute hypothese, en maakt men ruimte voor nieuwe hypotheses.

\paragraph{}
Op vaste tijdsintervallen zal de \emph{onderhandelaar} een deel van de hypotheses uit de \emph{ervaring-set} halen. Een deel van deze hypotheses vormen de nieuwe positieve set van de \emph{zoekruimte}. De overige worden doorgestuurd in de \emph{ervaring-set} van de andere processoren ge\"injecteerd. Een deel van de doorgestuurde hypotheses vormt ook een basis van de negatieve set van de \emph{zoekruimte}.

\paragraph{}
De figuur toont de verschillende stromen van informatie: volle lijnen duiden op informatiestromen die lokaal worden uitgevoerd. Streepjeslijnen duiden op informatie die de processor uitstuurt naar andere processoren. Stippellijnen tonen de informatie die de processor ontvangt van andere processoren.