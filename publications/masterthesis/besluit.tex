\chapter{Besluit}
\label{besluit}

\chapterquote{Man sage nicht, das schwerste sei die Tat; I da hilft der Mut, der Augenblick, die Regung; I das schwerste dieser Welt ist der Entschluss.}{Franz Grillparzer}

In \secref{conclusions} worden de verschillende besluiten die doorheen deze thesis werden getrokken opgesomd. In \secref{potentials} overlopen we de verdere potenti\"ele ontwikkelingen van dit systeem.

\section{Besluiten}
\seclab{conclusions}

\subsection{Hyperheuristieken}

Hyperheuristieken zijn een familie van benaderingsalgoritmen die voor een grote aantal optimalisatieprobleem een acceptabel antwoord proberen te genereren. De algoritmen werken op basis van een gegeven set transitiefuncties en een iteratieve Monte-Carlo simulatie.

\paragraph{}
Het onderzoek naar hyperheuristieken is vrij recent en mist momenteel nog een theoretische basis: de prestaties van een concrete hyperheuristiek zijn erg afhankelijk van de de onderliggende set transitiefuncties.

\paragraph{}
Er bestaan enkele implementaties van \emph{parallelle hyperheuristieken}. De meeste implementaties werken volgens het \emph{master-slave} paradigma. De prestaties van de \emph{hyperheuristieken} inzake \emph{speed-up} zijn eerder wisselvallig.

\subsection{\emph{ParHyFlex}}

\emph{ParHyFlex} is een systeem die de implementatie van hyperheuristieken ondersteunt. Het probeert een platform aan te bieden waarbij zowel de probleemafhankelijke heuristieken als de hyperheuristiek in kwestie zich minimaal bewust zijn van het feit dat het algoritme parallel wordt uitgevoerd.

\paragraph{}
Het systeem is georganiseerd volgens het \emph{peer-to-peer} communicatie paradigma: er bestaat geen hi\"erarchische relaties tussen de verschillende processoren. Het \emph{Island Model}, een model bij genetische algoritmen vormt een inspiratiebron voor het systeem: oplossingen worden uitgewisseld met behulp van asynchrone communicatie via een \emph{uitwisselingsstrategie}.

\paragraph{}
Om meer processoren te richten op interessante gebieden wordt een concept genaamd \emph{afdwingbare beperkingen} gebruikt: een set voorwaarden die bij iedere oplossing kunnen worden afgedwongen. Op geregelde tijdstippen onderhandelen processoren over de actieve \emph{afdwingbare beperkingen}. Hierdoor kan men rekenkracht focussen op interessante gebieden en vermijden dat processoren te gelijkaardige oplossingen onderzoeken.

\paragraph{}
\emph{Afdwingbare beperkingen} laten ook toe om ervaring te genereren. Op basis van gegenereerde oplossingen kan met deze \emph{beperkingen} genereren en evaluatie gebeurt door een analyse te maken naar de kwaliteit van de oplossingen die wel of niet aan deze \emph{beperking} voldoen.

\subsection{\emph{ParAdapHH}}

\emph{ParAdapHH} is een parallelle variant van \emph{AdapHH}, een hyperheuristiek ge\"implementeerd door Mustafa M\i{}s\i{}r.

\paragraph{}
Het systeem probeert heuristieken beter te evalueren doordat elke processor de waarde van de bijbehorende metrieken doorstuurt. De samengestelde van al deze metrieken wordt vervolgens meegenomen in de evaluatie.

\paragraph{}
De \emph{learning automaton} in het systeem probeert ook te leren op basis van meer data. De verschillende gewichten van de \emph{learning automata} van de processoren worden uitgewisseld en elke \emph{learning automaton} beschouwt een interpolatie tussen de eigen gewichten en de vreemde gewichten.

\paragraph{}
Om een nieuwe oplossing te accepteren wordt deze vergeleken met een lijst van historisch beste fitness-waarden. Door een gedistribueerde lijst uit te wisselen, is men in staat om sterkere convergentie naar een optimale oplossing af te dwingen. Om te voorkomen dat een hyperheuristiek te lang geen oplossing accepteert, bestaat een vast deel van de historische waarden uit fitness-waarden van oplossingen die lokaal gegenereerd werden.

\subsection{Resultaten}

\section{Potenti\"ele ontwikkelingen}
\seclab{potentials}

Vooral op het gebied van het \prbm{Finite Domain Costraint Optimization Problem (FDCOP)} zien we potentieel interessante ontwikkelingen. Er bestaan reeds verschillende pakketten die toelaten op optimalisatieproblemen in logische taal uit te drukken. We denken dan bijvoorbeeld aan \emph{ECLiPSe}. De meeste van deze bibliotheken werken met behulp van een \emph{branch-and-bound}-mechanisme.

\paragraph{}
\emph{Branch-and-bound} garandeert dat de optimale oplossing op termijn gevonden wordt, maar kan voor de meeste complexe problemen geen sterke oplossing in aanvaardbare tijd vinden. Door hyperheuristieken in het proces te betrekken zien we enkele voordelen.

\paragraph{}
Hyperheuristieken leveren altijd een oplossing af binnen een zekere termijn. Dit is ook mogelijk met het \emph{branch-and-bound} mechanisme. Dit laatste mechanisme zoekt in een beperkte tijd echter enkel een set van gelijkaardige oplossingen af. De kans is groot dat hyperheuristieken in deze tijd tot betere oplossingen komen omdat het zoekterrein meer divers is.

\paragraph{}
Een hyperheuristiek kan ook gebruikt worden als een vorm van \emph{preprocessing}. Door een hyperheuristiek eerst een benaderende oplossing te laten uitrekenen wordt een strenge \emph{bound} bepaald. Het aantal \emph{backtracking}-stappen in het effectieve zoekproces kan hierdoor gereduceerd worden.

\paragraph{}
Het parallel uitrekenen van hyperheuristieken kan resulteren in een softwarepakket die optimalisatieproblemen gespecificeerd in een logische taal kan uitrekenen. Door dit in een parallelle context uit te voeren kan men een oplossing met potentieel een arbitraire fout in een arbitraire tijd uitrekenen.

\paragraph{}
Omgekeerd denken we dat logisch programmeren ook een bijdrage kan leveren bij het ontwikkelen van metaheuristieken en hyperheuristieken. Door het uitdrukken van optimalisatieproblemen in logische taal, specificeert men enkel het probleem en is het de verantwoordelijkheid van het algoritme in kwestie om met zichzelf om te vormen tot een effici\"ent optimalisatiesysteem voor het specifieke probleem.

\paragraph{}
Tot slot kan men in een later stadium ook een brug maken met \emph{energy complexity}\cite{Roy:2013:ECM:2422436.2422470,conf/icpp/KorthikantiAG11}: het berekenen met hoeveel processoren men het effici\"entst een kwalitatieve oplossing kan berekenen.

%%% Local Variables: 
%%% mode: latex
%%% TeX-master: "masterproef"
%%% End: 
