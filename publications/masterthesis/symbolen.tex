\begin{flushleft}
  \renewcommand{\arraystretch}{1.1}
  \begin{tabularx}{\textwidth}{@{}p{20mm}X@{}}
    $\BoolSet$							& De verzameling van Booleaanse waarden: $\BoolSet=\accl{\mbox{\textbf{true}},\mbox{\textbf{false}}}$. \\
    $\NatSet$							& De verzameling van natuurlijke getallen: $\NatSet=\accl{0,1,2,\ldots}$ \\
    $\QbitSet{\bitvar}$					& De verzameling van rationale getallen die met $\bitvar$ bits kunnen worden voorgesteld. Hier gedefinieerd als:\\ &$\QbitSet{\bitvar}=\accl{x/2^{\bitvar/2},-x/2^{\bitvar/2}|x\in\NatSet:x<2^{\bitvar-1}}$ \\
    $\zeromatrix[m\times n]$			& Een nulmatrix met dimensies $m\times n$\\
    $\onematrix[m\times n]$				& Een matrix met dimensies $m\times n$ waarbij elk element gelijk is aan $1$\\
    $\identitymatrix[m\times n]$		& Een identiteitsmatrix met dimensies $m\times n$\\
    $\RealSet$							& De verzameling van re\"ele getallen\\
    $\mean[\fun{\calD}{x}]{\fun{f}{x}}$	& Het gemiddelde van een functie $f$ volgens een verdeling $\calD$\\
    $\Prob{e}$							& De kans op een gebeurtenis $e$\\
    $\OpProblem$						& Optimalisatieprobleem\\
    $\ConfigSet$						& De verzameling van mogelijke configuraties\\
    $\ConfigOpSet$						& De set van de globaal optimale oplossingen\\
    $\ConfigValSet$						& De verzameling van geldige configuraties:\\
										& $\ConfigValSet=\accl{x|x\in\ConfigSet:\fun{c}{x}=\true{}}$\\
    $\sol$								& Een globaal optimale oplossing: $\sol\in\ConfigSet$\\
    $\bestSol$							& Een globaal optimale oplossing: $\xstar\in\ConfigOpSet$\\
    $\SolSet$							& Een oplossingsruimte in de context van een metaheuristiek. $\SolSet\subseteq\ConfigValSet$\\
    $\PopSet$							& Een verzameling oplossingen ofwel \emph{populatie}\\
    $\hcfun$							& Harde beperkingen: $\funsig{\hcfun}{\ConfigSet}{\BoolSet}$\\
    $\evalfun$							& Evaluatiefunctie: $\funsig{\evalfun}{\ConfigSet}{\RealSet}$\\
    $\hittime$							& De raaktijd voor een optimalisatieprobleem in een sequenti\"ele context\\
    $\phittime$							& De raaktijd voor een optimalisatieprobleem in een parallelle context\\
    $\neighbr$							& Omgeving van een oplossing\\
    $\ev[i,A]$							& De $i$-de eigenwaarde van een matrix $A$. De eigenwaarden zijn gerangschikt van groot naar klein: $\abs{\ev[i,A]}\geq\abs{\ev[i+1,A]}$\\
    $\ev[A]$							& De dominante eigenwaarde van een matrix $A$\\
    $\evl[i,A]$							& De linkse eigenvector die bij de $i$-de eigenwaarde van matrix $A$ hoort\\
    $\evr[i,A]$							& De rechtse eigenvector die bij de $i$-de eigenwaarde van matrix $A$ hoort\\
    $\krdelta{x}$						& De Kroneckerdelta voor een gegeven Booleaanse expressie $x$\\
    $\TranSet$							& Set van transitiefuncties\\
    $p$									& Aantal processoren in een parallelle configuratie\\
    $\dabs[k]{\vec{v}}$					& $k$-norm van vector $\vec{v}$: $\dabs[k]{\vec{v}}=\sqrt[k]{\sum_i\abs{v_i}^k}$. Indien $k$ niet vermeld wordt is $k=2$.\\
    $\transpose{A}$						& De getransponeerde matrix van een matrix $A$\\
  \end{tabularx}
\end{flushleft}