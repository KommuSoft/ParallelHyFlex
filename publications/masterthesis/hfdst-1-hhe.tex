\section{Hyperheuristieken}

Deze masterthesis behandelt het parallelliseren van hyperheuristieken. Hyperheuristieken zijn een veralgemening van metaheuristieken. In \algref{metaheuristicGeneral} genereert men een nieuwe populatie uit de vorige populatie. Hoe dit concreet gebeurt is de verantwoordelijkheid van de metaheuristiek. Een hyperheuristiek gaat in de eerste plaats op zoek naar een sterke oplossingsmethode, niet een sterke oplossing.\cite{Burke03hyper-heuristics:an,Kendall01ahyperheuristic,Burke_aclassification}

\begin{definition}[Hyperheuristiek\cite{Burke_aclassification}]
Een hyperheuristiek is een automatische methodologie voor het selecteren of genereren van heuristieken om moeilijke computationele zoekproblemen op te lossen.
\end{definition}
De heuristieken worden ook wel \emph{low-level heuristieken} of \emph{transitiefuncties} genoemd.

\importtikz{hyperheuristic}{hhschema}{Schematische voorstelling van een hyperheuristiek.\cite{Burke03hyper-heuristics:an}}

\paragraph{}
De definitie zelf maakt reeds een onderscheid tussen twee mechanismes\cite{Burke_aclassification}:
\begin{enumerate}
 \item \emph{Heuristic selection}: de hyperheuristiek kiest een transitiefunctie die wordt toegepast op de populatie.
 \item \emph{Heuristic generation}: de hyperheuristiek ontwikkelt zelf transitiefuncties op basis van aangereikte componenten.
\end{enumerate}

Het selecteren of genereren van een hyperheuristiek kan op verschillende manieren gebeuren. Men maakt meestal een onderscheid tussen \emph{no-learning}, \emph{offline learning} en \emph{online learning}.

\subsection{Motivering}
Indien het gegeven probleem complex en een metaheuristiek ontwikkelen geen sinecure is, maakt men gebruik van hyperheuristieken. Meestal is dit het geval bij problemen waar de instanties vrij divers kunnen zijn, en men in het ontwerp van de metaheuristiek er niet in slaagt alle families van probleeminstanties te beschouwen.

\paragraph{}
Hyperheuristieken kunnen ook ingezet worden bij de ontwikkeling van een mechanisme die verschillende problemen kan oplossen. De set van transitiefuncties (of componenten) maakt immers deel uit van de invoer. Door andere functies in te voeren kan men andere problemen oplossen. Hyperheuristieken worden daarom vaak voorgesteld als een oplossingstechniek die er op termijn moet in slagen alle optimalisatieproblemen benaderend op te lossen.

\subsection{Het effect van de transitiefuncties}
\cite{DBLP:conf/ppsn/MisirVCB12} onderzoekt in welke mate het invoeren van andere transitiefuncties tot andere resultaten leidt. In plaats van verschillende transitiefuncties in te voeren, werden varianten op de bestaande transitiefuncties beschouwd: functies die een onderliggende transitiefunctie enkele malen herhalen.

\paragraph{}
Vermits de achterliggende transitiefuncties dezelfde zijn, verwacht men dat hyperheuristieken die in de studie werden onderzocht verhoudingsgewijs even sterk zullen scoren. Dit bleek echter niet altijd het geval te zijn. Hyperheuristiek die minder sterk scoorden op de originele transitiefuncties, scoorden soms beter op de afgeleide heuristieken. Dit resultaat vormt een indicatie dat de huidige hyperheuristieken momenteel onvoldoende leren om de effecten van de onderliggende heuristieken te kunnen begrijpen.

\subsection{Parallelle Hyperheuristieken}
\ssclab{defparhyhe}

Hyperheuristieken zijn doorgaans minder effici\"ent in het oplossen van specifieke optimalisatieproblemen waarvoor sterke metaheuristieken bestaan. De rede is dat eerst de sterktes en zwaktes van de verschillende heuristieken moeten worden geleerd. Vermits de heuristieken bovendien in eender welke sequentie kunnen voorkomen, kan men geen specifieke optimalisaties doorvoeren\footnote{We denken dan aan het feit dat een transitiefunctie tegelijk enkele variabelen kan uitrekenen die door een andere transitiefunctie alsnog zullen worden gebruikt.}. In deze context kan het nuttig zijn om een hyperheuristiek op verschillende processoren te laten werken.

\paragraph{}
Het idee is niet nieuw en werd reeds op enkele problemen toegepast. In \cite{conf/gecco/LeonMS08,conf/pdp/SeguraSL12} wordt een systeem ontwikkeld gebaseerd op genetische algoritmen. Het genetisch algoritme wordt onderverdeeld in enkele componenten: een \emph{master} bepaalt op basis van welke de componenten een \emph{worker} een lokaal genetisch algoritme zal laten werken. Wanneer een stopcriterium is bereikt, worden de resultaten doorgestuurd en zal de \emph{master} de \emph{worker} een nieuwe set componenten geven. In \cite{conf/pdp/SeguraSL12} voert men een studie uit op het adaptief vermogen van dit systeem. Er wordt aangetoond dat het systeem snel meer dan de helft van de beschikbare rekenkracht aan de sterkste genetische configuratie geeft. Het systeem maakt echter gebruik van een parameter en de robuustheid van deze parameter is nog niet aangetoond. Wanneer men met dit systeem het \prbm{Antenna Positioning Problem (APP)} oplost, stelt men vast dat de verbetering stagneert naarmate men het aantal processoren opdrijft. Bovendien boekt men op middellange termijn meestal slechtere prestaties met meerdere processoren.

\paragraph{}
Planningsproblemen werden opgelost met behulp van parallelle hyperheuristieken in \cite{Rattadilok04adistributed}. Men laat heuristieken uitvoeren op oplossingen door verschillende processoren. Deze processoren worden aangestuurd door een \emph{controller}. Verschillende \emph{controller}s communiceren met elkaar om ervaring uit te wisselen. Een probleem met deze benadering is dat de \emph{controller} zelf op een processor draait en meestal amper data verwerkt en bijgevolg ineffici\"ent omspringt met de aangeboden rekenkracht. Men stelt een consistente \emph{speed-up} vast maar de effici\"entie daalt naarmate het aantal processoren stijgt. Bovendien zijn de oplossing niet altijd kwalitatief beter.

\paragraph{}
\emph{Multiobjectivisation} -- het omzetten van \'e\'en evaluatiefunctie in verschillende evaluatiefuncties -- is de drijvende kracht in \cite{Luna08usinga}. Doordat elke processor een andere functie probeert te optimaliseren hoopt men tot een meer robuuste oplossingsmethode te komen.