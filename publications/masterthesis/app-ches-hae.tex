\section{\emph{HAEA: Hybrid Adaptive Evolutionary Algorithm} (\#10)}
\seclab{haea}

\subsection{Implementatie}
\emph{HAEA}\cite{chesc-haea,Gomez04selfadaptation} is een hyperheuristiek die werkt met behulp van drie oplossingen: de \emph{ouder}, het \emph{kind} (een oplossing gegenereerd uit de ouder) en de \emph{beste oplossing} tot dan toe. Het algoritme onderhoudt twee verzamelingen bij van heuristieken: \texttt{heu} houdt vier heuristieken bij uit de lijst van alle \abllhn{}. \texttt{loc} houdt vier heuristieken bij uit de lijst van \abls{} \abllhn{}.

\paragraph{}
Elke iteratie kiezen we een algemene \abh{} uit \texttt{heu} en een \abls{} \abllh{} uit \texttt{loc}. Deze \abhn{} worden na elkaar toegepast op de ouder. Indien dit kind beter is dan de ouder accepteren we het kind als nieuwe ouder en belonen we beide heuristieken. Dit doen we door de kans te verhogen dat ze de volgende maal opnieuw gekozen worden. Indien het kind niet beter presteert verlagen we deze kans. Na enkele iteraties zonder verbetering, worden er nieuwe elementen voor \texttt{heu} en \texttt{loc}. Indien een \abco{} \abllh{} wordt toegepast gebeurd dit met de ouder en de beste oplossing tot dusver.

\subsection{Kritiek}
\begin{itemize}
 \item \abco[C]{} wordt niet optimaal gebruikt: de kans is groot dat de huidige oplossing al dicht bij de beste oplossing zit.
 \item Beide \abhn{} worden beloond. Der oorzaak van een betere oplossing kan ook te wijten zijn aan de combinatie van de twee heuristieken.
 \item Wanneer nieuwe sets worden gekozen, gaat de opgedane ervaring verloren.
\end{itemize}