\chapter{Inleiding}
\chplab{inleiding}
\chapterquote{If a man will begin with certainties, he shall end in doubts; but if he will be content to begin with doubts, he shall end in certainties.}{Francis Bacon}

Optimalisatie is een belangrijk onderwerp in de artifici\"ele intelligentie. Bij de meeste algemene problemen in de artifici\"ele intelligentie, komen dan ook vormen van optimalisatie kijken. Het bepalen van de optimale parameters, het kortste pad of de kleinste correcte beslissingsboom zijn enkele voorbeelden.

\paragraph{}
Hoewel elke tak in de artifici\"ele intelligentie met deze problemen wordt geconfronteerd, bestaat er een specifieke tak die zich focust op deze problemen: ``Operationeel Onderzoek'' ofwel ``Operational Research''. De problemen die men tracht op te lossen zijn doorgaans moeilijk van aard: er is sprake van een groot aantal potenti\"ele oplossingen, het evalueren van een mogelijke oplossing is geen sinecure en de regels aan die een oplossing moet voldoen zijn vrij complex.

\section{Operationeel Onderzoek}

Operationeel onderzoek probeert doorgaans optimalisatieproblemen op te lossen in een zeer concrete en praktische setting. Belangrijke voorbeelden zijn bijvoorbeeld het genereren van productieplanningen, het optimaliseren van winstmarges, ... Omdat de zoekruimte dermate groot is, is het vinden van de exacte oplossing meestal niet mogelijk, men stelt zich in de meeste gevallen dan ook tevreden met een benaderende oplossing.

\paragraph{}
De laatste jaren is er een trend naar telkens meer ge\"integreerde systemen: algoritmen die onafhankelijk van het concrete probleem toch een oplossing kunnen uitrekenen. Deze evolutie is vooral te wijten aan de kost die de ontwikkeling van benaderingsalgoritmen met zich meebrengen. Omdat ontwikkelingskosten meestal hoger liggen dan de kosten ten gevolge door rekentijd is het financieel interessant om te investeren in systemen die met een minimale inspanning kunnen worden ontwikkeld. Belangrijk hierbij is de bouw van componenten, die men kan hergebruiken voor het effectief oplossen van andere problemen. Een belangrijke groep van dit soort systemen zijn \emph{Hyperheuristieken}.

\section{Parallelle algoritmen}

Het oplossen van problemen met probleemonafhankelijke algoritmen komt met een kostprijs: we verwachten dat op maat gemaakte algoritmen effici\"enter zullen werken en dus binnen een gegeven tijd betere oplossingen zullen voorstellen. Door te investeren in betere machines kan men dit verlies enigszins goedmaken. Men kan de kloksnelheid van een processor echter niet eindeloos opdrijven. Deze fysische beperking betekent bijgevolg een limiet op de resultaten die \'e\'en processor kan afleveren.

\paragraph{}
Parallelle algoritmen worden meestal ingezet wanneer de rekenkracht van \'e\'en processor tekortschiet. Door het programma op te splitsen in verschillende deelprogramma's die elk op \'e\'en processor draaien, hoopt men het rekenvermogen te kunnen opdrijven. Deze trend is bovendien ook zichtbaar door de introductie van meerdere \emph{kernen} ofwel \emph{cores} in moderne processoren. Vertaald naar optimalisatieproblemen hopen we dus met behulp van parallelle algoritmen tot betere oplossingen te komen die sneller uitgerekend kunnen worden.

\section{Onderzoeksvraag}

In deze masterthesis onderzoeken we of het mogelijk is om de prestaties van hyperheuristieken te verbeteren door deze parallel of verschillende processoren te laten draaien. Een belangrijk deel in deze onderzoeksvraag is welke componenten hiertoe kunnen bijdragen.

\paragraph{}
Vermits hyperheuristieken probleemonafhankelijk werken, moeten ze door middel van ervaring leren hoe men het probleem kan oplossen. Door verschillende processen ervaring met elkaar te laten uitwisselen, kan de tijd waarin men voornamelijk rekenkracht investeert in het opdoen van ervaring mogelijk worden verkort. Dit kan echter ook averechts werken: wanneer men te snel in een exploitatiefase komt, is het mogelijk dat men uitspraken doet op basis van een te kleine hoeveelheid ervaring. Bovendien kan men de kennis die \'e\'en processor heeft opgedaan niet altijd zomaar overdragen naar de andere processoren.

\section{Gevolgde methodiek}

In een eerste fase werd de beschikbare literatuur geraadpleegd. Er is veel literatuur te vinden rond het parallelliseren van metaheuristieken. Parallelle hyperheuristieken en hyperheuristieken in het algemeen zijn echter een vrij nieuw domein. Er bestaan dan ook slechts enkele concrete implementaties en weinig onderzoeken in verband die het effect van de verschillende paradigma testen. De belangrijkste werken worden dan ook vermeld in \sscref{defparhyhe}.

\paragraph{}
Om meer inzicht te verwerven in de structuur van hyperheuristieken werden 16 verschillende implementaties onderzocht. De resultaten van dit onderzoek worden gerapporteerd in \chpref{chesc} en \appref{chesc}. Op basis van deze studie werden er enkele hypotheses naar voren geschoven die mogelijk verklaren waarom sommige hyperheuristieken beter werken dan anderen.

\paragraph{}
Op basis van de opgedane kennis werd een systeem ge\"implementeerd die het parallelliseren van hyperheuristieken ondersteund. Dit systeem wordt uitvoerig besproken in \chpref{parhyf}. Ook werd een concrete hyperheuristiek die uit de vermelde studie, \emph{AdapHH}, aangepast zodat deze op dit systeem kan werken.

\paragraph{}
Om de invloed van de verschillende componenten te testen, werden twee problemen ge\"implementeerd. Door vervolgens de parameters aan te passen of componenten artificieel uit te schakelen, kunnen we de invloed van deze componenten op de prestaties van de hyperheuristiek nagaan. De resultaten van dit onderzoek staan in \chpref{resul}.

\paragraph{}
Enkele activiteiten met betrekking tot deze thesis werden niet opgenomen in dit werk. Dit komt omdat deze activiteiten een te kleine relevante bijdrage leverden. Het gaat hier in de eerste plaats om de implementatie van een systeem die het implementeren van een hyperheuristiek op een gestructureerde manier toelaat. Een programmeur kan door verschillende modulaire componenten samen te nemen een eigen hyperheuristiek bouwen. De bedoeling van dit systeem is om op termijn meer inzicht te verwerven in de invloed van sommige componenten op de prestaties van een hyperheuristiek.

\paragraph{}
Daarnaast werd een visualisatiesysteem genaamd \emph{ParVis} ge\"implementeerd. Dit systeem toont de toestand waarin de verschillende processoren zich bevinden samen met de berichten die onderling verstuurd worden. De bedoeling van dit softwarepakket is om de werking van een parallel algoritme beter te kunnen begrijpen. De broncode van dit softwaresysteem is te downloaden op \url{http://goo.gl/YPuMr}. Het softwarepakket omvat ook een voorbeeld die de werking van een parallel \algo{Sum-Product}-algoritme en een \emph{asynchrone GatherAll}\footnote{Zie \secref{mpimod}.} illustreert.


\section{Structuur}
In \chpref{defi} defini\"eren we het probleemdomein en de concepten hieromtrent. We bespreken in dit hoofdstuk ook de relevante literatuur samen met de huidige stand van zaken.

\paragraph{}
\chpref{chesc} omvat een onderzoek naar sequenti\"ele hyperheuristieken. We onderzoeken 16 verschillende implementaties die in 2011 werden voorgesteld op een competitie. Op elk van de implementaties wordt kritiek geleverd en op het einde schuiven we enkele hypotheses naar voren waarom sommige hyperheuristieken beter presteren dan anderen. Een deel van deze hypotheses wordt ook verder beargumenteerd met empirisch onderzoek.

\paragraph{}
We bespreken het \emph{ParHyFlex}-systeem in \chpref{parhyf}. Dit systeem ondersteund de bouw van parallelle hyperheuristieken en werd gebouwd op basis van de literatuurstudie in \chpref{defi} en de hypotheses die voortkomen uit \chpref{chesc}. Het systeem omvat drie grote componenten: \emph{uitwisselen van ervaring}, \emph{opdoen van ervaring} en \emph{afbakenen van een zoekruimte}.

\paragraph{}
In \chpref{paradaphh} werken we een concrete hyperheuristiek uit. \emph{ParAdapHH} is een parallelle variant van \emph{AdapHH}, de winnaar van de competitie die we in \chpref{chesc} hebben bestudeerd. We bespreken de werking van \emph{AdapHH}, maken een analyse over de verschillende bronnen van parallellisatie samen met de uiteindelijke implementatie.

\paragraph{}
In \chpref{resul} testen we het systeem met behulp van twee problemen: \prbm{Max-3Sat} en het \prbm{Finite Domain Constraint Optimization Problem}. Er worden verschillende testresultaten voorgesteld die de invloed van bepaalde componenten en parameters onderzoeken.



%%% Local Variables: 
%%% mode: latex
%%% TeX-master: "masterproef"
%%% End: 

%%% ASPELL CHECK 2013-05-20