\chapter{Inleiding}
\chplab{inleiding}

Optimalisatie is een belangrijk onderwerp in de artifici\"ele intelligentie. Allereerst komen bij heel wat algemene problemen in de artifici\"ele intelligentie vormen van optimalisatie kijken: we denken dan bijvoorbeeld aan het zoeken van een optimale set parameters bij een specifiek algoritme. Anderzijds kunnen we ook spreken van een specifieke tak in de artifici\"ele intelligentie die zich bezighoudt met optimalisatie problemen: ``Operationeel Onderzoek'' ofwel ``Operational Research''

\section{Operationeel Onderzoek}

Operationeel onderzoek probeert doorgaans optimalisatieproblemen op te lossen in een zeer concrete en praktische setting. Belangrijke voorbeelden zijn bijvoorbeeld het genereren van productieplanningen, het optimaliseren van winstmarges,... Al deze problemen worden altijd met algoritmen opgelost die een benadering voorstellen. De zoekruimte is immers veel te groot om volledig afgezocht te worden, en meestal volstaat de benadering om het probleem op te lossen.

\paragraph{}
De laatste jaren is er een trend om naar telkens meer ge\"integreerde systemen: algoritmen die onafhankelijk van het concrete probleem toch een oplossing kunnen uitrekenen. Deze evolutie is vooral te wijten aan de kost die de ontwikkeling van benaderingsalgoritmen met zich meebrengen. Door componenten slechts \'e\'enmaal te defini\"eren kunnen ontwikkelingskosten uitgespaard worden. Dergelijke systemen noemt men vaak \emph{Hyperheuristieken}.

\section{Parallelle algoritmen}

Het oplossen van problemen met probleem-onafhankelijke algoritmen komt met een kostprijs: we verwachten dat op maat gemaakte algoritmen effici\"enter zullen werken en daarom binnen een gegeven tijd betere oplossingen zullen voorstellen.

\paragraph{}
Parallelle algoritmen worden meestal ingezet wanneer de rekenkracht van \'e\'en processor tekortschiet om het probleem binnen een redelijke tijd op te lossen. Vertaald naar optimalisatieproblemen hopen we dus met behulp van parallelle algoritmen tot betere oplossingen te komen die sneller uitgerekend kunnen worden.

\section{Onderzoeksvraag}

In deze masterthesis onderzoeken we of het mogelijk is om de prestaties van hyperheuristieken te verbeteren door deze parallel of verschillende processoren te laten draaien. Een belangrijk component in deze onderzoeksvraag is welke componenten hiertoe kunnen bijdragen.

\paragraph{}
Vermits hyperheuristieken probleemonafhankelijk werken, moeten ze door middel van ervaring leren hoe men het probleem kan oplossen. Door verschillende processen ervaring met elkaar te laten uitwisselen kan de tijd waarin men voornamelijk rekenkracht investeert in het opdoen van ervaring mogelijk worden verkort. Dit kan echter ook averechts werken: wanneer men te snel in een exploitatiefase komt, is het mogelijk dat men uitspraken doet op basis van een te kleine hoeveelheid ervaring. Bovendien kan men de de kennis die \'e\'en processor heeft opgedaan niet zomaar overdragen naar de andere processoren.

\section{Gevolgde methodiek}


\section{Structuur}
In \chpref{1} defini\"eren we het probleemdomein en de concepten hieromtrent. Bij de meeste aspecten wordt ook relevante literatuur besproken samen met de huidige stand van zaken.

\paragraph{}
\chpref{2} omvat een onderzoek naar sequenti\"ele hyperheuristieken. We onderzoeken 16 verschillende implementaties die in 2011 werden voorgesteld op een competitie. Op elk van de implementaties wordt kritiek geleverd en op het einde schuiven we enkele hypotheses naar voren waarom sommige hyperheuristieken beter presteren dan anderen. Een deel van deze hypotheses wordt ook verder beargumenteerd met empirisch onderzoek.

\paragraph{}
We ontwikkelen een systeem genaamd \emph{ParHyFlex} in \chpref{3}. Dit systeem ondersteund de bouw van parallelle hyperheuristieken en werd gebouwd op basis van de literatuurstudie in \chpref{1} en de hypotheses die voortkomen uit \chpref{2}. Ook wordt een concrete hyperheuristiek uitgewerkt: \emph{ParAdapHH} is een parallelle variant van \emph{AdapHH}, de winnaar van de competitie die we in \chpref{2} beschouwen.

\paragraph{}
In \chpref{4} testen we het systeem met behulp van twee problemen: \prbm{Max-3Sat} en het \prbm{Finite Domain Constraint Optimization Problem}. Er worden verschillende testresultaten voorgesteld die de invloed van bepaalde componenten en parameters onderzoeken.



%%% Local Variables: 
%%% mode: latex
%%% TeX-master: "masterproef"
%%% End: 
