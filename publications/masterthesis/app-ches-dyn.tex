\section{\emph{DynILS: Dynamic Iterated Local Search} (\#13)}
\seclab{dyn-ils}

\subsection{Implementatie}
\emph{Dynamic Iterated Local Search}\cite{chesc-dynils,journals/orsnz/ksosils} is een hyperheuristiek die gebaseerd is op het gekende \emph{Iterative Local Search}\cite{Lourenco02iteratedlocal}-principe. Men kiest telkens uniform een \abpte{} \abllh{} en past vervolgens alle \abls{} \abhn{} toe.

\paragraph{}
Men introduceert met twee kleine wijzigingen: vermits de \abhn{} parameters hebben, probeert het algoritme deze parameters te optimaliseren. Hiertoe houdt men een vector voor een discrete set waardes voor deze parameters bij. Met behulp van deze vector rekent men dan de kans uit dat de parameter op een bepaalde waarde wordt gezet. Indien een \abllh{} tot een betere oplossing komt verhoogt men de kans van de gekozen parameterwaarde. In het andere geval neemt de kans af.

\paragraph{}
Een tweede aanpassing is de \emph{non-improvement bias}: de parameter wordt evenredig verhoogt met het aantal opeenvolgende iteraties waarin het algoritme faalt om de oplossing te verbeteren. Door de parameter te verhogen zoeken we een groter gebied af met \abls{} en muteren we de oplossing ook sterker. Hierdoor hoopt men sneller uit een lokaal optimum te ontsnappen.
\subsection{Kritiek}
\begin{itemize}
 \item Selectie van de \emph{perturbatie} is vrij eenvoudig: er wordt geen moeite gedaan om een bepaalde kansverdeling te leren.
 \item Vaste volgorde bij het toepassen van \abls{}: hierdoor kan de ene \abls{} \abh{} de andere consistent blokkeren.
\end{itemize}