\chapter{De eerste bijlage}
\label{app:A}
In de bijlagen vindt men de data terug die nuttig kunnen zijn voor de
lezer, maar die niet essentieel zijn om het betoog in de normale tekst te
kunnen volgen. Voorbeelden hiervan zijn bronbestanden,
configuratie-informatie, langdradige wiskundige afleidingen, enz.

In een bijlage kunnen natuurlijk ook verdere onderverdelingen voorkomen,
evenals figuren en referenties\cite{h2g2}.

%\section{Meer lorem}
%\lipsum[50]

%\subsection{Lorem 15--17}
%\lipsum[15-17]

%\subsection{Lorem 18--19}
%\lipsum[18-19]

%\section{Lorem 51}
%\lipsum[51]

%%% Local Variables: 
%%% mode: latex
%%% TeX-master: "masterproef"
%%% End: 
