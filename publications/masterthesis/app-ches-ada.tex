\section{\emph{AdapHH: Adaptive Hyper-heuristic} (\#1)}
\label{sss:adaphh}
\subsection{Implementatie}
AdapHH\cite{chesc-adaphh,chesc-adaphh2,348072} werkt met een populatie van \abhn{} die de \emph{Adaptive Dynamic Heuristic Set (ADHS)} wordt genoemd. Het is de bedoeling dat deze set heuristieken zich aanpast zodat telkens de op dat moment interessantste \abhn{} in de set zitten. Hiervoor deelt men de tijd op in fases. \abh[H]{} worden op het einde van een fase ge\"evalueerd als een gewogen som van vier metrieken: het aantal maal dat men een tot dan toe beste oplossing aanreikte, de totale fitness verbetering, de totale fitness verslechtering en de gespendeerde tijd. Bovendien maakt men een onderscheid tussen de bereikte resultaten in de fase en over de volledige termijn dat de hyperheuristiek draait: de gewichten van de effecten in de fase zijn wegen significant zwaarder door. Op het einde van een fase komt een deel van de \abhn{} in de ``\emph{Tabu List}''\cite{DBLP:journals/informs/Glover89}. Indien een \abh{} bij herintroductie onmiddellijk weer verwijdert wordt, wordt het aantal tabu-fases met \'e\'en verhoogt voor deze \abh{}. Indien deze teller een maximum bereikt wordt de heuristiek definitief geschrapt. Ook \abhn{} die tijdens de fase geen betere oplossing vonden, rust een tijdelijk tabu. Een tweede manier waarop men \abhn{} selecteert is de zogenaamde ``\emph{Relay Hybridisation}'': Elke \abhn{} houdt ook een lijstje van 10 andere metaheuristieken bij die na de deze \abh{} kunnen worden uitgevoerd. Dit systeem wordt vooral op het einde van een fase actief. Bij de \emph{relay hybridisation} worden dus twee \abhn{} na elkaar uitgevoerd. Een ander belangrijk component in het systeem is \emph{move acceptance}: het al dan niet beschouwen van het resultaat van de \abh{} als de nieuwe oplossing. Dit systeem wordt het \emph{Adaptive Iteration Limited List-based Threshold Accepting (AILLA) system} genoemd. Betere oplossingen worden altijd geaccepteerd. In het geval we een slechtere oplossing optekenen zullen we doorgaans weigeren, tenzij de oplossing beter is dan deze beste oplossing van enkele fases terug. Het aantal fases dat we terug mogen kijken wordt systematisch opgehoogd wanneer we een bepaalde beweging niet accepteren. Parameters ten slotte worden aangepast door de \abhn{} te categoriseren en op basis hiervan de \emph{Intensity of Mutation} of de \emph{Depth of Search} in kleine stappen te verhogen of te verlagen.