\section{\emph{AdapHH: Adaptive Hyper-heuristic} (\#1)}
\seclab{adaphh}
\subsection{Implementatie}
AdapHH\cite{chesc-adaphh,chesc-adaphh2,348072} deelt de beschikbare tijd op in \emph{fases}. De hyperheuristiek onderhoudt een set van actieve \abhn{} die men de \emph{Adaptive Dynamic Heuristic Set (ADHS)} noemt en werkt volgens de principes van een \emph{Tabu Set}\cite{DBLP:journals/informs/Glover89}. Op het einde van een \emph{fase} worden slecht presterende \abhn{} voor een paar fasen \emph{tabu}.De evaluatie werkt een gewogen som van vier metrieken: het aantal maal dat men een tot dan toe beste oplossing aanreikte, de totale fitness verbetering, de totale fitness verslechtering en de gespendeerde tijd. Men maakt ook een onderscheid tussen resultaten over de hele termijn en de resultaten die tijdens de laatste fase werden gerealiseerd: de laatste wegen significant zwaarder door. Indien een \abh{} bij herintroductie onmiddellijk weer verwijdert wordt, wordt het aantal \emph{tabu-fases} met \'e\'en verhoogt voor deze \abh{}. Indien deze teller een maximum bereikt wordt de heuristiek definitief geschrapt. \abhn[H] die tijdens de laatste fase geen betere oplossing vonden, maar wel veel tijd gebruikten, worden onder bepaalde omstandigheden ook tijdelijk uitgesloten.

\paragraph{}
Een tweede manier waarop men \abhn{} selecteert is de zogenaamde ``\emph{Relay Hybridisation}''. \emph{Relay Hybridisation} werkt met een \emph{Learning Automaton}\cite{RePEc:cla:levarc:481} waarbij de \emph{acties} de verschillende \abhn{} voorstellen. Op basis van deze \emph{Learning Automaton} wordt een eerste \abh{} geselecteerd. Elke \abh{} houdt een lijst bij van andere \abh{} die tot sterke resultaten leiden wanneer ze na de eerste \abh{} worden toegepast. Uit de lijst wordt een tweede \abh{} geselecteerd die dan onmiddellijk na de eerste \abh{} wordt toegepast. De \emph{Learning Automaton} beloont een \abh{} wanneer door dit systeem een nieuwe beste oplossing wordt gevonden.

\paragraph{}
Na het uitvoeren van \'e\'en van de twee mechanismen moet worden beslist of de oplossing de nieuwe actieve oplossing wordt. Dit is de taak van het \emph{Adaptive Iteration Limited List-based Threshold Accepting (AILLA)}-systeem. Betere oplossingen worden altijd geaccepteerd. Slechtere oplossing worden doorgaans geweigerd tenzij de oplossing beter is dan een historisch beste oplossing. Het aantal oplossingen die teruggekeken wordt hangt af van het aantal pogingen die al ondernomen zijn om een nieuwe oplossing te accepteren.

\paragraph{}
De \abh{}-parameters ten slotte worden aangepast door de \abhn{} te categoriseren op basis van hun prestaties. De \emph{Intensity of Mutation} en \emph{Depth of Search} worden in kleine stappen verhoogt of verlaagt op basis van de categorie waarin de \abh{} zich op dat moment bevindt.