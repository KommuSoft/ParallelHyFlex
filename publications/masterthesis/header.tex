\usepackage{amssymb,amsmath,slantsc,bbm,tikz,tocloft,etoolbox,hyphenat,pdfpages, amsthm,graphicx,cite,multicol}
\usepackage[algochapter,ruled,vlined]{algorithm2e}
\usepackage[showframe]{geometry}
\usepackage[all]{xy}

\setlength{\parskip}{1.3ex plus 0.2ex minus 0.2ex}
\setlength{\parindent}{0pt}

\SetKwInput{KwData}{Data}

\SetAlgorithmName{Algoritme}{willemvanonsem}{Lijst van algoritmes}

\usetikzlibrary{shapes,fit,calc}

\newcommand{\baselab}[2]{\label{#1:#2}}
\newcommand{\baseref}[3]{\textsc{\nohyphens{#3}~\textup{\ref{#1:#2}}}}

\newcommand{\deflab}[1]{\baselab{def}{#1}}
\newcommand{\defref}[1]{\baseref{def}{#1}{Definitie}}
\newcommand{\imglab}[1]{\baselab{fig}{#1}}
\newcommand{\imgref}[1]{\baseref{fig}{#1}{Figuur}}
\newcommand{\tbllab}[1]{\baselab{tbl}{#1}}
\newcommand{\tblref}[1]{\baseref{tbl}{#1}{Tabel}}
\newcommand{\alglab}[1]{\baselab{alg}{#1}}
\newcommand{\algref}[1]{\baseref{alg}{#1}{Algoritme}}
\newcommand{\eqnlab}[1]{\baselab{eqn}{#1}}
\newcommand{\eqnref}[1]{\textsc{Vergelijking~(\ref{eqn:#1})}}
\newcommand{\thelab}[1]{\baselab{the}{#1}}
\newcommand{\theref}[1]{\baseref{the}{#1}{Theorema}}
\newcommand{\chplab}[1]{\baselab{chp}{#1}}
\newcommand{\chpref}[1]{\baseref{chp}{#1}{Hoofdstuk}}
\newcommand{\applab}[1]{\baselab{app}{#1}}
\newcommand{\appref}[1]{\baseref{app}{#1}{Appendix}}
\newcommand{\seclab}[1]{\baselab{sec}{#1}}
\newcommand{\secref}[1]{\baseref{sec}{#1}{Sectie}}
\newcommand{\ssclab}[1]{\baselab{ssc}{#1}}
\newcommand{\sscref}[1]{\baseref{ssc}{#1}{Subsectie}}
\newcommand{\ssslab}[1]{\baselab{sss}{#1}}
\newcommand{\sssref}[1]{\baseref{sss}{#1}{Subsubsectie}}

\newtheorem{defnition}{Definitie}
\numberwithin{defnition}{chapter}
\newtheorem{theorem}{Theorema}

\newcommand{\listdefinitionname}{Lijst van definities}
\newlistof{X}{eX}{\listdefinitionname}
\newenvironment{definition}[1][]{
  \addcontentsline{eX}{figure}{\protect\numberline{\thechapter}#1}
\begin{defnition}[#1]}
{\end{defnition}}

\makeatletter
\preto\chapter{\addtocontents{def}{\protect\addvspace{10\p@}}}%
\makeatother

\newcommand{\mathbftxt}[1]{\ensuremath{\mbox{\textbf{#1}}}}

\newcommand{\brak}[1]{\ensuremath{\left(#1\right)}}
\newcommand{\fbrk}[1]{\ensuremath{\left[#1\right]}}
\newcommand{\tupl}[1]{\ensuremath{\left\langle #1\right\rangle}}
\newcommand{\accl}[1]{\ensuremath{\left\{ #1\right\}}}
\newcommand{\abs}[1]{\ensuremath{\left| #1\right|}}
\newcommand{\dabs}[2][]{\ensuremath{\left\| #2\right\|_{#1}}}
\newcommand{\ceil}[1]{\ensuremath{\left\lceil #1\right\rceil}}
\newcommand{\floor}[1]{\ensuremath{\left\lfloor #1\right\rfloor}}

\newcommand{\conditional}[2]{\begin{array}{cc}#1&\mbox{(#2)}\end{array}}
\newcommand{\guards}[1]{\left\{\begin{array}{ll}#1\end{array}\right.}

\newcommand{\fun}[2]{\ensuremath{#1\brak{#2}}}
\newcommand{\funf}[2]{\ensuremath{#1\fbrk{#2}}}
\newcommand{\funm}[2]{\fun{\mbox{#1}}{#2}}
\newcommand{\funsig}[3]{#1:#2\rightarrow #3}
\newcommand{\funsigimp}[5]{#1:#2\rightarrow #3:#4\mapsto #5}

\newcommand{\bigoh}[1]{\fun{\mathcal{O}}{#1}}

\newcommand{\xstar}{\ensuremath{x^{\star}}}
\newcommand{\xdot}{\ensuremath{x^{\circ}}}
\newcommand{\calXop}{\calX^{\star}}

\newcommand{\argmin}{\ensuremath{\mbox{argmin}}}
\newcommand{\mean}[2][]{\funf{\EEE_{#1}}{#2}}
\newcommand{\krdelta}[1]{\funf{\delta}{#1}}
\newcommand{\Prob}[1]{\funf{\Pr}{#1}}

\newcommand{\comp}[1]{\textsc{\mbox{#1}}}
\newcommand{\prbm}[1]{\textsc{#1}}
\newcommand{\algo}[1]{\nohyphens{\textsc{#1}}}

\newcommand{\AAA}{\mathbb{A}}
\newcommand{\BBB}{\mathbb{B}}
\newcommand{\CCC}{\mathbb{C}}
\newcommand{\DDD}{\mathbb{D}}
\newcommand{\EEE}{\mathbb{E}}
\newcommand{\FFF}{\mathbb{F}}
\newcommand{\GGG}{\mathbb{G}}
\newcommand{\HHH}{\mathbb{H}}
\newcommand{\III}{\mathbb{I}}
\newcommand{\JJJ}{\mathbb{J}}
\newcommand{\KKK}{\mathbb{K}}
\newcommand{\LLL}{\mathbb{L}}
\newcommand{\MMM}{\mathbb{M}}
\newcommand{\NNN}{\mathbb{N}}
\newcommand{\OOO}{\mathbb{O}}
\newcommand{\PPP}{\mathbb{P}}
\newcommand{\QQQ}{\mathbb{Q}}
\newcommand{\RRR}{\mathbb{R}}
\newcommand{\SSS}{\mathbb{S}}
\newcommand{\TTT}{\mathbb{T}}
\newcommand{\UUU}{\mathbb{U}}
\newcommand{\VVV}{\mathbb{V}}
\newcommand{\WWW}{\mathbb{W}}
\newcommand{\XXX}{\mathbb{X}}
\newcommand{\YYY}{\mathbb{Y}}
\newcommand{\ZZZ}{\mathbb{Z}}

\newcommand{\zeromatrix}[1][]{\boldsymbol{0}_{#1}}
\newcommand{\identitymatrix}[1][]{\boldsymbol{I}_{#1}}
\newcommand{\onematrix}[1][]{\boldsymbol{1}_{#1}}

\newcommand{\calA}{\mathcal{A}}
\newcommand{\calB}{\mathcal{B}}
\newcommand{\calC}{\mathcal{C}}
\newcommand{\calD}{\mathcal{D}}
\newcommand{\calE}{\mathcal{E}}
\newcommand{\calF}{\mathcal{F}}
\newcommand{\calG}{\mathcal{G}}
\newcommand{\calH}{\mathcal{H}}
\newcommand{\calI}{\mathcal{I}}
\newcommand{\calJ}{\mathcal{J}}
\newcommand{\calK}{\mathcal{K}}
\newcommand{\calL}{\mathcal{L}}
\newcommand{\calM}{\mathcal{M}}
\newcommand{\calN}{\mathcal{N}}
\newcommand{\calO}{\mathcal{O}}
\newcommand{\calP}{\mathcal{P}}
\newcommand{\calQ}{\mathcal{Q}}
\newcommand{\calR}{\mathcal{R}}
\newcommand{\calS}{\mathcal{S}}
\newcommand{\calT}{\mathcal{T}}
\newcommand{\calU}{\mathcal{U}}
\newcommand{\calV}{\mathcal{V}}
\newcommand{\calW}{\mathcal{W}}
\newcommand{\calX}{\mathcal{X}}
\newcommand{\calY}{\mathcal{Y}}
\newcommand{\calZ}{\mathcal{Z}}

\newcommand{\BoolSet}{\BBB}
\newcommand{\NatSet}{\NNN}
\newcommand{\RealSet}{\RRR}
\newcommand{\OpProblem}{\Pi}
\newcommand{\ConfigSet}{\calX}
\newcommand{\ConfigValSet}{\ConfigSet'}
\newcommand{\ConfigOpSet}{\ConfigSet^{\star}}
\newcommand{\sol}{x}
\newcommand{\bestSol}{\sol^{\star}}
\newcommand{\goodSol}{\sol^{\circ}}
\newcommand{\SolSet}{\calS}
\newcommand{\PopSet}{\calP}
\newcommand{\HypSet}{\calH}
\newcommand{\TranSet}{\calT}
\newcommand{\hcfun}{c}
\newcommand{\evalfun}{f}
\newcommand{\evalfuna}{f'}
\newcommand{\hittime}{\theta}
\newcommand{\phittime}{\Theta}
\newcommand{\neighbr}{\calN}
\newcommand{\nvar}{n}
\newcommand{\VarDom}{A}
\newcommand{\PopChain}{\mathfrak{P}}
\newcommand{\evl}[1][]{\ensuremath{\vec{\mu}_{#1}}}%left eigenvector
\newcommand{\evr}[1][]{\ensuremath{\vec{\nu}_{#1}}}%right eigenvector
\newcommand{\ev}[1][]{\ensuremath{\lambda_{#1}}}%right eigenvalue
\newcommand{\isdefinedas}{\ensuremath{:=}}%right eigenvalue
\newcommand{\smbox}[1]{\ensuremath{\mbox{\small{#1}}}}%right eigenvalue
\newcommand{\fitnessval}{\ensuremath{v}}
\newcommand{\ntarpop}{\ensuremath{g}}
\newcommand{\npop}{\ensuremath{N}}
\newcommand{\transmat}{\ensuremath{P}}
\newcommand{\bridgemat}{\ensuremath{B}}
\newcommand{\removmat}{\ensuremath{\hat{P}}}
\newcommand{\probdistmeta}[1]{\ensuremath{\vec{\alpha}_{#1}}}
\newcommand{\probdistgoalmeta}[1]{\ensuremath{\vec{\alpha}^G_{#1}}}
\newcommand{\probdistnormmeta}[1]{\ensuremath{\vec{\hat{\alpha}}_{#1}}}
\newcommand{\arbitraryval}{\ensuremath{m}}
\newcommand{\arbitraryvec}{\ensuremath{\vec{x}}}
\newcommand{\metapa}{\ensuremath{\sigma}}
\newcommand{\metapb}{\ensuremath{\lambda}}

\SetKw{true}{\mathbftxt{true}}
\SetKw{fals}{\mathbftxt{false}}
\SetKw{nult}{\mathbftxt{null}}

\SetKwBlock{ParFor}{parfor}{end}
\SetKwBlock{WhnRcv}{When receiving}{}
\SetKwBlock{myalg}{Algorithm}{end}
\SetKwBlock{myproc}{Procedure}{end}

\SetKwFunction{pinit}{Initialiseer}
\SetKwFunction{pgata}{GatherAll}
\SetKwFunction{predy}{Gereed}
\SetKwFunction{prest}{Reset}
\SetKwFunction{pzedt}{ZendData}
\SetKwFunction{psend}{send}
\SetKwFunction{pisnd}{isend}
\SetKwFunction{precv}{receive}
\SetKwFunction{parry}{array}

\SetKwData{dres}{resultaat}
\SetKwData{ddim}{dimensies}
\SetKwData{dbas}{basis}
\SetKwData{didq}{id}
\SetKwData{dsnd}{zender}
\SetKwData{drcv}{ontvanger}
\SetKwData{dmsg}{bericht}
\SetKwData{dodt}{eigenData}
\SetKwData{dprt}{partner}

\newcommand{\chapterquote}[2]{\begin{figure*}[htb]\centering\begin{tikzpicture}\node[text width=\textwidth-1.75 cm,anchor=center] (Q) at (0,0) {\Large\textit{#1}};\node[gray,anchor=north east] (Ql) at (Q.north west) {\Huge\textbf{``}};\node[gray,anchor=north west] (Qr) at (Q.south east) {\Huge\textbf{''}};\node[black!80,anchor=north east] (Qa) at (Qr.north west) {\small - #2};\end{tikzpicture}\end{figure*}}

\selectcolormodel{gray}
\newcommand{\importtikz}[4][1.00]{\begin{figure}[hbt]\begin{center}\def\sc{#1}\input{tikzpictures/#2}\caption{#4}\imglab{#3}\end{center}\end{figure}}
\newcommand{\importalgo}[3]{\begin{algorithm}[hbt]\input{algorithms/#1}\caption{#2}\alglab{#3}\end{algorithm}}

\newcommand{\drawmem}[3]{
\node[draw,rectangle,fill=gray!20,minimum width=\sc*0.5*#2 cm, minimum height=\sc*0.5cm] (#1) at (0,0.5) {};
\foreach\x in {2,...,#2} {
  \draw (0.5*\x-0.25*#2-0.5,0.25) -- ++(0,0.5);
}
\foreach\x in {#3,...,#2} {
  \draw (0.5*\x-0.25*#2-0.5,0.25) -- ++(0.5,0.5);
}
\foreach\x in {1,...,#2} {
  \node[minimum height=\sc*0.5 cm,minimum width=\sc*0.5 cm] (#1\x) at (0.5*\x-0.25*#2-0.25,0.5) {$s_{\x}$};
}
\draw (#1.north) node[anchor=south] {\small{Geheugen}};
}

\tikzset{hypothesis/.style={draw=black,regular polygon,regular polygon sides=3,scale=0.7,fill=gray,inner sep=0 pt},solution/.style={draw=black,circle,scale=0.7,inner sep=0 pt},outstream/.style={dashed},instream/.style={dotted},llh/.style={circle,fill=black,thick,draw=gray}}

\newcommand{\ubridge}[4][1]{\nbridge[-#1]{#2}{#3}{#4}}
\newcommand{\nbridge}[4][1]{\draw[#2] (#3) .. controls ($(#3)+(0,#1)$) and ($(#4)+(0,#1)$) .. (#4);}
\newcommand{\nubridge}[4][1]{\draw[#2] (#3) .. controls ($(#3)+(0,#1)$) and ($(#4)-(0,#1)$) .. (#4);}
\newcommand{\unbridge}[4][1]{\nubridge[-#1]{#2}{#3}{#4}}

\newcommand{\ubridgearrow}[3][1]{\ubridge[#1]{->}{#2}{#3}}
\newcommand{\nbridgearrow}[3][1]{\nbridge[#1]{->}{#2}{#3}}
\newcommand{\unbridgearrow}[3][1]{\unbridge[#1]{->}{#2}{#3}}
\newcommand{\nubridgearrow}[3][1]{\nubridge[#1]{->}{#2}{#3}}

\newcommand{\defrect}[5][]{\node[rectangle,draw=black,minimum width=#4, minimum height=#5,#1] (#2) at (#3) {};}

\newcommand{\drawcube}[2]{
\coordinate (A) at (0,0,0);
\coordinate (B) at (0,0,#1);
\coordinate (C) at (0,#1,0);
\coordinate (D) at (0,#1,#1);
\coordinate (E) at (#1,0,0);
\coordinate (F) at (#1,0,#1);
\coordinate (G) at (#1,#1,0);
\coordinate (H) at (#1,#1,#1);
\foreach \x/\t in {#2} {
  \fill (\x) circle (0.1 cm) node[anchor=south east]{\t};
}
\foreach \x/\y in {B/D,B/F,C/D,C/G,D/H,E/F,E/G,F/H,G/H} {
  \draw (\x) -- (\y);
}
\draw[dashed] (B) -- (A);
\draw (C) -- (C |- D);
\draw[dashed] (C |- D) -- (A);
\draw (E) -- (E -| F);
  \draw[dashed] (E -| F) -- (A);
}

\newcommand{\edgeinteraction}[3][0.25,0]{
\draw[<->] ($0.85*(#2)+0.15*(#3)+(#1)$) -- ($0.15*(#2)+0.85*(#3)+(#1)$);
}