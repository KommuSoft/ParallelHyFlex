\section{\emph{VNS-TW: Variable Neighborhood Search-based} (\#2)}
\label{sss:vns-tw}
\subsection{Implementatie}
\emph{VNS-TW}\cite{chesc-vns-tw} is zoals de naam doet vermoeden gebaseerd op ``\emph{Variable Neighborhood Search}''\cite{hom/vns}. Het is een populatie-gebaseerd algoritme en werkt op basis van vier stappen: \emph{shaking}, \emph{local search}, \emph{tabu} en \emph{vervangen-selectie}. Elke iteratie passen we toe op slechts \'e\'en oplossing. Aan het einde van de iteratie kunnen we eventueel van actieve oplossing veranderen. Bij stap 1 -- \emph{shaking} -- gebruikt men een toevallig gekozen heuristiek uit de set van \abmt{} of \abrr{} \abllhn{} en past men deze toe op de actieve oplossing. Vervolgens zullen we in de \emph{local search} fase een \abls{} \abh{} kiezen en deze toepassen op het resultaat van de \emph{shaking}-fase. De heuristiek wordt gekozen op basis van rang: eerst kiezen uit een set heuristieken die nog niet gekozen zijn of een beter resultaat opleverden (rang 1), daarna uit een reeks heuristieken die de vorige keer een gelijkwaardige (maar niet gelijke) oplossing opleverden (rang 0). Indien geen enkele heuristiek meer aan deze  voorwaarden voldoet, of we stellen na $c$ opeenvolgende pogingen geen verbetering vast, stoppen we de \emph{local search}-fase. In de volgende stap -- \emph{tabu} -- zullen we op basis van de afgeleverde oplossing, de \emph{shaking} heuristieken aanpassen. Dit doen we met behulp van ``\emph{Tabu Search}''\cite{DBLP:journals/informs/Glover89}: indien het eindresultaat slechter is dan het origineel komt de heuristiek in de \emph{tabu}-lijst terecht. Indien we tot een gelijkaardig resultaat komen doen we dit in 20\% van de gevallen. Tot slot passen we de populatie aan: indien het resultaat beter is dan het origineel neemt dit zijn plaats in in de populatie. Indien het resultaat slechter is, zal de slechte oplossing in de populatie plaats maken voor dit resultaat. We kiezen de nieuwe actieve oplossing met behulp van ``\emph{2-tournament selection}''\cite{Miller95geneticalgorithms} uit de populatie.
\subsection{Kritiek}
\begin{itemize}
 \item Er bestaat een kleine kans dat het algoritme vastloopt op een lokaal optimum: vermits enkel de beste oplossingen worden geaccepteerd in de populatie.
\end{itemize}