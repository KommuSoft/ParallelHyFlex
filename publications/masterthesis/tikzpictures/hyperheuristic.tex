\begin{tikzpicture}[scale=\sc]
\node[draw=black,rectangle,minimum width=0.8*\textwidth,minimum height=2 cm] (HH) at (0,1.4) {};
\node[draw=black,rectangle,minimum width=0.8*\textwidth] (DB) at (0,0) {Domein-barri\`ere};
\node[draw=black,rectangle,minimum width=0.8*\textwidth,minimum height=3.6 cm] (PD) at (0,-2.2) {};
\begin{scope}[xshift=-0.2*\textwidth,yshift=-1.9 cm]
\node (HEU) at (0,0) {heuristieken};
\foreach \a in {0,30,...,330} {
  \node[llh] (H\a) at ({2*cos(\a)},{sin(\a)}) {};
}
\node[draw=black,rectangle] (PDO) at (0.4*\textwidth,0) {\begin{minipage}{5cm}\begin{itemize}[noitemsep,topsep=0pt,parsep=0pt,partopsep=0pt]
 \item Probleemrepresentatie
 \item Probleeminstanties
 \item Evaluatiefunctie
 \item Initi\"ele oplossing
 \item Andere...
\end{itemize}\end{minipage}};
\end{scope}
\begin{scope}[yshift=1.2 cm]
\node (HI) at (0,0) {\begin{minipage}{10cm}Verzamelen en beheren van probleemonafhankelijke informatie: aantal heuristieken, veranderingen in de evaluatiefunctie, nieuwe oplossingen, afstand tussen de oplossingen,...\end{minipage}};
\end{scope}
\draw (HH.north) node[anchor=north] {\textbf{Hyperheuristiek}};
\draw (PD.south) node[anchor=south] {\textbf{Probleemdomein}};
\draw[thick,->] (HEU) -- (HI.south -| HEU);
\draw[thick,->] (HI.south -| PDO) -- (PDO);
\end{tikzpicture}