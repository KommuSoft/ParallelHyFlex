\section{\emph{EPH: Evolutionary Programming Hyper-heuristic} (\#5)}
\seclab{eph}
\subsection{Implementatie}
De \emph{Evolutionary Programming Hyper-Heuristic}\cite{chesc-eph} werkt met twee populaties: een populatie oplossingen en een populatie van sequenties van \abhn{}. Beide populaties evolueren tegelijk. De populatie oplossing wordt ``random'' ge\"initialiseerd worden.

\paragraph{}
Bij de evaluatie van de populatie van de sequenties worden de sequenties toegepast op de populatie oplossingen en worden op die manier nieuwe oplossingen gegenereerd. Telkens nadat zo'n sequentie is toegepast op een oplossing. Zal men proberen dit resultaat in de populatie proberen te injecteren. Deze injectie is succesvol indien het de oplossing een betere fitness waarde heeft dan minstens \'e\'en oplossing in de populatie en de fitness-waarde nog niet voorkomt. Ter compensatie wordt de slechtste oplossing uit de populatie gehaald.

\paragraph{}
Een sequentie van \abllhn{} bestaat uit twee delen: een \abpt{}-gedeelte met een maximale lengte van twee, en een \abls{} gedeelte van variabele lengte. Naast de \abllhn{} die we toepassen bevat een sequentie ook informatie over de parameters. Verder dient een sequentie aan volgende voorwaarden te voldoen: indien we twee \abpt{}-\abhn{} beschouwen dienen ze verschillend te zijn; een \abpt{} \abh{} kan enkel op de eerste plaats staan. Bij het \abls{} gedeelte wordt een heuristiek ofwel \'e\'enmaal ofwel volgens een ``\emph{Variable Neighborhood Descent schema}''\cite{hom/vns} uitgevoerd.

\paragraph{}
De populatie van de sequenties wordt initieel random bevolkt en evolueert door middel van mutatie en selectie. Hiervoor worden vier types mutaties gebruikt die met uniforme kans worden gekozen: modificeren van de \abpt{}-parameters, modificeren van de \abls{}-parameters, verwijderen/toevoegen van een \abpt{}, permutatie van de \abls{} \abhn{}. Op elke sequentie wordt een \abmt{} toegepast. Daarna wordt via ``\emph{2-tournament selection}''\cite{Miller95geneticalgorithms} de populatie opnieuw gehalveerd: twee toevallig gekozen sequenties nemen het in enkele rondes tegen elkaar op. In een ronde worden ze op eenzelfde individu in de oplossingsverzameling toegepast. De sequentie die na de rondes het vaakst met de beste nieuwe oplossing komt, wordt geselecteerd in de nieuwe generatie van sequenties.

\subsection{Kritiek}
\begin{itemize}
 \item Redundante aspecten in een sequentie worden niet noodzakelijk ge\"elimineerd: het algoritme is niet effici\"ent met tijdsgebruik.
 \item Indien de populatie van oplossingen niet correct is afgesteld wordt veel rekentijd ineffici\"ent gebruikt.
\end{itemize}