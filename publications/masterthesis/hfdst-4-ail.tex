\section{AILLA}
\seclab{ailla-exp}

Nadat men met behulp van de hierboven beschreven mechanismes (\emph{ADHS} of \emph{RH}) een nieuwe oplossing heeft berekend, wordt beslist of deze oplossing de nieuwe actieve oplossing wordt. Dit is de verantwoordelijkheid van \emph{AILLA}. \emph{AILLA} beschouwd twee verschillende gevallen:
\begin{enumerate}
 \item In het geval de gegenereerde oplossing beter is dan de originele oplossing, wordt deze altijd geaccepteerd.
 \item Indien dit niet het geval is wordt de oplossing alleen geaccepteerd wanneer deze sterker is dan een historisch beste oplossing. Hiervoor onderhoud \emph{AILLA} een lijst van de laatste globaal beste oplossingen. Het aantal mislukte pogingen op rij om een oplossing te accepteren bepaald vervolgens hoe diep men terug in het verleden kijkt.
\end{enumerate}

\paragraph{}
\importtikz[1.4]{ailla}{ailla}{Werkingsprincipe van \emph{AILLA}.}
\imgref{ailla} illustreert dit principe. Op de figuur worden de verschillende mogelijke configuraties voorgesteld door de horizontale as. De verticale as geeft de fitness-waarde van de overeenkomstige configuratie weer. Door oplossingen te berekenen worden de overeenkomstige fitness-waardes opgeslagen in het \emph{AILLA} systeem. Deze waardes worden door de dunne grijze horizontale lijnen voorgesteld. Een oplossing zal altijd geaccepteerd worden wanneer deze zich onder de tot dan toe onderste horizontale lijn bevindt. Wanneer we een oplossing vanuit $x_4$ genereren ($x_4'$) zien we dat aan deze eis niet wordt voldaan. Daarom zullen we de eerste maal de oplossing verwerpen. Wanneer we dit de tweede maal een oplossing genereren ($x_4''$), moet deze zich enkel onder de op \'e\'en na onderste lijn bevinden. Opnieuw voldoet de oplossing niet aan de voorwaarde. Tot slot bevindt de derde oplossing ($x_5$) zich onder de derde onderste lijn. Daarom wordt deze oplossing wel geaccepteerd. Merk op dat $x_5$ een minder kwalitatieve oplossing voorstelt dan $x_4'$.

\paragraph{}
We kunnen dit systeem verreiken door in de lijst van historisch beste fitness-waardes ook de resultaten van andere processoren op te nemen. Hierdoor streven we naar sterkere stijgingen. Deze fitness-waarden zijn bovendien op termijn te realiseren: er worden immers ook oplossingen uitgewisseld waar andere processoren dan gebruik van kunnen maken. Door strenge grenzen op te leggen kan een processor veel iteraties rekenen alvorens een nieuwe oplossing zal geaccepteerd worden. Daarom werd dit systeem aangepast zodat hooguit de helft van de lijst bestaat uit vreemde fitness-waardes.