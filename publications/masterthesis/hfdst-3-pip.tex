\section{Probleemonafhankelijk gedeelte}

Bovenop het probleemafhankelijke gedeelte werkt een hyperheuristiek, een component die we los van \emph{ParHyFlex} kunnen zien. Het softwaresysteem kan hier ondersteuning bieden met ingebouwde en aanpasbare componenten.  In \emph{ParHyFlex} werden daarom de volgende systemen ge\"implementeerd: \emph{uitwisselen van oplossingen}, \emph{afbakenen van zoekruimtes}, \emph{genereren van ervaring} en \emph{onderhandelen over een nieuwe zoekruimte}. In de volgende subsecties zullen we deze taken verder bespreken.

\subsection{Uitwisselen van oplossingen}

Elke processor werkt met een eigen lokaal geheugen, maar reserveert ook plaats voor de geheugens van de andere processoren. Op het moment dat een nieuwe oplossing naar een lokale geheugencel geschreven wordt, zal op basis van een \emph{uitwisselingsstrategie} beslist worden met welke processoren deze oplossing zal worden gedeeld. De taak van het verzenden en ontvangen van een oplossing samen met een reeks uitwisselingsstrategie\"en wordt ondersteund door \emph{ParHyFlex}.

\subsection{Afbakenen van de zoekruimte}
 
De zoekruimte bewaken is ook een verantwoordelijkheid van \emph{ParHyFlex}. Hiervoor voorziet men twee sets van afdwingbare beperkingen: positieve en negatieve. Telkens wanneer er een nieuwe oplossing wordt gegenereerd\footnote{Of via uitwisseling in het geheugen wordt ingeladen.} zal \emph{ParHyFlex} alle beperkingen in de positieve set afdwingen en \'e\'en beperking uit de negatieve set. Om te vermijden dat de zoekruimte vaak verandert is het daarom belangrijk dat men niet te veel afdwingbare beperkingen in de negatieve set plaatst.

\paragraph{}
Het afdwingen gebeurt in een willekeurige volgorde. Beperkingen kunnen immers met elkaar interfereren: een eerste beperking kan een variabele op \'e\'en waarde zetten waarna de volgende beperkingen deze wijziging weer ongedaan maakt. Men kan dit probleem proberen op te lossen door alle permutaties uit te proberen in de hoop dat \'e\'en mutatie toch tot het correcte resultaat leidt. Dit is echter niet noodzakelijk zo, en bovendien vereist een dergelijke oplossing exponenti\"ele tijd. We nemen aan dat de beperkingen meestal minimaal met elkaar interfereren en dat een zoekruimte niet strikt moet worden bewaakt. De hierboven vernoemde strategie is niet verplicht. Men kan door een interface te implementeren een andere strategie hanteren.

\subsection{Genereren van ervaring}
 
Telkens wanneer \'e\'en van de processoren een nieuwe oplossing voortbrengt, kan deze oplossing -- samen met andere oplossingen -- worden omgezet in een afdwingbare beperking. Een processor kan echter niet alle beperkingen blijven bewaren: het uitwisselen van ervaring dient snel te gebeuren, we dienen een voldoende grote zoekruimte te behouden en bovendien kunnen we net een beperking genereren die het zoeken de foute kant opstuurt. Daarom maken we gebruik van een \emph{ervaring-set}, een set van vaste grootte waar gegenereerde beperkingen in worden bewaard.

\paragraph{}
De elementen in de set worden telkens ge\"evalueerd: wanneer er een nieuwe oplossing wordt gegenereerd, zal de \emph{ervaring-set} analyseren aan welke beperkingen de oplossing voldoet. Op basis van de fitness-waarde van de oplossingen kunnen de beperkingen dan ge\"evalueerd worden. Door de lijst van fitness-waardes op te delen in waardes waarbij de beperking wordt gerespecteerd en waardes waarin dat niet het geval is, ontstaan twee sets aan fitness-waardes.

\paragraph{}
Met een online algoritme\cite[p. 232]{citeulike:175026} berekenen we voor beide sets het gemiddelde en de variantie\footnote{We doen dit online om te voorkomen dat alle fitness-waardes moeten worden bijgehouden.}. We beschouwen beide sets dan ook als normaal verdeeld. We kunnen de kans uitrekenen dat dat een waarde uit een normaal verdeelde verdeling kleiner is dan een waarde uit een andere normale verdeling. Naarmate de kans groter wordt dat de fitness-waarde van oplossingen kleiner is conform de beperking, maken we de assumptie dat deze beperkingen sterker zijn.

\paragraph{}
Gegenereerde hypotheses zijn niet per definitie juist, zelfs indien oplossingen conform de beperking een betere fitness-waarde hebben. Daarom dienen we de set regelmatig van nieuwe hypotheses te voorzien, dit proces heet \emph{amnesie}. \emph{Amnesie} wordt op geregelde tijdstippen toegepast: oude hypotheses worden uit de set gehaald op plaats te maken voor nieuwe hypotheses. We wensen dat sterke hypothese meer kans maken om te overleven maar wel de kans lopen om op termijn te verdwijnen. Daarom rangschikken we de beperkingen op basis van hun evaluatie. De kans dat de hypothese vervolgens uit de set wordt gehaald berekenen we vervolgens op basis van de Benford-verdeling\cite{citeulike:748130} die enkel afhangt van de gesorteerde index.

\subsection{Onderhandelen over een nieuwe zoekruimte}

Elke processor houdt een \emph{ervaring-set} bij. Het is de bedoeling dat deze ervaring omgezet wordt in een nieuwe zoekruimte. Bovendien kan ervaring uitgewisseld worden met andere processoren zodat deze later ook hun zoekruimtes aanpassen. Anderzijds wensen we te voorkomen dat de zoekruimtes te homogeen worden en dus potentieel sterke oplossingen genegeerd worden. Dit zijn de taken van de \emph{onderhandelaar}.

\paragraph{}
De \emph{onderhandelaar} is een component die af en toe geactiveerd wordt. Een deel van de afdwingbare beperkingen worden uit de \emph{ervaring-set} gehaald om opgenomen te worden in het positieve component van de \emph{zoekruimte}. Deze beperkingen worden via groepscommunicatie doorgestuurd naar de andere processoren. Een deel van de ontvangen beperkingen wordt ge\"injecteerd in de \emph{ervaring-set} van de ontvangers. Een andere deel vormt de basis van het negatieve gedeelte van de \emph{zoekruimte}. Omdat een deel van de afdwingbare beperkingen vanaf dan in de \emph{ervaring-set} van de andere processoren wordt ge\"evalueerd (met een andere zoekruimte), is men in staat om zo'n beperking op een objectievere manier te evalueren\footnote{Sommige afdwingbare beperkingen leiden immers enkel tot sterke resultaten in een bepaalde \emph{zoekruimte}.}.