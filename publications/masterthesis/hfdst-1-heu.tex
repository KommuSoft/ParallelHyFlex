\section{Heuristieken}

Hoewel de meeste optimalisatieproblemen \comp{NP-hard} zijn, is in een praktische context de configuratie met een optimale evaluatiefunctie net van cruciaal belang. Voor de meeste toepassingen is een configuratie die aan de harde beperkingen voldoet en de fitness-waarde van de echte oplossing benadert voldoende. In dat geval wordt meestal een heuristiek ge\"implementeerd:

\begin{definition}[Heuristiek]
Een heuristiek is een programma die gegeven een optimalisatieprobleem $\OpProblem=\tupl{\ConfigSet,\hcfun,\evalfun}$ een oplossing berekent $\goodSol$ in een redelijke tijd. Doorgaans voldoet deze oplossing aan de harde beperkingen ($\fun{\hcfun}{\goodSol}=\true$) en ligt de voorgestelde oplossing $\goodSol$ in fitness-waarde niet ver van de werkelijke oplossing $\bestSol$.
\end{definition}

Deze definitie blijft redelijk vaag en geeft dan ook veel ruimte voor interpretatie. Doorgaans verwachten we dat het algoritme stop in polynomiale tijd en in de meeste gevallen worden er ook beperkingen gezet op hoe ver de fitness-waarde $\fun{\evalfun}{\goodSol}$ mag afwijken van de optimale fitness-waarde $\fun{\evalfun}{\bestSol}$, al zijn beide voorwaarden niet strikt noodzakelijk.