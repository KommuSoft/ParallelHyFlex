\batchmode
\documentclass[master=cws,masteroption=ai]{kulemt}
\RequirePackage{ifthen}

%
\providecommand{\abseq}[1][s]{#1equentieel}%
\providecommand{\abseqe}[1][s]{#1equenti\"ele}%
\providecommand{\abpar}[1][p]{#1arallel}%
\providecommand{\abpare}[1][p]{#1arallelle}%
\providecommand{\abh}[1][h]{#1euristiek}%
\providecommand{\abhn}[1][h]{#1euristieken}%
\providecommand{\abmh}[1][m]{#1etaheuristiek}%
\providecommand{\abmhn}[1][m]{#1etaheuristieken}%
\providecommand{\abhh}[1][h]{#1yperheuristiek}%
\providecommand{\abhhn}[1][h]{#1yperheuristieken}%
\providecommand{\abllh}[1][l]{#1ow-level heuristiek}%
\providecommand{\abllhn}[1][l]{#1ow-level heuristieken}%
\providecommand{\abrr}[1][r]{#1uin-recreate}%
\providecommand{\abls}[1][l]{#1ocal search}%
\providecommand{\abmt}[1][m]{#1utation}%
\providecommand{\abco}[1][c]{#1rossover}%
\providecommand{\abeff}[1][e]{#1ffici\"ent}%
\providecommand{\abeffe}[1][e]{#1ffici\"ente}%
\providecommand{\abieff}[1][i]{#1neffici\"ent}%
\providecommand{\abieffe}[1][i]{#1neffici\"ente}%
\providecommand{\abhf}{\emph{HyFlex}}%
\providecommand{\abjava}{\emph{Java}}%
\providecommand{\abchesc}{\emph{CHeSC}}%
\providecommand{\abchescy}{\emph{CHeSC2011}}%
\providecommand{\abpt}[1][p]{#1erturbatie}%
\providecommand{\abpts}[1][p]{#1erturbaties}%
\providecommand{\abpte}{perturberende}%
\providecommand{\abalg}[1][a]{#1lgoritme}%
\providecommand{\absu}[1][s]{\emph{#1peed-up}}%
\providecommand{\absus}[1][s]{\emph{#1peed-ups}}%
\providecommand{\abslsu}[1][s]{\emph{#1uperlineaire speed-up}}%
\providecommand{\ablsu}[1][l]{\emph{#1ineaire speed-up}} 
\setup{title={Parallelle Hyperheuristieken},
  author={Willem~Van~Onsem},
  promotor={Prof.\,dr.\ Bart~Demoen},
  assessor={Prof.\,dr.\,ir.\, Maurice~Bruynooghe\and Dr.\,ir.\ Marko van Dooren},%Ir.\,W. Eetveel\and W. Eetrest
  assistant={Prof.\,dr.\ Bart~Demoen}}%Ir.\ A.~Assistent \and D.~Vriend}
\setup{filingcard,
  translatedtitle={Parallel Hyper-heuristics},
  udc=681.3,
  shortabstract={
  Hyperheuristieken zijn een familie van algoritmen die een optimalisatieprobleem benaderend oplossen met behulp van Monte-Carlo simulaties. Op basis van een set transitiefuncties manipuleert het algoritme op een probleemonafhankelijke manier een oplossing tot deze van acceptabele kwaliteit is.\\
  Voor complexe optimalisatieproblemen vereist dit mechanisme veel rekentijd. Door het algoritme op verschillende processoren uit te voeren kunnen we deze rekentijd reduceren. Hiervoor stellen een systeem genaamd \emph{ParHyFlex} voor. Het systeem ondersteunt de implementatie van parallelle hyperheuristieken met behulp van verschillende concepten. Tussentijdse oplossingen worden uitgewisseld om zodat sterke eigenschappen van verschillende oplossingen kunnen worden gecombineerd. Daarnaast zal het systeem op basis van eerder beschouwde tussentijdse oplossingen ervaring opdoen over welke eigenschappen tot goede oplossingen leiden. Deze ervaring wordt met behulp van \emph{afdwingbare beperkingen} omgezet in een zoekruimte: een subset van de mogelijke configuraties die waarschijnlijk tot sterke oplossingen kan komen. Het systeem probeert te voorkomen dat te veel processoren op termijn in hetzelfde gebied naar oplossingen zoeken.\\Verder stellen we in dit werk ook \emph{ParAdapHH} voor, een parallelle variant van \emph{AdapHH}, een hyperheuristiek ontwikkeld door Mustafa M\i{}s\i{}r. De effectiviteit van het systeem en de verschillende deelcomponenten wordt vervolgens getest aan de hand van \prbm{MAX-3SAT}-problemen.}}%Hier komt een heel bondig abstract van hooguit 500 woorden. \LaTeX\ commando's mogen hier gebruikt worden. Blanco lijnen (of het commando \texttt{\string\pa r}) zijn wel niet toegelaten! \endgraf%en \prbm{Finite Domain Constraint Optimisation}
\usepackage{amssymb,amsmath,slantsc,bbm,tikz,tocloft,etoolbox,hyphenat,pdfpages,amsthm,graphicx,cite,multicol,ltablex,enumitem,todonotes,rotating,graphicx,tocvsec2}
\usepackage[algochapter,ruled,vlined]{algorithm2e}
\usepackage[all]{xy}



\setlength{\parskip}{1.3ex plus 0.2ex minus 0.2ex} 

\setlength{\parindent}{0pt} 
\setcounter{tocdepth}{2}


\SetKwInput{KwData}{Data}


\SetAlgorithmName{Algoritme}{willemvanonsem}{Lijst van algoritmen}


\usetikzlibrary{shapes,fit,calc}



\newcounter{tmpC} 
%
\providecommand{\baselab}[2]{\label{#1:#2}}%
\providecommand{\baseref}[3]{\textsc{\nohyphens{#3}~\textup{\ref{#1:#2}}}}%
\providecommand{\baserefs}[3]{
  \setcounter{tmpC}{0}
  \foreach \x in {#2} {\addtocounter{tmpC}{1}}
  \textsc{\nohyphens{#3}
	\foreach \x in {#2} {\addtocounter{tmpC}{-1}\ifthenelse{\value{tmpC} > 0}{\textup{\ref{#1:\x}},}{en \textup{\ref{#1:\x}}}}
  }~
}%
\providecommand{\baserefq}[3]{\textsc{\textup{\ref{#1:#2}}}} 

%
\providecommand{\rot}[1]{\begin{sideways}#1\end{sideways}} 

%
\providecommand{\deflab}[1]{\label{def:#1}}%
\providecommand{\defref}[1]{\textsc{\nohyphens{Definitie}~\textup{\ref{def:#1}}}}%
\providecommand{\imglab}[1]{\label{fig:#1}}%
\providecommand{\imgref}[1]{\textsc{\nohyphens{Figuur}~\textup{\ref{fig:#1}}}}%
\providecommand{\sfglab}[1]{\label{sfg:#1}}%
\providecommand{\sfgref}[1]{\textsc{\nohyphens{Deelfiguur}~\textup{\ref{sfg:#1}}}}%
\providecommand{\tbllab}[1]{\label{tbl:#1}}%
\providecommand{\tblref}[1]{\textsc{\nohyphens{Tabel}~\textup{\ref{tbl:#1}}}}%
\providecommand{\tblrefs}[1]{
  \setcounter{tmpC}{0}
  \foreach \x in {#1} {\addtocounter{tmpC}{1}}
  \textsc{\nohyphens{Tabellen}
	\foreach \x in {#1} {\addtocounter{tmpC}{-1}\ifthenelse{\value{tmpC} > 0}{\textup{\ref{tbl:\x}},}{en \textup{\ref{tbl:\x}}}}
  }~
}%
\providecommand{\alglab}[1]{\label{alg:#1}}%
\providecommand{\algref}[1]{\textsc{\nohyphens{Algoritme}~\textup{\ref{alg:#1}}}}%
\providecommand{\eqnlab}[1]{\label{eqn:#1}}%
\providecommand{\eqnref}[1]{\textsc{Vergelijking~(\ref{eqn:#1})}}%
\providecommand{\thelab}[1]{\label{the:#1}}%
\providecommand{\theref}[1]{\textsc{\nohyphens{Theorema}~\textup{\ref{the:#1}}}}%
\providecommand{\chplab}[1]{\label{chp:#1}}%
\providecommand{\chpref}[1]{\textsc{\nohyphens{Hoofdstuk}~\textup{\ref{chp:#1}}}}%
\providecommand{\applab}[1]{\label{app:#1}}%
\providecommand{\appref}[1]{\textsc{\nohyphens{Appendix}~\textup{\ref{app:#1}}}}%
\providecommand{\seclab}[1]{\label{sec:#1}}%
\providecommand{\secref}[1]{\textsc{\nohyphens{Sectie}~\textup{\ref{sec:#1}}}}%
\providecommand{\secrefq}[1]{\textsc{\textup{\ref{sec:#1}}}}%
\providecommand{\ssclab}[1]{\label{ssc:#1}}%
\providecommand{\sscref}[1]{\textsc{\nohyphens{Subsectie}~\textup{\ref{ssc:#1}}}}%
\providecommand{\ssslab}[1]{\label{sss:#1}}%
\providecommand{\sssref}[1]{\textsc{\nohyphens{Subsubsectie}~\textup{\ref{sss:#1}}}}%
\providecommand{\prblab}[1]{\label{prb:#1}} 



\newtheorem{defnition}{Definitie} 
\numberwithin{defnition}{chapter} 

\newtheorem{theorem}{Theorema} 
%
\providecommand{\listdefinitionname}{Lijst van definities} 
\newlistof{X}{eX}{Lijst van definities}

%
\newenvironment{definition}[1][]{
  \addcontentsline{eX}{figure}{\protect\numberline{\thechapter}#1}
\begin{theorem_type}[defnition][defnition][][][][]
[#1]}
{\end{theorem_type}
} 


\makeatletter
\preto\chapter{\addtocontents{def}{\protect\addvspace{10\p@}}}%
\makeatother

%
\providecommand{\mathbftxt}[1]{\ensuremath{\mbox{\textbf{#1}}}} 

%
\providecommand{\brak}[1]{\ensuremath{\left(#1\right)}}%
\providecommand{\fbrk}[1]{\ensuremath{\left[#1\right]}}%
\providecommand{\tupl}[1]{\ensuremath{\left\langle #1\right\rangle}}%
\providecommand{\accl}[1]{\ensuremath{\left\{ #1\right\}}}%
\providecommand{\abs}[1]{\ensuremath{\left| #1\right|}}%
\providecommand{\dabs}[2][]{\ensuremath{\left\| #2\right\|_{#1}}}%
\providecommand{\ceil}[1]{\ensuremath{\left\lceil #1\right\rceil}}%
\providecommand{\floor}[1]{\ensuremath{\left\lfloor #1\right\rfloor}}%
\providecommand{\transpose}[1]{\ensuremath{{#1}^{\top}}} 

%
\providecommand{\conditional}[2]{\begin{array}{cc}#1&\mbox{(#2)}\end{array}}%
\providecommand{\guards}[1]{\left\{\begin{array}{ll}#1\end{array}\right.} 

%
\providecommand{\fun}[2]{\ensuremath{#1\ensuremath{\left(#2\right)}}}%
\providecommand{\funf}[2]{\ensuremath{#1\ensuremath{\left[#2\right]}}}%
\providecommand{\funm}[2]{\ensuremath{\mbox{#1}\ensuremath{\left(#2\right)}}}%
\providecommand{\funsig}[3]{#1:#2\rightarrow #3}%
\providecommand{\funsigimp}[5]{#1:#2\rightarrow #3:#4\mapsto #5} 

%
\providecommand{\bigoh}[1]{\ensuremath{\mathcal{O}\ensuremath{\left(#1\right)}}} 

%
\providecommand{\xstar}{\ensuremath{x^{\star}}}%
\providecommand{\xdot}{\ensuremath{x^{\circ}}}%
\providecommand{\calXop}{\mathcal{X}^{\star}} 

%
\providecommand{\argmin}{\ensuremath{\mbox{argmin}}}%
\providecommand{\mean}[2][]{\ensuremath{\mathbb{E}_{#1}\ensuremath{\left[#2\right]}}}%
\providecommand{\krdelta}[1]{\ensuremath{\delta\ensuremath{\left[#1\right]}}}%
\providecommand{\Prob}[1]{\ensuremath{\Pr\ensuremath{\left[#1\right]}}} 

%
\providecommand{\comp}[1]{\textsc{\mbox{#1}}}%
\providecommand{\algo}[1]{\nohyphens{\textsc{#1}}}%
\providecommand{\auth}[2][]{\emph{\nohyphens{#2}}\ifthenelse{\equal{#1}{}}{}{\cite{#1}}}%
\providecommand{\prob}[1]{\textup{\textsc{#1}}}%
\providecommand{\prbm}[1]{\hyperref[prb:#1]{\textup{\textsc{#1}}}}%
\providecommand{\work}[1]{``\emph{#1}''} 

%
\providecommand{\AAA}{\mathbb{A}}%
\providecommand{\BBB}{\mathbb{B}}%
\providecommand{\CCC}{\mathbb{C}}%
\providecommand{\DDD}{\mathbb{D}}%
\providecommand{\EEE}{\mathbb{E}}%
\providecommand{\FFF}{\mathbb{F}}%
\providecommand{\GGG}{\mathbb{G}}%
\providecommand{\HHH}{\mathbb{H}}%
\providecommand{\III}{\mathbb{I}}%
\providecommand{\JJJ}{\mathbb{J}}%
\providecommand{\KKK}{\mathbb{K}}%
\providecommand{\LLL}{\mathbb{L}}%
\providecommand{\MMM}{\mathbb{M}}%
\providecommand{\NNN}{\mathbb{N}}%
\providecommand{\OOO}{\mathbb{O}}%
\providecommand{\PPP}{\mathbb{P}}%
\providecommand{\QQQ}{\mathbb{Q}}%
\providecommand{\RRR}{\mathbb{R}}%
\providecommand{\SSS}{\mathbb{S}}%
\providecommand{\TTT}{\mathbb{T}}%
\providecommand{\UUU}{\mathbb{U}}%
\providecommand{\VVV}{\mathbb{V}}%
\providecommand{\WWW}{\mathbb{W}}%
\providecommand{\XXX}{\mathbb{X}}%
\providecommand{\YYY}{\mathbb{Y}}%
\providecommand{\ZZZ}{\mathbb{Z}} 

%
\providecommand{\zeromatrix}[1][]{\boldsymbol{0}_{#1}}%
\providecommand{\identitymatrix}[1][]{\boldsymbol{I}_{#1}}%
\providecommand{\onematrix}[1][]{\boldsymbol{1}_{#1}} 

%
\providecommand{\calA}{\mathcal{A}}%
\providecommand{\calB}{\mathcal{B}}%
\providecommand{\calC}{\mathcal{C}}%
\providecommand{\calD}{\mathcal{D}}%
\providecommand{\calE}{\mathcal{E}}%
\providecommand{\calF}{\mathcal{F}}%
\providecommand{\calG}{\mathcal{G}}%
\providecommand{\calH}{\mathcal{H}}%
\providecommand{\calI}{\mathcal{I}}%
\providecommand{\calJ}{\mathcal{J}}%
\providecommand{\calK}{\mathcal{K}}%
\providecommand{\calL}{\mathcal{L}}%
\providecommand{\calM}{\mathcal{M}}%
\providecommand{\calN}{\mathcal{N}}%
\providecommand{\calO}{\mathcal{O}}%
\providecommand{\calP}{\mathcal{P}}%
\providecommand{\calQ}{\mathcal{Q}}%
\providecommand{\calR}{\mathcal{R}}%
\providecommand{\calS}{\mathcal{S}}%
\providecommand{\calT}{\mathcal{T}}%
\providecommand{\calU}{\mathcal{U}}%
\providecommand{\calV}{\mathcal{V}}%
\providecommand{\calW}{\mathcal{W}}%
\providecommand{\calX}{\mathcal{X}}%
\providecommand{\calY}{\mathcal{Y}}%
\providecommand{\calZ}{\mathcal{Z}} 

%
\providecommand{\BoolSet}{\mathbb{B}}%
\providecommand{\NatSet}[1][]{\ifthenelse{\equal{#1}{}}{\mathbb{N}}{\ensuremath{\mathbb{N}\ensuremath{\left[#1\right]}}}}%
\providecommand{\RealSet}{\mathbb{R}}%
\providecommand{\QbitSet}[1]{\ensuremath{\mathbb{Q}\ensuremath{\left[#1\right]}}}%
\providecommand{\OpProblem}{\Pi}%
\providecommand{\ConfigSet}{\mathcal{X}}%
\providecommand{\ConfigValSet}{\mathcal{X}'}%
\providecommand{\ConfigOpSet}{\mathcal{X}^{\star}}%
\providecommand{\sol}{x}%
\providecommand{\bestSol}{x^{\star}}%
\providecommand{\goodSol}{x^{\circ}}%
\providecommand{\SolSet}{\mathcal{S}}%
\providecommand{\PopSet}{\mathcal{P}}%
\providecommand{\HypSet}{\mathcal{H}}%
\providecommand{\TranSet}{\mathcal{T}}%
\providecommand{\hcfun}{c}%
\providecommand{\evalfun}{f}%
\providecommand{\evalfuna}{f'}%
\providecommand{\hittime}{\theta}%
\providecommand{\phittime}{\Theta}%
\providecommand{\neighbr}{\mathcal{N}}%
\providecommand{\nvar}{n}%
\providecommand{\VarDom}{A}%
\providecommand{\PopChain}{\mathfrak{P}}%
\providecommand{\evl}[1][]{\ensuremath{\vec{\mu}_{#1}}}%left eigenvector%
\providecommand{\evr}[1][]{\ensuremath{\vec{\nu}_{#1}}}%right eigenvector%
\providecommand{\ev}[1][]{\ensuremath{\lambda_{#1}}}%right eigenvalue%
\providecommand{\isdefinedas}{\ensuremath{:=}}%right eigenvalue%
\providecommand{\smbox}[1]{\ensuremath{\mbox{\small{#1}}}}%right eigenvalue%
\providecommand{\fitnessval}{\ensuremath{v}}%
\providecommand{\ntarpop}{\ensuremath{g}}%
\providecommand{\npop}{\ensuremath{N}}%
\providecommand{\transmat}{\ensuremath{P}}%
\providecommand{\bridgemat}{\ensuremath{B}}%
\providecommand{\removmat}{\ensuremath{\hat{P}}}%
\providecommand{\probdistmeta}[1]{\ensuremath{\vec{\alpha}_{#1}}}%
\providecommand{\probdistgoalmeta}[1]{\ensuremath{\vec{\alpha}^G_{#1}}}%
\providecommand{\probdistnormmeta}[1]{\ensuremath{\vec{\hat{\alpha}}_{#1}}}%
\providecommand{\arbitraryval}{\ensuremath{m}}%
\providecommand{\arbitraryvec}{\ensuremath{\vec{x}}}%
\providecommand{\metapa}{\ensuremath{\sigma}}%
\providecommand{\metapb}{\ensuremath{\lambda}}%
\providecommand{\bitvar}{\ensuremath{L}} 


\SetKw{true}{\ensuremath{\mbox{\textbf{true}}}}
\SetKw{fals}{\ensuremath{\mbox{\textbf{false}}}}
\SetKw{nult}{\ensuremath{\mbox{\textbf{null}}}}


\SetKwBlock{ParFor}{parfor}{end}
\SetKwBlock{WhnRcv}{When receiving}{}
\SetKwBlock{myalg}{Algorithm}{end}
\SetKwBlock{myproc}{Procedure}{end}


\SetKwFunction{pinit}{Initialiseer}
\SetKwFunction{pgata}{GatherAll}
\SetKwFunction{predy}{Gereed}
\SetKwFunction{prest}{Reset}
\SetKwFunction{pzedt}{ZendData}
\SetKwFunction{psend}{send}
\SetKwFunction{pisnd}{isend}
\SetKwFunction{precv}{receive}
\SetKwFunction{parry}{array}


\SetKwData{dres}{resultaat}
\SetKwData{ddim}{dimensies}
\SetKwData{dbas}{basis}
\SetKwData{didq}{id}
\SetKwData{dsnd}{zender}
\SetKwData{drcv}{ontvanger}
\SetKwData{dmsg}{bericht}
\SetKwData{dodt}{eigenData}
\SetKwData{dprt}{partner}

%
\providecommand{\chapterquote}[2]{\begin{figure*}[htb]\centering\begin{tikzpicture}\node[text width=\textwidth-1.75 cm,anchor=center] (Q) at (0,0) {\Large\textit{\nohyphens{#1}}};\node[gray,anchor=north east] (Ql) at (Q.north west) {\Huge\textbf{``}};\node[gray,anchor=north west] (Qr) at (Q.south east) {\Huge\textbf{''}};\node[black!80,anchor=north east] (Qa) at (Qr.north west) {\small - #2};\end{tikzpicture}\end{figure*}} 


\selectcolormodel{gray}%
\providecommand{\importtikz}[4][1.00]{\begin{figure}[hbt]\begin{center}\def \sc{#1}%
\input{tikzpictures/#2}\caption{#4}\label{fig:#3}\end{center}\end{figure}}%
\providecommand{\importalgo}[3]{\begin{algorithm}[hbt]%
\input{algorithms/#1}\caption{#2}\label{alg:#3}\end{algorithm}}%
\providecommand{\importgraphsub}[3]{
  \foreach \f/\o/\t/\l in {#1} {
	\subfigures[\t] {
	  \label{sfg:\l}
	  \includegraphics[\o]{grafieken/\f}
	}
  }
} 

%
\providecommand{\showgraph}[4][]{
 \begin{figure}[hbt]
  \centering
  \includegraphics[#1]{grafieken/#2.pdf}
  \caption{#3}
  \label{fig:#4}
 \end{figure}
} 

%
\providecommand{\inputresult}[2]{
 \begin{table}[hbt]
 %
\input{tables/data-#1.tex}
 \caption{#2}
 \label{tbl:#1}
 \end{table}
} 

%
\providecommand{\inputtable}[2]{
 \begin{table}[hbt]
 \centering
 %
\input{tables/#1.tex}
 \caption{#2}
 \label{tbl:#1}
 \end{table}
} 

%
\providecommand{\drawmem}[3]{
\node[draw,rectangle,fill=gray!20,minimum width=\sc*0.5*#2 cm, minimum height=\sc*0.5cm] (#1) at (0,0.5) {};
\foreach\x in {2,...,#2} {
  \draw (0.5*\x-0.25*#2-0.5,0.25) -- ++(0,0.5);
}
\foreach\x in {#3,...,#2} {
  \draw (0.5*\x-0.25*#2-0.5,0.25) -- ++(0.5,0.5);
}
\foreach\x in {1,...,#2} {
  \node[minimum height=\sc*0.5 cm,minimum width=\sc*0.5 cm] (#1\x) at (0.5*\x-0.25*#2-0.25,0.5) {$s_{\x}$};
}
\draw (#1.north) node[anchor=south] {\small{Geheugen}};
} 


\tikzset{hypothesis/.style={draw=black,regular polygon,regular polygon sides=3,scale=0.7,fill=gray,inner sep=0 pt},solution/.style={draw=black,circle,scale=0.7,inner sep=0 pt},outstream/.style={dashed},instream/.style={dotted},llh/.style={circle,fill=black,thick,draw=gray}}

%
\providecommand{\ubridge}[4][1]{\draw[#2] (#3) .. controls ($(#3)+(0,-#1)$) and ($(#4)+(0,-#1)$) .. (#4);}%
\providecommand{\nbridge}[4][1]{\draw[#2] (#3) .. controls ($(#3)+(0,#1)$) and ($(#4)+(0,#1)$) .. (#4);}%
\providecommand{\nubridge}[4][1]{\draw[#2] (#3) .. controls ($(#3)+(0,#1)$) and ($(#4)-(0,#1)$) .. (#4);}%
\providecommand{\unbridge}[4][1]{\draw[#2] (#3) .. controls ($(#3)+(0,-#1)$) and ($(#4)-(0,-#1)$) .. (#4);} 

%
\providecommand{\ubridgearrow}[3][1]{\draw[->] (#2) .. controls ($(#2)+(0,-#1)$) and ($(#3)+(0,-#1)$) .. (#3);}%
\providecommand{\nbridgearrow}[3][1]{\draw[->] (#2) .. controls ($(#2)+(0,#1)$) and ($(#3)+(0,#1)$) .. (#3);}%
\providecommand{\unbridgearrow}[3][1]{\draw[->] (#2) .. controls ($(#2)+(0,-#1)$) and ($(#3)-(0,-#1)$) .. (#3);}%
\providecommand{\nubridgearrow}[3][1]{\draw[->] (#2) .. controls ($(#2)+(0,#1)$) and ($(#3)-(0,#1)$) .. (#3);} 

%
\providecommand{\defrect}[5][]{\node[rectangle,draw=black,minimum width=#4, minimum height=#5,#1] (#2) at (#3) {};} 

%
\providecommand{\drawcube}[2]{
\coordinate (A) at (0,0,0);
\coordinate (B) at (0,0,#1);
\coordinate (C) at (0,#1,0);
\coordinate (D) at (0,#1,#1);
\coordinate (E) at (#1,0,0);
\coordinate (F) at (#1,0,#1);
\coordinate (G) at (#1,#1,0);
\coordinate (H) at (#1,#1,#1);
\foreach \x/\t in {#2} {
  \fill (\x) circle (0.1 cm) node[anchor=south east]{\t};
}
\foreach \x/\y in {B/D,B/F,C/D,C/G,D/H,E/F,E/G,F/H,G/H} {
  \draw (\x) -- (\y);
}
\draw[dashed] (B) -- (A);
\draw (C) -- (C |- D);
\draw[dashed] (C |- D) -- (A);
\draw (E) -- (E -| F);
  \draw[dashed] (E -| F) -- (A);
} 

%
\providecommand{\edgeinteraction}[3][0.25,0]{
\draw[<->] ($0.85*(#2)+0.15*(#3)+(#1)$) -- ($0.15*(#2)+0.85*(#3)+(#1)$);
} 


\setup{font=lm}


\usepackage[pdfusetitle,colorlinks,plainpages=false]{hyperref}




\usepackage[dvips]{color}


\pagecolor[gray]{.7}

\usepackage[latin1]{inputenc}



\makeatletter
\AtBeginDocument{\makeatletter
\input /home/kommusoft/Projects/parallelhyflex/ParallelHyFlex/publications/masterthesis/masterproef.aux
\makeatother
}
\AtBeginDocument{\makeatletter
\input /home/kommusoft/Projects/parallelhyflex/ParallelHyFlex/publications/masterthesis/inleiding.aux
\makeatother
}
\AtBeginDocument{\makeatletter
\input /home/kommusoft/Projects/parallelhyflex/ParallelHyFlex/publications/masterthesis/hfdst-1.aux
\makeatother
}
\AtBeginDocument{\makeatletter
\input /home/kommusoft/Projects/parallelhyflex/ParallelHyFlex/publications/masterthesis/hfdst-2.aux
\makeatother
}
\AtBeginDocument{\makeatletter
\input /home/kommusoft/Projects/parallelhyflex/ParallelHyFlex/publications/masterthesis/hfdst-3.aux
\makeatother
}
\AtBeginDocument{\makeatletter
\input /home/kommusoft/Projects/parallelhyflex/ParallelHyFlex/publications/masterthesis/hfdst-4.aux
\makeatother
}
\AtBeginDocument{\makeatletter
\input /home/kommusoft/Projects/parallelhyflex/ParallelHyFlex/publications/masterthesis/hfdst-5.aux
\makeatother
}
\AtBeginDocument{\makeatletter
\input /home/kommusoft/Projects/parallelhyflex/ParallelHyFlex/publications/masterthesis/besluit.aux
\makeatother
}
\AtBeginDocument{\makeatletter
\input /home/kommusoft/Projects/parallelhyflex/ParallelHyFlex/publications/masterthesis/app-ches.aux
\makeatother
}
\AtBeginDocument{\makeatletter
\input /home/kommusoft/Projects/parallelhyflex/ParallelHyFlex/publications/masterthesis/app-comm.aux
\makeatother
}
\AtBeginDocument{\makeatletter
\input /home/kommusoft/Projects/parallelhyflex/ParallelHyFlex/publications/masterthesis/app-lege.aux
\makeatother
}
\AtBeginDocument{\makeatletter
\input /home/kommusoft/Projects/parallelhyflex/ParallelHyFlex/publications/masterthesis/app-arti.aux
\makeatother
}
\AtBeginDocument{\makeatletter
\input /home/kommusoft/Projects/parallelhyflex/ParallelHyFlex/publications/masterthesis/app-post.aux
\makeatother
}

\makeatletter
\count@=\the\catcode`\_ \catcode`\_=8 
\newenvironment{tex2html_wrap}{}{}%
\catcode`\<=12\catcode`\_=\count@
\newcommand{\providedcommand}[1]{\expandafter\providecommand\csname #1\endcsname}%
\newcommand{\renewedcommand}[1]{\expandafter\providecommand\csname #1\endcsname{}%
  \expandafter\renewcommand\csname #1\endcsname}%
\newcommand{\newedenvironment}[1]{\newenvironment{#1}{}{}\renewenvironment{#1}}%
\let\newedcommand\renewedcommand
\let\renewedenvironment\newedenvironment
\makeatother
\let\mathon=$
\let\mathoff=$
\ifx\AtBeginDocument\undefined \newcommand{\AtBeginDocument}[1]{}\fi
\newbox\sizebox
\setlength{\hoffset}{0pt}\setlength{\voffset}{0pt}
\addtolength{\textheight}{\footskip}\setlength{\footskip}{0pt}
\addtolength{\textheight}{\topmargin}\setlength{\topmargin}{0pt}
\addtolength{\textheight}{\headheight}\setlength{\headheight}{0pt}
\addtolength{\textheight}{\headsep}\setlength{\headsep}{0pt}
\setlength{\textwidth}{349pt}
\newwrite\lthtmlwrite
\makeatletter
\let\realnormalsize=\normalsize
\global\topskip=2sp
\def\preveqno{}\let\real@float=\@float \let\realend@float=\end@float
\def\@float{\let\@savefreelist\@freelist\real@float}
\def\liih@math{\ifmmode$\else\bad@math\fi}
\def\end@float{\realend@float\global\let\@freelist\@savefreelist}
\let\real@dbflt=\@dbflt \let\end@dblfloat=\end@float
\let\@largefloatcheck=\relax
\let\if@boxedmulticols=\iftrue
\def\@dbflt{\let\@savefreelist\@freelist\real@dbflt}
\def\adjustnormalsize{\def\normalsize{\mathsurround=0pt \realnormalsize
 \parindent=0pt\abovedisplayskip=0pt\belowdisplayskip=0pt}%
 \def\phantompar{\csname par\endcsname}\normalsize}%
\def\lthtmltypeout#1{{\let\protect\string \immediate\write\lthtmlwrite{#1}}}%
\newcommand\lthtmlhboxmathA{\adjustnormalsize\setbox\sizebox=\hbox\bgroup\kern.05em }%
\newcommand\lthtmlhboxmathB{\adjustnormalsize\setbox\sizebox=\hbox to\hsize\bgroup\hfill }%
\newcommand\lthtmlvboxmathA{\adjustnormalsize\setbox\sizebox=\vbox\bgroup %
 \let\ifinner=\iffalse \let\)\liih@math }%
\newcommand\lthtmlboxmathZ{\@next\next\@currlist{}{\def\next{\voidb@x}}%
 \expandafter\box\next\egroup}%
\newcommand\lthtmlmathtype[1]{\gdef\lthtmlmathenv{#1}}%
\newcommand\lthtmllogmath{\dimen0\ht\sizebox \advance\dimen0\dp\sizebox
  \ifdim\dimen0>.95\vsize
   \lthtmltypeout{%
*** image for \lthtmlmathenv\space is too tall at \the\dimen0, reducing to .95 vsize ***}%
   \ht\sizebox.95\vsize \dp\sizebox\z@ \fi
  \lthtmltypeout{l2hSize %
:\lthtmlmathenv:\the\ht\sizebox::\the\dp\sizebox::\the\wd\sizebox.\preveqno}}%
\newcommand\lthtmlfigureA[1]{\let\@savefreelist\@freelist
       \lthtmlmathtype{#1}\lthtmlvboxmathA}%
\newcommand\lthtmlpictureA{\bgroup\catcode`\_=8 \lthtmlpictureB}%
\newcommand\lthtmlpictureB[1]{\lthtmlmathtype{#1}\egroup
       \let\@savefreelist\@freelist \lthtmlhboxmathB}%
\newcommand\lthtmlpictureZ[1]{\hfill\lthtmlfigureZ}%
\newcommand\lthtmlfigureZ{\lthtmlboxmathZ\lthtmllogmath\copy\sizebox
       \global\let\@freelist\@savefreelist}%
\newcommand\lthtmldisplayA{\bgroup\catcode`\_=8 \lthtmldisplayAi}%
\newcommand\lthtmldisplayAi[1]{\lthtmlmathtype{#1}\egroup\lthtmlvboxmathA}%
\newcommand\lthtmldisplayB[1]{\edef\preveqno{(\theequation)}%
  \lthtmldisplayA{#1}\let\@eqnnum\relax}%
\newcommand\lthtmldisplayZ{\lthtmlboxmathZ\lthtmllogmath\lthtmlsetmath}%
\newcommand\lthtmlinlinemathA{\bgroup\catcode`\_=8 \lthtmlinlinemathB}
\newcommand\lthtmlinlinemathB[1]{\lthtmlmathtype{#1}\egroup\lthtmlhboxmathA
  \vrule height1.5ex width0pt }%
\newcommand\lthtmlinlineA{\bgroup\catcode`\_=8 \lthtmlinlineB}%
\newcommand\lthtmlinlineB[1]{\lthtmlmathtype{#1}\egroup\lthtmlhboxmathA}%
\newcommand\lthtmlinlineZ{\egroup\expandafter\ifdim\dp\sizebox>0pt %
  \expandafter\centerinlinemath\fi\lthtmllogmath\lthtmlsetinline}
\newcommand\lthtmlinlinemathZ{\egroup\expandafter\ifdim\dp\sizebox>0pt %
  \expandafter\centerinlinemath\fi\lthtmllogmath\lthtmlsetmath}
\newcommand\lthtmlindisplaymathZ{\egroup %
  \centerinlinemath\lthtmllogmath\lthtmlsetmath}
\def\lthtmlsetinline{\hbox{\vrule width.1em \vtop{\vbox{%
  \kern.1em\copy\sizebox}\ifdim\dp\sizebox>0pt\kern.1em\else\kern.3pt\fi
  \ifdim\hsize>\wd\sizebox \hrule depth1pt\fi}}}
\def\lthtmlsetmath{\hbox{\vrule width.1em\kern-.05em\vtop{\vbox{%
  \kern.1em\kern0.8 pt\hbox{\hglue.17em\copy\sizebox\hglue0.8 pt}}\kern.3pt%
  \ifdim\dp\sizebox>0pt\kern.1em\fi \kern0.8 pt%
  \ifdim\hsize>\wd\sizebox \hrule depth1pt\fi}}}
\def\centerinlinemath{%
  \dimen1=\ifdim\ht\sizebox<\dp\sizebox \dp\sizebox\else\ht\sizebox\fi
  \advance\dimen1by.5pt \vrule width0pt height\dimen1 depth\dimen1 
 \dp\sizebox=\dimen1\ht\sizebox=\dimen1\relax}

\def\lthtmlcheckvsize{\ifdim\ht\sizebox<\vsize 
  \ifdim\wd\sizebox<\hsize\expandafter\hfill\fi \expandafter\vfill
  \else\expandafter\vss\fi}%
\providecommand{\selectlanguage}[1]{}%
\makeatletter \tracingstats = 1 
\providecommand{\Beta}{\textrm{B}}
\providecommand{\Mu}{\textrm{M}}
\providecommand{\Kappa}{\textrm{K}}
\providecommand{\Rho}{\textrm{R}}
\providecommand{\Epsilon}{\textrm{E}}
\providecommand{\Chi}{\textrm{X}}
\providecommand{\Iota}{\textrm{J}}
\providecommand{\omicron}{\textrm{o}}
\providecommand{\Zeta}{\textrm{Z}}
\providecommand{\Eta}{\textrm{H}}
\providecommand{\Omicron}{\textrm{O}}
\providecommand{\Nu}{\textrm{N}}
\providecommand{\Tau}{\textrm{T}}
\providecommand{\Alpha}{\textrm{A}}


\begin{document}
\pagestyle{empty}\thispagestyle{empty}\lthtmltypeout{}%
\lthtmltypeout{latex2htmlLength hsize=\the\hsize}\lthtmltypeout{}%
\lthtmltypeout{latex2htmlLength vsize=\the\vsize}\lthtmltypeout{}%
\lthtmltypeout{latex2htmlLength hoffset=\the\hoffset}\lthtmltypeout{}%
\lthtmltypeout{latex2htmlLength voffset=\the\voffset}\lthtmltypeout{}%
\lthtmltypeout{latex2htmlLength topmargin=\the\topmargin}\lthtmltypeout{}%
\lthtmltypeout{latex2htmlLength topskip=\the\topskip}\lthtmltypeout{}%
\lthtmltypeout{latex2htmlLength headheight=\the\headheight}\lthtmltypeout{}%
\lthtmltypeout{latex2htmlLength headsep=\the\headsep}\lthtmltypeout{}%
\lthtmltypeout{latex2htmlLength parskip=\the\parskip}\lthtmltypeout{}%
\lthtmltypeout{latex2htmlLength oddsidemargin=\the\oddsidemargin}\lthtmltypeout{}%
\makeatletter
\if@twoside\lthtmltypeout{latex2htmlLength evensidemargin=\the\evensidemargin}%
\else\lthtmltypeout{latex2htmlLength evensidemargin=\the\oddsidemargin}\fi%
\lthtmltypeout{}%
\makeatother
\setcounter{page}{1}
\onecolumn

% !!! IMAGES START HERE !!!

\setcounter{tocdepth}{2}
\stepcounter{paragraph}
\stepcounter{paragraph}
\stepcounter{paragraph}
\stepcounter{paragraph}
\stepcounter{paragraph}
\begingroup \endgroup
\begingroup \endgroup
\stepcounter{chapter}

\renewcommand{\arraystretch}{1.1}

\renewcommand{\arraystretch}{1.1}
\stepcounter{chapter}
\stepcounter{paragraph}
\stepcounter{section}
\stepcounter{paragraph}
\stepcounter{section}
\stepcounter{paragraph}
\stepcounter{section}
\stepcounter{paragraph}
\stepcounter{section}
\stepcounter{paragraph}
\stepcounter{paragraph}
\stepcounter{paragraph}
\stepcounter{paragraph}
\stepcounter{paragraph}
\stepcounter{paragraph}
\stepcounter{section}
\stepcounter{paragraph}
\stepcounter{paragraph}
\stepcounter{paragraph}
\stepcounter{paragraph}
\stepcounter{paragraph}
\stepcounter{paragraph}
\stepcounter{paragraph}
\stepcounter{paragraph}
\stepcounter{chapter}
\stepcounter{section}
\stepcounter{subsubsection}
\stepcounter{paragraph}
\stepcounter{paragraph}
\stepcounter{paragraph}
\stepcounter{paragraph}
\stepcounter{subsubsection}
\stepcounter{paragraph}
\stepcounter{paragraph}
\stepcounter{paragraph}
\stepcounter{subsubsection}
\stepcounter{paragraph}
\stepcounter{subsection}
\stepcounter{subsubsection}
\stepcounter{paragraph}
\stepcounter{subsubsection}
\stepcounter{paragraph}
\stepcounter{section}
\stepcounter{paragraph}
\stepcounter{paragraph}
\stepcounter{subsection}
\stepcounter{paragraph}
\stepcounter{paragraph}
\stepcounter{paragraph}
\stepcounter{section}
\stepcounter{paragraph}
\stepcounter{paragraph}
\stepcounter{section}
\stepcounter{subsection}
\stepcounter{paragraph}
\stepcounter{paragraph}
\stepcounter{subsection}
{\newpage\clearpage
\lthtmlfigureA{algorithm3955}%
\begin{algorithm}
% latex2html id marker 3955
[hbt]Bereken een initi\"ele set van oplossingen $\PopSet_1$ en hun fitness-waarde\;
 $b_1\leftarrow\displaystyle\argmin_{\sol\in\PopSet_1}{\fun{\evalfun}{\sol}}$\;
 \Repeat{het stopcriterium is bereikt}{
  Genereer op basis van $\PopSet_t$ en $t$ stochastisch een nieuwe set oplossingen $\PopSet_{t+1}$ samen met hun fitness-waarde\;
  $b_{t+1}\leftarrow\displaystyle\argmin_{\sol\in\PopSet_{t+1}\cup\accl{b_t}}{\fun{\evalfun}{\sol}}$\;
  $t\leftarrow t+1$\;
 }
 \KwRet{$b_t$}\caption{Hoog niveau beschrijving van een metaheuristiek\cite{DBLP:journals/jc/ShonkwilerV94}.}
\end{algorithm}%
\lthtmlfigureZ
\lthtmlcheckvsize\clearpage}

\stepcounter{subsubsection}
\stepcounter{paragraph}
\stepcounter{subsection}
\stepcounter{subsubsection}
\stepcounter{subsubsection}
\stepcounter{subsubsection}
\stepcounter{subsubsection}
\stepcounter{subsubsection}
\stepcounter{subsection}
\stepcounter{subsection}
\stepcounter{subsubsection}
\stepcounter{paragraph}
\stepcounter{paragraph}
\stepcounter{paragraph}
\stepcounter{paragraph}
\stepcounter{paragraph}
\stepcounter{paragraph}
\stepcounter{paragraph}
\stepcounter{subsubsection}
\stepcounter{subsubsection}
\stepcounter{paragraph}
\stepcounter{paragraph}
\stepcounter{paragraph}
\stepcounter{subsubsection}
\stepcounter{paragraph}
\stepcounter{subsubsection}
\stepcounter{paragraph}
\stepcounter{paragraph}
\stepcounter{paragraph}
\stepcounter{paragraph}
\stepcounter{subsubsection}
\stepcounter{paragraph}
\stepcounter{subsection}
\stepcounter{paragraph}
\stepcounter{paragraph}
\stepcounter{paragraph}
{\newpage\clearpage
\lthtmlinlinemathA{tex2html_wrap_inline22662}%
$ 1.0$%
\lthtmlinlinemathZ
\lthtmlcheckvsize\clearpage}

\stepcounter{paragraph}
\stepcounter{paragraph}
\stepcounter{paragraph}
\stepcounter{subsubsection}
\stepcounter{subsubsection}
\stepcounter{paragraph}
\stepcounter{subsubsection}
\stepcounter{section}
\stepcounter{paragraph}
\stepcounter{paragraph}
\stepcounter{subsection}
\stepcounter{paragraph}
\stepcounter{subsection}
\stepcounter{paragraph}
\stepcounter{subsection}
\stepcounter{paragraph}
\stepcounter{paragraph}
\stepcounter{paragraph}
\stepcounter{section}
\stepcounter{paragraph}
\stepcounter{paragraph}
\stepcounter{paragraph}
\stepcounter{paragraph}
\stepcounter{paragraph}
\stepcounter{paragraph}
\stepcounter{paragraph}
\stepcounter{chapter}
\stepcounter{section}
\stepcounter{subsection}
\stepcounter{paragraph}
\stepcounter{paragraph}
\stepcounter{paragraph}
\stepcounter{paragraph}
\stepcounter{paragraph}
\stepcounter{paragraph}
\stepcounter{subsection}
\stepcounter{paragraph}
\stepcounter{section}
\stepcounter{section}
\stepcounter{subsection}
\stepcounter{paragraph}
\stepcounter{paragraph}
\stepcounter{subsection}
\stepcounter{paragraph}
\stepcounter{paragraph}
\stepcounter{subsection}
\stepcounter{paragraph}
\stepcounter{subsection}
\stepcounter{paragraph}
\stepcounter{chapter}
\stepcounter{paragraph}
\stepcounter{section}
\stepcounter{paragraph}
\stepcounter{paragraph}
\stepcounter{paragraph}
\stepcounter{paragraph}
\stepcounter{paragraph}
\stepcounter{paragraph}
\stepcounter{paragraph}
\stepcounter{section}
\stepcounter{subsection}
\stepcounter{subsection}
\stepcounter{subsection}
\stepcounter{subsection}
\stepcounter{subsection}
\stepcounter{paragraph}
\stepcounter{paragraph}
\stepcounter{section}
\stepcounter{subsection}
\stepcounter{subsection}
\stepcounter{paragraph}
\stepcounter{subsection}
\stepcounter{paragraph}
\stepcounter{paragraph}
\stepcounter{paragraph}
\stepcounter{subsection}
\stepcounter{paragraph}
\stepcounter{section}
\stepcounter{paragraph}
\stepcounter{paragraph}
\stepcounter{paragraph}
\stepcounter{paragraph}
\stepcounter{paragraph}
\stepcounter{chapter}
\stepcounter{section}
\stepcounter{paragraph}
\stepcounter{section}
\stepcounter{paragraph}
\stepcounter{paragraph}
\stepcounter{section}
{\newpage\clearpage
\lthtmlinlinemathA{tex2html_wrap_inline23008}%
$ 25\%$%
\lthtmlinlinemathZ
\lthtmlcheckvsize\clearpage}

\stepcounter{paragraph}
\stepcounter{section}
\stepcounter{paragraph}
{\newpage\clearpage
\lthtmlinlinemathA{tex2html_wrap_inline23018}%
$ x_4'$%
\lthtmlinlinemathZ
\lthtmlcheckvsize\clearpage}

\stepcounter{paragraph}
\stepcounter{section}
\stepcounter{section}
\stepcounter{chapter}
\stepcounter{section}
\stepcounter{subsection}
\stepcounter{paragraph}
\stepcounter{paragraph}
\stepcounter{paragraph}
\stepcounter{subsection}
\stepcounter{section}
\stepcounter{paragraph}
\stepcounter{paragraph}
\stepcounter{paragraph}
\stepcounter{paragraph}
\stepcounter{paragraph}
\stepcounter{paragraph}
\stepcounter{section}
\stepcounter{subsection}
{\newpage\clearpage
\lthtmlinlinemathA{tex2html_wrap_inline23146}%
$ 3$%
\lthtmlinlinemathZ
\lthtmlcheckvsize\clearpage}

\stepcounter{subsection}
\stepcounter{subsubsection}
\stepcounter{paragraph}
{\newpage\clearpage
\lthtmlinlinemathA{tex2html_wrap_inline23176}%
$ 700$%
\lthtmlinlinemathZ
\lthtmlcheckvsize\clearpage}

\stepcounter{subsubsection}
\stepcounter{paragraph}
{\newpage\clearpage
\lthtmlinlinemathA{tex2html_wrap_inline23190}%
$ a=0$%
\lthtmlinlinemathZ
\lthtmlcheckvsize\clearpage}

\stepcounter{paragraph}
\stepcounter{subsubsection}
\stepcounter{paragraph}
\stepcounter{paragraph}
\stepcounter{paragraph}
\stepcounter{subsubsection}
\stepcounter{paragraph}
\stepcounter{subsubsection}
\stepcounter{chapter}
\stepcounter{section}
\stepcounter{subsection}
\stepcounter{paragraph}
\stepcounter{paragraph}
\stepcounter{subsection}
\stepcounter{paragraph}
\stepcounter{paragraph}
\stepcounter{paragraph}
\stepcounter{subsection}
\stepcounter{paragraph}
\stepcounter{paragraph}
\stepcounter{paragraph}
\stepcounter{subsection}
\stepcounter{paragraph}
\stepcounter{paragraph}
\stepcounter{paragraph}
\stepcounter{paragraph}
\stepcounter{paragraph}
\stepcounter{section}
\stepcounter{paragraph}
\stepcounter{paragraph}
\stepcounter{paragraph}
\appendix
\stepcounter{chapter}
\stepcounter{section}
\stepcounter{subsection}
\stepcounter{paragraph}
\stepcounter{subsection}
\stepcounter{section}
\stepcounter{subsection}
\stepcounter{subsection}
\stepcounter{section}
\stepcounter{subsection}
\stepcounter{paragraph}
\stepcounter{subsection}
\stepcounter{section}
\stepcounter{subsection}
\stepcounter{subsection}
\stepcounter{section}
\stepcounter{subsection}
\stepcounter{subsection}
\stepcounter{section}
\stepcounter{subsection}
\stepcounter{paragraph}
\stepcounter{paragraph}
\stepcounter{subsection}
\stepcounter{section}
\stepcounter{subsection}
\stepcounter{subsection}
\stepcounter{section}
\stepcounter{subsection}
\stepcounter{paragraph}
\stepcounter{paragraph}
\stepcounter{subsection}
\stepcounter{section}
\stepcounter{subsection}
\stepcounter{paragraph}
\stepcounter{subsection}
\stepcounter{section}
\stepcounter{subsection}
\stepcounter{paragraph}
\stepcounter{subsection}
\stepcounter{section}
\stepcounter{subsection}
\stepcounter{subsection}
\stepcounter{section}
\stepcounter{subsection}
\stepcounter{paragraph}
\stepcounter{paragraph}
\stepcounter{paragraph}
\stepcounter{subsection}
\stepcounter{section}
\stepcounter{subsection}
\stepcounter{paragraph}
\stepcounter{paragraph}
\stepcounter{paragraph}
\stepcounter{subsection}
\stepcounter{section}
\stepcounter{subsection}
\stepcounter{paragraph}
\stepcounter{paragraph}
\stepcounter{paragraph}
\stepcounter{subsection}
\stepcounter{section}
\stepcounter{subsection}
\stepcounter{paragraph}
\stepcounter{paragraph}
\stepcounter{paragraph}
\stepcounter{section}
\stepcounter{subsection}
\stepcounter{paragraph}
\stepcounter{paragraph}
\stepcounter{paragraph}
\stepcounter{section}
\stepcounter{chapter}
\stepcounter{section}
\stepcounter{subsection}
\stepcounter{paragraph}
\stepcounter{paragraph}
\stepcounter{paragraph}
\stepcounter{subsection}
\stepcounter{paragraph}
\stepcounter{paragraph}
\stepcounter{paragraph}
\stepcounter{paragraph}
\stepcounter{section}
\stepcounter{subsection}
\stepcounter{paragraph}
\stepcounter{subsection}
\stepcounter{paragraph}
\stepcounter{subsection}
\stepcounter{subsection}
\stepcounter{section}
\stepcounter{paragraph}
\stepcounter{subsection}
\stepcounter{paragraph}
\stepcounter{paragraph}
{\newpage\clearpage
\lthtmlfigureA{algorithm9402}%
\begin{algorithm}
% latex2html id marker 9402
[hbt]$\dres\leftarrow\mbox{eigen deel van de data}$\;
 \For{$i=0\mbox{ \textbf{\emph{to}} }d-1$}{
  $\dprt\leftarrow\didq\mbox{ XOR }2^i$\;
  $\psend{\dprt,\dres}$\;
  $\dmsg\leftarrow\precv{\dprt}$\;
  $\dres\leftarrow\dres\cup\dmsg$\;
 }\caption{GatherAll\cite{books/bc/KumarGGK94}.}
\end{algorithm}%
\lthtmlfigureZ
\lthtmlcheckvsize\clearpage}

\stepcounter{paragraph}
\stepcounter{subsection}
{\newpage\clearpage
\lthtmlfigureA{algorithm9415}%
\begin{algorithm}
% latex2html id marker 9415
[hbt]\myproc(\pinit{}){
 $\dres\leftarrow\parry{p,\nult}$\;
 $\ddim\leftarrow\parry{d+1,\fals}$\;
 $z\leftarrow0$\;
}
\WhnRcv($\tupl{\dsnd,\dmsg}$){
 $k\leftarrow\didq\mbox{ XOR }\dsnd$\;
 $\dbas\leftarrow\dsnd\mbox{ AND }\brak{\mbox{NEG } k-1}$\;
 $\ddim\fbrk{1+\log_2k}\leftarrow\true$\;
 $l\leftarrow\funm{min}{k,p-\dbas}$\;
 $\dres\fbrk{\dbas:\dbas+l-1}\leftarrow\dmsg\fbrk{0:l-1}$\;
 \pzedt{}\;
}
\myproc(\pgata{\dodt}){
 $\dres\fbrk{\didq}\leftarrow\dodt$\;
 $\ddim\fbrk{0}\leftarrow\true$\;
 \pzedt{}\;
}
\myproc(\predy{}){
 \Return{$z\geq d\wedge\ddim\fbrk{d}$}\;
}
\myproc(\prest{}){
 $\ddim\leftarrow\parry{d+1,\fals}$\;
 $z\leftarrow 0$\;
}{}
\myproc(\pzedt{}){
 \While{\emph{$z<d\wedge\ddim\fbrk{z}$}}{
   $l\leftarrow 2^z$\;
   $\drcv\leftarrow \didq\mbox{ XOR }l$\;
   \eIf{\emph{$\drcv < p$}}{
	 $\dbas\leftarrow\didq\mbox{ AND }\brak{\mbox{NEG } l-1}$\;
	 $m\gets\funm{min}{l,p-\dbas}$\;
	 $\dmsg\leftarrow\parry{m}$\;
	 $\dmsg\fbrk{0:m-1}\leftarrow\dres\fbrk{\dbas:\dbas+m-1}$\;
	 $\pisnd{\drcv,\dmsg}$\;
   }{
	 $\ddim\fbrk{z+1}\leftarrow\true$
   }
   $z\leftarrow z+1$\;
 }
}\caption{Asynchrone GatherAll.}
\end{algorithm}%
\lthtmlfigureZ
\lthtmlcheckvsize\clearpage}

\stepcounter{paragraph}
\stepcounter{paragraph}
\stepcounter{paragraph}
\stepcounter{paragraph}
\stepcounter{paragraph}
\stepcounter{section}
\stepcounter{paragraph}
\stepcounter{paragraph}
\stepcounter{chapter}
\stepcounter{section}
\stepcounter{section}
\stepcounter{subsection}
\stepcounter{subsection}
\stepcounter{subsection}
\stepcounter{subsection}
\stepcounter{subsection}
\stepcounter{subsection}
\stepcounter{subsection}
\stepcounter{subsection}
\stepcounter{subsection}
\stepcounter{subsection}
\stepcounter{subsection}
\stepcounter{subsection}
\stepcounter{section}
\stepcounter{chapter}
\stepcounter{chapter}

\end{document}
