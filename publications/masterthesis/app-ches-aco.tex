\section{\emph{ACO-HH: Ant Colony Optimization} (\#11)}
\label{sss:aco-hh}
\subsection{Implementatie}
\emph{ACO-HH}\cite{chesc-aco-hh} maakt gebruik van de ``\emph{Ant-Colony Optimization}''\cite{hom/aco} techniek. Net als bij Ant-Q (zie \ref{sss:ant-q}) beschouwen we een graaf waarbij knopen de \abllhn{} voorstellen. Een fundamenteel verschil is echter dat de graaf voorgesteld wordt als een tabel met $n$ kolommen en $H$ rijen (met $H$ het aantal metaheuristieken en $n$ een parameter genaamd de \emph{padlengte}). Elke \abllh{} komt in deze graaf juist $n$ keer voor. Verder beschouwen we enkel bogen tussen twee verschillende kolommen. Het is de bedoeling dat agenten een pad afleggen met padlengte $n$ en telkens bij een knoop de overeenkomstige \abllh{} toepassen. Elke agent start met dezelfde initi\"ele oplossing en legt een pad van $n$ stappen af. De volgende knoop wordt telkens gekozen op basis van labels op de bogen die een niet-genormaliseerde kans aangeven. Wanneer alle agenten de laatste kolom bereikt hebben worden de resultaten ge\"evalueerd. Zo worden alle bogen aangepast waarover een agent heeft gelopen op basis van het verschil in fitnesswaarde van de oorspronkelijke en uiteindelijke oplossing. Uit de set van finale oplossingen dient men ook een oplossing te kiezen die de initi\"ele oplossing van de volgende fase voorstelt. Doorgaans neemt men de beste oplossing over alle fases heen, tenzij de huidige fase deze oplossing heeft gegenereerd.
\subsection{Kritiek}
\begin{itemize}
 \item Het systeem kan in een lange tijd in een lokaal optimum terechtkomen: sommige paden kunnen in het begin tot sterke kansen komen, waardoor het moeilijk is om na verloop van tijd een sterker pad te belonen.
 \item Redundante aspecten in een sequentie worden niet noodzakelijk ge\"elimineerd: elke pad is bijvoorbeeld even lang en zolang een pad van dezelfde lengte niet met sterkere resultaten komt is er geen reden kunnen er ineffici\"ente heuristieken in het meest populaire pad zitten. Men zou in de graaf bijvoorbeeld ook \emph{nul-operaties} kunnen voorstellen. Een agent die in zo'n operatie terecht komt voert nadien geen operaties meer uit tot het einde van de iteratie. Men dient dan ook paden die zo'n nul-operatie introduceren extra te belonen.
\end{itemize}