\section{\emph{ACO-HH: Ant Colony Optimization} (\#11)}
\seclab{aco-hh}

\subsection{Implementatie}
\emph{ACO-HH}\cite{chesc-aco-hh} maakt gebruik van een ``\emph{Ant-Colony Optimization}''\cite{hom/aco} techniek. Net als bij \emph{Ant-Q} (zie \secref{ant-q}) beschouwen we een graaf waarbij knopen de \abllhn{} voorstellen. Het algoritme verschilt van \emph{Ant-Q} omdat de knopen geordend zijn in een tabel met $n$ kolommen en $H$ rijen (met $H$ het aantal metaheuristieken en $n$ een parameter genaamd de \emph{padlengte}). Elke \abllh{} komt in deze graaf juist $n$ keer voor. Verder beschouwen we enkel bogen tussen de knopen in twee verschillende kolommen. Agenten leggen een pad af met padlengte $n$. Telkens een agent een knoop bereikt wordt de overeenkomstige \abllh{} toepassen.

\paragraph{}
Elke agent start met dezelfde initi\"ele oplossing en legt een pad van $n$ stappen af. De volgende knoop wordt telkens gekozen op basis van labels op de bogen die een niet-genormaliseerde kans aanduiden. Wanneer alle agenten de laatste kolom bereikt hebben worden de resultaten ge\"evalueerd.

\paragraph{}
Alle bogen waarover een agent heeft gelopen worden aangepast. Dit gebeurt op basis van het verschil in fitnesswaarde van de oorspronkelijke en uiteindelijke oplossing. Uit de set van finale oplossingen dient men ook een oplossing te kiezen die de initi\"ele oplossing van de volgende fase voorstelt. Doorgaans neemt men de beste oplossing over alle fases heen, tenzij de huidige fase deze oplossing heeft gegenereerd.

\subsection{Kritiek}
\begin{itemize}
 \item Het systeem kan in een lange tijd in een lokaal optimum terechtkomen: sommige paden kunnen in het begin met sterke kansen gelabeld worden. Het is moeilijk om een pad dat pas na een zekere termijn iets oplevert daarvoor voldoende te belonen.
 \item Redundante aspecten in een sequentie worden niet noodzakelijk ge\"elimineerd: elke pad is even lang. In sterke paden kunnen heuristieken zitten die weinig opleveren maar geen specifiek nadeel vormen. Dit zou men kunnen oplossen door bijvoorbeeld \emph{nul-operaties} te introduceren.Agent die in een \emph{nul-operatie} terecht komt voeren geen operaties uit tot het einde van de iteratie. Paden met een \emph{nul-operatie} kunnen extra beloond worden omdat ze effici\"enter werken.
\end{itemize}