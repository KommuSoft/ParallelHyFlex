\chapter{Definities en State-of-the-Art}
\label{hoofdstuk:1}

In de inleiding hebben we de kort de verschillende concepten besproken die in deze thesis een belangrijke rol zullen spelen. In dit hoofdstuk gaan we hier dieper op in: we formaliseren de concepten in definities en geven een kort overzicht van de belangrijkste wetmatigheden rond deze concepten.

\section{Optimalisatieproblemen}

We beginnen deze sectie met een formele definitie van een optimalisatieprobleem:

\begin{definition}[Optimalisatieprobleem]%, harde beperkingen, evaluatiefunctie, fitness-waarde
Een optimalisatieprobleem $\Pi$ is een tuple $\Pi=\tupl{X=A_1\times A_2\times\ldots\times A_n,c,f}$ waarbij $X$ een verzameling is van een set configuraties voor $n$ variabelen, $c:X\rightarrow\BBB$ een afbeelding van zo'n configuratie naar een Booleaanse waarde, die bepaald of de configuratie voldoet aan de ``harde beperkingen''. $f:X\rightarrow\RRR$ stelt een evaluatiefunctie voor die bepaald in welke mate een configuratie wenselijk is. De waarde van de evaluatie van een configuratie \fun{f}{x} wordt ook wel de fitness-waarde genoemd.
\end{definition}
Bij een optimalisatieprobleem gaan we op zoek naar een configuratie $x\in X$ die aan de harde beperkingen voldoet en de evaluatiefunctie optimaliseert. Meestal maakt men het onderscheid tussen een minimalisatie en een maximalisatie. In deze thesis zullen we altijd we een bij een optimalisatieprobleem altijd streven naar een configuratie $x\in X$ met een zo laag mogelijk evaluatie \fun{f}{x}. We kunnen echter eenvoudig elk maximalisatieprobleem $\tupl{X,c,f}$ omzetten in een minimalisatieprobleem $\tupl{X,c,f'}$ met $f':X\rightarrow\RRR:x\mapsto-\fun{f}{x}$. Formeel zoeken we dus naar een configuratie \xstar{}.

\begin{equation}
\xstar=\displaystyle\argmin_{x\in X'}\fun{f}{x}\mbox{ where }X'=\accl{x|\forall x\in X:\fun{c}{x}=\true}
\end{equation}

In een algemeen geval kunnen de domeinen $A_i$ van de variabelen $x_i$ oneindig groot zijn en bijvoorbeeld $\RRR$ omvatten. Geen enkele machine met een eindig geheugen kan echter alle elementen uit een domein met oneindig veel elementen voorstellen. We zullen daarom altijd de domeinen $A_i$ als eindig beschouwen. In het geval het domein van een variabele in werkelijkheid oneindig is, discretiseren we dus dit domein en beperken we het aantal elementen met een onder- en bovengrens. Indien er door discretisatie fouten worden ge\"introduceerd, kunnen we deze oplossen door het domein fijner te discretiseren.
\paragraph{}
Vermits zowel de harde beperkingen $c$ als de evaluatiefunctie $f$ hier een ``\emph{blackbox}'' zijn, zullen we om $\xstar$ te berekenen, over een significant deel van de verzameling $X$ moeten itereren. We verwachten dus dat de tijdscomplexiteit om een dergelijke oplossing te vinden gelijk is aan:
\begin{equation}
\bigoh{\abs{X}}=\bigoh{\displaystyle\prod_{i=1}^n\abs{A_i}}
\end{equation}
Indien we de assumptie maken dat alle domeinen dezelfde zijn dan bekomen we:
\begin{equation}
\bigoh{\abs{X}}=\bigoh{\abs{A_1}^n}\mbox{ indien }\forall A_i,A_j: A_i=A_j
\end{equation}
We zien dus dat deze tijdscomplexiteit exponentieel stijgt met het aantal variabelen~$n$. Optimalisatieproblemen in het algemeen liggen dan ook in \comp{NP-hard}.

\subsection{Complexiteit van optimalisatieproblemen}

Sommige optimalisatieproblemen liggen in \comp{P}. \algo{Karmarkar's algoritme}\cite{linearProgrammingInP} bijvoorbeeld lost het \prbm{lineaire optimalisatie} probleem op in \bigoh{n^{3.5}L} met $n$ het aantal variabelen en $L$ de diepte van de discretisatie in bits. Dit komt omdat we beperkingen plaatsen op de vorm van de evaluatiefunctie $f$ en de harde beperkingen $c$. Bij lineair programmeren betekent dit dat de evaluatiefunctie kan geschreven worden als het inwendig product tussen de vector van de variabelen en een vector met constanten. Het harde beperkingen moeten voor te stellen zijn zodat wanneer we de vector met de variabelen vermenigvuldigen met een matrix met constante elementen, alle elementen in de resulterende vector kleiner zijn dan een andere vector met constante elementen. Ook andere optimalisatieproblemen zoals bijvoorbeeld \prbm{Maximum Flow} en \prbm{Minimum Spanning Tree} zijn problemen die met polynomiale algoritmen kunnen worden opgelost.

\paragraph{}
Toch is er weinig ruimte voor optimisme. Een logische veralgemening van \prbm{Lineaire optimalisatie} is immers \prbm{Kwadratische optimalisatie}. Onder sommige omstandigheden kunnen we dit probleem reduceren naar een geval van \prbm{Lineaire Optimalisatie}\cite{Kozlov1980223}, maar een algemeen \prbm{Kwadratisch optimalisatie} probleem ligt in \comp{NP-hard}\cite{qpInNP}. Ook andere bekende optimalisatieproblemen zoals \prbm{Travelling Salesman Problem (TSP)} en \prbm{Integer Programming (IP)} liggen in \comp{NP-hard}.

\paragraph{}
Tot slot dient men in de context van optimalisatieproblemen een kanttekening maken dat een polynomiaal algoritme meestal niet meteen impliceert dat dit ook op kleine gevallen sneller werkt dan zijn exponenti\"ele tegenhangers. Een populaire methode bij het oplossen van \prbm{Lineaire optimalisatie} is bijvoorbeeld het \algo{Simplex}-algoritme. Klee en Minty\cite{klee:1972} construeerden echter een een geval waarbij het algoritme exponentieel veel tijd vraagt. Toch is \algo{Simplex} in de meeste gevallen sneller dan \algo{Karmarkar's algoritme}.

\section{Heuristieken}

Hoewel de meeste optimalisatieproblemen \comp{NP-hard} zijn, is in een praktische context de configuratie met een optimale evaluatiefunctie net van cruciaal belang. Voor de meeste toepassingen is een configuratie die aan de harde beperkingen voldoet en de fitness-waarde van de echte oplossing benadert voldoende. In dat geval wordt meestal een heuristiek ge\"implementeerd:

\begin{definition}[Heuristiek]
Een heuristiek is een programma die gegeven een optimalisatieprobleem $\Pi=\tupl{X,c,f}$ een oplossing berekent $\xdot$ in een redelijke tijd. Doorgaans voldoet deze oplossing aan de harde beperkingen ($\fun{c}{\xdot}=\true$) en ligt de voorgestelde oplossing $\xdot$ niet ver van de werkelijke oplossing $\xstar$.
\end{definition}

Deze definitie blijft redelijk vaag en geeft dan ook veel ruimte voor interpretatie. Doorgaans verwachten we dat het algoritme stop in polynomiale tijd en in de meeste gevallen worden er ook beperkingen gezet op hoe ver de fitness-waarde \fun{f}{\xdot} mag afwijken van de optimale fitness-waarde \fun{f}{\xstar}, al zijn beide voorwaarden niet strikt noodzakelijk. Minsky\cite{minskyHeuristic} schrijf hierover:
\begin{quote}
``Hints'', ``suggestions'', or ``rules of thumb'', which only usually work are called heuristics. A program which works on such a basis is called a heuristic program. It is difficult to give a more precise definition of heuristic program - this is to be expected in the light of Turing's demonstration that there is no systematic procedure which can distinguish between algorithms (programs that always work) and programs that do not always work.
\end{quote}
In deze paper komen enkele methodes aan bod hoe men heuristieken kan ontwikkelen.
\section{Besluit van dit hoofdstuk}
Als je in dit hoofdstuk tot belangrijke resultaten of besluiten gekomen
bent, dan is het ook logisch om het hoofdstuk af te ronden met een
overzicht ervan. Voor hoofdstukken zoals de inleiding en het
literatuuroverzicht is dit niet strikt nodig.

%%% Local Variables: 
%%% mode: latex
%%% TeX-master: "masterproef"
%%% End: 
