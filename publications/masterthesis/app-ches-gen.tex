\section{\emph{GenHive: Genetic Hive Hyperheuristic} (\#12)}
\label{sss:genhive}
\subsection{Implementatie}
\emph{GenHive}\cite{chesc-genhive} is een hyperheuristiek die werkt op basis van een genetisch algoritme. Een individu in dit genetische algoritme is een sequentie van metaheuristieken die op het probleem worden toegepast. Enkele van deze individuen zijn actief: ze worden toegekend aan een oplossing in de zoekruimte en worden er bij een iteratie op toegepast. Nadien worden de resultaten beoordeelt. De beste strategie\"en blijven actief. De overige worden passieve individuen. Men dient echter aan elke oplossing een strategie toekennen. Hiertoe worden individuen die in de vorige iteratie passief waren gerecombineerd met de beste individuen en worden deze geactiveerd.
\subsection{Kritiek}
\begin{itemize}
 \item Algoritme bevat veel parameters: populatiegrootte, aantal individuen actief, aantal individuen die na de iteratie behouden blijft,... Het vinden van een correcte configuratie is niet triviaal.
 \item De beste individuen blijven aan dezelfde oplossingen gelinkt, terwijl men deze zou kunnen gebruiken om andere oplossingen ook significant te verbeteren. Bovendien verwachten we dat een strategie de tweede maal minder effectief zal zijn.
 \item Redundante aspecten in een oplossing (een sequentie oproepen die nooit een beter resultaat kunnen genereren) wordt niet ge\"elimineerd: hierdoor is men niet effici\"ent met tijdsgebruik.
 \item Het is ineffici\"ent om altijd alle oplossingen verder te ontwikkelen in een iteratie.
\end{itemize}