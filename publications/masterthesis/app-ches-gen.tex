\section{\emph{GenHive: Genetic Hive Hyperheuristic} (\#12)}
\seclab{genhive}

\subsection{Implementatie}
\emph{GenHive}\cite{chesc-genhive} werkt met behulp van een genetisch algoritme. Een \emph{individu} is in dit genetische algoritme een sequentie van heuristieken die op \'e\'en voor \'e\'en op een oplossing worden toegepast. Enkele van deze \emph{individuen} zijn actief: ze worden toegekend aan een oplossing in de zoekruimte en worden er bij een iteratie op toegepast. Nadien worden de resultaten beoordeelt. De beste \emph{individuen} blijven actief, de overige worden passieve individuen. De oplossingen die niet meer gebonden zijn aan een \abh{} krijgen een nieuw \emph{individu} toegekend. Deze individuen worden gecre\"eerd door de beste \emph{individuen} te kruisen met passieve \emph{individuen}. Een deel van de \emph{individuen} verdwijnt ook na iedere iteratie.

\subsection{Kritiek}
\begin{itemize}
 \item Het algoritme bevat veel parameters: populatiegrootte, aantal individuen actief, aantal individuen die na de iteratie behouden blijft,... Het vinden van een correcte configuratie is niet triviaal.
 \item De beste \emph{individuen} blijven aan dezelfde oplossingen gelinkt. Door deze \emph{individuen} ook op andere oplossingen toe te passen, verwachten we dat deze ook sterk zullen verbeteren. Een sequentie een tweede maal toepassen op een oplossing zal meestal minder gunstige effecten geven.
 \item Deelsequenties in \emph{individu} die ineffici\"ent werken worden niet ge\"elimineerd.
 \item Het is ineffici\"ent om altijd alle oplossingen verder te ontwikkelen in een iteratie.
\end{itemize}