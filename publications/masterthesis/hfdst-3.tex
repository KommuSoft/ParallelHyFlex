\chapter{\emph{ParHyFlex}: Parallel \emph{HyFlex}}
\chplab{parhyf}

\chapterquote{A skilled transition team leader will set the general goals for a transition and then confer on the other team leaders working with him the power to implement those goals.}{Richard V. Allen}

Op basis van de analyse in \chpref{chesc}, werd een systeem genaamd \emph{ParHyFlex} ge\"implementeerd die het mogelijk maakt om hyperheuristieken te ontwikkelen in een parallelle context. De broncode van dit systeem is te vinden onder \mbox{\url{http://goo.gl/3YNnp}} en wordt verder actief ontwikkeld.

\paragraph{}
Het beschrijven van parallelle algoritmes is vaak complex. In dit hoofdstuk zullen we eerst een algemeen overzicht geven van het volledige systeem. Daarna bespreken we welke probleemafhankelijke componenten die mogelijk een voordeel opleveren in een parallelle context. Vervolgens geven we een overzicht van de probleemonafhankelijke componenten die werden ge\"implementeerd. We eindigen met een bondig overzicht en bespreken welke inzichten uit het vorige hoofdstuk relevant waren voor de implementatie van dit systeem.

\section{Overzicht}

Vermits de onderliggende heuristieken heel divers kunnen zijn qua tijdsgebruik, lenen hyperheuristiek zich niet goed tot parallellisatie met een hoge granulariteit: indien \'e\'en van de processoren een \abls{} \abh{} kiest, zal deze veel rekentijd in beslag nemen. \abmt[M]{} \abhn{} daarentegen worden meestal vrij snel uitgevoerd. Op het moment dat een processor die een mutatie uitvoert, zijn opdracht heeft voltooid moet deze wachten op een processor met een computationeel veel zwaardere taak. Deze onbalans leidt er bijgevolg toe dat we maar weinig \emph{speed-up} mogen verwachten bij een dergelijke methode.

\paragraph{}
We kunnen er voor opteren om bepaalde heuristieken zelf in parallel uit te voeren, wat valt onder \emph{type 1}-parallellisatie van \emph{Crainic and Toulouse}\cite{crainicAndToulouse}. Het nadeel van deze aanpak is dat de onderliggende heuristieken zelf aangepast moeten worden voor een parallelle context. Bovendien verliezen we rekenkracht bij het opstarten en aflopen van een heuristiek: opdrachten moeten naar de processoren worden gecommuniceerd en de resultaten moeten terug naar een centrale processor vloeien. De aanwezig rekenkracht wordt op deze momenten niet benut waardoor deze methode vrij ineffici\"ent werkt. Sommige heuristieken zijn bovendien niet eenvoudig te parallelliseren: in het geval van een mutatie bijvoorbeeld zal het doorsturen van een oplossing meer tijd vragen dan de mutatie zelf uit te voeren.

\paragraph{}
Daarom hebben we geopteerd voor een model waarbij elke processor zelf een hyperheuristiek uitvoert. Deze vorm valt dus onder \emph{Type 3}-parallellisatie. De processen werken echter niet volledig onafhankelijk: oplossingen kunnen worden uitgewisseld naar andere processoren. Op die manier kunnen met behulp van een kruisingsoperator sterke delen uit een oplossing worden uitgewisseld. Oplossing worden uitgewisseld zonder dat de processor wacht tot de oplossing is doorgestuurd. We hopen hiermee de effici\"entie verder te kunnen opdrijven.

\paragraph{}
Met behulp van \emph{afdwingbare beperkingen}, een concept die we in de volgende sectie introduceren kunnen we ook de zoekruimte beperken. Dit is het \emph{Type 2}-parallellisme: per processor wordt er een op geregelde tijdstippen een nieuwe zoekruimte berekend waarin dan meer geconcentreerd wordt gezocht. Men dient echter te voorkomen dat teveel processoren hetzelfde deel van de zoekruimte afzoeken. Daarom vinden onderhandelingen tussen de verschillende processoren plaats. Ook deze onderhandeling vinden asynchroon plaats.

\paragraph{}
Communicatie kan een significante overhead voor een algoritme betekenen. Vooral het \emph{serialiseren} en \emph{deserialiseren} van data is een taak die aan processoren wordt uitbesteed en dus rekenkracht vergt. In de meeste bestudeerde implementaties verloopt de communicatie ofwel synchroon (de uitvoering wordt geblokkeerd op verschillende punten in de tijd om een boodschap te verzenden of te ontvangen), ofwel met behulp van verschillende ``\emph{threads}''. Wanneer men met \emph{threads} werkt, kunnen manipulaties op gegevensstructuren tegelijk plaatsvinden, wat tot inconsistentie kan leiden. Een oplossing is gebruik maken van \emph{locks} of andere synchronisatie-directieven. Deze directieven introduceert nieuwe vormen van verlies aan rekenkracht.

\paragraph{}
Dit is de motivatie om in het geval van \emph{ParHyFlex} met \'e\'en \emph{thread} te werken en communicatie asynchroon te verwerken. Doordat het toepassen van heuristieken de kern is van iedere hyperheuristiek kunnen we de assumptie maken dat gedurende het hele proces deze methodes zullen worden opgeroepen. Nadat een heuristiek is uitgevoerd controleert het systeem of er nieuwe berichten zijn toegekomen en wordt er \'e\'en bericht verwerkt. Sommige berichten veroorzaken het uitsturen van nieuwe berichten, dit kan een domino-effect teweeg brengt. Door slechts \'e\'en bericht te verwerken hopen we de communicatie beter in de hand te kunnen houden. Bovendien garanderen we dat een minimale hoeveelheid rekentijd zeker in de heuristieken wordt ge\"investeerd. De details in verband me de communicatie worden verder toegelicht in \appref{communication}.

\paragraph{}%TODO: paragaaf behouden?
Als we de assumptie maken dat er hoogstens \'e\'en bericht zal toekomen tijdens de uitvoer van een heuristiek zou dit bovendien niet tot significant ander gedrag mogen leiden. De hyperheuristiek maakt een keuze op basis van eigen ervaring en ervaring die door middel van boodschappen wordt doorgestuurd. Vermits deze boodschappen worden verwerkt alvorens de hyperheuristiek opnieuw controle krijgt over de processor, zal de hyperheuristiek met dezelfde data een nieuwe keuze maken. Ook wanneer men een operator toepast terwijl men een oplossing in het geheugen inleest verandert er niks: de operator zal immers altijd de oplossing gebruiken die op dat moment in het geheugen zit. Een bericht die ertoe leidt dat de zoekruimte wordt aangepast heeft wel effect: als we de zoekruimte pas achteraf aanpassen zal dit andere resultaten opleveren dan wanneer we dit doen wanneer het bericht binnenkomt.

\paragraph{}
\emph{ParHyFlex} werkt volgens het \emph{peer-to-peer} paradigma. Het systeem omvat echter \'e\'en hi\"erarchisch aspect: er is \'e\'en processor die het probleem inleest en naar de verschillende processoren stuurt. Wanneer de tijd afloopt zal deze processor ook de oplossingen van de verschillende processoren verzamelen en de beste oplossing rapporteren. Naast deze communicatie werkt het systeem volledig in een \emph{peer-to-peer} mode. Heel wat implementaties werken wel volgens het \emph{master-slave} model\cite{conf/gecco/LeonMS08,conf/pdp/SeguraSL12}. Dit heeft tot gevolg dat \'e\'en van de processoren zich meer met communicatie bezighoudt wat tot een onbalans leidt in het systeem. In andere implementaties\cite{Rattadilok04adistributed} stelt men een processor aan die zich uitsluitend met communicatie bezighoudt. De meeste publicaties rapporteren dat deze processor meestal 90\% van de tijd wacht op binnenkomende berichten. Dit bemoeilijkt de zaak om resultaten te rapporteren: het weglaten van deze processor in de berekeningen leidt tot optimistische resultaten.

\section{Probleemafhankelijk gedeelte}

We werken echter in een parallelle context. Daarom is het interessant dat het probleemafhankelijke gedeelte meer functionaliteiten ter beschikking stelt die uitgebuit kunnen worden door het bovenliggende systeem. Concreet denken we hierbij aan vier zaken: \emph{afstandmetrieken}, \emph{ervaring-generatoren}, \emph{zoekruimte beperkers} en \emph{multi-objectieven}

\subsection{Afstandmetrieken}
\abhf{} laat problemen een functie aanbieden die twee oplossingen met elkaar kan vergelijken. Deze functie kunnen we theoretisch omvormen tot een afstandsmetriek (we stellen de afstand tussen twee dezelfde oplossingen gelijk aan 0, en tussen twee verschillende aan een arbitraire constante groter dan 0). Deze metriek levert echter weinig informatie op. In een sequenti\"ele context gebruikt men soms het aantal mutaties die tussen een oplossing en \'e\'en van zijn voorouders om de afstand af te schatten. Dit is natuurlijk slechts een benadering. In het geval van parallelle uitvoer zullen we bovendien meestal niet over deze informatie beschikken. Daarom is het nuttig om de afstand tussen twee oplossingen te kunnen inschatten. In het geval de afstand geen triviaal gegeven is, kan men verschillende afstandsmetrieken defini\"eren en beslist de bovenliggende hyperheuristiek over de waarde van de metrieken. Een afstandmetriek is dus gedefinieerd als:
 \begin{equation}
  \delta_i:\SolSet^2\rightarrow\RealSet^+
 \end{equation}
 
\subsection{Ervaring-generatoren}
Elk proces draait een eigen hyperheuristiek en komt een sequentie heuristieken tegen. Uitwisselen van de sequentie kan potentieel een voordeel opleveren omdat de hyperheuristieken met meer kennis van zaken kunnen beslissen. Het doorsturen van alle heuristieken is doorgaans niet mogelijk omdat dit een te grote druk op het netwerk zet en bovendien de overige processoren te veel rekenkracht zouden investeren in het analyseren van de ontvangen oplossingen. Door het uitwisselen van ervaring, een compacte voorstelling van beschouwde oplossingen, zouden we dit probleem kunnen oplossen.

\subsection{Zoekruimte-beperkers}
Wanneer processoren oplossingen met elkaar uitwisselen lopen we de kans dat de verschillende processoren op termijn vergelijkbare populaties onderhouden. Dit laatste is nuttig wanneer sterke oplossingen in de buurt liggen van de oplossingen in de populatie. Indien de populaties echter rond eenzelfde lokaal optimum liggen, is dit nefast. In dat geval proberen alle processoren het lokale optimum te zoeken in een eenzelfde gebied, en wordt migratie naar mogelijk betere oplossingen in een ander gebied minder evident. Het introduceren van een component die diversificatie afdwingt kan helpen te voorkomen dat we op ijle populaties stuiten.

\subsection{Multi-objectieven}
Alle processoren proberen hetzelfde optimalisatieprobleem op te lossen. Door extra objectieven te introduceren, kunnen we echter een meer divers zoekproces aanbieden. Deze extra objectieven zijn eerder virtueel en dienen meer als een \emph{tie-breaker} in bijvoorbeeld gevallen waarbij twee oplossingen dezelfde fitness-waarde hebben.

\subsection{Afdwingbare beperkingen als probleemonafhankelijke ervaring}

Een probleem bij het genereren van \emph{ervaring} en het beperken van de \emph{zoekruimte} is dat dit op een probleemonafhankelijke manier dient te gebeuren: de bovenliggende hyperheuristiek heeft geen details over de structuur van de configuraties en kan bijgevolg niet zelf de zoekruimte beperken of conclusies genereren. We kunnen ervaring voorstellen als een object waar de hyperheuristiek de specificaties niet van kent, maar in dat geval moet ervaring wel enkele algemene functionaliteiten kunnen aanbieden die nuttig zijn. Om dit probleem op te lossen voeren we het concept van een \emph{afdwingbare beperking} in.

\begin{definition}[Afdwingbare beperking]
Een \emph{afdwingbare beperking} is een 3-tuple: $\tupl{c,c^+,c^-}$. $c:\SolSet\rightarrow\BoolSet$ is hierbij een functie die controleert of een gegeven oplossing aan een bepaalde voorwaarde voldoet. $c^+:\SolSet\rightarrow\SolSet$ is een functie die een gegeven oplossing minimaal kan aanpassen zodat deze aan de voorwaarde voldoet. $c^-:\SolSet\rightarrow\SolSet$ past oplossingen minimaal aan zodat ze niet aan de voorwaarde voldoen. De set van alle afdwingbare beperkingen die we op een probleem kunnen toepassen noteren we als $\HypSet$.
\end{definition}

We kunnen een afdwingbare beperking als een vorm van ervaring zijn. In de loop der tijd kunnen we immers een hypothese ontwikkelen dat sterke oplossingen aan een bepaalde voorwaarde voldoen (bijvoorbeeld een variabele in het probleem krijgt een vaste waarde). We kunnen dan oplossingen aantrekken naar de hypothese door $c^+$ op willekeurige oplossingen toe te passen. Anderzijds kunnen we ook oplossingen afstoten van de hypothese met $c^-$. De bovenliggende hyperheuristiek dient echter niet op de hoogte te zijn welke voorwaarden een concrete afdwingbare beperking stelt, zolang deze maar de oplossingen kan manipuleren.

\paragraph{}
Afdwingbare beperkingen kunnen we eveneens gebruiken om de zoekruimtes te beperken. Elk proces kan immers een aantal afdwingbare beperkingen gebruiken om een bepaalde zoekruimte te beschouwen, terwijl het de afdwingbare beperkingen van de andere processoren gebruikt om uit de buurt van de andere zoekruimtes te blijven.

\paragraph{}
De hyperheuristieken zelf kunnen geen ervaring genereren, ze hebben immers geen weet van de structuur van een oplossing. Daarom zal het specifieke probleem dus een set functies defini\"eren die we \emph{hypothese-generatoren (hypogen)} noemen:
\begin{definition}[Hypothese-generator]
Een \emph{hypothese-generator (hypogen)} $g_i:\SolSet^{n_{g_i}}\rightarrow\HypSet$ is een functie die op basis van een set oplossingen een afdwingbare beperking kan genereren.	
\end{definition}

\section{Probleemonafhankelijk gedeelte}

Bovenop het probleemafhankelijke gedeelte werkt een hyperheuristiek, een component die we los van \emph{ParHyFlex} kunnen zien. Het softwaresysteem kan hier ondersteuning bieden met ingebouwde en aanpasbare componenten.  In \emph{ParHyFlex} werden daarom de volgende systemen ge\"implementeerd: \emph{uitwisselen van oplossingen}, \emph{afbakenen van zoekruimtes}, \emph{genereren van ervaring} en \emph{onderhandelen over een nieuwe zoekruimte}. In de volgende subsecties zullen we deze taken verder bespreken.

\subsection{Uitwisselen van oplossingen}

Elke processor werkt met een eigen lokaal geheugen, maar reserveert ook plaats voor de geheugens van de andere processoren. Op het moment dat een nieuwe oplossing naar een lokale geheugencel geschreven wordt, zal op basis van een \emph{uitwisselingsstrategie} beslist worden met welke processoren deze oplossing zal worden gedeeld. De taak van het verzenden en ontvangen van een oplossing samen met een reeks uitwisselingsstrategie\"en wordt ondersteund door \emph{ParHyFlex}.

\subsection{Afbakenen van de zoekruimte}
 
De zoekruimte bewaken is ook een verantwoordelijkheid van \emph{ParHyFlex}. Hiervoor voorziet men twee sets van afdwingbare beperkingen: positieve en negatieve. Telkens wanneer er een nieuwe oplossing wordt gegenereerd\footnote{Of via uitwisseling in het geheugen wordt ingeladen.} zal \emph{ParHyFlex} alle beperkingen in de positieve set afdwingen en \'e\'en beperking uit de negatieve set. Om te vermijden dat de zoekruimte vaak verandert is het daarom belangrijk dat men niet te veel afdwingbare beperkingen in de negatieve set plaatst.

\paragraph{}
Het afdwingen gebeurt in een willekeurige volgorde. Beperkingen kunnen immers met elkaar interfereren: een eerste beperking kan een variabele op \'e\'en waarde zetten waarna de volgende beperkingen deze wijziging weer ongedaan maakt. Men kan dit probleem proberen op te lossen door alle permutaties uit te proberen in de hoop dat \'e\'en mutatie toch tot het correcte resultaat leidt. Dit is echter niet noodzakelijk zo, en bovendien vereist een dergelijke oplossing exponenti\"ele tijd. We nemen aan dat de beperkingen meestal minimaal met elkaar interfereren en dat een zoekruimte niet strikt moet worden bewaakt. De hierboven vernoemde strategie is niet verplicht. Men kan door een interface te implementeren een andere strategie hanteren.

\subsection{Genereren van ervaring}
 
Telkens wanneer \'e\'en van de processoren een nieuwe oplossing voortbrengt, kan deze oplossing -- samen met andere oplossingen -- worden omgezet in een afdwingbare beperking. Een processor kan echter niet alle beperkingen blijven bewaren: het uitwisselen van ervaring dient snel te gebeuren, we dienen een voldoende grote zoekruimte te behouden en bovendien kunnen we net een beperking genereren die het zoeken de foute kant opstuurt. Daarom maken we gebruik van een \emph{ervaring-set}, een set van vaste grootte waar gegenereerde beperkingen in worden bewaard.

\paragraph{}
De elementen in de set worden telkens ge\"evalueerd: wanneer er een nieuwe oplossing wordt gegenereerd, zal de \emph{ervaring-set} analyseren aan welke beperkingen de oplossing voldoet. Op basis van de fitness-waarde van de oplossingen kunnen de beperkingen dan ge\"evalueerd worden. Door de lijst van fitness-waardes op te delen in waardes waarbij de beperking wordt gerespecteerd en waardes waarin dat niet het geval is, ontstaan twee sets aan fitness-waardes.

\paragraph{}
Met een online algoritme\cite[p. 232]{citeulike:175026} berekenen we voor beide sets het gemiddelde en de variantie\footnote{We doen dit online om te voorkomen dat alle fitness-waardes moeten worden bijgehouden.}. We beschouwen beide sets dan ook als normaal verdeeld. We kunnen de kans uitrekenen dat dat een waarde uit een normaal verdeelde verdeling kleiner is dan een waarde uit een andere normale verdeling. Naarmate de kans groter wordt dat de fitness-waarde van oplossingen kleiner is conform de beperking, maken we de assumptie dat deze beperkingen sterker zijn.

\paragraph{}
Gegenereerde hypotheses zijn niet per definitie juist, zelfs indien oplossingen conform de beperking een betere fitness-waarde hebben. Daarom dienen we de set regelmatig van nieuwe hypotheses te voorzien, dit proces heet \emph{amnesie}. \emph{Amnesie} wordt op geregelde tijdstippen toegepast: oude hypotheses worden uit de set gehaald op plaats te maken voor nieuwe hypotheses. We wensen dat sterke hypothese meer kans maken om te overleven maar wel de kans lopen om op termijn te verdwijnen. Daarom rangschikken we de beperkingen op basis van hun evaluatie. De kans dat de hypothese vervolgens uit de set wordt gehaald berekenen we vervolgens op basis van de Benford-verdeling\cite{citeulike:748130} die enkel afhangt van de gesorteerde index.

\subsection{Onderhandelen over een nieuwe zoekruimte}

Elke processor houdt een \emph{ervaring-set} bij. Het is de bedoeling dat deze ervaring omgezet wordt in een nieuwe zoekruimte. Bovendien kan ervaring uitgewisseld worden met andere processoren zodat deze later ook hun zoekruimtes aanpassen. Anderzijds wensen we te voorkomen dat de zoekruimtes te homogeen worden en dus potentieel sterke oplossingen genegeerd worden. Dit zijn de taken van de \emph{onderhandelaar}.

\paragraph{}
De \emph{onderhandelaar} is een component die af en toe geactiveerd wordt. Een deel van de afdwingbare beperkingen worden uit de \emph{ervaring-set} gehaald om opgenomen te worden in het positieve component van de \emph{zoekruimte}. Deze beperkingen worden via groepscommunicatie doorgestuurd naar de andere processoren. Een deel van de ontvangen beperkingen wordt ge\"injecteerd in de \emph{ervaring-set} van de ontvangers. Een andere deel vormt de basis van het negatieve gedeelte van de \emph{zoekruimte}. Omdat een deel van de afdwingbare beperkingen vanaf dan in de \emph{ervaring-set} van de andere processoren wordt ge\"evalueerd (met een andere zoekruimte), is men in staat om zo'n beperking op een objectievere manier te evalueren\footnote{Sommige afdwingbare beperkingen leiden immers enkel tot sterke resultaten in een bepaalde \emph{zoekruimte}.}.

%\section{Invloed van eerdere studies}
\todo{Paginalimiet! Misschien weglaten.}

\section{Besluit van dit hoofdstuk}

\importtikz[1.4]{parhyflexstructure}{parhyflexstructure}{Structuur van \emph{ParHyFlex}.}
Op \imgref{parhyflexstructure} geven we schematisch de structuur van \emph{ParHyFlex} weer.	De componenten die gemarkeerd worden met een asterisk, zijn component die niet aanwezig zijn in \emph{HyFlex}. Het grijze blok stelt de kern van het \emph{ParHyFlex} systeem voor.%TODO

\paragraph{}
Een deel van de geheugencellen is gemarkeerd met een schuine streep. Deze geheugencellen stellen vreemd geheugen voor waarvan er lokaal een kopie wordt bijgehouden. De geheugencellen kunnen uitgelezen worden, maar er kan geen oplossing naar geschreven worden.

\paragraph{}
\importtikz[1.4]{parhyflexwerking}{parhyflexwerking}{Schematische voorstelling van de kern van \emph{ParHyFlex}.}
Op \imgref{parhyflexwerking} beschrijven we kort het proces die een berekende of ontvangen oplossing doormaakt. Deze oplossing -- op de figuur $s_1^{(0)}$ -- wordt eerst aangepast door de zoekruimte: alle positieve hypotheses en \'e\'en negatieve hypothese worden toegepast op de oplossing en wordt aangepast tot $s_1^{(E)}$ die binnen de zoekruimte valt. De fitness-waarde wordt berekend en de evaluaties van de reeds aanwezige hypotheses in de ervaring-set worden aangepast (de data wordt voor elke hypothese opgenomen in \'e\'en van de twee normale verdelingen). Verder wordt met behulp van \'e\'en van de hypothesegeneratoren  een hypothese gegenereerd die met een bepaalde kans opgenomen wordt in de ervaring-set. De oplossing wordt vervolgens in het geheugen opgenomen en eventueel doorgestuurd naar andere processoren.

\paragraph{}
Op geregelde tijdstippen treed er amnesie op in de \emph{ervaring-set}: een deel van de hypothese worden uit de set verwijdert. Dit gebeurt op basis van de twee normale verdelingen per hypothese. Op die manier kan men zich ontdoen van foute hypothese, en maakt men ruimte voor nieuwe hypotheses.

\paragraph{}
Op vaste tijdsintervallen zal de \emph{onderhandelaar} een deel van de hypotheses uit de \emph{ervaring-set} halen. Een deel van deze hypotheses vormen de nieuwe positieve set van de \emph{zoekruimte}. De overige worden doorgestuurd in de \emph{ervaring-set} van de andere processoren ge\"injecteerd. Een deel van de doorgestuurde hypotheses vormt ook een basis van de negatieve set van de \emph{zoekruimte}.

\paragraph{}
De figuur toont de verschillende stromen van informatie: volle lijnen duiden op informatiestromen die lokaal worden uitgevoerd. Streepjeslijnen duiden op informatie die de processor uitstuurt naar andere processoren. Stippellijnen tonen de informatie die de processor ontvangt van andere processoren.