\documentclass[master=elt,masteroption=ge]{kulemt}
\usepackage{amssymb,amsmath,slantsc,bbm,tikz,tocloft,etoolbox,hyphenat,pdfpages, amsthm,graphicx,cite,multicol,ltablex,enumitem,todonotes}
\usepackage[algochapter,ruled,vlined]{algorithm2e}
%\usepackage[showframe]{geometry}
\usepackage[all]{xy}

\setlength{\parskip}{1.3ex plus 0.2ex minus 0.2ex}
\setlength{\parindent}{0pt}

\SetKwInput{KwData}{Data}

\SetAlgorithmName{Algoritme}{willemvanonsem}{Lijst van algoritmes}

\usetikzlibrary{shapes,fit,calc}

\newcommand{\baselab}[2]{\label{#1:#2}}
\newcommand{\baseref}[3]{\textsc{\nohyphens{#3}~\textup{\ref{#1:#2}}}}

\newcommand{\deflab}[1]{\baselab{def}{#1}}
\newcommand{\defref}[1]{\baseref{def}{#1}{Definitie}}
\newcommand{\imglab}[1]{\baselab{fig}{#1}}
\newcommand{\imgref}[1]{\baseref{fig}{#1}{Figuur}}
\newcommand{\tbllab}[1]{\baselab{tbl}{#1}}
\newcommand{\tblref}[1]{\baseref{tbl}{#1}{Tabel}}
\newcommand{\alglab}[1]{\baselab{alg}{#1}}
\newcommand{\algref}[1]{\baseref{alg}{#1}{Algoritme}}
\newcommand{\eqnlab}[1]{\baselab{eqn}{#1}}
\newcommand{\eqnref}[1]{\textsc{Vergelijking~(\ref{eqn:#1})}}
\newcommand{\thelab}[1]{\baselab{the}{#1}}
\newcommand{\theref}[1]{\baseref{the}{#1}{Theorema}}
\newcommand{\chplab}[1]{\baselab{chp}{#1}}
\newcommand{\chpref}[1]{\baseref{chp}{#1}{Hoofdstuk}}
\newcommand{\applab}[1]{\baselab{app}{#1}}
\newcommand{\appref}[1]{\baseref{app}{#1}{Appendix}}
\newcommand{\seclab}[1]{\baselab{sec}{#1}}
\newcommand{\secref}[1]{\baseref{sec}{#1}{Sectie}}
\newcommand{\ssclab}[1]{\baselab{ssc}{#1}}
\newcommand{\sscref}[1]{\baseref{ssc}{#1}{Subsectie}}
\newcommand{\ssslab}[1]{\baselab{sss}{#1}}
\newcommand{\sssref}[1]{\baseref{sss}{#1}{Subsubsectie}}

\newtheorem{defnition}{Definitie}
\numberwithin{defnition}{chapter}
\newtheorem{theorem}{Theorema}

\newcommand{\listdefinitionname}{Lijst van definities}
\newlistof{X}{eX}{\listdefinitionname}
\newenvironment{definition}[1][]{
  \addcontentsline{eX}{figure}{\protect\numberline{\thechapter}#1}
\begin{defnition}[#1]}
{\end{defnition}}

\makeatletter
\preto\chapter{\addtocontents{def}{\protect\addvspace{10\p@}}}%
\makeatother

\newcommand{\mathbftxt}[1]{\ensuremath{\mbox{\textbf{#1}}}}

\newcommand{\brak}[1]{\ensuremath{\left(#1\right)}}
\newcommand{\fbrk}[1]{\ensuremath{\left[#1\right]}}
\newcommand{\tupl}[1]{\ensuremath{\left\langle #1\right\rangle}}
\newcommand{\accl}[1]{\ensuremath{\left\{ #1\right\}}}
\newcommand{\abs}[1]{\ensuremath{\left| #1\right|}}
\newcommand{\dabs}[2][]{\ensuremath{\left\| #2\right\|_{#1}}}
\newcommand{\ceil}[1]{\ensuremath{\left\lceil #1\right\rceil}}
\newcommand{\floor}[1]{\ensuremath{\left\lfloor #1\right\rfloor}}
\newcommand{\transpose}[1]{\ensuremath{{#1}^{\top}}}

\newcommand{\conditional}[2]{\begin{array}{cc}#1&\mbox{(#2)}\end{array}}
\newcommand{\guards}[1]{\left\{\begin{array}{ll}#1\end{array}\right.}

\newcommand{\fun}[2]{\ensuremath{#1\brak{#2}}}
\newcommand{\funf}[2]{\ensuremath{#1\fbrk{#2}}}
\newcommand{\funm}[2]{\fun{\mbox{#1}}{#2}}
\newcommand{\funsig}[3]{#1:#2\rightarrow #3}
\newcommand{\funsigimp}[5]{#1:#2\rightarrow #3:#4\mapsto #5}

\newcommand{\bigoh}[1]{\fun{\mathcal{O}}{#1}}

\newcommand{\xstar}{\ensuremath{x^{\star}}}
\newcommand{\xdot}{\ensuremath{x^{\circ}}}
\newcommand{\calXop}{\calX^{\star}}

\newcommand{\argmin}{\ensuremath{\mbox{argmin}}}
\newcommand{\mean}[2][]{\funf{\EEE_{#1}}{#2}}
\newcommand{\krdelta}[1]{\funf{\delta}{#1}}
\newcommand{\Prob}[1]{\funf{\Pr}{#1}}

\newcommand{\comp}[1]{\textsc{\mbox{#1}}}
\newcommand{\algo}[1]{\nohyphens{\textsc{#1}}}
\newcommand{\auth}[2][]{\emph{\nohyphens{#2}}\ifthenelse{\equal{#1}{}}{}{\cite{#1}}}
\newcommand{\prob}[1]{\textup{\textsc{#1}}}
\newcommand{\prbm}[1]{\textup{\textsc{#1}}}
\newcommand{\work}[1]{``\emph{#1}''}

\newcommand{\AAA}{\mathbb{A}}
\newcommand{\BBB}{\mathbb{B}}
\newcommand{\CCC}{\mathbb{C}}
\newcommand{\DDD}{\mathbb{D}}
\newcommand{\EEE}{\mathbb{E}}
\newcommand{\FFF}{\mathbb{F}}
\newcommand{\GGG}{\mathbb{G}}
\newcommand{\HHH}{\mathbb{H}}
\newcommand{\III}{\mathbb{I}}
\newcommand{\JJJ}{\mathbb{J}}
\newcommand{\KKK}{\mathbb{K}}
\newcommand{\LLL}{\mathbb{L}}
\newcommand{\MMM}{\mathbb{M}}
\newcommand{\NNN}{\mathbb{N}}
\newcommand{\OOO}{\mathbb{O}}
\newcommand{\PPP}{\mathbb{P}}
\newcommand{\QQQ}{\mathbb{Q}}
\newcommand{\RRR}{\mathbb{R}}
\newcommand{\SSS}{\mathbb{S}}
\newcommand{\TTT}{\mathbb{T}}
\newcommand{\UUU}{\mathbb{U}}
\newcommand{\VVV}{\mathbb{V}}
\newcommand{\WWW}{\mathbb{W}}
\newcommand{\XXX}{\mathbb{X}}
\newcommand{\YYY}{\mathbb{Y}}
\newcommand{\ZZZ}{\mathbb{Z}}

\newcommand{\zeromatrix}[1][]{\boldsymbol{0}_{#1}}
\newcommand{\identitymatrix}[1][]{\boldsymbol{I}_{#1}}
\newcommand{\onematrix}[1][]{\boldsymbol{1}_{#1}}

\newcommand{\calA}{\mathcal{A}}
\newcommand{\calB}{\mathcal{B}}
\newcommand{\calC}{\mathcal{C}}
\newcommand{\calD}{\mathcal{D}}
\newcommand{\calE}{\mathcal{E}}
\newcommand{\calF}{\mathcal{F}}
\newcommand{\calG}{\mathcal{G}}
\newcommand{\calH}{\mathcal{H}}
\newcommand{\calI}{\mathcal{I}}
\newcommand{\calJ}{\mathcal{J}}
\newcommand{\calK}{\mathcal{K}}
\newcommand{\calL}{\mathcal{L}}
\newcommand{\calM}{\mathcal{M}}
\newcommand{\calN}{\mathcal{N}}
\newcommand{\calO}{\mathcal{O}}
\newcommand{\calP}{\mathcal{P}}
\newcommand{\calQ}{\mathcal{Q}}
\newcommand{\calR}{\mathcal{R}}
\newcommand{\calS}{\mathcal{S}}
\newcommand{\calT}{\mathcal{T}}
\newcommand{\calU}{\mathcal{U}}
\newcommand{\calV}{\mathcal{V}}
\newcommand{\calW}{\mathcal{W}}
\newcommand{\calX}{\mathcal{X}}
\newcommand{\calY}{\mathcal{Y}}
\newcommand{\calZ}{\mathcal{Z}}

\newcommand{\BoolSet}{\BBB}
\newcommand{\NatSet}{\NNN}
\newcommand{\RealSet}{\RRR}
\newcommand{\QbitSet}[1]{\funf{\QQQ}{#1}}
\newcommand{\OpProblem}{\Pi}
\newcommand{\ConfigSet}{\calX}
\newcommand{\ConfigValSet}{\ConfigSet'}
\newcommand{\ConfigOpSet}{\ConfigSet^{\star}}
\newcommand{\sol}{x}
\newcommand{\bestSol}{\sol^{\star}}
\newcommand{\goodSol}{\sol^{\circ}}
\newcommand{\SolSet}{\calS}
\newcommand{\PopSet}{\calP}
\newcommand{\HypSet}{\calH}
\newcommand{\TranSet}{\calT}
\newcommand{\hcfun}{c}
\newcommand{\evalfun}{f}
\newcommand{\evalfuna}{f'}
\newcommand{\hittime}{\theta}
\newcommand{\phittime}{\Theta}
\newcommand{\neighbr}{\calN}
\newcommand{\nvar}{n}
\newcommand{\VarDom}{A}
\newcommand{\PopChain}{\mathfrak{P}}
\newcommand{\evl}[1][]{\ensuremath{\vec{\mu}_{#1}}}%left eigenvector
\newcommand{\evr}[1][]{\ensuremath{\vec{\nu}_{#1}}}%right eigenvector
\newcommand{\ev}[1][]{\ensuremath{\lambda_{#1}}}%right eigenvalue
\newcommand{\isdefinedas}{\ensuremath{:=}}%right eigenvalue
\newcommand{\smbox}[1]{\ensuremath{\mbox{\small{#1}}}}%right eigenvalue
\newcommand{\fitnessval}{\ensuremath{v}}
\newcommand{\ntarpop}{\ensuremath{g}}
\newcommand{\npop}{\ensuremath{N}}
\newcommand{\transmat}{\ensuremath{P}}
\newcommand{\bridgemat}{\ensuremath{B}}
\newcommand{\removmat}{\ensuremath{\hat{P}}}
\newcommand{\probdistmeta}[1]{\ensuremath{\vec{\alpha}_{#1}}}
\newcommand{\probdistgoalmeta}[1]{\ensuremath{\vec{\alpha}^G_{#1}}}
\newcommand{\probdistnormmeta}[1]{\ensuremath{\vec{\hat{\alpha}}_{#1}}}
\newcommand{\arbitraryval}{\ensuremath{m}}
\newcommand{\arbitraryvec}{\ensuremath{\vec{x}}}
\newcommand{\metapa}{\ensuremath{\sigma}}
\newcommand{\metapb}{\ensuremath{\lambda}}
\newcommand{\bitvar}{\ensuremath{L}}

\SetKw{true}{\mathbftxt{true}}
\SetKw{fals}{\mathbftxt{false}}
\SetKw{nult}{\mathbftxt{null}}

\SetKwBlock{ParFor}{parfor}{end}
\SetKwBlock{WhnRcv}{When receiving}{}
\SetKwBlock{myalg}{Algorithm}{end}
\SetKwBlock{myproc}{Procedure}{end}

\SetKwFunction{pinit}{Initialiseer}
\SetKwFunction{pgata}{GatherAll}
\SetKwFunction{predy}{Gereed}
\SetKwFunction{prest}{Reset}
\SetKwFunction{pzedt}{ZendData}
\SetKwFunction{psend}{send}
\SetKwFunction{pisnd}{isend}
\SetKwFunction{precv}{receive}
\SetKwFunction{parry}{array}

\SetKwData{dres}{resultaat}
\SetKwData{ddim}{dimensies}
\SetKwData{dbas}{basis}
\SetKwData{didq}{id}
\SetKwData{dsnd}{zender}
\SetKwData{drcv}{ontvanger}
\SetKwData{dmsg}{bericht}
\SetKwData{dodt}{eigenData}
\SetKwData{dprt}{partner}

\newcommand{\chapterquote}[2]{\begin{figure*}[htb]\centering\begin{tikzpicture}\node[text width=\textwidth-1.75 cm,anchor=center] (Q) at (0,0) {\Large\textit{\nohyphens{#1}}};\node[gray,anchor=north east] (Ql) at (Q.north west) {\Huge\textbf{``}};\node[gray,anchor=north west] (Qr) at (Q.south east) {\Huge\textbf{''}};\node[black!80,anchor=north east] (Qa) at (Qr.north west) {\small - #2};\end{tikzpicture}\end{figure*}}

\selectcolormodel{gray}
\newcommand{\importtikz}[4][1.00]{\begin{figure}[hbt]\begin{center}\def\sc{#1}\input{tikzpictures/#2}\caption{#4}\imglab{#3}\end{center}\end{figure}}
\newcommand{\importalgo}[3]{\begin{algorithm}[hbt]\input{algorithms/#1}\caption{#2}\alglab{#3}\end{algorithm}}

\newcommand{\drawmem}[3]{
\node[draw,rectangle,fill=gray!20,minimum width=\sc*0.5*#2 cm, minimum height=\sc*0.5cm] (#1) at (0,0.5) {};
\foreach\x in {2,...,#2} {
  \draw (0.5*\x-0.25*#2-0.5,0.25) -- ++(0,0.5);
}
\foreach\x in {#3,...,#2} {
  \draw (0.5*\x-0.25*#2-0.5,0.25) -- ++(0.5,0.5);
}
\foreach\x in {1,...,#2} {
  \node[minimum height=\sc*0.5 cm,minimum width=\sc*0.5 cm] (#1\x) at (0.5*\x-0.25*#2-0.25,0.5) {$s_{\x}$};
}
\draw (#1.north) node[anchor=south] {\small{Geheugen}};
}

\tikzset{hypothesis/.style={draw=black,regular polygon,regular polygon sides=3,scale=0.7,fill=gray,inner sep=0 pt},solution/.style={draw=black,circle,scale=0.7,inner sep=0 pt},outstream/.style={dashed},instream/.style={dotted},llh/.style={circle,fill=black,thick,draw=gray}}

\newcommand{\ubridge}[4][1]{\nbridge[-#1]{#2}{#3}{#4}}
\newcommand{\nbridge}[4][1]{\draw[#2] (#3) .. controls ($(#3)+(0,#1)$) and ($(#4)+(0,#1)$) .. (#4);}
\newcommand{\nubridge}[4][1]{\draw[#2] (#3) .. controls ($(#3)+(0,#1)$) and ($(#4)-(0,#1)$) .. (#4);}
\newcommand{\unbridge}[4][1]{\nubridge[-#1]{#2}{#3}{#4}}

\newcommand{\ubridgearrow}[3][1]{\ubridge[#1]{->}{#2}{#3}}
\newcommand{\nbridgearrow}[3][1]{\nbridge[#1]{->}{#2}{#3}}
\newcommand{\unbridgearrow}[3][1]{\unbridge[#1]{->}{#2}{#3}}
\newcommand{\nubridgearrow}[3][1]{\nubridge[#1]{->}{#2}{#3}}

\newcommand{\defrect}[5][]{\node[rectangle,draw=black,minimum width=#4, minimum height=#5,#1] (#2) at (#3) {};}

\newcommand{\drawcube}[2]{
\coordinate (A) at (0,0,0);
\coordinate (B) at (0,0,#1);
\coordinate (C) at (0,#1,0);
\coordinate (D) at (0,#1,#1);
\coordinate (E) at (#1,0,0);
\coordinate (F) at (#1,0,#1);
\coordinate (G) at (#1,#1,0);
\coordinate (H) at (#1,#1,#1);
\foreach \x/\t in {#2} {
  \fill (\x) circle (0.1 cm) node[anchor=south east]{\t};
}
\foreach \x/\y in {B/D,B/F,C/D,C/G,D/H,E/F,E/G,F/H,G/H} {
  \draw (\x) -- (\y);
}
\draw[dashed] (B) -- (A);
\draw (C) -- (C |- D);
\draw[dashed] (C |- D) -- (A);
\draw (E) -- (E -| F);
  \draw[dashed] (E -| F) -- (A);
}

\newcommand{\edgeinteraction}[3][0.25,0]{
\draw[<->] ($0.85*(#2)+0.15*(#3)+(#1)$) -- ($0.15*(#2)+0.85*(#3)+(#1)$);
}
\setup{title={Parallelle Hyperheuristieken},
  author={Willem Van Onsem},
  promotor={Prof.\,dr.\ Bart Demoen},
  assessor={Ir.\,W. Eetveel\and W. Eetrest},
  assistant={Ir.\ A.~Assistent \and D.~Vriend}}
% De volgende \setup mag verwijderd worden als geen fiche gewenst is.
\setup{filingcard,
  translatedtitle={The best master thesis ever},
  udc=681.3,
  shortabstract={Hier komt een heel bondig abstract van hooguit 500
    woorden. \LaTeX\ commando's mogen hier gebruikt worden. Blanco lijnen
    (of het commando \texttt{\string\pa r}) zijn wel niet toegelaten!
    \endgraf \lipsum[2]}}
% Verwijder de "%" op de volgende lijn als je de kaft wil afdrukken
%\setup{coverpageonly}
% Verwijder de "%" op de volgende lijn als je enkel de eerste pagina's wil
% afdrukken en de rest bv. via Word aanmaken.
%\setup{frontpagesonly}

% Kies de fonts voor de gewone tekst, bv. Latin Modern
\setup{font=lm}

% Hier kun je dan nog andere pakketten laden of eigen definities voorzien

% Tenslotte wordt hyperref gebruikt voor pdf bestanden.
% Dit mag verwijderd worden voor de af te drukken versie.
\usepackage[pdfusetitle,colorlinks,plainpages=false]{hyperref}

%%%%%%%
% Om wat tekst te genereren wordt hier het lipsum pakket gebruikt.
% Bij een echte masterproef heb je dit natuurlijk nooit nodig!
\IfFileExists{lipsum.sty}%
 {\usepackage{lipsum}\setlipsumdefault{11-13}}%
 {\newcommand{\lipsum}[1][11-13]{\par Hier komt wat tekst: lipsum ##1.\par}}
%%%%%%%

%\includeonly{hfdst-n}
\begin{document}

\begin{preface}
\chapterquote{Begegnet uns jemand, der uns Dank schuldig ist, gleich f\"alt uns ein. Wie oft k\"onnen wir jemand begegnen, dem wir Dank schuldig sind, ohne daran zu denken.}{Johann Wolfgang von Goethe}
  Bij deze wil ik iedereen bedanken die het afgelopen academiejaar aan mij gedacht heeft. In de eerste plaats omdat dit waarde hiervan niet opwoog tegen het denken aan iets anders. \cite{gastonVanCamp} stelt dat een mens in 1971 biologisch een waarde heeft van 180 frank ofwel 30.00 euro in 2013.
\end{preface}

\tableofcontents*

\begin{abstract}
  In dit \texttt{abstract} environment wordt een al dan niet uitgebreide
  samenvatting van het werk gegeven. De bedoeling is wel dat dit tot
  1~bladzijde beperkt blijft.

  \lipsum[1]
\end{abstract}

% Een lijst van figuren en tabellen is optioneel
%\listoffigures
%\listoftables
% Bij een beperkt aantal figuren en tabellen gebruik je liever het volgende:
\listoffiguresandtables
% De lijst van symbolen is eveneens optioneel.
% Deze lijst moet wel manueel aangemaakt worden, bv. als volgt:
\chapter{Lijst van afkortingen en symbolen}
\section*{Afkortingen}
\begin{flushleft}
  \renewcommand{\arraystretch}{1.1}
  \begin{tabularx}{\textwidth}{@{}p{22mm}X@{}}%TODO: MODIFIED!!!
    CHeSC	& Cross-domain Heuristic Search Challenge \\
    LLH		&Low-level heuristics \\
    MCHH-S	&Markov Chain Hyper-Heuristic\\
    GISS	&Generic Iterative Simulated Annealing Search\\
    DynILS	&Dynamic Iterated Local Search\\
    ACO-HH	&Ant Colony Optimization Hyper-Heuristic\\
    HAEA	&Hybrid Adaptive Evolutionary Algorithm\\
    ISEA	&Iterated Search by Evolutionary Algorithm\\
    EPH		&Evolutionary Programming Hyper-heuristic\\
    VNS		&Variable Neighborhood Search\\
    CBM		&Coalition Based Metaheuristic\\
    ACO		&Ant Colony Optimization\\
    SA		&Simulated Annealing\\
    GA		&Genetic Algorithm\\
    LS		&Local Search\\
    ILS		&Iterated Local Search\\
    VNS		&Variable Neighborhood Search\\
    LMS		&Local Search Metaheuristics
  \end{tabularx}
\end{flushleft}
\section*{Symbolen}
\begin{flushleft}
  \renewcommand{\arraystretch}{1.1}
  \begin{tabularx}{\textwidth}{@{}p{22mm}X@{}}
    $\BoolSet$					& De verzameling van Booleaanse waarden: $\mathbb{B}=\left\{\mbox{\textbf{true}},\mbox{\textbf{false}}\right\}$. \\
    $\NatSet$					& De verzameling van natuurlijke getallen: $\mathbb{N}=\left\{0,1,2,\ldots\right\}$ \\
    $\zeromatrix[m\times n]$			& Een nulmatrix met dimensies $m\times n$\\
    $\onematrix[m\times n]$			& Een matrix vol enen met dimensies $m\times n$\\
    $\identitymatrix[m\times n]$		& Een identiteitsmatrix met dimensies $m\times n$\\
    $\RealSet$					& De verzameling van re\"ele getallen\\
    $\mean[\fun{\calD}{x}]{\fun{f}{x}}$	& Het gemiddelde van een functie $f$ volgens een verdeling $\calD$\\%: $\mean[\fun{\calD}{x}]{\fun{f}{x}}=\displaystyle\int\fun{f}{x}\fun{\calD}{x}\ dx$\\
    $\Prob{e}$					& De kans op een gebeurtenis $e$\\
    $\OpProblem$				& Optimalisatieprobleem\\
    $\ConfigSet$				& De verzameling van mogelijke configuraties\\
    $\ConfigOpSet$				& De set van de globaal optimale oplossingen\\
    $\ConfigValSet$				& De verzameling van geldige configuraties: \\
    $\sol$					& Een globaal optimale oplossing: $\sol\in\ConfigSet$\\
    $\bestSol$					& Een globaal optimale oplossing: $\xstar\in\ConfigOpSet$\\
    $\SolSet$					& Een oplossingsruimte in de context van een metaheuristiek. $S\subseteq X$\\
    $\PopSet$					& Een verzameling oplossingen ofwel \emph{populatie}\\
    $\hcfun$					& Harde beperkingen: $\funsig{\hcfun}{\ConfigSet}{\BoolSet}$\\
    $\evalfun$					& Evaluatiefunctie: $\funsig{\evalfun}{\ConfigSet}{\RealSet}$\\
    $\hittime$					& De raaktijd voor een optimalisatieprobleem in een sequenti\"ele context\\
    $\phittime$					& De raaktijd voor een optimalisatieprobleem in een parallelle context\\
    $\neighbr$					& Omgeving van een oplossing\\
    $\ev[i,A]$					& De $i$-de eigenwaarde van een matrix $A$. De eigenwaardes zijn gerangschikt van groot naar klein: $\abs{\ev[i,A]}\geq\abs{\ev[i+1,A]}$\\
    $\ev[A]$					& De dominante eigenwaarde van een matrix $A$\\
    $\evl[i,A]$					& De linkse eigenvector die bij de $i$-de eigenwaarde van matrix $A$ hoort\\
    $\evr[i,A]$					& De rechtse eigenvector die bij de $i$-de eigenwaarde van matrix $A$ hoort\\
    $\krdelta{x}$				& De Kr\"onecker-delta voor een gegeven Booleaanse expressie $x$\\
  \end{tabularx}
\end{flushleft}

% Nu begint de eigenlijke tekst
\mainmatter

\chapter{Inleiding}
\chplab{inleiding}
\chapterquote{If a man will begin with certainties, he shall end in doubts; but if he will be content to begin with doubts, he shall end in certainties.}{Francis Bacon}

Optimalisatie is een belangrijk onderwerp in de artifici\"ele intelligentie. Allereerst komen bij heel wat algemene problemen in de artifici\"ele intelligentie vormen van optimalisatie kijken. Het bepalen van de optimale parameters, het kortste pad of de kleinste beslissingsboom die een bepaalde fractie van de data juist classificeert zijn enkele voorbeelden.

\paragraph{}
Hoewel elke tak in de artifici\"ele intelligentie met deze problemen wordt geconfronteerd, bestaat er een specifieke tak die zich focust op deze problemen: ``Operationeel Onderzoek'' ofwel ``Operational Research''. Problemen die men tracht op te lossen zijn doorgaans moeilijk van aard: er is sprake van een groot aantal potenti\"ele oplossingen, het evalueren van een mogelijke oplossing is geen sinecure en de regels aan dewelke een oplossing moet voldoen zijn vrij complex.

\section{Operationeel Onderzoek}

Operationeel onderzoek probeert doorgaans optimalisatieproblemen op te lossen in een zeer concrete en praktische setting. Belangrijke voorbeelden zijn bijvoorbeeld het genereren van productieplanningen, het optimaliseren van winstmarges,... Omdat de zoekruimte dermate groot is, is het vinden van de exacte oplossing meestal niet mogelijk, men stelt zich in de meeste gevallen dan ook tevreden met een benaderende oplossing.

\paragraph{}
De laatste jaren is er een trend naar telkens meer ge\"integreerde systemen: algoritmen die onafhankelijk van het concrete probleem toch een oplossing kunnen uitrekenen. Deze evolutie is vooral te wijten aan de kost die de ontwikkeling van benaderingsalgoritmen met zich meebrengen. Omdat ontwikkelingskosten meestal hoger liggen dan de kosten ten gevolge door rekentijd is het financieel interessant om te investeren in systemen die met een minimale inspanning kunnen worden ontwikkeld. Belangrijk hierbij is de bouw van componenten die men kan hergebruiken voor het effectief oplossen van andere problemen. Een belangrijke groep van dit soort systemen zijn \emph{Hyperheuristieken}.

\section{Parallelle algoritmen}

Het oplossen van problemen met probleemonafhankelijke algoritmen komt met een kostprijs: we verwachten dat op maat gemaakte algoritmen effici\"enter zullen werken en dus binnen een gegeven tijd betere oplossingen zullen voorstellen. Door te investeren in betere machines kan men dit verlies enigszins goedmaken. Men kan de kloksnelheid van een processor echter niet eindeloos opdrijven. Deze fysische beperking betekent bijgevolg een limiet op de resultaten die \'e\'en processor kan afleveren.

\paragraph{}
Parallelle algoritmen worden meestal ingezet wanneer de rekenkracht van \'e\'en processor tekortschiet. Door het programma op te splitsen in verschillende deelprogramma's die elk op \'e\'en processor draaien hoopt men het rekenvermogen te kunnen opdrijven. Deze trend is bovendien ook zichtbaar door de introductie van meerdere \emph{kernen} ofwel \emph{cores} in moderne processoren. Vertaald naar optimalisatieproblemen hopen we dus met behulp van parallelle algoritmen tot betere oplossingen te komen die sneller uitgerekend kunnen worden.

\section{Onderzoeksvraag}

In deze masterthesis onderzoeken we of het mogelijk is om de prestaties van hyperheuristieken te verbeteren door deze parallel of verschillende processoren te laten draaien. Een belangrijk deel in deze onderzoeksvraag is welke componenten hiertoe kunnen bijdragen.

\paragraph{}
Vermits hyperheuristieken probleemonafhankelijk werken, moeten ze door middel van ervaring leren hoe men het probleem kan oplossen. Door verschillende processen ervaring met elkaar te laten uitwisselen kan de tijd waarin men voornamelijk rekenkracht investeert in het opdoen van ervaring mogelijk worden verkort. Dit kan echter ook averechts werken: wanneer men te snel in een exploitatiefase komt, is het mogelijk dat men uitspraken doet op basis van een te kleine hoeveelheid ervaring. Bovendien kan men de de kennis die \'e\'en processor heeft opgedaan niet altijd zomaar overdragen naar de andere processoren.

\section{Gevolgde methodiek}

In een eerste fase werd de beschikbare literatuur geraadpleegd. Er is veel literatuur te vinden rond het parallelliseren van metaheuristieken. Parallelle hyperheuristieken en hyperheuristieken in het algemeen zijn echter een vrij nieuw domein. Er bestaan dan ook slechts enkele concrete implementaties en weinig onderzoeken in verband die het effect van de verschillende paradigma testen. De belangrijkste werken worden dan ook vermeld in \sscref{defparhyhe}.

\paragraph{}
Om meer inzicht te verwerven in de structuur van hyperheuristieken werden 16 verschillende implementaties onderzocht. De resultaten van dit onderzoek worden gerapporteerd in \chpref{chesc} en \appref{chesc}. Op basis van deze studie werden er enkele hypotheses naar voren geschoven die mogelijk verklaren waarom sommige hyperheuristieken beter werken dan anderen.

\paragraph{}
Op basis van de opgedane kennis werd een systeem ge\"implementeerd die het parallelliseren van hyperheuristieken ondersteund. Dit systeem wordt uitvoerig besproken in \chpref{parhyf}. Ook werd een concrete hyperheuristiek die uit de vermelde studie, \emph{AdapHH}, aangepast zodat deze op dit systeem kan werken.

\paragraph{}
Om de invloed van de verschillende componenten te testen, werden twee problemen ge\"implementeerd. Door vervolgens de parameters aan te passen of componenten artificieel uit te schakelen, kunnen we de invloed van deze componenten op de prestaties van de hyperheuristiek nagaan. De resultaten van dit onderzoek staan in \chpref{resul}.

\paragraph{}
Enkele activiteiten met betrekking tot deze thesis werden niet opgenomen in dit werk. Dit komt omdat deze activiteiten een te kleine relevante bijdrage leverden. Het gaat hier in de eerste plaats om de implementatie van een systeem die het implementeren van een hyperheuristiek op een gestructureerde manier toelaat. Een programmeur kan door verschillende modulaire componenten samen te nemen een eigen hyperheuristiek bouwen. De bedoeling van dit systeem is om op termijn meer inzicht te verwerven in de invloed van sommige componenten op de prestaties van een hyperheuristiek.

\paragraph{}
Daarnaast werd een visualisatiesysteem genaamd \emph{ParVis} ge\"implementeerd. Dit systeem toont de toestand waarin de verschillende processoren zich bevinden samen met de berichten die onderling verstuurd worden. De bedoeling van dit softwarepakket is om de werking van een parallel algoritme beter te kunnen begrijpen. De broncode van dit softwaresysteem is te downloaden op \url{http://goo.gl/YPuMr}. Het softwarepakket omvat ook een voorbeeld die de werking van een parallel \algo{Sum-Product}-algoritme en een \emph{asynchrone GatherAll}\footnote{Zie \secref{mpimod}.} illustreert.


\section{Structuur}
In \chpref{defi} defini\"eren we het probleemdomein en de concepten hieromtrent. We bespreken in dit hoofdstuk ook de relevante literatuur samen met de huidige stand van zaken.

\paragraph{}
\chpref{chesc} omvat een onderzoek naar sequenti\"ele hyperheuristieken. We onderzoeken 16 verschillende implementaties die in 2011 werden voorgesteld op een competitie. Op elk van de implementaties wordt kritiek geleverd en op het einde schuiven we enkele hypotheses naar voren waarom sommige hyperheuristieken beter presteren dan anderen. Een deel van deze hypotheses wordt ook verder beargumenteerd met empirisch onderzoek.

\paragraph{}
We bespreken het \emph{ParHyFlex}-systeem in \chpref{parhyf}. Dit systeem ondersteund de bouw van parallelle hyperheuristieken en werd gebouwd op basis van de literatuurstudie in \chpref{defi} en de hypotheses die voortkomen uit \chpref{chesc}. Het systeem omvat drie grote componenten: \emph{uitwisselen van ervaring}, \emph{opdoen van ervaring} en \emph{afbakenen van een zoekruimte}.

\paragraph{}
In \chpref{paradaphh} werken we een concrete hyperheuristiek uit. \emph{ParAdapHH} is een parallelle variant van \emph{AdapHH}, de winnaar van de competitie die we in \chpref{chesc} hebben bestudeerd. We bespreken de werking van \emph{AdapHH}, maken een analyse over de verschillende bronnen van parallellisatie samen met de uiteindelijke implementatie.

\paragraph{}
In \chpref{resul} testen we het systeem met behulp van twee problemen: \prbm{Max-3Sat} en het \prbm{Finite Domain Constraint Optimization Problem}. Er worden verschillende testresultaten voorgesteld die de invloed van bepaalde componenten en parameters onderzoeken.



%%% Local Variables: 
%%% mode: latex
%%% TeX-master: "masterproef"
%%% End: 

%%% ASPELL CHECK 2013-05-20
\chapter{Definities en State-of-the-Art}
\label{hoofdstuk:1}

In de inleiding hebben we de kort de verschillende concepten besproken die in deze thesis een belangrijke rol zullen spelen. In dit hoofdstuk gaan we hier dieper op in: we formaliseren de concepten in definities en geven een kort overzicht van de belangrijkste wetmatigheden rond deze concepten.

\section{Parallelle algoritmes}

\subsubsection{Motivatie}

De ``Wet van Moore''\cite{4785860} stelt dat het aantal transistors verdubbelt elke achttien maanden. De klokfrequentie stagneert echter en is door fysische grenzen niet langer makkelijk te verhogen. Het gevolg van deze evolutie is dat complexe problemen enkel uitgerekend kunnen worden indien men voldoende rekentijd beschikbaar stelt.

\paragraph{}
Een reactie is de introductie van het ``\emph{parallel rekenen}'' ofwel ``\emph{parallel computing}''. Een probleem wordt opgelost door programma's die op verschillende processoren tegelijk te draaien en elk een deel van het probleem oplossen. Door berichten uit te wisselen tussen de verschillende processoren of door een gemeenschappelijk geheugen aan te bieden communiceren de programma's die op de verschillende processoren draaien met elkaar.

\paragraph{}
Door met verschillende processoren te werken hoopt men de rekentijd drastisch naar beneden te kunnen halen. Dit is echter niet de enige motivatie: doorgaans verbruiken zogenaamde multi-core processoren - processoren die andere hardware zoals geheugens delen - ook minder energie. Men kan dus ook beogen de aanwezige hardware effici\"enter te gebruiken. Sommige problemen kunnen bovendien enkel opgelost worden met parallelle algoritmen: weersimulaties gebruiken bijvoorbeeld grote hoeveelheden geheugen, meestal kan men niet al de data in het geheugen van \'e\'en machine opslaan. Door elke machine een deel van de data te laten opslaan en manipuleren kan dit probleem worden opgelost. Tot slot worden soms parallelle algoritmes ingeschakeld om problemen op te lossen die berusten op onafhankelijke entiteiten. Indien verschillende bedrijven bijvoorbeeld een globale planning willen opmaken, maar geen details over de interne werking openbaar wensen te maken, kunnen parallelle algoritmes een oplossing bieden.

\subsubsection{Prestaties van parallelle algoritmen}

De prestaties van een parallel algoritme worden meestal gemeten op basis van \emph{speed-up}: de mate waarin we virtueel de kloksnelheid kunnen opdrijven. We defini\"eren eerst enkele concepten waarna we kort een taxonomie hieromtrent toelichten.

\begin{definition}[Wall time, Speed-up, Effici\"entie]
We defini\"eren de \emph{wall time} $T$ als de tijd tussen de start van het algoritme en het moment waarop het algoritme stopt. De \emph{wall time} maakt abstractie van het aantal processoren. De \emph{speed-up} is de verhouding tussen de \emph{wall-time} bij \'e\'en processor $T_1$ en de \emph{wall-time} bij $p$ processoren $T_p$:
\begin{equation}
\funm{speed-up}{p}=\displaystyle\frac{T_1}{T_p}
\end{equation}
In het geval $\funm{speed-up}{p}\approx p$ spreken we over \emph{lineaire speed-up}. In het geval $\funm{speed-up}{p}>\funm{speed-up}{p+1}$ is er sprake van \emph{negatieve speed-up}. Wanneer we de \emph{speed-up} delen door het aantal processoren $p$ spreken we over de \emph{effici\"entie}.
\begin{equation}
\funm{effici\"entie}{p}=\funm{speed-up}{p}/p
\end{equation}
\end{definition}

\paragraph{}
Indien men een voor een ineffici\"ent algoritme een parallel equivalent schrijft, stelt men vaak een lineaire speed-up vast. Dit komt omdat een na\"ief algoritme meestal geen relaties tussen de verschillende componenten in de invoer zal uitbuiten. De lineaire speed-up is bijgevolg eerder optimistisch. Een sterkere metriek is daarom de \emph{strong speed-up}. In dit geval wordt de snelste parallelle implementatie tegenover de sterkste sequenti\"ele implementatie geplaatst. Omdat algoritmes in een parallelle context meestal maar een beperkt deel van de data ter beschikking hebben, wordt het ontwikkelen van lineaire speed-up bijgevolg complexer en in sommige gevallen zelfs onmogelijk.

\paragraph{}
In eerder uitzonderlijke omstandigheden stelt men effectief superlineaire versnellingen vast. Dit is te wijten aan het effici\"enter gebruik van de aanwezig hardware. Cache is hierbij een uitstekend voorbeeld: bij multi-core processoren hebben de verschillende ``\emph{kernen}'' ofwel ``\emph{cores}'' meestal een eigen segment aan cache. Sommige gegevens worden gebruikt door alle processoren, bijgevolg wordt deze data in de gemeenschappelijke cache opgeslagen. Een gevolg is dat er meer ruimte beschikbaar is in de aparte caches voor specifieke data. Men noemt het effect van superlineaire versnellingen dan ook meestal het ``\emph{cache-effect}''\cite{cacheEffect}.

\subsubsection{Beperkingen en oplossingen}

Het verhogen van de snelheid met behulp van parallelle algoritmen is meestal een feit. Het optekenen van lineaire speed-up is echter geen evidentie. Dit komt onder meer door de ``\emph{Wet van Amadahl}''\cite{Amdahl:1967:VSP:1465482.1465560}. De meeste algoritmes hebben een zekere nood aan het sequentieel uitvoeren van bepaalde instructies. De Wet van Amadahl stelt dat gegeven de fractie van het aantal verplicht sequenti\"ele instructies, er een plafond bestaat en dat men met een willekeurig aantal processoren altijd onder dit plafond zal blijven. De ``\emph{Wet van Gustafson}''\cite{Gustafson:1988:RAL:42411.42415} stelt dan weer dat wanneer we de probleemgrootte arbitrair kunnen opdrijven, programma's wel een lineaire speed-up kunnen halen. Dit komt omdat de fractie aan de nodige sequenti\"ele instructies verwaarloosbaar klein wordt. Vermits we vooral ge\"interesseerd in het parallelliseren van grote optimalisatieproblemen, behoort linaire speedup misschien alsnog tot de mogelijkheden. Een belangrijke opmerking is dat beide theorie\"en geen rekening houden met de eventuele overhead die bij parallelle algoritmen komt kijken zoals bijvoorbeeld het doorsturen van data over een netwerk.

\subsection{De belangrijkste parallellisatie paradigma}

Meestal beschouwd men twee verschillende modellen met betrekking tot parallellisatie: \emph{Message Passing Interface (MPI)} en \emph{Tuple Space}. We geven hieronder een korte samenvatting.

\subsubsection{Message Passing Interface}

In het geval van \emph{MPI} beschouwen we een set \emph{agents}: processoren die onafhankelijk werken op een eigen stuk geheugen. Deze agenten interagerend door het uitwisselen van berichten over een netwerk. Dit hoeft niet te betekenen dat de agenten ook effectief draaien op verschillende machines: men kan op elke \emph{core} van een machine een onafhankelijke agent laten werken en vervolgens de boodschappen laten bezorgen met behulp van de faciliteiten die de meeste besturingssystemen aanbieden.

\paragraph{}
In het geval de \emph{agents} wel degelijk op verschillende machines draaien werken ze meestal volgens een bepaalde \emph{topologie}: een structuur die bepaalt welke machines met elkaar verbonden zijn. Een belangrijk aspect in \emph{MPI} is ``\emph{Group Communication}'': agenten die met een significant gedeelte van de andere agenten tegelijk communiceren. Voor elke topologie zijn er dan ook algoritmes uitgewerkt om het aantal berichten tot een minimum te beperken.

\subsubsection{Tuple Space}

Een \emph{Tuple Space} is een model waarbij agenten kennis hebben van een globale wereld: de \emph{Tuple Space}. Elke agent kan tuples toevoegen in de tuple space en een query op de ruimte uitvoeren. Door tuples toe te voegen die dan door andere agenten worden verwerkt ontstaat er een communicatieprincipe. Een \emph{tuple space} verschilt echter van een \emph{message passing interface} omdat men meer abstractie maakt van de bestemming van een boodschap.

\section{Optimalisatieproblemen}

We beginnen deze sectie met een formele definitie van een optimalisatieprobleem:

\begin{definition}[Optimalisatieprobleem]%, harde beperkingen, evaluatiefunctie, fitness-waarde
Een optimalisatieprobleem $\OpProblem$ is een tuple $\OpProblem=\tupl{\ConfigSet=\VarDom_1\times \VarDom_2\times\ldots\times \VarDom_{\nvar},\hcfun,\evalfun}$ waarbij $\ConfigSet$ een verzameling is van een set configuraties voor $\nvar$ variabelen, $\funsig{\hcfun}{\ConfigSet}{\BBB}$ een afbeelding van zo'n configuratie naar een Booleaanse waarde, die bepaald of de configuratie voldoet aan de ``harde beperkingen''. $\funsig{\evalfun}{\ConfigSet}{\RRR}$ stelt een evaluatiefunctie voor die bepaald in welke mate een configuratie wenselijk is. De waarde van de evaluatie van een configuratie \fun{f}{x} wordt ook wel de fitness-waarde genoemd.
\end{definition}
Bij een optimalisatieprobleem gaan we op zoek naar een configuratie $x\in\ConfigSet$ die aan de harde beperkingen voldoet en de evaluatiefunctie optimaliseert. Meestal maakt men het onderscheid tussen een minimalisatie en een maximalisatie. In deze thesis zullen we altijd we een bij een optimalisatieprobleem altijd streven naar een configuratie $x\in\ConfigSet$ met een zo laag mogelijk evaluatie \fun{f}{x}. We kunnen echter eenvoudig elk maximalisatieprobleem $\tupl{\ConfigSet,\hcfun,\evalfun}$ omzetten in een minimalisatieprobleem $\tupl{\ConfigSet,\hcfun,\evalfuna}$ met $\funsigimp{\evalfuna}{\ConfigSet}{\RealSet}{x}{-\fun{\evalfun}{x}}$. Formeel zoeken we dus naar een configuratie \xstar{} die we het \emph{globaal optimum noemen}.

\begin{definition}[Globaal optimum $\bestSol$]
Een globaal optimum voor een zoekprobleem $\OpProblem=\tupl{\ConfigSet,\hcfun,\evalfun}$ is een configuratie $\bestSol$ waarbij:
\begin{equation}
\bestSol=\displaystyle\argmin_{\sol\in\ConfigValSet}\fun{\evalfun}{\sol}\mbox{ met }\ConfigValSet=\accl{\sol|\forall\sol\in\ConfigSet:\fun{\hcfun}{\sol}=\true}
\end{equation}
\end{definition}

Het is niet ongewoon dat er verschillende configuraties zijn met een gelijkaardige fitness-waarde. Dit geldt tevens voor het globaal optimum. Daarom defini\"eren we ook een optimum-set $\calXop$: een set met alle configuraties met een minimale fitness-waarde voor het probleem.

\begin{definition}[Optimum-set $\ConfigOpSet$]
Een optimum-set $\ConfigOpSet$ voor een zoekprobleem $\OpProblem=\tupl{\ConfigSet,\hcfun,\evalfun}$ is een set van geldige configuraties $\sol\in\ConfigValSet$ waarvoor geldt:
\begin{equation}
\calXop=\accl{\sol|\sol\in\ConfigValSet\wedge\fun{\evalfun}{\sol}=\fun{\evalfun}{\bestSol}}
\end{equation}
\end{definition}

\paragraph{}
In een algemeen geval kunnen de domeinen $A_i$ van de variabelen $x_i$ oneindig groot zijn en bijvoorbeeld $\RRR$ omvatten. Geen enkele machine met een eindig geheugen kan echter alle elementen uit een domein met oneindig veel elementen voorstellen. We zullen daarom altijd de domeinen $A_i$ als eindig beschouwen. In het geval het domein van een variabele in werkelijkheid oneindig is, discretiseren we dus dit domein en beperken we het aantal elementen met een onder- en bovengrens. Indien er door discretisatie fouten worden ge\"introduceerd, kunnen we deze oplossen door het domein fijner te discretiseren.
\paragraph{}
Vermits zowel de harde beperkingen $\hcfun$ als de evaluatiefunctie $\evalfun$ hier een ``\emph{blackbox}'' zijn, zullen we om $\bestSol$ te berekenen, over een significant deel van de verzameling $\ConfigSet$ moeten itereren. We verwachten dus dat de tijdscomplexiteit om een dergelijke oplossing te vinden gelijk is aan:
\begin{equation}
\bigoh{\abs{\ConfigSet}}=\bigoh{\displaystyle\prod_{i=1}^\nvar\abs{\VarDom_i}}
\end{equation}
Indien we de assumptie maken dat alle domeinen dezelfde zijn dan bekomen we:
\begin{equation}
\bigoh{\abs{\ConfigSet}}=\bigoh{\abs{\VarDom_1}^\nvar}\mbox{ indien }\forall \VarDom_i,\VarDom_j: \VarDom_i=\VarDom_j
\end{equation}
We zien dus dat deze tijdscomplexiteit exponentieel stijgt met het aantal variabelen~$\nvar$. Optimalisatieproblemen in het algemeen liggen dan ook in \comp{NP-hard}.

\subsection{Complexiteit van optimalisatieproblemen}

Sommige optimalisatieproblemen liggen in \comp{P}. \algo{Karmarkar's algoritme}\cite{linearProgrammingInP} bijvoorbeeld lost het \prbm{lineaire optimalisatie} probleem op in \bigoh{\nvar^{3.5}L} met $\nvar$ het aantal variabelen en $L$ de diepte van de discretisatie in bits. Dit komt omdat we beperkingen plaatsen op de vorm van de evaluatiefunctie $\evalfun$ en de harde beperkingen $\hcfun$. Bij lineair programmeren betekent dit dat de evaluatiefunctie kan geschreven worden als het inwendig product tussen de vector van de variabelen en een vector met constanten. Het harde beperkingen moeten voor te stellen zijn zodat wanneer we de vector met de variabelen vermenigvuldigen met een matrix met constante elementen, alle elementen in de resulterende vector kleiner zijn dan een andere vector met constante elementen. Ook andere optimalisatieproblemen zoals bijvoorbeeld \prbm{Maximum Flow} en \prbm{Minimum Spanning Tree} zijn problemen die met polynomiale algoritmen kunnen worden opgelost.

\paragraph{}
Toch is er weinig ruimte voor optimisme. Een logische veralgemening van \prbm{Lineaire optimalisatie} is immers \prbm{Kwadratische optimalisatie}. Onder sommige omstandigheden kunnen we dit probleem reduceren naar een geval van \prbm{Lineaire Optimalisatie}\cite{Kozlov1980223}, maar een algemeen \prbm{Kwadratisch optimalisatie} probleem ligt in \comp{NP-hard}\cite{qpInNP}. Ook andere bekende optimalisatieproblemen zoals \prbm{Travelling Salesman Problem (TSP)} en \prbm{Integer Programming (IP)} liggen in \comp{NP-hard}.

\paragraph{}
Tot slot dient men in de context van optimalisatieproblemen een kanttekening maken dat een polynomiaal algoritme meestal niet meteen impliceert dat dit ook op kleine gevallen sneller werkt dan zijn exponenti\"ele tegenhangers. Een populaire methode bij het oplossen van \prbm{Lineaire optimalisatie} is bijvoorbeeld het \algo{Simplex}-algoritme. Klee en Minty\cite{klee:1972} construeerden echter een een geval waarbij het algoritme exponentieel veel tijd vraagt. Toch is \algo{Simplex} in de meeste gevallen sneller dan \algo{Karmarkar's algoritme}.

\section{Heuristieken}

Hoewel de meeste optimalisatieproblemen \comp{NP-hard} zijn, is in een praktische context de configuratie met een optimale evaluatiefunctie net van cruciaal belang. Voor de meeste toepassingen is een configuratie die aan de harde beperkingen voldoet en de fitness-waarde van de echte oplossing benadert voldoende. In dat geval wordt meestal een heuristiek ge\"implementeerd:

\begin{definition}[Heuristiek]
Een heuristiek is een programma die gegeven een optimalisatieprobleem $\OpProblem=\tupl{\ConfigSet,\hcfun,\evalfun}$ een oplossing berekent $\goodSol$ in een redelijke tijd. Doorgaans voldoet deze oplossing aan de harde beperkingen ($\fun{\hcfun}{\goodSol}=\true$) en ligt de voorgestelde oplossing $\goodSol$ in fitness-waarde niet ver van de werkelijke oplossing $\bestSol$.
\end{definition}

Deze definitie blijft redelijk vaag en geeft dan ook veel ruimte voor interpretatie. Doorgaans verwachten we dat het algoritme stop in polynomiale tijd en in de meeste gevallen worden er ook beperkingen gezet op hoe ver de fitness-waarde $\fun{\evalfun}{\goodSol}$ mag afwijken van de optimale fitness-waarde $\fun{\evalfun}{\bestSol}$, al zijn beide voorwaarden niet strikt noodzakelijk.

%Minsky\cite{minskyHeuristic} schrijf hierover:
% \begin{quote}
% ``Hints'', ``suggestions'', or ``rules of thumb'', which only usually work are called heuristics. A program which works on such a basis is called a heuristic program. It is difficult to give a more precise definition of heuristic program - this is to be expected in the light of Turing's demonstration that there is no systematic procedure which can distinguish between algorithms (programs that always work) and programs that do not always work.
% \end{quote}

% Een belangrijke theorema stelt dat fout onmogelijk constant kan zijn:
% \begin{theorem}
% Voor geen enkele heuristiek bestaat geen getal $E$ waarvoor geldt:
% \begin{quote}
% dat voor elke probleem-instantie van het probleem, $\fun{f}{\xdot}-\fun{f}{L}\leq L$.
% \end{quote}
% \end{theorem}


\section{Metaheuristieken}

\subsection{Problemen met heuristieken}

Heuristieken bieden een antwoord door een algoritme uit te voeren die in polynomiale tijd een oplossing zal uitrekenen. In het geval het algoritme snel genoeg is, en we kunnen leven met de garanties die dit algoritme biedt, is dit een acceptabele methode. In de meeste gevallen is dit echter niet zo. Praktische problemen zijn groot en complex en doorgaans kan een heuristiek weinig garanties bieden. Een ander probleem met heuristieken is dat ze weinig aanpasbaar zijn: stel dat we een bepaalde tijd ter beschikking stellen, dan zal het algoritme zich doorgaans niet aan deze beperking houden: ofwel loopt het algoritme eerder af en maakt het dus geen gebruik van alle beschikbaargestelde rekentijd, indien het algoritme niet afloopt binnen de gespecificeerde tijd is er ook geen sprake van een parti\"ele oplossing.

\subsection{Formele definitie}

Metaheuristieken proberen deze problemen op te lossen. We geven eerst een formele definitie van een metaheuristiek waarna we relevante terminologie invoeren.

\begin{definition}[Metaheuristiek]
\deflab{metaheuristic}
Een metaheuristiek is een algoritme die een oplossingsruimte $\SolSet\subseteq\accl{\sol|\forall \sol\in \ConfigSet:\fun{\hcfun}{\sol}}\subseteq\ConfigSet$ beschouwd met $\bestSol\in\SolSet$. Verder beschouwd het \'e\'en of meer overgangsfuncties $h_i:\SolSet^{k_i}\times\RealSet^{l_i}\rightarrow\SolSet$. De metaheuristiek werkt door \'e\'en of meerdere instanties $s_1,s_2,\ldots,s_j$ uit $\SolSet$ te genereren. En vervolgens herhaaldelijk deze overgangsfuncties toe te passen op deze instanties. De resultaten van deze functietoepassingen kun gebruikt worden in andere toepassingen van de overgangsfuncties. Het algoritme stopt wanneer aan een bepaalde stopconditie voldaan is (bijvoorbeeld: het algoritme draait een bepaalde tijd op de machine). Hierna wordt de oplossing met de beste fitness-waarde als uitvoer teruggegeven.
\end{definition}

Op basis van deze definitie kunnen we ook een algoritme op hoog niveau opstellen zoals beschreven in \algref{metaheuristicGeneral}.

\begin{algorithm}[H]
 \SetAlgoLined
 Bereken een initi\"ele set van oplossingen $\PopSet_1$ en hun fitness-waarde\;
 $b_1\leftarrow\displaystyle\argmin_{\sol\in\PopSet_1}{\fun{\evalfun}{\sol}}$\;
 \Repeat{het stopcriterium is bereikt}{
  Genereer op basis van $\PopSet_t$ en $t$ stochastisch een nieuwe set oplossingen $\PopSet_{t+1}$ samen met hun fitness-waarde\;
  $b_{t+1}\leftarrow\displaystyle\argmin_{\sol\in\PopSet_{t+1}\cup\accl{b_t}}{\fun{\evalfun}{\sol}}$\;
  $t\leftarrow t+1$\;
 }
 \KwRet{$b_t$}
 \caption{Hoog niveau beschrijving van een metaheuristiek\cite{DBLP:journals/jc/ShonkwilerV94}.}
 \alglab{metaheuristicGeneral}
\end{algorithm}

Typisch aan metaheuristieken is een (sterke) aanwezigheid van toevalsgetallen. Zo hebben we bij de signatuur van de overgangsfuncties ook een set re\"ele getallen opgenomen. Deze getallen worden door het toeval gegenereerd en bepalen mee het resultaat van de functie. Een belangrijk concept die we hiermee kunnen introduceren is de omgeving ofwel ``\emph{neighborhood}'' van een overgangsfunctie:

\begin{definition}[Omgeving van een overgangsfunctie]
De omgeving van een overgangsfunctie is de verzameling van alle oplossingen die we kunnen genereren vanuit een gegeven set van oplossingen ongeacht de waarde van de toevalsfactoren. De omgeving van een functie $h_i$ is dus:
\begin{equation}
\fun{\neighbr_{h_i}}{s_1,s_2,\ldots,s_{k_i}}=\accl{s|s=\fun{h_i}{s_1,s_2,\ldots,s_{k_i},\xi_1,\xi_2,\ldots,\xi_{l_i}}}
\end{equation}
\end{definition}

\subsubsection{Local Search}
Het concept van een omgeving is belangrijke omdat het meteen ook een populaire zoekstrategie voor metaheuristieken introduceert: \emph{lokaal zoeken} ofwel ``\emph{local search (LS)}''. Deze zoekstrategie vertrekt van een gegeven oplossing en zoekt - volgens een bepaalde omgeving - alle oplossingen in de buurt af op zoek naar een betere oplossing. In het geval we zo'n oplossing vinden wordt deze oplossing de nieuwe oplossing en beginnen we vervolgens een zoektocht rond de nieuwe oplossing. Local Search komt voor in twee smaken: ``\emph{first-improvement}'' en ``\emph{best-improvement}''. In het geval van \emph{first-improvement} wordt de eerste oplossing die beter is dan de huidige oplossing de nieuwe actieve oplossing. In het geval van \emph{best-improvement} doorzoeken we de volledige omgeving en migreren we naar de beste oplossing in de omgeving.
\paragraph{}
Local Search is een krachtige optimalisatietechniek die doorgaans tot acceptabele resultaten kan leiden. Het probleem zit hem in het idempotente karakter van local search: \'e\'enmaal we local search op een oplossing hebben toegepast heeft een tweede maal local search toepassen met dezelfde omgevings-definitie geen zin: we weten immers dat er in de omgeving geen betere oplossingen gevonden zullen worden want anders was het algoritme de vorige keer niet bij deze oplossing gestopt. Omwille van deze reden kan local search zelf geen volwaardige metaheuristiek worden genoemd. We verwachten immers dat als we meer tijd investeren, we op termijn altijd tot betere resultaten $\xdot$ ofwel de echte oplossing $\xstar$ zullen komen. Local search vormt echter de basis voor een groot aantal metaheuristieken die doorgaans een mechanisme implementeren om de oplossing uit het lokaal optimum te laten migreren.

\subsection{De ``Zoo van de Metaheuristieken''}

\defref{metaheuristic} is vrij abstract. Daarom zullen we in deze sectie de belangrijkste families van metaheuristieken kort beschouwen.

\subsubsection{Genetische Algoritmes}

\emph{Genetische algoritmen} ofwel ``\emph{Genetic Algorithms (GA)}'' zijn de eerste familie van metaheuristieken en werken op basis van een populatie: een set van oplossingen. Uit deze populatie worden twee of meer oplossingen geselecteerd. De selectie is meestal gerelateerd aan de fitness-waarde van deze oplossingen. Vervolgens past men een recombinatie-operator toe: een overgangsfunctie die twee of meer oplossingen als invoer neemt en op basis hiervan een nieuwe oplossing genereert die eigenschappen van alle ouders deelt. Doorgaans past men op deze oplossing ook een mutatie toe: men zal de oplossing lichtjes aanpassen door middel van een andere overgangsfunctie. Vervolgens de oplossing onder bepaalde voorwaarden in de populatie ge\"injecteerd. In ruil voor de nieuwe oplossing zal de populatie ook \'e\'en of meer oplossingen uit de populatie verwijderen.

\subsubsection{Simulated Annealing}

\emph{Simulated Annealing (SA)} is de eerste gepubliceerde metaheuristiek. Het is een schema gebaseerd op het \algo{Algoritme van Metropolis}. Het werkt aan de hand van \'e\'en overgangsfunctie en \'e\'en oplossing die soms de actieve oplossing wordt genoemd. Het algoritme voert telkens de overgangsfunctie uit op de actieve oplossing. De resulterende oplossing beschouwen we als de nieuwe oplossing met een kans:
\begin{equation}
\fun{p}{\mbox{accept $s_1\rightarrow s_2$}}=\fun{\min}{1,e^{\brak{\fun{\evalfun}{s_2}-\fun{\evalfun}{s_1}}/T}}
\end{equation}

Deze formule stelt dat indien de nieuwe oplossing beter is dan de oude oplossing, we altijd accepteren. Indien de nieuwe oplossing minder gunstig is accepteren we met een bepaalde kans die kleiner wordt naarmate het verschil in fitness-waarde groeit. In de formule staat ook een onbekende parameter $T$ ofwel \emph{temperatuur}. De temperatuur bepaalt hoe sterk de kans daalt naarmate de kloof groeit. Het is een variabele die initieel op een positief getal wordt gezet en gedurende het zoekproces langzaam naar 0 zakt. Dit betekent dat we in het begin sterk geneigd zijn om een slechtere oplossing te accepteren. Op het einde accepteren we bijna uitsluitend oplossing die beter zijn dan hun ouder. Hoe de temperatuur concreet evolueert kan vrij ingesteld worden en introduceert dan ook een groot aantal varianten.

\subsubsection{Iterated Local Search}

In de vorige sectie bespraken we \emph{Local Search}. Het probleem met \emph{Local Search} is echter dat het convergeert naar een lokaal minimum en vanaf dat moment geen vooruitgang meer kan maken. \emph{Iterated Local Search (ILS)} is een zoekproces dat afwisselt tussen twee fases: in de \emph{local search}-fase optimaliseert het programma de oplossing tot het in een lokaal optimum terecht komt; in de \emph{perturbatie}-fase voert men een overgangsfunctie uit die met een zekere kans in staat moet zijn om een oplossing te genereren die uit het lokale optimum ontsnapt.

\subsubsection{Variable Neighbourhood Search}

\emph{Variable Neighbourhood Search (VNS)} is een metaheuristiek die verder bouwt op \emph{Iterated Local Search}. In het geval van \emph{Variable Neighbourhood Search} is er echter sprake van verschillende definities voor een omgeving $\neighbr_i$ in een iteratie zullen in de \emph{shake}-fase migreren naar een random oplossing in de omgeving $\neighbr_i$ en vervolgens passen we \emph{local search} toe op basis van deze omgeving. In het geval we tot een betere oplossing komen, wordt de eerste omgeving $\neighbr_1$ opnieuw de actieve omgeving. In het andere geval kiezen we de volgende omgeving $\neighbr_{i+1}$. Op het moment dat alle omgevingen zijn uitgeput beginnen we ook opnieuw bij $\neighbr_1$.%in het geval we een lokaal optimum volgens omgeving $\calN_i$ bereiken, zullen we \emph{local search} toepassen op basis van de volgende omgeving $\calN_i$ in de hoop dat deze omgeving ons naar een betere oplossing zal leiden. In het geval alle

\subsubsection{Tabu Search}
In het geval van \emph{Tabu Search} is de omgeving de unie van een reeks deelomgevingen. Onder bepaalde voorwaarden kan men beslissen een deel van de omgeving of oplossingsruimte tijdelijk ontoegankelijk te verklaren. De acties of locaties vallen op dat moment onder ``\emph{tabu}''. Dit kan bijvoorbeeld nuttig zijn om eerdere acties niet plots ongedaan te maken. Acties of oplossingsruimtes vallen maar tijdelijk onder tabu: na een aantal iteraties worden de elementen weer toegankelijk.

\subsection{Classificatie in de ``Zoo van de Metaheuristieken''}

\subsection{Omzetten van beperkingen naar evaluaties}

Soms is $c$ een complexe functie met een weinig voorspelbaar gedrag. Dit heeft een implicatie op de transitiefuncties die doorgaans veel rekenkracht nodig hebben om een oplossing te genereren die aan de harde beperkingen voldoet.

\paragraph{}
Een oplossing voor dit probleem is het omzetten van een deel de harde beperkingen $c$ in de evaluatiefunctie $f$. Oplossingen die niet aan een deel van de harde beperkingen voldoen, zullen we toch als geldige oplossingen aanzien, maar met een hoge fitness-waarde. De procedure die berekend welke oplossing wordt ook aangepast zodat enkel een geldige oplossing kan worden teruggegeven. Een algoritme die een optimalisatieprobleem $\tupl{X,c_1\wedge c_2,f}$ oplost, kan dus inwendig een optimalisatieprobleem $\tupl{X,c_1,f-K\cdot\krdelta{c_2}}$ oplossen met $K$ een grote constante.

\subsection{Modelleren van Metaheuristieken}

Elke metaheuristiek die we hierboven beschreven hebben dient telkens drie belangrijke vragen in het achterhoofd te houden\cite{DBLP:journals/jc/ShonkwilerV94}:
\begin{enumerate}
 \item kan het globale optimum altijd gevonden worden door de metaheuristiek,
 \item hoe kunnen we identificeren dat we het globale optimum gevonden hebben, en
 \item hoe lang zal het duren alvorens we dit optimum gevonden hebben
\end{enumerate}

Wanneer we geen details kennen in verband met de vorm van de evaluatiefunctie $\evalfun$ spreekt het voor zich dat we enkel zeker kunnen zijn dat we een globaal optimum hebben gevonden door middel van \emph{exhaustive search}. Ook wat betreft de derde vraag verwachten we een \emph{exhaustive search} proces alvorens we zeker zijn dat we het optimum hebben gevonden. De verwachtte tijd alvorens dit gebeurt kan echter sterk verschillen van de tijd die we dienen te besteden aan het doorzoeken van een significant deel van de zoekruimte. Berekenen wanneer we gemiddeld dit optimum zullen bereiken staat bekend als het \prbm{Hitting Time Problem}. Om dit probleem te formaliseren dienen we eerst de notie van de raaktijd van een Markov-keten te defini\"eren:

\begin{definition}[Raaktijd $\hittime$]
We defini\"eren de raaktijd $\hittime$ van een Markov-keten $\PopChain$ die bestaat uit populaties $\PopSet_t$ als:
\begin{equation}
\fun{\hittime}{\PopChain}=\min\accl{t|\PopSet_t\in\PopChain\wedge\ConfigOpSet\cap \PopSet_t\neq\emptyset}
\end{equation}
We kunnen deze definitie ook uitbreiden naar de raaktijd voor een zekere fitness-waarde $v$:
\begin{equation}
\fun{\hittime}{\PopChain,v}=\min\accl{t|\PopSet_t\in\PopChain:\exists \sol\in \PopSet_t:\fun{\evalfun}{\sol}\leq v}
\end{equation}
\end{definition}

Het \prbm{Hitting Time Problem} gaat vervolgens op zoek naar de raaktijd van een gemiddelde Markov-keten.

\subsubsection{Verwachtte tijd van een sequentieel proces}

Op basis van \algref{metaheuristicGeneral} kunnen we uitspraken doen, zowel over de verwachtte raaktijd wanneer we het zoeken uitbesteden aan \'e\'en processor of uitbesteden aan verschillende processoren die elk onafhankelijk een eigen zoekproces laten lopen. Onderzoek naar de raaktijd is gepubliceerd in \cite{DBLP:journals/jc/ShonkwilerV94}. Hiervoor zal men een Markov-model beschouwen. De knopen van dit Markov-model stellen populaties voor: een verzameling van oplossingen. We kunnen hier dus denken aan een verzameling $\PopSet_i$ in \algref{metaheuristicGeneral}. Als beperking stellen we dat de populatie een vaste grootte heeft. Deze beperking is redelijk vermits elke praktische machine een eindig geheugen heeft en dus ook maar een eindig aantal oplossingen tegelijk kan beschouwen. Verder bevat ons Markov-model ook overgangskansen $\fun{p}{t,i,j}$: de kans om van een populatie $\PopSet_i$ naar een populatie $\PopSet_j$ te gaan op tijdstip $t$. Indien de waarde van $\fun{p}{i,j,t}$ constant blijft over de tijd $t$ dan spreken we over een \emph{stationair} proces (dit is typisch voor bijvoorbeeld een genetisch algoritme), indien de kansen in de tijd veranderen beschouwen we een \emph{niet-stationair} proces.

\paragraph{}

Indien een populatie een oplossing bevat met een fitness-waarde die kleiner of gelijk is aan de waarde die we zoeken $v$, spreken we over een \emph{doel-populatie}, in het andere geval zullen we spreken over een \emph{normale populatie}. We kunnen zonder verlies van algemeenheid de populaties zo ordenen dat de populaties $\PopSet_1,\PopSet_2,\ldots,\PopSet_g$ doel-populaties zijn, en de populaties $\PopSet_{g+1},\PopSet_{g+2},\ldots,\PopSet_N$ populaties waar er geen gewenste oplossing in zit. In dat geval kunnen we de transitiematrix $P$ opdelen in submatrices:
\begin{equation}
P=\brak{\begin{array}{cc}
J&H\\
B&\hat{P}
\end{array}}
\end{equation}
Hierbij stelt $J$ een $g\times g$ voor, $H$ een $g\times N-g$, $B$ een $N-g\times g$ matrix en $\hat{P}$ logischerwijs een $N-g\times N-g$ matrix. Het spreekt voor zich dat de waarde van $J$ en $H$ niet zo interessant zijn: eenmaal we ons in een doel-populatie bevinden kunnen we immers stoppen met het algoritme uit te voeren. Daarom zullen we stellen dat $J=\identitymatrix[g\times g]$ en $H=\zeromatrix[g\times N-g]$. We streven er verder naar om met ons algoritme de waardes in $B$ -- ook wel de \emph{bridge} genoemd -- zo hoog mogelijk te maken: deze matrix bevat immers de kansen om van naar een doel-populatie te migreren. $\hat{P}$ wordt ook wel de \emph{verwijderde transitiematrix} genoemd. De waardes in de matrix worden be\"invloed door het probleem (via de definities van de omgevingen) en het zoekalgoritme (de metaheuristiek in kwestie).
\paragraph{}
De kwaliteit van een zoekmethode na $t$ stappen kunnen we uitdrukken met een probabiliteitsvector $\vec{\alpha}_t$. $\alpha_{t,i}$ geeft aan met hoeveel kans het algoritme in tijdstap $t$ de populatie $\PopSet_i$ beschouwt. Net als bij de matrix delen we de probabiliteitsvector op in twee delen: een gedeelte $\vec{\alpha}^G_t$ met dimensie $g$ en een gedeelte $\vec{\hat{\alpha}}_t$. Op basis van de transitiematrix kunnen we de probabiliteitsvector na $t$ stappen berekenen in functie van $\alpha_0$:
\begin{equation}
\vec{\alpha}_t=\brak{\vec{\alpha}_t^G\ |\ \vec{\hat{\alpha}}_t}=\vec{\alpha}_{t-1}\cdot P=\brak{\vec{\alpha}_{t-1}^G+B\cdot\vec{\hat{\alpha}}_{t-1}\ |\ \vec{\hat{\alpha}}_{t-1}\cdot\hat{P}}
\eqnlab{oneStepTransition}
\end{equation}
We zijn vooral ge\"interesseerd in de fractie van Markov-ketens die in stap $t$ naar een doel-populatie migreren ofwel de kans dat we stap $t$ een gewenste oplossing bereiken:
\begin{equation}
\Prob{\theta=t}=\displaystyle\sum_{i=g+1}^{N}\brak{\alpha_{t-1,i}\cdot\displaystyle\sum_{j=1}^{g}\fun{p}{t,i,j}}=\vec{\hat{\alpha}}_{t-1}\cdot B\cdot\onematrix[1\times N-g]
\eqnlab{exhaust}
\end{equation}
Een ander logische gevolg uit \eqnref{oneStepTransition} zijn de transities van normale populaties naar andere normale populaties. De som van de kansen die in deze deelvector aanwezig zijn, is de kans dat we na deze $t$ stappen nog steeds geen acceptabele oplossing gevonden hebben:
\begin{eqnarray}
\vec{\hat{\alpha}}_t=\vec{\hat{\alpha}}_{t-1}\cdot\hat{P}=\vec{\hat{\alpha}}_{t-2}\cdot\hat{P}^2=\ldots=\vec{\hat{\alpha}}_{0}\cdot\hat{P}^t\eqnlab{remainder}\\
\Prob{\theta>t}=\displaystyle\sum_{i=g+1}^{N}\alpha_{t,i}=\vec{\hat{\alpha}}_t\cdot\onematrix[1\times N-g]=\vec{\hat{\alpha}}_{0}\cdot\hat{P}^t\cdot\onematrix[1\times N-g]\eqnlab{nothit}
\end{eqnarray}
Op basis van \eqnref{remainder} en \eqnref{nothit} kunnen we nu de verwachte stap berekenen waarop we een gewenste oplossing zullen berekenen:
\begin{equation}
\mean{\theta}=1+\displaystyle\sum_{i=0}^{\infty}{\Prob{\theta>i}}=1+\displaystyle\sum_{i=0}^{\infty}{\vec{\hat{\alpha}}_{0}\cdot\hat{P}^i\cdot\onematrix[1\times N-g]}
\eqnlab{meanHit}
\end{equation}
In deze vergelijking tellen we elke kans $\Prob{\theta=k}$ juist $k$ keer.
\paragraph{}
Vermits $\hat{P}$ een matrix met kansen voorstelt is elke element $\hat{p}_{i,j}\geq 0$. We maken verder de assumptie dat we na een arbitrair aantal stappen $m$, vanuit elke normale populatie $\calP_i$ een kans bestaat dat we naar een andere normale populatie $\calP_j$ kunnen migreren. Vermits hierdoor de matrix primitief is, en er minstens \'e\'en rij van $\hat{P}$ niet sommeert naar 1 (anders zouden we nooit in een doel-populatie kunnen terechtkomen), weten we dat de grootste eigenwaarde $\ev[\hat{P}]<1$.

% Een theorema stelt:
% \begin{theorem}
% stel de twee grootste eigenwaardes $\lambda_1>\abs{\lambda_2}$ van een $n\times n$ matrix $A$. Dan bestaat er een vaste veelterm $h$ met een graad $0\leq\funm{deg}{h}< n-1$ zodat:
% \begin{equation}
% \forall k=1,2,\ldots:\abs{\displaystyle\frac{\vec{x}\cdot A\cdot\onematrix[1\times n]}{\lambda_1^k}-\vec{x}\cdot\vec{v}}<\fun{h}{k}\cdot\abs{\displaystyle\frac{\lambda_2}{\lambda_1}}^k
% \end{equation}
% Met $\vec{x}$ een willekeurige vector en $\vec{v}$ de rechtse eigenvector van $\lambda_1$.
% \thelab{eigenValuesMatrix}
% \end{theorem}
\paragraph{}
We kunnen een eigenvector benaderen door een $\onematrix$-vector een groot aantal maal toe te passen op de matrix in kwestie en vervolgens te normaliseren. Daarom kunnen we stellen dat voor een grote $t$:
\begin{equation}
\begin{array}{cc}
\vec{x}\cdot\hat{P}^t\cdot\onematrix[1\times N-g]\approx\vec{x}\cdot\ev[1,\hat{P}]^t\cdot\evr[\hat{P}]=\ev[1,\hat{P}]^t\cdot\sigma&\mbox{($t\gg 1$)}
\end{array}
\eqnlab{longTermAlpha}
\end{equation}
Wanneer we als $\vec{x}$ de begindistributie $\vec{\hat{\alpha}}_0$ nemen, kunnen we het dot-product tussen $\vec{\hat{\alpha}}_0$ en $\evr[\hat{P}]$ defini\"eren als $\sigma$. We kunnen vervolgens \eqnref{meanHit} en \eqnref{longTermAlpha} samennemen in een nieuwe vergelijking die
%ons met behulp van \theref{eigenValuesMatrix}
een benaderende waarde voor de gemiddelde raaktijd oplevert:
\begin{equation}
\mean{\theta}\approx 1+\vec{\hat{\alpha}}_0\cdot\onematrix+\sigma\cdot\brak{\displaystyle\sum_{i=1}^{\infty}\lambda_{1,\hat{P}}^i}=1+\vec{\hat{\alpha}}_0\cdot\onematrix+\displaystyle\frac{\sigma\cdot \lambda_{1,\hat{P}}}{1-\lambda_{1,\hat{P}}}
\end{equation}
Net zoals we de termen van dit gemiddelde benadert hebben aan de hand van \eqnref{longTermAlpha}, kunnen we de eerste termen benaderen door deze vergelijking omgekeerd toe te passen. We stellen dus dat $1+\vec{\hat{\alpha}}_0\cdot\onematrix=\sigma\cdot\brak{1+\lambda^{-1}}$. Dit levert ons een ruwe schatting voor de raaktijd op:
\begin{equation}
\mean{\hittime}\approx\displaystyle\frac{\sigma}{\brak{1-\lambda_{1,\hat{P}}}\cdot\lambda_{1,\hat{P}}}
\eqnlab{hitTimeSingle}
\end{equation}

\subsubsection{Onafhankelijke parallelle Monte-Carlo simulaties}

\cite{DBLP:journals/jc/ShonkwilerV94} toont ook een logische stap tot parallellisatie: $p$ processoren draaien elk onafhankelijk van elkaar een Monte-Carlo simulatie. We kunnen dit fenomeen modelleren met behulp van een Markov-keten waarbij elke knoop van de keten een $p$-tuple voorstelt met de populaties van de verschillende processoren in de tijdstap. Vervolgens defini\"eren we \emph{parallelle raaktijd} $\phittime$ als volgt:

\begin{definition}[Parallelle raaktijd $\phittime$]
We defini\"eren de parallelle raaktijd $\phittime$ van een Markov-keten $\PopChain$ die bestaat uit $p$-tuples van populaties $\tupl{\PopSet_{t,1},\PopSet_{t,2},\ldots,\PopSet_{t,p}}$ als:
\begin{equation}
\fun{\phittime}{\PopChain}=\min\accl{t|\tupl{\PopSet_{t,1},\PopSet_{t,2},\ldots,\PopSet_{t,p}}\in\PopChain\wedge\exists i:\ConfigOpSet\cap \PopSet_{t,i}\neq\emptyset}
\end{equation}
We kunnen deze definitie ook uitbreiden naar de raaktijd voor een zekere fitness-waarde $v$:
\begin{equation}
\fun{\phittime}{\PopChain,v}=\min\accl{t|\tupl{\PopSet_{t,1},\PopSet_{t,2},\ldots,\PopSet_{t,p}}\in\PopChain\wedge\exists i\exists \sol\in\PopSet_{t,i}:\fun{\evalfun}{\sol}\leq v}
\end{equation}
\end{definition}
Vermits de verschillende processoren een onafhankelijk en equivalent proces uitvoeren, kunnen we gebruik maken van het theorema van de onafhankelijke kansen met de onafhankelijke EN-regel:
\begin{equation}
\begin{aligned}
\Prob{\phittime\geq t}&=\Prob{\hittime_1\geq t\wedge\hittime_2\geq t\wedge\ldots\wedge\hittime_p\geq t}\\
&=\Prob{\hittime_1\geq t}\cdot\Prob{\hittime_2\geq t}\cdot\ldots\cdot\Prob{\hittime_p\geq t}\\
&=\Prob{\hittime\geq t}\cdot\Prob{\hittime\geq t}\cdot\ldots\cdot\Prob{\hittime\geq t}=\Prob{\hittime\geq t}^p
\end{aligned}
\eqnlab{probabilityPower}
\end{equation}
Op basis van deze regel kunnen we de gemiddelde parallelle raaktijd bepalen op basis van \eqnref{meanHit}:
\begin{equation}
\begin{aligned}
\mean{\phittime}&=1+\displaystyle\sum_{i=0}^{\infty}{\Prob{\phittime>i}}=1+\displaystyle\sum_{i=0}^{\infty}{\Prob{\hittime>i}^p}\\
%&=1+\displaystyle\sum_{i=0}^{\infty}{\vec{\hat{\alpha}}_{0}\cdot\hat{P}^i\cdot\onematrix[1\times N-g]}\\
&\approx 1+\brak{\vec{\hat{\alpha}}_0\cdot\onematrix}^p+\sigma^p\cdot\brak{\displaystyle\sum_{i=1}^{\infty}\lambda_{1,\hat{P}}^{p\cdot i}}=1+\brak{\vec{\hat{\alpha}}_0\cdot\onematrix}^p+\displaystyle\frac{\sigma^p\cdot\lambda_{1,\hat{P}}^p}{1-\lambda_{1,\hat{P}}^p}\\
&\approx\displaystyle\frac{\sigma^p}{\brak{1-\lambda_{1,\hat{P}}^p}\cdot\lambda_{1,\hat{P}}^{p}}
%\Prob{\phittime\geq t}&=\Prob{\hittime_1\geq t\wedge\hittime_2\geq t\wedge\ldots\wedge\hittime_p\geq t}\\
%&=\Prob{\hittime_1\geq t}\cdot\Prob{\hittime_2\geq t}\cdot\ldots\cdot\Prob{\hittime_p\geq t}\\
%&=\Prob{\hittime\geq t}\cdot\Prob{\hittime\geq t}\cdot\ldots\cdot\Prob{\hittime\geq t}=\Prob{\hittime\geq t}^p
\end{aligned}
\eqnlab{probabilityPower}
\end{equation}
De \emph{speed-up} is bijgevolg gelijk aan:
\begin{equation}
\mbox{speed-up}=\displaystyle\frac{\mean{\hittime}}{\mean{\phittime}}=\displaystyle\frac{\lambda_{1,\hat{P}}^{p-1}}{\sigma^{p-1}}\cdot\displaystyle\frac{1-\lambda_{1,\hat{P}}^{p}}{1-\lambda_{1,\hat{P}}}\approx\displaystyle\frac{p\cdot\lambda^{p-1}}{\sigma^{p-1}}
\eqnlab{speedupMetaheuristic}
\end{equation}
In het laatste deel van de vergelijking vervangen we ook $\ev[\hat{P}]$ door $\lambda$. Dit is louter om de formules die hiervan afgeleid zijn leesbaar te houden.

\subsubsection{Semantiek van parameters $\sigma$ en $\lambda$}

\eqnref{speedupMetaheuristic} formaliseert de verwachte \emph{speed-up} bij een probleem. Deze formule bevat 3 parameters en kan daarom niet zomaar ge\"interpreteerd worden. Behalve het aantal parameters $p$ is het niet triviaal om de overige parameters te interpreteren.

\paragraph{}
We hebben in \eqnref{longTermAlpha} een benadering geformuleerd voor de distributie over normale populaties. Elke initi\"ele distributie zal bijgevolg altijd convergeren naar een distributie proportioneel aan de sterkste linkse eigenvector $\evl[\hat{P}]$. Eenmaal $\vec{\hat{\alpha}}_t\approx\evl[\hat{P}]$ kunnen we stellen dat:
\begin{equation}
\conditional{\vec{\hat{\alpha}}_{t+1}=\vec{\hat{\alpha}}_t\cdot\hat{P}=\lambda\cdot\vec{\hat{\alpha}}_{t}}{$t\gg 1$}
\end{equation}
Bijgevolg kunnen we $\lambda$ interpreteren als de kans op lange termijn dat we vanuit een normale populatie de volgende stap niet naar een doel-populatie migreren. Voor elk zinvol probleem kunnen we stellen dat $\lambda$ dicht bij $N-g/N\approx 1$ zal blijven (dit is ook de oorzaak van de laatste vereenvoudiging in \eqnref{speedupMetaheuristic}). Verder geldt altijd: $0\leq\lambda<1$.

\paragraph{}

We kunnen ook de rechtse dominante eigenvector $\evr[\hat{P}]$ beschouwen. Vermits eigenvectoren geen karakteristieke lengte hebben, beschouwen we $\dabs{\evr[\hat{P}]}_1=1$. Vermits het volgende geldt:
\begin{equation}
\conditional{A^t\approx\ev[A]^t\cdot\evr[A]\cdot\evl[A]^T}{$t\gg1$}
\end{equation}
Kunnen we stellen dat de genormaliseerde rechtse eigenvector een probabiliteitsvector is die aangeeft in welke mate populaties $i$ bijdragen tot het behoudt in een normale populatie. Wanneer we dus de $i$-de gewone populatie beschouwen en het $i$-de element van $\evr[\hat{P}]$ is klein, is er een grote kans dat we in een doel-toestand terecht zullen komen.

\paragraph{}

De interpretatie van de rechtse eigenvector vormt de basis voor de interpretatie van $\sigma$. Vermits $\sigma$ het dot-product is tussen de probabiliteitsvector $\vec{\hat{\alpha}}_0$ en $\evr[\hat{P}]$. Dit dot-product is proportioneel met de cosinus van de hoek tussen beide vectoren:
\begin{equation}
\fun{\cos}{\vec{\hat{\alpha}}_0,\evr[\hat{P}]}=\displaystyle\frac{\vec{\hat{\alpha}}_0^T\cdot\evr[\hat{P}]}{\dabs{\vec{\hat{\alpha}}_0}\cdot\dabs{\evr[\hat{P}]}}=\displaystyle\frac{\sigma}{\dabs{\vec{\hat{\alpha}}_0}\cdot\dabs{\evr[\hat{P}]}}
\end{equation}
Uit de definitie van de cosinus kunnen we bovendien stellen dat:
\begin{equation}
\dabs{\evr[\hat{P}]}=\displaystyle\frac{\dabs{\evl[\hat{P}]}}{\fun{\cos}{\evl[\hat{P}],\evr[\hat{P}]}}
\end{equation}
Bijgevolg kunnen we stellen dat:
\begin{equation}
\sigma=\displaystyle\frac{\dabs{\hat{\alpha}_0}\cdot\fun{\cos}{\vec{\hat{\alpha}}_0,\evr[\hat{P}]}}{\dabs{\evl[\hat{P}]}\cdot\fun{\cos}{\evl[\hat{P}],\evr[\hat{P}]}}
\eqnlab{sigmaMeaning}
\end{equation}
We kunnen dus stellen dat $\sigma$ de ratio is tussen de projecties van enerzijds de initi\"ele kansverdeling $\vec{\hat{\alpha}}$ en anderzijds de linkse eigenvector $\evl[\hat{P}]$ op de rechtse eigenvectoren $\evr[\hat{P}]$. Wanneer $\sigma$ klein is verwachten we sneller een oplossing te vinden. De factoren die hiertoe bijdragen zijn dus:
\begin{itemize}
 \item Een kleine $\dabs{\vec{\hat{\alpha}}_0}$. Vermits de som van $\vec{\hat{\alpha}}_0$ neerkomt om de kans dat onze beginpopulatie geen doelpopulatie is, streven we er naar om deze zo laag mogelijk te houden.
 \item Wanneer $\vec{\hat{\alpha}}_0$ en $\evr[\hat{P}]$ niet gelijkaardig zijn, dan zijn de populaties waar we met een grote kans in starten niet de populaties die met een grote kans naar een normale populatie migreren in de volgende stap. Wanneer de cosinus tussen de vectoren dus klein is, verwachten we in de eerste stappen naar een doel-populatie te migreren.
 \item Wanneer de linkse en de rechtse eigenvector gelijk zijn, is de matrix $\hat{P}$ symmetrisch. Indien de matrix symmetrisch is, verwachtten we doorgaans minder populaties waarnaar we kunnen migreren, maar vervolgens niet uit kunnen geraken. Indien de matrix immers symmetrisch is, is de kans om vanuit een populatie in een andere populatie te migreren immers dezelfde om er vervolgens terug uit te geraken. Daarom streven we naar een zo minimaal mogelijk hoek tussen de twee.
\end{itemize}
Uit de formule kunnen we afleiden dan $\sigma$ aan slechts \'e\'en kant begrensd is: $0<\sigma$.

\subsubsection{Superlineaire Speed-up}

Vermits $0\leq\lambda<1$ en $0<\sigma$, kunnen we met behulp van \eqnref{speedupMetaheuristic} bepalen welke versnelling mogelijk is. Dit is mogelijk wanneer $\sigma<\lambda$. Een opmerkelijk feit is dat deze grenzen superlineaire speed-up niet uitsluiten. Sterker nog, \cite{DBLP:journals/jc/ShonkwilerV94} ontwikkelen in hun paper enkele metaheuristieken voor problemen waar men ook effectief superlineaire speed-up kan vaststellen.

\paragraph{}

Superlineaire speed-up zorgt echter in de literatuur voor veel ophef\cite{superlinearSpeedup}. Theoretisch kan men immers een parallel proces op een sequenti\"ele processor emuleren. Deze processor voert dan afwisselend het werk uit van \'e\'en van de kernen. In de praktijk neemt men soms superlineaire speed-up waar en dit zelfs op een consistente manier. Doorgaans wijt men dit echter aan het \emph{cache-effect}: als we een algoritme op verschillende processoren uitvoeren beschikken we ook over meer snelle geheugens (\emph{cache}). Bovendien kan een deel van het verwerken van data soms uitbesteed worden aan de netwerkapparatuur (bijvoorbeeld bepalen welke processoren op de hoogte moeten worden gebracht van een bepaalde gebeurtenis). Vermits we echter in bovenstaande redenering abstractie van deze extra bronnen van speed-up kan dit de oorzaak niet zijn.

\paragraph{}

De anomalie zit dan ook waarschijnlijk in het feit dat de parallelle variant van de Monte-Carlo simulatie geen zuivere metaheuristiek is: wanneer we immers het aantal processoren opdrijven, drijven we ook het aantal initi\"ele populaties op. Zoals we uit \eqnref{sigmaMeaning} kunnen afleiden wordt $\sigma$ klein indien de beginverdeling $\vec{\hat{\alpha}}$ klein is of sterk afwijkt van $\evr[\hat{P}]$. In dat geval loont het dus eerder de moeite van telkens nieuwe populaties te genereren dan deze populaties te laten evolueren. We kunnen dus stellen dat in het geval $\sigma<\lambda$, de metaheuristiek eenvoudigweg slecht is opgesteld. De voorbeelden die \cite{DBLP:journals/jc/ShonkwilerV94} aanhaalt zijn dan ook problemen in \comp{P} of de metaheuristieken zijn niet correct afgesteld (bijvoorbeeld te lokaal) op het probleem.

\subsubsection{A flaw in the plan: anomalie\"en in het model}

Het voorgestelde model vertoond enkele anomalie\"en. Hieronder geven we een bondig overzicht.

\paragraph{}
We hebben een stationair Markov-systeem gemodelleerd. Metaheuristieken hebben eerder een dynamisch karakter: op basis van kennis uit het verleden, veranderen de transitiematrices per tijdstap. In het geval van bijvoorbeeld \emph{Simulated Annealing} gebeurt dit volgens een vast patroon en kunnen we sommige effecten modelleren. Meestal echter veranderen de kansen op basis van de bezochte populaties. In dat geval is het minder evident om een model op te stellen.

\paragraph{}
In het model vraagt het genereren van een populatie dezelfde hoeveelheid tijd als een transitie. In de praktijk verwachten we dat in heel wat gevallen de tweede stap sneller zal verlopen (omdat de configuraties en populaties maar gedeeltelijk aangepast zullen worden). Dit is ook een aspect die pleit voor metaheuristieken: we besparen tijd door de transities sneller uit te voeren dan het genereren van nieuwe configuraties.

\paragraph{}
Niet elke transitie verloopt even snel: sommige transities vragen inherent meer tijd omdat bijvoorbeeld een groter deel van de populatie wordt aangepast of er veranderen meer variabelen in een configuratie van waarde. Vermits het programma draait op een processor die maar een eindig aantal bits in constante tijd kan veranderen\footnote{We beschouwen geen vector computer.} zullen sommige transities meer tijd kosten.

\paragraph{}
De meeste metaheuristieken de voorkeur aan migratie naar sterke populaties. Hiervoor worden algoritmes zoals Local Search gebruikt. Het spreekt voor zich dat local search transities meestal meer tijd kosten omdat de evaluatiefuncties onderweg voortdurend moeten worden uitgerekend. We kunnen dit probleem oplossen door het introduceren van dummy-populaties: populaties die de toestanden onderweg voorstellen en zo dus meer tijdstappen genereren. Het is echter minder evident om de parameters $\sigma$ en $\lambda$ te berekenen.

\subsubsection{Minimale Speed-up}

Metaheuristieken worden ge\"implementeerd als een evoluerende populatie (mogelijk bestaat de populatie uit \'e\'en individu) met transitiefuncties. De reden is dat men verwacht dat men iets uit het verleden kan leren en goede oplossingen meestal minimaal van elkaar verschillen (dit kan beargumenteerd worden met een studie in \cite{kirkPatrick}). Indien we aan deze assumptie geen geloof hechten kunnen we op een andere manier de optimale oplossing proberen te vinden. Vermits de variabelen en de bijbehorende domeinen op voorhand gekend zijn, kan een programma een toevallige oplossing genereren. Bovendien kunnen we een programma zo ontwerpen dat het uitsluitend oplossingen genereert die nog niet eerder gegenereerd werden in dit in een tijdscomplexiteit onafhankelijk van het aantal eerder gegenereerde oplossingen.
\paragraph{}
Stel dat het domein exact \'e\'en optimale oplossing bevat, dan is de kans dat we in iteratie $t$ deze waarde vinden gelijk aan:
\begin{equation}
\Prob{\theta=t|\theta\geq t}=\guards{
0&\mbox{if }t>N\\
1/N-t+1&\mbox{otherwise}
}
\end{equation}
Met $N$ het aantal mogelijke configuraties ($N=\bigoh{\prod_i\abs{A_i}}$). In dat geval is gemiddelde raaktijd:
\begin{equation}
\mean{\hittime}=\displaystyle\sum_{i=1}^N\Prob{\hittime\geq i}=\displaystyle\sum_{i=1}^N\displaystyle\frac{i}{N}=\displaystyle\frac{N+1}{2}
\end{equation}
In het geval dat we dit algoritme op $p$ machines laten draaien bekomen we:
\begin{equation}
\mean{\phittime}=\displaystyle\sum_{i=1}^N\Prob{\phittime\geq i}=\displaystyle\sum_{i=1}^N\Prob{\hittime\geq i}^p=\displaystyle\frac{1}{N^p}\displaystyle\sum_{i=1}^Ni^p\approx\displaystyle\frac{N}{p+1}
\end{equation}
De laatste benadering bekomen we omdat de som een veelterm van graad $p+1$ is met $1/p+1$ als leidende co\"effici\"ent. Bijgevolg is de \emph{speed-up} bij benadering gelijk aan:
\begin{equation}
\mbox{speed-up}=\displaystyle\frac{\mean{\hittime}}{\mean{\phittime}}\approx\displaystyle\frac{\brak{N+1}\cdot\brak{p+1}}{2\cdot N}\approx\displaystyle\frac{p+1}{2}
\eqnlab{minimalSpeedupMetaheuristic}
\end{equation}
De formule is verder te veralgemenen naar $g$ gewenste oplossingen maar leidt tot dezelfde speed-up. Men dient wel op te merken dat er doorgaans betere algoritmen bestaan om naar de oplossing te zoeken dan louter per toeval configuraties te genereren en in dat geval is het meestal minder evident om dezelfde speed-up te realiseren.



\subsection{Parallelliseren van metaheuristieken in de praktijk}

Een referentiewerk hoe men metaheuristieken in de praktijk op verschillende processoren laat werken is ``\emph{Parallel Metaheuristics: A New Class of Algorithms}'' van Alba\cite{Alba2005book}. In het boek maakt men het eerder vermelde onderscheid tussen \emph{Local Search Metaheuristics (LMS)} en anderzijds \emph{Evolutionary Algorithms (EA)}.
\paragraph{}
Bij de \emph{Local Search Metaheuristics} ziet men drie bronnen tot parallellisme:
\begin{enumerate}
 \item \emph{Parallel multistart}: de processoren starten met een verschillende initi\"ele populatie en werken min of meer onafhankelijk. Dit is het parallellisatie-paradigma die we in de modellering hebben beschouwd.
 \item \emph{Parallel moves}: we beschouwen \'e\'en globale actieve oplossing. Wanneer we een migratie uitvoeren op een sequenti\"ele implementatie genereren we echter \'e\'en oplossing. We kunnen elke processor een migratie laten uitvoeren en vervolgens uit de verzameling van oplossingen uit de omgeving een nieuwe actieve oplossing kiezen. In het geval we een \emph{Local Search} stap uitvoeren kunnen we bijvoorbeeld ook elke processor een deel van de volledige omgeving laten afzoeken.
 \item \emph{Move acceleration}: een migratie kan er soms uit bestaan dat een significant deel van de oplossing verandert. Bovendien kan de evaluatiefunctie zelf ook moeilijk uit te rekenen zijn. Bij \emph{move acceleration} zijn we in staat om de oplossing op te delen in min of meer onafhankelijke delen. Elke processor past een deel van de configuratie aan en rekent de verandering van fitness-waarde met betrekking tot dit deel uit. Soms is er nog postprocessing vereist om de uiteindelijke fitness-waarde te berekenen. In andere modellen laat men fouten toe in de evaluatiefunctie en berekend men de fitness-waarde maar sporadisch correct.
\end{enumerate}

\paragraph{}
In het geval van \emph{Evolutionary Algorithms} ziet men drie bronnen tot parallellisme:
\begin{enumerate}
 \item \emph{Parallel computation}: meestal dient men operaties toe te passen op een significant deel van de populatie (bijvoorbeeld het berekenen van fitness-waardes). Vermits oplossingen nog steeds onafhankelijke entiteiten zijn, is dit een inherente vorm van parallellisatie.
 \item \emph{Parallel population}: men kan ook op elke processor en onafhankelijke populatie beschouwen waarop de processor dan werkt. Meestal gaat dit enigszins gepaard met kruisbestuiving: het uitwisselen van sterke oplossingen. Meestal noemt een genetisch algoritme waarbij men subpopulaties beschouwd een \emph{Island Model}\cite{islandModel}.
\end{enumerate}

\paragraph{}
In de bovenstaande modellen tot parallellisatie zijn de bronnen geordend in stijgende granulariteit. Meestal introduceert een fijnere granulariteit echter ook problemen in verband met mogelijke speed-ups: na elke stap worden de processoren opnieuw gesynchroniseerd. Op het moment dat een processor klaar is met zijn werk dient er gewacht te worden tot ook andere processoren deze \emph{barrier} bereiken. Dit introduceert dan ook een vorm van overhead. \cite{conf/glvlsi/HaldarNCB00} rapporteert bijvoorbeeld een negatieve speed-up door het grote aantal synchronisatiestappen.

\paragraph{}
Crainic en Toulouse\cite{crainicAndToulouse} delen de parallellisatie van metaheuristieken op in drie algemene types. Onder \emph{Type 1} vallen de \emph{Move acceleration}, \emph{Parallel moves} en \emph{Parallel computation} paradigma's van Alba. \cite{crainicAndToulouse} stellen dat dit type van parallellisatie uitsluitend probeert om operaties of evaluaties te versnellen. Wanneer men echter het algoritme op een sequenti\"ele processor draait met eenzelfde aantal iteraties, verwachten we dat dezelfde elementen in de oplossingsruimte werden onderzocht. \emph{Type 3} komt overeen met \emph{Parallel multistart} en \emph{Parallel population}. Crainic en Toulouse argumenteren hier dat al deze technieken leiden tot parallelle verkenning van dezelfde zoekruimte. \emph{Type 2} is een vorm van parallellisme die in Alba minder aan bod komt: het opdelen van de zoekruimte. Dit gebeurt door een deel van de variabelen vaste waardes te laten aannemen, de overige variabelen worden dan geoptimaliseerd door de verschillende processoren. Meestal kiest men per processor andere variabelen. Op vaste tijdstippen zal een processor vervolgens de configuraties van de verschillende processoren samenvoegen in \'e\'en globale oplossing. Omdat dit systeem sommige delen van de configuratieruimte onbezocht kan laten worden op geregelde tijdstippen nieuwe oplossingsruimtes aan de processoren toegekend.

\paragraph{}
Crainic en Toulouse maken ook een onderscheid in \emph{type 3} tussen \emph{interagerende processen} en \emph{niet-interagerende processen}. In het tweede geval wisselen de processen ook bepaalde aspecten in het zoekproces. Men verwacht dat met behulp van deze informatie er betere strategie\"en kunnen worden ontwikkeld die een metaheuristiek sneller naar de oplossing zullen leiden.

\subsubsection{Empirische resultaten}

Alba geeft in zijn boek een uitgebreid overzicht over verschillende pogingen om metaheuristieken te parallelliseren. In een groot aantal families van metaheuristieken is lineaire speed-up geen evidentie. In het geval van \emph{Ant Colony Optimisation (ACO)} bijvoorbeeld rapporteren zowel \cite{alba8-42,alba8-38,alba8-7,alba8-16} lage effici\"entie (meestal rond de 0.40) die meestal daalt naarmate het aantal processoren toeneemt. Andere paradigma's, zoals bijvoorbeeld \emph{Tabu Search}, halen wel sterke resultaten die meestal lineaire speed-up benaderen.

\subsubsection{Speed-up als een goede metriek?}
Empirische resultaten tonen dat de empirische speed-up sterk kan vari\"eren naargelang het paradigma van de metaheuristiek. Ook wanneer men een sterke \emph{speed-up} realiseert, ontbreken harde garanties dat dit op alle probleeminstanties het geval zal zijn. \cite{crainicAndToulouse} plaats daarom vraagtekens bij de betekenis van \emph{speed-up}. Omdat we meestal de exacte oplossing niet kennen is speed-up bovendien moeilijk te verifi\"eren: een parallel algoritme kan snel sterke oplossingen genereren, maar zal misschien niet significant sneller de echte oplossing vinden. Bovendien zal men meestal de uitvoer stopzetten alvorens een metaheuristiek de oplossing heeft gevonden. \cite{crainicAndToulouse} stelt daarom drie andere metrieken voor: kwalitatief sterkere oplossingen gedurende een significant deel van het zoekproces, robuustere systemen die meer garantie bieden op een sterke oplossing, of een algoritme die de algemene tijdscomplexiteit naar omlaag haalt.

\subsubsection{Parallellisatie met behulp van aangepast hardware}
Het boek besteed ook aandacht aan hardware-parallellisatie: simulaties tonen aan dat een processor soms een significante hoeveelheid rekentijd besteed aan communicatie met andere processoren. Door hardware te ontwikkelen kan men de rekentijd verder beperken. Sommige metaheuristieken zijn bijvoorbeeld op een \emph{Field Programmable Gate Array (FPGA)} ge\"implementeerd\cite{conf/fpt/GuntschMSDESS02,journals/gpem/Martin01}.

\section{Hyperheuristieken}

Deze masterthesis behandelt het parallelliseren van hyperheuristieken. Hyperheuristieken zijn een veralgemening van metaheuristieken. In \algref{metaheuristicGeneral} genereert het algoritme een nieuwe populatie uit de vorige populatie. Hoe dit concreet gebeurt is de verantwoordelijkheid van de metaheuristiek. Een hyperheuristiek werkt met een set transitiefuncties zonder de details van de functies te kennen\footnote{In de meeste implementaties kent de hyperheuristiek wel enkele gegevens over de transitiefunctie (zo worden transitiefuncties meestal onderverdeelt in verschillende types).}. \algref{hyperheuristicsGeneral} geeft een beschrijving op hoog niveau hoe een hyperheuristiek met een oplossing zoekt door een sequentie transitiefuncties toe te passen op een initi\"ele populatie. De hyperheuristiek maakt dan gebruik van on-line machine learning technieken om een zo optimaal mogelijke sequentie te genereren.

\begin{algorithm}[H]
 \SetAlgoLined
 Bereken een initi\"ele set van oplossingen $\PopSet_1$ en hun fitness-waarde\;
 $b_1\leftarrow\displaystyle\argmin_{\sol\in\PopSet_1}{\fun{\evalfun}{\sol}}$\;
 \Repeat{het stopcriterium is bereikt}{
  Voer een transitiefunctie $h_i\in\TranSet$ op \'e\'en of meer oplossingen in $\PopSet_{t}$ en genereer hiermee $\PopSet_{t+1}$\;
  $b_{t+1}\leftarrow\displaystyle\argmin_{\sol\in\PopSet_{t+1}\cup\accl{b_t}}{\fun{\evalfun}{\sol}}$\;
  $t\leftarrow t+1$\;
 }
 \KwRet{$b_t$}
 \caption{Hoog niveau beschrijving van een hyperheuristiek.}
 \alglab{hyperheuristicsGeneral}
\end{algorithm}

\paragraph{}
Het doel van een hyperheuristiek is om de doeltreffendheid te verhogen tegenover een metaheuristiek. Metaheuristieken zijn geprogrammeerd met een bepaald probleem in het achterhoofd. Speciale gevallen van het probleem zijn soms minder effectief op te lossen. Hyperheuristieken kunnen in dat geval een oplossing bieden.

\paragraph{}
Naast het oplossen van een groter aantal probleeminstanties, kunnen hyperheuristieken ook worden ingezet om een grote familie aan optimalisatieproblemen op te lossen. Vermits de transitiefuncties een vorm van invoer zijn, kan men een ander probleem oplossen door andere transitiefuncties te injecteren. Op termijn hoopt men dat men hiermee een algoritme kan ontwikkelen die eender welk optimalisatieprobleem kan oplossen.

\subsubsection{Het effect van de transitiefuncties als een metriek voor de huidige stand van zaken}
Sinds een tiental jaar doet men actief onderzoek naar hyperheuristieken.

\subsubsection{Parallelle Hyperheuristieken}



% \begin{algorithm}[H]
%  \SetAlgoLined
%  \KwData{}
%  \KwResult{how to write algorithm with \LaTeX2e }
%  \ParFor($i=1\ldots p$){
%   Bereken een initi\"ele set van oplossingen $\PopSet_{1,i}$ en hun fitness-waarde\;
%   $b_{1,i}\leftarrow\displaystyle\argmin_{\sol\in\PopSet_{1,i}}{\fun{\evalfun}{\sol}}$\;
%  }
%  $B_1\leftarrow\displaystyle\argmin_{\sol\in\accl{b_{1,1},b_{1,2},\ldots,b_{1,p}}}{\fun{\evalfun}{\sol}}$\;
%  $t\leftarrow1$\;
%  \Repeat{het stopcriterium is bereikt}{
%    \ParFor($i=1\ldots p$){
% 	Genereer op basis van $\tupl{\PopSet_{t,1},\PopSet_{t,2},\ldots,\PopSet_{t,p}}$ en $t$ stochastisch een nieuwe set oplossingen $\PopSet_{t+1,i}$ samen met hun fitness-waarde\;
% 	$b_{t+1,i}\leftarrow\displaystyle\argmin_{\sol\in\PopSet_{t+1,i}\cup\accl{b_{t,i}}}{\fun{\evalfun}{\sol}}$\;
%    }
%    $B_{t+1}\leftarrow\displaystyle\argmin_{\sol\in\accl{b_{t+1,1},b_{t+1,2},\ldots,b_{t+1,p}}}{\fun{\evalfun}{\sol}}$\;
%    $t\leftarrow t+1$\;
%  }
%  \KwRet{$B_t$}
%  \caption{Hoog niveau beschrijving van een parallelle metaheuristiek.}
%  \alglab{parallelMetaheuristicGeneral}
% \end{algorithm}

% \begin{figure}
% \centering
% \begin{tikzpicture}
% \foreach \a/\x/\y/\t in {0/0/2/Doel,1/5/0/Niet-doel,2/7/0/Initieel,3/9/0/Limiet} {
%   \node (L\a) at (\x,-\y+0.5) {\t};
% }
% \foreach \x/\t in {2/1,3/2,5/g} {
%   \node[draw=black,minimum width=1.5 cm,rectangle] (G\x) at (0,-\x) {$\PopSet_{\t}$};
% }
% \node[draw=black,dashed,fit=(L0) (G2) (G5),minimum width=1.5 cm,inner sep=10pt,rectangle] (GA) {};
% \foreach \x/\t/\u in {0/g+1/1,1/g+2/2,2/g+3/3,4/i/i-g,6/j/j-g,8/N+g/N} {
%   \node[draw=black,minimum width=1.5 cm,rectangle] (P\x) at (5,-\x) {$\PopSet_{\t}$};
%   \node (IP\x) at (P\x -| L2) {$\hat{\alpha}_{\u}$};
%   \node (EP\x) at (P\x -| L3) {$w_{\u}$};
%   \draw[->] (P\x.west) to node[above,midway,sloped]{$1-\hat{v}_{\u}$} (GA);
% }
% \node[draw=black,dashed,fit= (L1) (L3) (P0) (P8),minimum width=1.5 cm,inner sep=10pt,rectangle] (PA) {};
% \coordinate (LC) at ($(PA.south)+(0,-0.25)$);
% \coordinate (TR) at ($(PA.north east)+(0.25,0.25)$);
% \draw[->] (PA.south) -- ++(0,-0.25) to node[above,midway,sloped]{$1-\lambda$} (GA.south |- LC) -- (GA.south);
% \draw[->] (PA.north) -- ++(0,0.25) to node[above,midway,sloped]{$\lambda$} (TR) |- (PA.east);
% \end{tikzpicture}
% \caption{Schematische voorstelling van het effect van de parameters}
% \end{figure}

% \begin{figure}
% \begin{tikzpicture}[scale=4,domain=0.1:0.9]
% \draw[very thin,color=gray] (-0.1,-0.1) grid (1.1,1.1);
% \draw[->] (-0.2,0) -- (1.2,0) node[right] {$\lambda$};
% \draw[->] (0,-0.2) -- (0,1.2) node[above] {$\mean{\theta}$};
% \foreach \s/\t in {0.1/,0.05/dashed,0.025/dotted} {
%   \draw[\t] plot[id=x] function{\s/(x*(1-x))} node[right] {$\sigma=\s$};
% }
% \end{tikzpicture}
% \caption{Invloed van de parameter $\lambda$ op de raaktijd}
% \end{figure}
\section{Besluit van dit hoofdstuk}
Als je in dit hoofdstuk tot belangrijke resultaten of besluiten gekomen
bent, dan is het ook logisch om het hoofdstuk af te ronden met een
overzicht ervan. Voor hoofdstukken zoals de inleiding en het
literatuuroverzicht is dit niet strikt nodig.

%%% Local Variables:
%%% mode: latex
%%% TeX-master: "masterproef"
%%% End:

\chapter{Studie naar Sequenti\"ele Hyperheuristieken (CHeSC2011)}
\label{hoofdstuk:2}

Alvorens zelf hyperheuristieken te implementeren, is het interessant om te analyseren aan welke eigenschappen een goede hyperheuristiek moet voldoen. We verwachten immers dat bij het effici\"ent paralleliseren van deze algoritmen sommige wetmatigheden kunnen worden overgenomen. Vandaar deze studie naar \abseqe{} \abhhn{}.

\section{Implementatie-set}

\subsection{\abhf{}}

In 2010 publiceren \auth[5586064]{Burke et al.} een paper waarin ze \abhf{} voorstellen. \abhf{} is een klassenbibliotheek geschreven in \abjava{}. Het laat toe dat de gebruiker hyperheuristieken implementeert zonder dat het algoritme de details kent van het onderliggende probleem en de \abllhn{}. Het systeem werkt op basis van geheugen: een lijst waarin tussentijdse oplossingen worden opgeslagen en uitgelezen. Het systeem biedt vervolgens de mogelijkheid om een \abllh{} met een specifieke index toe te passen op de oplossing op een specifieke plaats in het geheugen en het resultaat op een specifieke plaats op te slaan. Daarnaast kunnen gebruikers ook de objectiviteitswaarde van een bepaalde oplossing in het geheugen opvragen en vragen of twee oplossingen in het geheugen equivalent zijn.

\paragraph{}
Het programma heeft dan wel geen idee wat de onderliggende \abllhn{} precies doen, ze worden wel gecategoriseerd in 4 verschillende types:
\begin{enumerate}
 \item \abmt{}: hierbij wordt de oplossing op een toevallig manier aangepast. Dit soort heuristieken heeft geen componenten die inschatten of de verandering onmiddellijk of op termijn tot betere resultaten zal leiden.
 \item \abco{}: dit zijn de enige heuristieken die twee oplossingen recombineren in een nieuwe oplossing. Het is uiteraard de bedoeling dat de nieuwe oplossing karakteristieken gemeen heeft met beide ``ouders''.
 \item \abrr{}: deze heuristieken breken een deel van de oplossing af, om ze dan vervolgens met behulp van bijvoorbeeld een gretig algoritme terug op te bouwen.
 \item \abls{}: dit is een familie van algoritmen die herhaaldelijk mutaties uitvoeren indien deze mutaties ook per stap winst opleveren. Indien geen enkele mutatie meer tot een beter resultaat leidt stopt het algoritme.
\end{enumerate}
Het softwaresysteem houdt ook de mogelijkheid open om een \abllh{} te classificeren onder ``other'', maar voor zover ons bekend zijn er nog geen heuristieken in \abhf{} geschreven die onder deze categorie vallen. Verder kan er verwarring optreden over het feitelijke verschil tussen \abrr{} en \abls{}. We kunnen immers bij beide families verwachten dat ze minstens een oplossing opleveren die beter is dan het origineel (\abrr{} zal immers tot in dat geval het afgebroken gedeelte reconstrueren zoals het origineel). Een verschil die men doorgaans maakt is dat \abls{} een operator is die idempotent is.

\subsection{\abchescy{}}

Om meer aandacht te vestigen op \abhf{} werd in 2011 een wedstrijd georganiseerd door de universiteit van Nottingham: de ``\emph{Cross-domain Heuristic Search Challenge (\abchescy)}''. De verschillende programma's krijgen een set van verschillende problemen en worden gequoteerd op basis van de kwaliteit van de oplossingen die ze na 10 minuten uitvoer afleveren.%TODO(check)
De wedstrijd omvatte problemen uit zes verschillende domeinen: \prob{Maximal Satisfiability}, \prob{Bin Packing}, \prob{Personnel Scheduling}, \prob{Flow Shop}, \prob{Travelling Salesman Problem} en het \prob{Vehicle Routing Problem}.

\paragraph{}
In totaal telde de competitie 20 teams. We hebben met onze studie de zestien implementaties die gedocumenteerd werden bestudeerd. Tabel \ref{tbl:chescParticipants} bevat een lijst met de verschillende implementaties en geeft aan welke systemen in de studie opgenomen werden.

\begin{table}[hbt]
  \centering
  \begin{tabular}{rllrc} \toprule
    \#&Naam&Auteur/Team&Score&Bestudeerd\\\midrule
    1&	AdapHH\cite{chesc-adaphh,chesc-adaphh2}&		Mustafa M\i{}s\i{}r&	181.00&	$\checkmark$\\
    2&	VNS-TW\cite{chesc-vns-tw}&				Mathieu Larose&		134.00&	$\checkmark$\\
    3&	ML\cite{chesc-ml,chesc-ml2}&				Mustafa M\i{}s\i{}r&	131.50&	$\checkmark$\\
    4&	PHUNTER\cite{chesc-phunter}&				Fan Xue&		93.25&	$\checkmark$\\
    5&	EPH\cite{chesc-eph}&					David Meignan&		89.75&	$\checkmark$\\
    6&	HAHA&							Andreas Lehrbaum&	75.75&	\\
    7&	NAHH&							MFranco Mascia&		75.00&	\\
    8&	ISEA\cite{chesc-isea}&					Jiri Kubalik&		71.00&	$\checkmark$\\
    9&	KSATS-HH\cite{chesc-ksats-hh}&				Kevin Sim&		66.50&	$\checkmark$\\
    10&	HAEA\cite{chesc-haea}&					Jonatan Gomez&		53.50&	$\checkmark$\\
    11&	ACO-HH\cite{chesc-aco-hh}&				Jos\'e Luis N\'u\~nez&	39.00&	$\checkmark$\\
    12&	GenHive\cite{chesc-genhive}&				CS-PUT&			36.50&	$\checkmark$\\
    13&	DynILS\cite{chesc-dynils}&				Mark Johnston&		27.00&	$\checkmark$\\
    14&	SA-ILS&							He Jiang&		24.25&	\\
    15&	XCJ&							Kamran Shafi&		22.50&	\\
    16&	AVEG-Nep\cite{chesc-aveg-nep}&				Thommaso Urli&		21.00&	$\checkmark$\\
    17&	GISS\cite{chesc-giss}&					Alberto Acu\~na&	16.75&	$\checkmark$\\
    18&	SelfSearch\cite{chesc-selfsearch}&			Jawad Elomari&		7.00&	$\checkmark$\\
    19&	MCHH-S\cite{chesc-mchh-s}&				Kent McClymont&		4.75&	$\checkmark$\\
    20&	Ant-Q\cite{chesc-ant-q}&				Imen Khamassi&		0.00&	$\checkmark$\\
    \bottomrule
  \end{tabular}
  \caption{Deelnemers van de \abchescy{} competitie.}
  \label{tbl:chescParticipants}
\end{table}


%De ``\emph{Cross-domain Heuristic Search Challenge (CHeSC)}'' was een wedstrijd georganiseerd in 2011. De wedstrijd spitste zich toe op het ontwikkelen van hyperheuristieken in ``\abhyfl{}''\cite{hyflex2012,5586064}. \abhyfl{} is een klassenbibliotheek geschreven in Java 

\subsection{Een item}
Een tekst staat nooit alleen. Dit wil zeggen dat er zeker ook referenties
nodig zijn. Dit kan zowel naar on-line documenten\cite{wiki} als naar
boeken\cite{pratchett06:_good_omens}.

\section{Tabellen}
Tabellen kunnen gebruikt worden om informatie op een overzichtelijke te
groeperen. Een tabel is echter geen rekenblad! Vergelijk maar eens
tabel~\ref{tab:verkeerd} en tabel~\ref{tab:juist}. Welke tabel vind jij het
duidelijkst?

\begin{table}
  \centering
  \begin{tabular}{||l|lr||} \hline
    gnats     & gram      & \$13.65 \\ \cline{2-3}
              & each      & .01 \\ \hline
    gnu       & stuffed   & 92.50 \\ \cline{1-1} \cline{3-3}
    emu       &           & 33.33 \\ \hline
    armadillo & frozen    & 8.99 \\ \hline
  \end{tabular}
  \caption{Een tabel zoals het niet moet.}
  \label{tab:verkeerd}
\end{table}

\begin{table}
  \centering
  \begin{tabular}{@{}llr@{}} \toprule
    \multicolumn{2}{c}{Item} \\ \cmidrule(r){1-2}
    Animal    & Description & Price (\$)\\ \midrule
    Gnat      & per gram    & 13.65 \\
              & each        & 0.01 \\
    Gnu       & stuffed     & 92.50 \\
    Emu       & stuffed     & 33.33 \\
    Armadillo & frozen      & 8.99 \\ \bottomrule
  \end{tabular}
  \caption{Een tabel zoals het beter is.}
  \label{tab:juist}
\end{table}

\section{Lorem ipsum}
Tenslotte gaan we hier nog wat tekst voorzien zodat er minstens een
bijkomende bladzijde aangemaakt wordt. Dat geeft de gelegenheid om eens te
zien hoe de koptekst en de voettekst zich gedragen.

\section{Besluit van dit hoofdstuk}
Als je in dit hoofdstuk tot belangrijke resultaten of besluiten gekomen
bent, dan is het ook logisch om het hoofdstuk af te ronden met een
overzicht ervan. Voor hoofdstukken zoals de inleiding en het
literatuuroverzicht is dit niet strikt nodig.

%%% Local Variables: 
%%% mode: latex
%%% TeX-master: "masterproef"
%%% End: 

\chapter{\emph{ParHyFlex}: Parallel \emph{HyFlex}}
\label{hoofdstuk:3}

\chapterquote{A skilled transition team leader will set the general goals for a transition and then confer on the other team leaders working with him the power to implement those goals.}{Richard V. Allen}

Op basis van de analyse in hoofdstuk \ref{hoofdstuk:2}, werd een systeem genaamd \emph{ParHyFlex} ge\"implementeerd die het mogelijk maakt om hyperheuristieken te implementeren in een parallelle context. De broncode van dit systeem is te vinden onder \url{https://www.github.com/KommuSoft/ParallelHyFlex} en wordt nog verder actief ontwikkeld. Daar het beschrijven van parallelle algoritmes complex is, geven we eerst een algemeen overzicht. Daarna bespreken we welke probleemafhankelijke componenten we hebben toegevoegd om dit systeem beter te doen werken. Vervolgens beschrijven we de principes achter het parallel uitvoeren van de hyperheuristieken. We eindigen met een bondig overzicht en bespreken welke inzichten uit het vorige hoofdstuk relevant waren voor de implementatie van dit systeem.

\section{Systeem}

\paragraph{}
Zoals al eerder aangehaald maakt een hyperheuristiek een keuze uit verschillende transitie-functies. Deze transitiefuncties kunnen parallel ge\"implementeerd worden. Zo kunnen we bijvoorbeeld \emph{Local Search} heuristieken over verschillende processoren verdelen. Een nadeel van deze %TODO

\paragraph{}
Zowel \emph{HyFlex} als \emph{ParHyFlex} maken een duidelijk onderscheid tussen enerzijds het probleemafhankelijke gedeelte (de kennis die een de gebruiker zelf dient te injecteren om het systeem te laten werken) en het probleemonafhankelijke gedeelte (de strategie die bepaalt welke heuristieken we uitvoeren).

\subsection{Probleemafhankelijk gedeelte}

We werken echter in een parallelle context. Daarom is het interessant dat het probleemafhankelijke gedeelte meer functionaliteiten ter beschikking stelt die uitgebuit kunnen worden door het bovenliggende systeem. Concreet denken we hierbij aan vier zaken: \emph{afstandmetrieken}, \emph{ervaring-generatoren}, \emph{zoekruimte beperkers} en \emph{multi-objectieven}

\subsubsection{Afstandmetrieken}
\abhf{} laat problemen een functie aanbieden die twee oplossingen met elkaar kan vergelijken. Deze functie kunnen we theoretisch omvormen tot een afstandsmetriek (we stellen de afstand tussen twee dezelfde oplossingen gelijk aan 0, en tussen twee verschillende aan een arbitraire constante groter dan 0). Deze metriek levert echter weinig informatie op. In een sequenti\"ele context gebruikt men soms het aantal mutaties die tussen een oplossing en \'e\'en van zijn voorouders om de afstand af te schatten. Dit is natuurlijk slechts een benadering. In het geval van parallelle uitvoer zullen we bovendien meestal niet over deze informatie beschikken. Daarom is het nuttig om de afstand tussen twee oplossingen te kunnen inschatten. In het geval de afstand geen triviaal gegeven is, kan men verschillende afstandsmetrieken defini\"eren en beslist de bovenliggende hyperheuristiek over de waarde van de metrieken. Een afstandmetriek is dus gedefinieerd als:
 \begin{equation}
  \delta_i:\SolSet^2\rightarrow\RealSet^+
 \end{equation}
 
\subsubsection{Ervaring-generatoren}
Elk proces draait een eigen hyperheuristiek en komt een sequentie heuristieken tegen. Uitwisselen van de sequentie kan potentieel een voordeel opleveren omdat de hyperheuristieken met meer kennis van zaken kunnen beslissen. Het doorsturen van alle heuristieken is doorgaans niet mogelijk omdat dit een te grote druk op het netwerk zet en bovendien de overige processoren te veel rekenkracht zouden investeren in het analyseren van de ontvangen oplossingen. Door het uitwisselen van ervaring, een compacte voorstelling van beschouwde oplossingen, zouden we dit probleem kunnen oplossen.

\subsubsection{Zoekruimte-beperkers}
Wanneer processoren oplossingen met elkaar uitwisselen lopen we de kans dat de verschillende processoren op termijn vergelijkbare populaties onderhouden. Dit laatste is nuttig wanneer sterke oplossingen in de buurt liggen van de oplossingen in de populatie. Indien de populaties echter rond eenzelfde lokaal optimum liggen, is dit nefast. In dat geval proberen alle processoren het lokale optimum te zoeken in een eenzelfde gebied, en wordt migratie naar mogelijk betere oplossingen in een ander gebied minder evident. Het introduceren van een component die diversificatie afdwingt kan helpen te voorkomen dat we op ijle populaties stuiten.

\subsubsection{Multi-objectieven}
Alle processoren proberen hetzelfde optimalisatieprobleem op te lossen. Door extra objectieven te introduceren, kunnen we echter een meer divers zoekproces aanbieden. Deze extra objectieven zijn eerder virtueel en dienen meer als een \emph{tie-breaker} in bijvoorbeeld gevallen waarbij twee oplossingen dezelfde fitness-waarde hebben.

\subsubsection{Afdwingbare beperkingen als probleemonafhankelijke ervaring}

Een probleem bij het genereren van \emph{ervaring} en het beperken van de \emph{zoekruimte} is dat dit op een probleemonafhankelijke manier dient te gebeuren: de bovenliggende hyperheuristiek heeft geen details over de structuur van de configuraties en kan bijgevolg niet zelf de zoekruimte beperken of conclusies genereren. We kunnen ervaring voorstellen als een object waar de hyperheuristiek de specificaties niet van kent, maar in dat geval moet ervaring wel enkele algemene functionaliteiten kunnen aanbieden die nuttig zijn. Om dit probleem op te lossen voeren we het concept van een \emph{afdwingbare beperking} in.

\begin{definition}[Afdwingbare beperking]
Een \emph{afdwingbare beperking} is een 3-tuple: $\tupl{c,c^+,c^-}$. $c:\SolSet\rightarrow\BoolSet$ is hierbij een functie die controleert of een gegeven oplossing aan een bepaalde voorwaarde voldoet. $c^+:\SolSet\rightarrow\SolSet$ is een functie die een gegeven oplossing minimaal kan aanpassen zodat deze aan de voorwaarde voldoet. $c^-:\SolSet\rightarrow\SolSet$ past oplossingen minimaal aan zodat ze niet aan de voorwaarde voldoen. De set van alle afdwingbare beperkingen die we op een probleem kunnen toepassen noteren we als $\HypSet$.
\end{definition}

We kunnen een afdwingbare beperking als een vorm van ervaring zijn. In de loop der tijd kunnen we immers een hypothese ontwikkelen dat sterke oplossingen aan een bepaalde voorwaarde voldoen (bijvoorbeeld een variabele in het probleem krijgt een vaste waarde). We kunnen dan oplossingen aantrekken naar de hypothese door $c^+$ op willekeurige oplossingen toe te passen. Anderzijds kunnen we ook oplossingen afstoten van de hypothese met $c^-$. De bovenliggende hyperheuristiek dient echter niet op de hoogte te zijn welke voorwaarden een concrete afdwingbare beperking stelt, zolang deze maar de oplossingen kan manipuleren.

\paragraph{}
Afdwingbare beperkingen kunnen we eveneens gebruiken om de zoekruimtes te beperken. Elk proces kan immers een aantal afdwingbare beperkingen gebruiken om een bepaalde zoekruimte te beschouwen, terwijl het de afdwingbare beperkingen van de andere processoren gebruikt om uit de buurt van de andere zoekruimtes te blijven.

\paragraph{}
De hyperheuristieken zelf kunnen geen ervaring genereren, ze hebben immers geen weet van de structuur van een oplossing. Daarom zal het specifieke probleem dus een set functies defini\"eren die we \emph{hypothese-generatoren (hypogen)} noemen:
\begin{definition}
Een \emph{hypothese-generator (hypogen)} $g_i:\SolSet^{n_{g_i}}\rightarrow\HypSet$ is een functie die op basis van een set oplossingen een afdwingbare beperking kan genereren.	
\end{definition}

\subsection{Probleemonafhankelijk gedeelte}

Het probleemonafhankelijke gedeelte wordt ge\"implementeerd door de hyperheuristiek en kan dus los gezien worden van \emph{ParHyFlex}. Het probleemafhankelijke gedeelte biedt echter functionaliteiten aan waarvoor we ondersteuning kunnen bieden in het probleemonafhankelijke gedeelte.

\paragraph{}
In \emph{ParHyFlex} werden daarom de volgende componenten ge\"implementeerd: \emph{uitwisselen van oplossingen}, \emph{afbakenen van zoekruimtes}, \emph{genereren van ervaring} en \emph{onderhandelen over een nieuwe zoekruimte}. In de volgende subsubsecties zullen we deze taken verder bespreken.

\subsubsection{Uitwisselen van oplossingen}

Elke processor werkt met een eigen lokaal geheugen, maar reserveert ook plaats voor de geheugens van de andere processoren. Op het moment dat een nieuwe oplossing naar een lokale geheugencel geschreven wordt, zal op basis van een \emph{uitwisselingsstrategie} beslist worden met welke processoren deze oplossing zal worden gedeeld. De taak van het verzenden en ontvangen van een oplossing samen met een reeks uitwisselingsstrategie\"en wordt ondersteund door \emph{ParHyFlex}.

\subsubsection{Afbakenen van de zoekruimte}
 
De zoekruimte bewaken is ook een verantwoordelijkheid van \emph{ParHyFlex}. Hiervoor voorziet men twee sets van afdwingbare beperkingen: positieve en negatieve. Telkens wanneer er een nieuwe oplossing wordt gegenereerd\footnote{Of via uitwisseling in het geheugen wordt ingeladen.} zal \emph{ParHyFlex} alle beperkingen in de positieve set afdwingen en \'e\'en beperking uit de negatieve set. Het afdwingen gebeurt in een willekeurige volgorde. Dit komt omdat de beperkingen met elkaar kunnen interfereren: een eerste beperking kan een variabele op \'e\'en waarde zetten waarna de volgende beperkingen deze wijziging weer ongedaan maakt. Men kan dit probleem proberen op te lossen door alle permutaties uit te proberen in de hoop dat \'e\'en mutatie toch tot het correcte resultaat leidt. Dit is echter niet noodzakelijk zo, en bovendien vereist een dergelijke oplossing exponenti\"ele tijd. We nemen aan dat de beperkingen meestal minimaal met elkaar interfereren en dat een zoekruimte niet strikt moet worden bewaakt. De hierboven vernoemde strategie is niet verplicht. Men kan door een interface te implementeren een andere strategie hanteren.

\subsubsection{Genereren van Ervaring}
 
Telkens wanneer \'e\'en van de processoren een nieuwe oplossing voortbrengt, kan hij deze oplossing -- samen met andere oplossingen -- omzetten in een afdwingbare beperking. Een processor kan echter niet alle beperkingen blijven bewaren: het uitwisselen van ervaring dient snel te gebeuren, we dienen een voldoende grote zoekruimte te behouden en bovendien kunnen we net een beperking genereren die het zoeken de foute kant opstuurt. Daarom maken we gebruik van een \emph{ervaring-set}, een set van vaste grootte waar gegenereerde beperkingen in worden bewaard. De elementen in de set worden telkens ge\"evalueerd: telkens wanneer er een nieuwe oplossing wordt gegenereerd, zal de \emph{ervaring-set} kijken aan welke beperkingen de oplossing voldoet. Op basis van de fitness-waarde van de oplossingen kunnen de beperkingen dan ge\"evalueerd worden. Door de lijst van fitness-waardes op te delen in waardes waarbij de beperking wordt gerespecteerd en waardes waarbij dat niet het geval is, ontstaan twee sets aan punten. Met een online algoritme\cite[p. 232]{citeulike:175026} berekenen we voor beide sets het gemiddelde en de variantie. Door de evaluaties van beide sets als onafhankelijke normale verdelingen te beschouwen, kunnen we de kans uitrekenen dat een fitness-waarde van een oplossing die aan de voorwaarde voldoet kleiner is dan de oplossing die niet aan de voorwaarde voldoet. Naarmate de kans groter wordt maken we de assumptie dat de beperking beter is. Omdat de gegenereerde beperkingen ook fout kunnen zijn, dienen we de set regelmatig van nieuwe hypotheses te voorzien, dit proces heet \emph{amnesie}. \emph{Amnesie} wordt op geregelde tijdstippen toegepast: oude hypotheses worden uit de set gehaald op plaats te maken voor nieuwe hypotheses. We wensen dat sterke hypothese meer kans maken om te overleven maar wel de kans lopen om te verdwijnen. Daarom rangschikken we de beperkingen op basis van hun evaluatie. De kans dat de hypothese vervolgens uit de set wordt gehaald berekenen we vervolgens op basis van de Benford-verdeling\cite{citeulike:748130}.

\subsubsection{Onderhandelen over een nieuwe zoekruimte}

Elke processor houdt een \emph{ervaring-set} bij. Het is de bedoeling dat deze ervaring wordt gebruikt om een nieuwe zoekruimte af te dwingen. Bovendien kan ervaring uitgewisseld worden met andere processoren zodat deze later ook hun zoekruimtes aanpassen. Tegelijk willen we voorkomen dat de zoekruimtes te homogeen worden en dus potentieel sterke oplossingen genegeerd worden. Dit zijn de taken van de \emph{onderhandelaar}. De \emph{onderhandelaar} is een component die af en toe geactiveerd wordt. Een deel van de afdwingbare beperkingen worden uit de \emph{ervaring-set} gehaald om opgenomen te worden in het positieve component van de \emph{zoekruimte}. Deze beperkingen worden via groepscommunicatie doorgestuurd naar de andere processoren. Een deel van de ontvangen beperkingen komen terecht in de \emph{ervaring-set} een andere deel vormt de basis van het negatieve gedeelte van de \emph{zoekruimte}. Omdat een deel van de afdwingbare beperkingen vanaf dan in de \emph{ervaring-set} van de andere processoren wordt ge\"evalueerd (met een andere zoekruimte), is men in staat om zo'n beperking op een objectievere manier te evalueren\footnote{Sommige afdwingbare beperkingen leiden immers enkel tot sterke resultaten in een bepaalde \emph{zoekruimte}.}.

\subsubsection{Invloed van eerdere studies}



\subsection{Overzicht}

\importtikz[1.4]{parhyflexstructure}{parhyflexstructure}{Structuur van \emph{ParHyFlex}.}
Op \imgref{parhyflexstructure} geven we schematisch de structuur van \emph{ParHyFlex} weer.	De componenten die gemarkeerd worden met een asterisk, zijn component die niet aanwezig zijn in \emph{HyFlex}. Het grijze blok stelt de kern van het \emph{ParHyFlex} systeem voor.%TODO

\paragraph{}
Een deel van de geheugencellen is gemarkeerd met een schuine streep. Deze geheugencellen stellen vreemd geheugen voor waarvan er lokaal een kopie wordt bijgehouden. De geheugencellen kunnen uitgelezen worden, maar er kan geen oplossing naar geschreven worden.

\paragraph{}
\importtikz[1.4]{parhyflexwerking}{parhyflexwerking}{Schematische voorstelling van de kern van \emph{ParHyFlex}.}
Op \imgref{parhyflexwerking} beschrijven we kort het proces die een berekende of ontvangen oplossing doormaakt. Deze oplossing -- op de figuur $s_1^{(0)}$ -- wordt eerst aangepast door de zoekruimte: alle positieve hypotheses en \'e\'en negatieve hypothese worden toegepast op de oplossing en wordt aangepast tot $s_1^{(E)}$ die binnen de zoekruimte valt. De fitness-waarde wordt berekend en de evaluaties van de reeds aanwezige hypotheses in de ervaring-set worden aangepast (de data wordt voor elke hypothese opgenomen in \'e\'en van de twee normale verdelingen). Verder wordt met behulp van \'e\'en van de hypothesegeneratoren  een hypothese gegenereerd die met een bepaalde kans opgenomen wordt in de ervaring-set. De oplossing wordt vervolgens in het geheugen opgenomen en eventueel doorgestuurd naar andere processoren.

\paragraph{}
Op geregelde tijdstippen treed er amnesie op in de \emph{ervaring-set}: een deel van de hypothese worden uit de set verwijdert. Dit gebeurt op basis van de twee normale verdelingen per hypothese. Op die manier kan men zich ontdoen van foute hypothese, en maakt men ruimte voor nieuwe hypotheses.

\paragraph{}
Op vaste tijdsintervallen zal de \emph{onderhandelaar} een deel van de hypotheses uit de \emph{ervaring-set} halen. Een deel van deze hypotheses vormen de nieuwe positieve set van de \emph{zoekruimte}. De overige worden doorgestuurd in de \emph{ervaring-set} van de andere processoren ge\"injecteerd. Een deel van de doorgestuurde hypotheses vormt ook een basis van de negatieve set van de \emph{zoekruimte}.

\section{\emph{ParAdapHH}}

Naast het ontwikkelen van een systeem om hyperheuristieken op verschillende processoren te kunnen laten werken, vereist het testen van het systeem sowieso dat we een concrete hyperheuristiek ontwikkelen. Een logische keuze is om \emph{AdapHH}, de hyperheuristiek voorgesteld door Mustafa M\i{}s\i{}r te implementeren. We geven eerst een motivatie voor deze hyperheuristiek. Daarna bespreken we in meer detail de werking van de hyperheuristiek samen met de wijzigingen om het systeem op verschillende processoren te laten werken.

\subsection{Motiviatie}

\subsection{Werking}

In het vorige hoofdstuk hebben we de werking van \emph{AdapHH} al op hoog niveau beschouwd. In deze subsectie zullen we de werking in meer detail bespreken en de wijzigingen in de context van een parallel systeem bespreken.

\paragraph{}
\emph{AdapHH} is een hyperheuristiek die de gegeven tijd opdeelt in fases. Het maakt gebruik van twee mechanismes die een effici\"ente sequentie genereren: \emph{Adaptive Dynamic Heuristic Set (ADHS)} en \emph{Relay Hybridisation (RH)}. \emph{ADHS} onderhoudt een tabu-set van heuristieken die in het verleden tot sterke resultaten hebben geleid. Heuristieken die niet aan dit criterium voldoen, worden enkele fases uit de set gehaald en daarna opnieuw ge\"introduceerd. Het \emph{RH} component werkt met een \emph{learning automaton}\cite{learningAutomaton} en voert twee heuristieken na elkaar uit. De twee mechanismes worden door elkaar gebruikt. Telkens wanneer \'e\'en van de twee mechanismen een nieuwe oplossing berekend, zal het \emph{Adaptive Iteration Limited List-based Threshold Accepting (AILLA)}-component beslissen of de nieuwe oplossing als actieve oplossing wordt aanvaard.

\subsubsection{ADHS}
\emph{AdapHH} probeert een set van sterke heuristieken te onderhouden. Hiervoor werkt het algoritme in fases. Een heuristiek wordt beoordeelt volgens de prestaties die het sinds de start heeft afgelegd, maar de prestaties in de laatste fase wegen zwaarder door in het besluit of een hyperheuristiek in de set blijft of enkele fases niet meer wordt gebruikt. Om heuristieken te evalueren wordt van volgende metriek gebruik gemaakt:
\begin{align*}
\funm{eval}{h_i}\isdefinedas{}&w_1\cdot\fbrk{\brak{1+\fun{C_{f,\smbox{best}}}{h_i}}\cdot t_{\smbox{rem.}}/t_{f,\smbox{spent}}}\cdot b+\\
&w_2\cdot\fbrk{\fun{f_{f,\smbox{imp}}}{h_i}/t_{f,\smbox{spent}}}-w_3\cdot\fbrk{\fun{f_{f,\smbox{wrs}}}{h_i}/t_{f,\smbox{spent}}}+\\
&w_4\cdot\fbrk{\fun{f_{\smbox{imp}}}{h_i}/t_{\smbox{spent}}}-w_5\cdot\fbrk{\fun{f_{\smbox{wrs}}}{h_i}/t_{\smbox{spent}}}\\\\
b\isdefinedas{}&\krdelta{\exists h_j:\fun{C_{f,\smbox{best}}}{h_j}>0}
\end{align*}
Met $C_{\smbox{best}}$ het aantal globaal betere oplossingen die de heuristiek heeft gevonden, $f_{\smbox{imp}}$ en $f_{\smbox{wrs}}$ de totale verbetering en verslechtering die de heuristiek veroorzaakt heeft. $t_{\smbox{spent}}$ houdt de totale rekentijd van een specifieke heuristiek bij. Indien er een subscript $f$ bij de metrieken wordt geplaatst, gaat het om de metriek in de laatste fase.

\paragraph{}
Een logische stap naar parallellisatie is het doorsturen van van de componenten van de metriek en vervolgens een uitspraak doen op basis van meer gegevens. Dit wordt echter bemoeilijkt door het feit dat de fases niet synchroon verlopen en dit bovendien de semantiek van een fase onderuit zou halen: uitspraken doen over hoe goed de heuristieken werken op een set gelijkaardige populaties. De metriek bevat echter ook enkele componenten die niet minder afhankelijk zijn van de laatste fase. Daarom introduceren we twee nieuwe termen die een uitspraak doen over het globale plaatje:
\begin{align*}
\funm{eval'}{h_i}\isdefinedas{}&w_1\cdot\fbrk{\brak{1+\fun{C_{f,\smbox{best}}}{h_i}}\cdot t_{\smbox{rem.}}/t_{f,\smbox{spent}}}\cdot b+\\
&w_2\cdot\fbrk{\fun{f_{f,\smbox{imp}}}{h_i}/t_{f,\smbox{spent}}}-w_3\cdot\fbrk{\fun{f_{f,\smbox{wrs}}}{h_i}/t_{f,\smbox{spent}}}+\\
&w_4\cdot\fbrk{\fun{f_{\smbox{imp}}}{h_i}/t_{\smbox{spent}}}-w_5\cdot\fbrk{\fun{f_{\smbox{wrs}}}{h_i}/t_{\smbox{spent}}}\\
&w_6\cdot\fbrk{\fun{f_{g,\smbox{imp}}}{h_i}/p\cdot t_{g,\smbox{spent}}}-w_7\cdot\fbrk{\fun{f_{g,\smbox{wrs}}}{h_i}/p\cdot t_{g,\smbox{spent}}}
\end{align*}
Waarbij het subscript $g$ betekent dat het gaat over de som van de gegevens van alle processoren. De gegevens worden op geregelde tijdstippen doorgestuurd om minder bandbreedte en rekenwerk aan boekhoudkundige taken toe te wijzen.

\paragraph{}
Op basis van de evaluatie worden de heuristieken gerangschikt met een kwaliteitsindex. De heuristieken met een kwaliteitsindex die onder het gemiddelde ligt, worden uit voor een periode van $\sqrt{2\cdot n}$ uit de set verwijdert (met $n$ het aantal heuristieken). Heuristieken die tijdelijk niet meer tot de set behoren worden hebben allemaal een kwaliteitsindex van $1$. Indien een heuristiek de fase nadat deze terug in de set werd ge\"introduceerd opnieuw wordt verwijdert, neemt het aantal fases toe. Indien het aantal tabu-fases verdubbelt is, wordt de heuristiek definitief verwijdert.

\paragraph{}


\subsubsection{RH}
Naast \emph{ADHS} is \emph{RH} ook een mechanisme om heuristieken te selecteren. Per iteratie in de fase zal met stijgende kans dit mechanisme geactiveerd worden. Op basis van een \emph{learning automaton} wordt een heuristiek gesecteerd die wordt toegepast op de actieve oplossing. Elke heuristiek onderhoud een lijst met heuristieken die effectief bleken als tweede transitiefunctie. Met een bepaalde kans wordt een heuristiek uit deze lijst geselecteerd. In het andere geval wordt er een toevallige heuristiek geselecteerd. Die tweede heuristiek wordt dan toegepast op het resultaat van de eerste heuristiek. De \emph{learning automaton} gebruik een \emph{lineair reward-interaction update schema} waardoor combinaties die globaal betere oplossingen vinden meer kans maken in de volgende iteraties.

\paragraph{}
In het geval van een \emph{learning automaton}, is \emph{mimetism} een populaire oplossing: het nabootsen van de toestanden van de andere processen. Na elke fases stuurt het proces de verschillen in de kansvector door naar de andere processoren. Deze verschillen worden gedeeltelijk doorgerekend. Door de wijzigingen slechts gedeeltelijk door te rekenen, hopen we voorkomen dat de kansvectoren uit de hand kunnen lopen.

\subsubsection{AILLA}
Nadat \'e\'en van de twee mechanismes (\emph{ADHS} of \emph{RH}) een nieuwe oplossing heeft gegenereerd, zal de heuristiek beslissen of deze oplossing de nieuwe actieve oplossing wordt. Dit is de verantwoordelijkheid van \emph{AILLA}. \emph{AILLA} beschouwd twee verschillende gevallen:
\begin{enumerate}
 \item In het geval de gegenereerde oplossing beter is dan de originele oplossing, wordt deze altijd geaccepteerd.
 \item In het andere geval wordt de oplossing alleen geaccepteerd wanneer de fitness-waarde beter is dan de fitness-waarde een historisch beste oplossing. Hiervoor onderhoud \emph{AILLA} een lijst van de laatste globaal beste oplossingen. Het aantal maal tot nu toe na elkaar het accepteren van een oplossing werd geweigerd bepaald hoe diep er terug in het verleden wordt gekeken om een oplossing alsnog te accepteren.
\end{enumerate}

\importtikz[1.4]{ailla}{ailla}{Werkingsprincipe van \emph{AILLA}.}

\imgref{ailla} illustreert dit principe. Op de figuur worden de verschillende mogelijke configuraties voorgesteld door de horizontale as. De verticale as geeft de fitness-waarde van de overeenkomstige configuratie weer. Wanneer we reeds enkele oplossingen hebben overlopen, hebben we enkele fitness-waardes. Deze waardes worden door de dunne grijze horizontale lijnen voorgesteld. Een oplossing zal hier geaccepteerd worden wanneer deze zich onder de tweede horizontale lijn zit: het is immers al de tweede maal op rij dat we proberen een nieuwe oplossing te accepteren. %TODO: figuur aanpassen

\paragraph{}
We kunnen dit systeem verreiken door in de lijst van beste fitness-waardes ook de resultaten van andere processoren op te nemen. Hierdoor streven we naar sterkere stijgingen. Bovendien verwachten we dat dit op termijn gehaald zal worden: er worden immers ook oplossingen uitgewisseld waar andere processoren dan gebruik van kunnen maken. Door echter te strenge grenzen op te leggen kan een processor veel iteraties nodig hebben alvorens een nieuwe oplossing zal geaccepteerd worden. Daarom werd dit systeem aangepast zodat hooguit de helft van de lijst bestaat uit vreemde fitness-waardes.

\subsubsection{Andere componenten}

\paragraph{Kruisingsheuristieken}
In het geval \emph{AdapHH} een kruisingsoperator selecteert, dient men een andere oplossing te selecteren. In dat geval kiest \emph{AdapHH} een historisch beste oplossing. Dit hoeven niet noodzakelijk de laatste beste oplossingen te zijn. Telkens wanneer er immers een nieuwe beste oplossing wordt gevonden, wordt dit op een willekeurige plaats in een lijst met vaste grootte gezet. We kunnen het uitwisselingsmechanisme van \emph{ParHyFlex} uitbuiten: in plaats van uitsluitend te kiezen uit de lijst van lokale oplossingen, kunnen ook oplossingen die door een andere machine werden gegenereerd en doorgestuurd worden gebruikt.
\chapter{Resultaten en Speed-up}
\label{hoofdstuk:4}
S
\section{Probleemset}

\subsection{\prbm{Max3-Sat}}

\prbm{Max3-Sat} is een algemeen gekend probleem. Het is afgeleid van \prbm{Sat} waar gegeven een set Booleaanse variabelen en een set Booleaanse expressies met uitsluitend deze variabelen, we op zoek gaan naar een configuratie waardoor alle Booleaanse expressies waar zijn. \prbm{3Sat} beperkt de expressiviteit van de expressies tot een disjunctie van drie atomen\footnote{Een atoom stelt de waarde van een variabele zelf voor, of zijn negatie}.

\paragraph{}
We kunnen \prbm{3Sat} omvormen tot het optimalisatieprobleem \prbm{Max3-Sat} door het aantal falende expressies als de fitness-waarde voor een configuratie te defini\"eren.

\subsubsection{Motivatie}

\prbm{Sat} is een \comp{NP-compleet} probleem en bijgevolg kan men elk probleem in \comp{NP} omzetten naar \prbm{Sat}. Door de prestaties van \prbm{Sat} te analyseren, analyseren we dus impliciet een grote set van problemen die allemaal omgezet kunnen worden.

\paragraph{}
\prbm{Sat} is een probleem die geen directe connectie met een praktisch probleem heeft. Hierdoor zijn de heuristieken doorgaans vrij inherent. Hierdoor zijn resultaten ook minder het gevolg van de set van onderliggende heuristieken. Dit laat ons beter toe de prestaties van de parallelle hyperheuristiek te evalueren.

\paragraph{}
In het verleden is er al veel onderzoek verricht naar \prbm{Sat}. \cite{satDifficult} stelt bijvoorbeeld dat \prbm{SAT} problemen moeilijk op te lossen zijn wanneer de verhouding van het aantal beperkingen tegenover het aantal variabelen dicht bij $4.2$ ligt. Deze wetmatigheid lijkt te gelden voor elk algoritme die \prbm{Sat} problemen oplost. \cite{satHardness} maakten bovendien een studie die ook andere parameters onthulde die in grote mate de verwachtte rekentijd kunnen schatten.

\paragraph{}
Een laatste argument die we aanhalen in verband met \prbm{Sat} is de compactheid: moderne processoren hebben een instructie-set en geheugen-structuur waardoor \prbm{Sat}-problemen compact voor te stellen zijn en snel berekend kunnen worden. Hierdoor kunnen slechte resultaten niet verklaard worden door slechte serialisatie van de data over het netwerk.

\subsubsection{Implementatie}

In deze subsubsectie beschouwen we kort de verschillende ge\"implementeerde functies (transitie-functies, afstandsmetrieken,...) in het \prbm{Sat} probleem. Een oplossing wordt voorgesteld als een bitvector: elke variabele heeft een vaste index en de $i$-de bit in de vector bepaalt de overeenkomstige waarde van de $i$-de variabele.

\begin{itemize}
 \item Initialisatie ($I$):
\begin{description}
 \item [$I_1$] Genereert een willekeurige bitvector.
\end{description}
 \item Afstandsfuncties ($\delta$):
\begin{description}
 \item [$\delta_1$] De Hamming-afstand tussen de twee verschillende bitvectors.
 \item [$\delta_2$] Het aantal beperkingen waaraan in juist \'e\'en van de twee oplossingen voldoet.
\end{description}
 \item Mutaties (M):
\begin{description}
 \item [$M_1$] Het omwisselen van een willekeurige Booleaanse variabele.
 \item [$M_2$] 
 \item [$M_3$] 
\end{description}
 \item Local Search (LS):
\begin{description}
 \item [$LS_1$] Alle Booleaanse variabelen worden overlopen. Indien door het omwisselen van een variabele meer Booleaanse variabelen op \true{} worden gezet, wordt de variabele omgewisseld. Indien zo'n wissel plaatsvindt worden alle Booleaanse variabelen nogmaals overlopen.
 \item [$LS_2$] 
 \item [$LS_3$] 
\end{description}
 \item Ruin-Recreate (RR):
\begin{description}
 \item [$RR_1$] 
 \item [$RR_2$] 
 \item [$RR_3$] 
\end{description}
\item Crossover (CO):
\begin{description}
 \item [$CO_1$] Twee configuraties worden puntgewijs gerecombineerd: op basis van de fitness-waarde van de configuraties levert een configuratie met een bepaalde kans de waarde voor een variabele.
 \item [$CO_2$] 
\end{description}
\item Objectieven (O):
\begin{description}
 \item [$O_1$] Het aantal beperkingen waaraan een oplossing voldoet.
 \item [$O_2$] Het aantal variabelen die op \true{} staat.
\end{description}
\item Hypothesegeneratoren (HG):
\begin{description}
 \item [$HG_1$] 
 \item [$HG_2$] 
\end{description}
\end{itemize}


\subsection{\prbm{Finite Domain Constraint Optimization Problem (FDCOP)}}

\prbm{Max3Sat} is een typisch probleem bij de optimalisatie in een eindig domein. In plaats van echter voor elk typisch probleem een eigen set van heuristieken, evaluatie-functies, afstandsfuncties,... te implementeren, kunnen we ook op zoek gaan naar algemene oplossingsmethode die elk eindigdomein optimalisatieprobleem kan uitvoeren.

\paragraph{}
Als basis hiervoor gebruiken we \emph{Logische specificatie}: in een logische programmeertaal specificeert men een set variabelen, de respectievelijke domeinen, onderlinge beperkingen en een set optimalisatiefuncties. Het is de bedoeling dat door dit logische programma te analyseren men tot de nodige componenten (heuristieken, afstandsfuncties,...) komt en vervolgens het probleem in \emph{ParHyFlex} oplost.

\paragraph{}
\emph{ECLiPSe} -- een uitbreiding op \emph{Prolog} -- bevat een bibliotheek \texttt{fd} waarmee optimalisatieproblemen ook kunnen gespecificeerd worden. Deze bibliotheek gebruikt echter \emph{backtracking} om het probleem op te lossen. Momenteel kan de implementatie van het \prbm{Finite Domain Constraint Optimization Problem} in \emph{ParHyFlex} eenvoudige problemen oplossen die hoofdzakelijk in \comp{P} liggen. Het is echter de bedoeling om dit pakket verder te ontwikkelen en expressiever te maken.

\subsubsection{Voorbeeld}

Om het concept meer concreet te maken geven we een voorbeeld. Onderstaand programma kan bijvoorbeeld zoeken naar een optimale combinatie van een rij getallen waarbij het gemiddelde van de lopende sommen minimaal is:

\begin{verbatim}
A in {2}u{3}u{5}u{8}u{13}u{21}
B in {2}u{3}u{5}u{8}u{13}u{21}
C in {2}u{3}u{5}u{8}u{13}u{21}
A #!= B
A #!= C
B #!= C
minimzing A + B + C
\end{verbatim}
Dit probleem kan opgelost worden met behulp van een \emph{gretig algoritme} die altijd het correcte antwoord levert. Naarmate de taal echter meer expressief wordt verwachten we ook praktische problemen te kunnen oplossen.

\subsubsection{Motivatie}

\subsubsection{Implementatie}
\chapter{Het laatste hoofdstuk}
\label{hoofdstuk:n}
Een hoofdstuk behandelt een samenhangend geheel dat min of meer op zichzelf
staat. Het is dan ook logisch dat het begint met een inleiding, namelijk
het gedeelte van de tekst dat je nu aan het lezen bent.

\section{Eerste onderwerp in dit hoofdstuk}
De inleidende informatie van dit onderwerp.

\subsection{Een item}
De bijbehorende tekst. Denk eraan om de paragrafen lang genoeg te maken en
de zinnen niet te lang.

Een paragraaf omvat een gedachtengang en bevat dus steeds een paar zinnen.
Een paragraaf die maar \'e\'en lijn lang is, is dus uit den boze.

\section{Tweede onderwerp in dit hoofdstuk}
Er zijn in een hoofdstuk verschillende onderwerpen. We zullen nu
veronderstellen dat dit het laatste onderwerp is.

\section{Besluit van dit hoofdstuk}
Als je in dit hoofdstuk tot belangrijke resultaten of besluiten gekomen
bent, dan is het ook logisch om het hoofdstuk af te ronden met een
overzicht ervan. Voor hoofdstukken zoals de inleiding en het
literatuuroverzicht is dit niet strikt nodig.

%%% Local Variables: 
%%% mode: latex
%%% TeX-master: "masterproef"
%%% End: 

\chapter{Besluit}
\label{besluit}

\chapterquote{Man sage nicht, das schwerste sei die Tat; I da hilft der Mut, der Augenblick, die Regung; I das schwerste dieser Welt ist der Entschluss.}{Franz Grillparzer}

\section{Conclusies}

\subsection{Metaheuristieken}
Metaheuristieken zijn een familie van algoritmen die 

\subsection{Hyperheuristieken}

Hyperheuristieken veralgemenen metaheuristieken door probleemonafhankelijk te werken.

\subsection{\emph{ParHyFlex} en \emph{ParAdapHH}}

\emph{ParHyFlex} 

\section{Potenti\"ele ontwikkelingen}

Vooral op het gebied van het \prbm{Finite Domain Costraint Optimization Problem (FDCOP)} zien we potentieel interessante ontwikkelingen. Er bestaan reeds verschillende pakketten die toelaten op optimalisatieproblemen in logische taal uit te drukken. We denken dan bijvoorbeeld aan \emph{ECLiPSe}. De meeste van deze bibliotheken werken met behulp van een \emph{branch-and-bound}-mechanisme. Dit mechanisme garandeert dat de optimale oplossing op termijn gevonden wordt, maar kan in de meeste gevallen complexe problemen niet in aanvaardbare tijd oplossen. Door hyperheuristieken in het proces te betrekken zien we enkele voordelen.

\paragraph{}
Hyperheuristieken leveren altijd een oplossing afleveren binnen een zekere termijn. Dit is ook mogelijk met het \emph{branch-and-bound} principe, maar dit mechanisme zal in een beperkte tijd meestal een zoekruimte van gelijkaardige oplossingen hebben afgezocht. De kans is groot dat hyperheuristieken in deze tijd tot betere oplossingen komen.

\paragraph{}
Daarnaast kan men een hyperheuristiek ook gebruiken als een vorm van \emph{preprocessing}. Door een hyperheuristiek eerst een benaderende oplossing te laten uitrekenen, wordt een strenge \emph{bound} bepaald. Het aantal \emph{backtracking}-stappen in het effectieve zoekproces kan hierdoor gereduceerd worden.

\paragraph{}
Het parallel uitrekenen van hyperheuristieken kan resulteren in een softwarepakket die optimalisatieproblemen gespecificeerd in een logische taal met uitrekent. Door dit in een parallelle context uit te voeren kan men een oplossing met een arbitraire fout in een arbitraire tijd uitrekenen.

\paragraph{}
Tot slot kan men ook een brug maken met \emph{energy complexity}\cite{Roy:2013:ECM:2422436.2422470,conf/icpp/KorthikantiAG11}: het berekenen met hoeveel processoren men het effici\"entst een kwalitatieve oplossing kan berekenen.

%%% Local Variables: 
%%% mode: latex
%%% TeX-master: "masterproef"
%%% End: 


% Indien er bijlagen zijn:
\appendixpage*          % indien gewenst
\appendix
\chapter{Communicatiemodel van \emph{ParHyFlex}}
\applab{a}

\chapterquote{The most important thing in communication is to hear what isn't being said.}{Peter F. Drucker}

Bij de uitwisseling van oplossingen en ervaring, het onderhandelen over een nieuwe zoekruimte en het uitwisselen van toestanden is communicatie vereist. \emph{ParHyFlex} gebruikt hiervoor twee vormen van communicatie: \emph{Message Passing Interface (MPI)} en \emph{User Datagram Protocol (UDP)}.

\subsection{Motivatie}

\subsubsection{\emph{Message Passing Interface (MPI)}}

\emph{MPI} is een standaard communicatieprotocol speciaal ontwikkeld voor parallelle algoritmen. Het omvat zowel directieven voor \emph{point-to-point} communicatie en \emph{collectieve} communicatie. Zowel de zender en de ontvanger kunnen kiezen om deze communicatie op een synchrone of asynchrone manier af te handelen.

\paragraph{}
Een groot voordeel van \emph{MPI} is dat er voor de meeste configuraties een implementatie beschikbaar is. Bovendien heeft men veel onderzoek ge\"investeerd in effici\"ente implementaties voor groepscommunicatie op verschillende netwerkstructuren (\emph{hypercube}, \emph{cycle},...). Er zijn verschillende netwerkkaarten ge\"implementeerd waar de \emph{MPI} commando's rechtstreeks in de hardware worden ge\"implementeerd en op die manier de processor ontlasten van een groot deel van de communicatie-aspecten.

\paragraph{}
\emph{MPI} legt weinig voorwaarden op inzake hoe de commando's ge\"implementeerd worden. De meeste implementaties werken op een manier die vergelijkbaar is met \emph{TCP} waarbij men meestal enkele optimalisaties inzake groepscommunicatie.

\paragraph{}
\emph{MPI} kent verschillende versies. %TODO

\subsubsection{\emph{User Datagraph Protocol (UDP)}}

\emph{UDP} is een protocol in de transportlaag die op een onbetrouwbare manier boodschappen doorstuurt. Boodschappen worden in pakketten doorgestuurd: sequenties aan bytes die op zichzelf staan.

\paragraph{}
Onbetrouwbare communicatie is vrij ongewoon in een parallelle context. De meeste parallelle algoritmen falen immers wanneer de resultaten die door andere processoren niet worden doorgestuurd. In het geval van metaheuristieken en hyperheuristieken is het echter niet noodzakelijk om informatie uit te wisselen: stel dat een oplossing niet wordt doorgestuurd, kan de potenti\"ele ontvanger nog altijd de oude oplossing gebruiken om verder te rekenen.

\paragraph{}
Werken met onbetrouwbare communicatie levert bovendien ook enkele voordelen op. Wanneer men moet verzekeren dat informatie wel degelijk de ontvanger bereikt, moet men een systeem implementeren waarbij de ontvanger een bericht terugstuurt dat de boodschap is aangekomen\footnote{De zogenaamde \emph{ACK}-pakketten.}. Heel wat implementaties zullen deze boodschappen effici\"ent proberen te implementeren. Toch zal men een deel van de beschikbare bandbreedte altijd gebruikt worden om de communicatie betrouwbaar te maken. Zeker in de context van een lokaal netwerk -- een configuratie waarbij een groot deel van de pakketten sowieso toekomt -- is dit een niet onbelangrijke kostprijs.

\paragraph{}
\emph{UDP} maakt ook het gebruik van \emph{multicast} pakketten eenvoudiger. Een \emph{multicast} pakket wordt naar meerdere ontvangers tegelijk gestuurd om zo de bandbreedte te sparen. Omdat betrouwbaarheid geen vereiste is, zal een pakket een constante kost teweegbrengen in het netwerk. In het geval we dit op een betrouwbare manier doen (bijvoorbeeld over het \emph{Transmission Control Protocol (TCP)}) moeten er bevestigingspakketten worden teruggestuurd die in totaal een kost teweegbrengen die schaalt met het aantal ontvangers. In het geval van \emph{TCP} werken we bovendien met een \emph{sliding window protocol}: slechts een deel van de fragmenten van een boodschap zijn tegelijk in omloop zijn. Indien \'e\'en ontvanger dus niet antwoordt -- bijvoorbeeld omdat deze op dat moment andere pakketten ontvangt -- kan dit ertoe leiden dat andere ontvangers geen verdere boodschappen meer ontvangen. \emph{Multicast} onder \emph{TCP} is bovendien geen sinecure\cite{dshp}: ontvangers moeten zichzelf eerst toevoegen aan een \emph{multicast group} om pakketten te ontvangen.

\paragraph{}
Een beperking aan \emph{UDP} is de pakketgrootte. Elk pakket heeft een maximale grote van 65'527 bytes aan data. Deze beperking is ingevoerd om te voorkomen dat een pakket lange tijd een communicatielijn kan opeisen waardoor andere entiteiten in het netwerk niet meer aan bod komen\cite{Tanen2003}. Vermits het niet zeker is dat een \emph{UDP} pakket wel degelijk op de bestemming toekomt, moet alle informatie dus in \'e\'en pakket worden opgeslagen. Het gevolg is dat sommige data niet uitwisselbaar is met behulp van \emph{UDP}. Door enkele eigenschappen van \emph{TCP} over te nemen in een nieuw protocol kan men dit probleem oplossen.

\section{Overzicht van de Communicatie}

In het algemeen beschouwen we volgende vormen van communicatie in \emph{ParHyFlex}:
\begin{enumerate}
 \item de probleeminstantie;
 \item geheugenconfiguratie;
 \item oplossingen;
 \item onderhandelingen over een nieuwe zoekruimte; en
 \item een deel van de toestand van de hyperheuristiek.
\end{enumerate}
in de volgende subsecties zullen we de verschillende vormen verder bespreken.

\subsubsection{De probleeminstantie en geheugenconfiguratie}

Wanneer \emph{ParHyFlex} wordt opgestart, zal \'e\'en van de processoren het optimalisatieprobleem inlezen. Het is de bedoeling dat het probleem vervolgens ook door de andere processoren wordt ingeladen. Hiervoor maken we gebruik van synchrone communicatie over \emph{MPI}. De probleeminstantie is immers een noodzakelijk deel van de data en als processoren reeds andere communicatie zouden aangaan met de processoren zijn de effecten oncontroleerbaar. Bovendien betreft het een eenmalige uitwisseling in het begin van het proces. Hierdoor zijn de communicatiekosten van minder belang.

\paragraph{}
Ook de geheugenconfiguratie wordt uitgewisseld. Hieronder verstaan we het aantal oplossingen die een processor tegelijk opslaat samen met enkele instellingen hoe oplossingen zullen worden gecommuniceerd. \emph{ParHyFlex} laat enkel toe dat de hyperheuristiek bij aanvang het geheugen juist afstelt. Daarom verloopt ook deze communicatie over een synchrone \emph{MPI} verbinding. De argumentatie is dezelfde als bij het uitwisselen van de probleeminstantie.

\subsubsection{Uitwisselen van oplossingen}

Het uitwisselen van oplossingen %TODO

\subsubsection{Onderhandelingen over een nieuwe zoekruimte}

Op geregelde tijdstippen vat \emph{ParHyFlex} een proces aan waarbij men over een nieuwe zoekruimte onderhandelt. Hiervoor dienen de verschillende processoren een voorstel voor hun eigen zoekruimte aan de andere processen mee te delen. Dit is een typisch voorbeeld van een \texttt{GatherAll}-operatie: een vorm van groepscommunicatie waarbij elke processor een deel van de data bezit. Op het einde van de operatie bezitten alle processoren alle delen. De meeste van \emph{MPI} bevatten geen implementatie voor een asynchrone \texttt{GatherAll}-operatie\footnote{De versies die dit niet ondersteunen zijn 1.0\cite{mpi10}, 1.3\cite{mpi13}, 2.0\cite{conf/europar/GeistGHLLSSS96,mpi20}, 2.1\cite{mpi21} en 2.2\cite{mpi22}. Versie 3.0\cite{mpi30} ondersteund dit commando wel.}, daarom hebben we deze zelf ge\"implementeerd. De details van deze implementatie staan in \secref{mpimod}.

\subsubsection{De toestand van de hyperheuristiek}

Een hyperheuristiek houdt meestal een toestand bij waarin hij bijvoorbeeld de prestaties van de heuristieken in opslaat. Daar de waarde van deze parameters sterk afhangt van het aantal gevallen waaruit deze data is opgebouwd, kan het interessant zijn deze data uit te wisselen: we doen dan immers een uitspraak op basis van meer gegevens. Voor deze taak werd een component ge\"implementeerd ter ondersteuning. De hyperheuristiek beschrijft in het begin welke data uitgewisseld moet worden en schuift de lokale data vervolgens in het systeem. Het systeem zal stuurt regelmatig de data rond waardoor de data van de andere systemen ook uitgelezen kan worden in het lokaal systeem. Ook hiervoor maken we gebruik van een asynchrone \texttt{gather all} implementatie (zie \secref{mpimod}).

\section{Aangepast \emph{MPI}-model: \emph{Non-blocking Gather All}}
\seclab{mpimod}

Het \texttt{GatherAll} commando zorgt voor de uitwisseling van gegevens tussen meerdere processoren. Voordat het commando wordt opgeroepen beschikt elke processor over een deel van de data. Na de operatie zitten bij alle processoren alle delen in het geheugen. Een \texttt{GatherAll} instructie kan men dus bekijken als een collectieve \texttt{Broadcast}: elke processor stuurt zijn deel van data naar alle andere processoren.

\paragraph{}De na\"ieve implementatie waarbij men inderdaad elke processor een \texttt{Broadcast} operatie laat ondernemen, vereist dat we $\bigoh{p\cdot\brak{p-1}/2}=\bigoh{p^2}$ berichten over het netwerkt sturen. Men kan stellen dat een voordeel van deze implementatie is dat de berichten in \bigoh{1} over het netwerk worden verstuurd. Als we echter de assumptie maken dat elke machine slechts \'e\'en bericht tegelijk kan ontvangen of de communicatielijnen de berichten sequentieel doorsturen, vereist deze operatie dus \bigoh{p} tijd.

\subsection{\texttt{GatherAll} met \bigoh{p\log p} berichten}
Een implementatie die minder berichten oplevert ordent de processoren in een \emph{hypercube}\cite[algoritme 4.7]{books/bc/KumarGGK94}. In het geval van $p$ processoren. Stel $d=\ceil{\log_2p}$, dan kunnen we de processoren ordenen in een $d$-dimensionale kubus. Elke processor heeft hierbij ofwel $d$ ofwel $d-1$ buren: processoren die slechts in \'e\'en dimensie van elkaar verschillen.

\paragraph{}
Processoren kunnen informatie uitwisselen met de buur van een bepaalde dimensie: zelf zend de processor alle data door waarover men op dat moment beschikt naar deze buur. Vermits de buur-relatie voor een specifieke dimensie symmetrisch is, zal deze buur ook alle data waarover hij beschikt doorsturen.

\paragraph{}
De totale tijdscomplexiteit van dit algoritme blijft minimaal \bigoh{p}: we sturen minder berichten door, maar de berichten zelf zijn langer. Een netwerk schaalt echter beter volgens de grootte van de pakketten dan het aantal pakketten. Bovendien verwerkt \emph{ParHyFlex} \'e\'en bericht per oproep naar een heuristiek. Door meer gegevens te groeperen in \'e\'en pakket wordt de \texttt{GatherAll}-instructie dus effici\"enter uitgevoerd.

\subsection{Algoritme}

\importtikz[1]{asynchronegatherall}{asynchronegatherall}{Werking van een \texttt{GatherAll} operatie op een \emph{HyperCube}.}
\paragraph{}
\imgref{asynchronegatherall} toont dat door incrementeel informatie uit te wisselen met de buur van een telkens hogere dimensie, na $d$ stappen alle processoren over alle informatie beschikken. Een formele beschrijving van dit algoritme staat in \algref{gatherallsequential}.

\importalgo{syncga}{GatherAll\cite{books/bc/KumarGGK94}.}{gatherallsequential}

\subsection{Asynchrone aspecten}

De implementatie in \algref{gatherallsequential} werkt met synchrone communicatie: processen worden geblokkeerd tot een succesvolle uitwisseling plaatsvind. Het algoritme die we willen implementeren zal met asynchrone communicatie werken. Men kan dit implementeren door dit proces bijvoorbeeld op een aparte \emph{thread} te laten werken: een proces brengt de data onder in de context van de \emph{thread} en werkt verder. Nadien wordt regelmatig gecontroleerd over de \emph{thread} al de nodige informatie heeft verzameld.

\importalgo{asyncga}{Asynchrone GatherAll.}{gatherallasync}

\paragraph{}
\emph{ParHyFlex} werkt met \'e\'en \emph{thread}. Ontvangen berichten worden verwerkt telkens wanneer een heuristiek is uitgevoerd. \algref{gatherallasync} is een aangepaste versie van het synchrone algoritme. Het algoritme werkt met twee lijsten: \dres en \ddim. \dres houdt de tot dusver ontvangen gegevens bij, deze kunnen dan later uitgelezen worden en verder uitgestuurd worden. $\ddim$ slaat per dimensie op of de bijbehorende buur reeds zijn deel van data heeft opgestuurd. $z$ bevat de kleinste dimensie waar we nog geen data naar hebben gestuurd. Wanneer we een bericht ontvangen van een buur, berekenen we waar deze data moet worden opgeslagen en markeren we het relevante item in de $\ddim$-lijst. Dit doen we ook wanneer we de lokale data aangeboden krijgen\footnote{Vermits het algoritme asynchroon verloopt kunnen er berichten vanuit de andere processoren worden gestuurd alvorens de lokale processor de \texttt{GatherAll}-instructie oproept.}.

\paragraph{}
Bij beide gebeurtenissen controleren we of we op dat moment zelf data kunnen uitsturen: de \pzedt{}-functie. Deze functie controleert per dimensie of we over voldoende data beschikken: dit wil zeggen dat alle buren met een lagere dimensie de data al hebben doorgestuurd. Vervolgens stellen we het pakket met de relevante data samen en wordt dit asynchroon verstuurd. De operatie is uitgevoerd wanneer we naar alle buren data hebben verstuurd en ontvangen hebben.

\paragraph{}
Een aantal problemen worden ge\"introduceerd door het asynchrone karakter: processoren kunnen al beginnen met het sturen van berichten, terwijl andere processoren nog niet tot het punt gekomen zijn waarop de \texttt{GatherAll} instructie begint. Omdat het proces in \'e\'en \emph{thread} werkt, moet het ontvangen bericht toch verwerkt worden. Indien men dus toch een bericht ontvangt zal het systeem zelf de datastructuren aanmaken en de berichten opslaan in afwachting van de \texttt{GatherAll}-instructie.

\paragraph{}
Indien meerdere \texttt{GatherAll} instructies door elkaar kunnen worden uitgevoerd is het niet duidelijk welk bericht voor welke \texttt{GatherAll}-instructie bedoelt is. \emph{MPI} laat toe om een \emph{tag} te plaatsen op een bericht en op die manier een betekenis aan het bericht te geven. Soms is het niet mogelijk om deze \emph{tags} te bepalen: indien we bijvoorbeeld niet weten welke \emph{tags} er op dat moment nog in gebruik zijn. In dat geval zal men eerst een \emph{tag} moeten bepalen of simpelweg niet toelaten dat er meerdere \texttt{GatherAll}-instructies tegelijk plaatsvinden.

\paragraph{}
Een voordeel van een asynchrone \texttt{GatherAll} instructie is dat men kan werken op gedeeltelijke data. Stel dat een parallel algoritme alle data op termijn nodig heeft, maar ook kan rekenen wanneer al een deel van de data beschikbaar is. Met behulp van groepscommunicatie reduceert men de impact op het netwerk. Het feit dat de communicatie asynchroon gebeurt laat dan weer toe om geregeld te testen of al een deel van de data werd ontvangen. Wanneer alle cruciale data beschikbaar is voor een deeltaak kan de processor verder rekenen.


\subsection{Onbetrouwbare communicatie}



%%% Local Variables: 
%%% mode: latex
%%% TeX-master: "masterproef"
%%% End: 

% ... en zo verder tot
\chapter{De laatste bijlage}
\label{app:n}
In de bijlagen vindt men de data terug die nuttig kunnen zijn voor de
lezer, maar die niet essentieel zijn om het betoog in de normale tekst te
kunnen volgen. Voorbeelden hiervan zijn bronbestanden,
configuratie-informatie, langdradige wiskundige afleidingen, enz.

\section{Lorem 20-24}
\lipsum[20-24]

\section{Lorem 25-27}
\lipsum[25-27]

%%% Local Variables: 
%%% mode: latex
%%% TeX-master: "masterproef"
%%% End: 


\backmatter
% Na de bijlagen plaatst men nog de bibliografie.
% Je kan de  standaard "abbrv" bibliografiestijl vervangen door een andere.
\bibliographystyle{abbrv}
\bibliography{referenties}

\end{document}

%%% Local Variables:
%%% mode: latex
%%% TeX-master: t
%%% End: