\documentclass[master=elt,masteroption=ge]{kulemt}
\usepackage{amssymb}
\newtheorem{definition}{Definitie}
\setup{title={Parallelle Hyperheuristieken},
  author={Willem Van Onsem},
  promotor={Prof.\,dr.\ Bart Demoen},
  assessor={Ir.\,W. Eetveel\and W. Eetrest},
  assistant={Ir.\ A.~Assistent \and D.~Vriend}}
% De volgende \setup mag verwijderd worden als geen fiche gewenst is.
\setup{filingcard,
  translatedtitle={The best master thesis ever},
  udc=681.3,
  shortabstract={Hier komt een heel bondig abstract van hooguit 500
    woorden. \LaTeX\ commando's mogen hier gebruikt worden. Blanco lijnen
    (of het commando \texttt{\string\pa r}) zijn wel niet toegelaten!
    \endgraf \lipsum[2]}}
% Verwijder de "%" op de volgende lijn als je de kaft wil afdrukken
%\setup{coverpageonly}
% Verwijder de "%" op de volgende lijn als je enkel de eerste pagina's wil
% afdrukken en de rest bv. via Word aanmaken.
%\setup{frontpagesonly}

% Kies de fonts voor de gewone tekst, bv. Latin Modern
\setup{font=lm}

% Hier kun je dan nog andere pakketten laden of eigen definities voorzien

% Tenslotte wordt hyperref gebruikt voor pdf bestanden.
% Dit mag verwijderd worden voor de af te drukken versie.
\usepackage[pdfusetitle,colorlinks,plainpages=false]{hyperref}

%%%%%%%
% Om wat tekst te genereren wordt hier het lipsum pakket gebruikt.
% Bij een echte masterproef heb je dit natuurlijk nooit nodig!
\IfFileExists{lipsum.sty}%
 {\usepackage{lipsum}\setlipsumdefault{11-13}}%
 {\newcommand{\lipsum}[1][11-13]{\par Hier komt wat tekst: lipsum ##1.\par}}
%%%%%%%

%\includeonly{hfdst-n}
\begin{document}

\begin{preface}
\chapterquote{Begegnet uns jemand, der uns Dank schuldig ist, gleich f\"alt uns ein. Wie oft k\"onnen wir jemand begegnen, dem wir Dank schuldig sind, ohne daran zu denken.}{Johann Wolfgang von Goethe}
  Bij deze wil ik iedereen bedanken die het afgelopen academiejaar aan mij gedacht heeft. In de eerste plaats omdat dit waarde hiervan niet opwoog tegen het denken aan iets anders. \cite{gastonVanCamp} stelt dat een mens in 1971 biologisch een waarde heeft van 180 frank ofwel 30.00 euro in 2013.
\end{preface}

\tableofcontents*

\begin{abstract}
  In dit \texttt{abstract} environment wordt een al dan niet uitgebreide
  samenvatting van het werk gegeven. De bedoeling is wel dat dit tot
  1~bladzijde beperkt blijft.

  \lipsum[1]
\end{abstract}

% Een lijst van figuren en tabellen is optioneel
%\listoffigures
%\listoftables
% Bij een beperkt aantal figuren en tabellen gebruik je liever het volgende:
\listoffiguresandtables
% De lijst van symbolen is eveneens optioneel.
% Deze lijst moet wel manueel aangemaakt worden, bv. als volgt:
\chapter{Lijst van afkortingen en symbolen}
\section*{Afkortingen}
\begin{flushleft}
  \renewcommand{\arraystretch}{1.1}
  \begin{tabularx}{\textwidth}{@{}p{22mm}X@{}}%TODO: MODIFIED!!!
    CHeSC	& Cross-domain Heuristic Search Challenge \\
    LLH		&Low-level heuristics \\
    MCHH-S	&Markov Chain Hyper-Heuristic\\
    GISS	&Generic Iterative Simulated Annealing Search\\
    DynILS	&Dynamic Iterated Local Search\\
    ACO-HH	&Ant Colony Optimization Hyper-Heuristic\\
    HAEA	&Hybrid Adaptive Evolutionary Algorithm\\
    ISEA	&Iterated Search by Evolutionary Algorithm\\
    EPH		&Evolutionary Programming Hyper-heuristic\\
    VNS		&Variable Neighborhood Search\\
    CBM		&Coalition Based Metaheuristic\\
    ACO		&Ant Colony Optimization\\
    SA		&Simulated Annealing\\
    GA		&Genetic Algorithm\\
    LS		&Local Search\\
    ILS		&Iterated Local Search\\
    VNS		&Variable Neighborhood Search\\
    LMS		&Local Search Metaheuristics
  \end{tabularx}
\end{flushleft}
\section*{Symbolen}
\begin{flushleft}
  \renewcommand{\arraystretch}{1.1}
  \begin{tabularx}{\textwidth}{@{}p{22mm}X@{}}
    $\BoolSet$					& De verzameling van Booleaanse waarden: $\mathbb{B}=\left\{\mbox{\textbf{true}},\mbox{\textbf{false}}\right\}$. \\
    $\NatSet$					& De verzameling van natuurlijke getallen: $\mathbb{N}=\left\{0,1,2,\ldots\right\}$ \\
    $\zeromatrix[m\times n]$			& Een nulmatrix met dimensies $m\times n$\\
    $\onematrix[m\times n]$			& Een matrix vol enen met dimensies $m\times n$\\
    $\identitymatrix[m\times n]$		& Een identiteitsmatrix met dimensies $m\times n$\\
    $\RealSet$					& De verzameling van re\"ele getallen\\
    $\mean[\fun{\calD}{x}]{\fun{f}{x}}$	& Het gemiddelde van een functie $f$ volgens een verdeling $\calD$\\%: $\mean[\fun{\calD}{x}]{\fun{f}{x}}=\displaystyle\int\fun{f}{x}\fun{\calD}{x}\ dx$\\
    $\Prob{e}$					& De kans op een gebeurtenis $e$\\
    $\OpProblem$				& Optimalisatieprobleem\\
    $\ConfigSet$				& De verzameling van mogelijke configuraties\\
    $\ConfigOpSet$				& De set van de globaal optimale oplossingen\\
    $\ConfigValSet$				& De verzameling van geldige configuraties: \\
    $\sol$					& Een globaal optimale oplossing: $\sol\in\ConfigSet$\\
    $\bestSol$					& Een globaal optimale oplossing: $\xstar\in\ConfigOpSet$\\
    $\SolSet$					& Een oplossingsruimte in de context van een metaheuristiek. $S\subseteq X$\\
    $\PopSet$					& Een verzameling oplossingen ofwel \emph{populatie}\\
    $\hcfun$					& Harde beperkingen: $\funsig{\hcfun}{\ConfigSet}{\BoolSet}$\\
    $\evalfun$					& Evaluatiefunctie: $\funsig{\evalfun}{\ConfigSet}{\RealSet}$\\
    $\hittime$					& De raaktijd voor een optimalisatieprobleem in een sequenti\"ele context\\
    $\phittime$					& De raaktijd voor een optimalisatieprobleem in een parallelle context\\
    $\neighbr$					& Omgeving van een oplossing\\
    $\ev[i,A]$					& De $i$-de eigenwaarde van een matrix $A$. De eigenwaardes zijn gerangschikt van groot naar klein: $\abs{\ev[i,A]}\geq\abs{\ev[i+1,A]}$\\
    $\ev[A]$					& De dominante eigenwaarde van een matrix $A$\\
    $\evl[i,A]$					& De linkse eigenvector die bij de $i$-de eigenwaarde van matrix $A$ hoort\\
    $\evr[i,A]$					& De rechtse eigenvector die bij de $i$-de eigenwaarde van matrix $A$ hoort\\
    $\krdelta{x}$				& De Kr\"onecker-delta voor een gegeven Booleaanse expressie $x$\\
  \end{tabularx}
\end{flushleft}

% Nu begint de eigenlijke tekst
\mainmatter

\chapter{Inleiding}
\chplab{inleiding}
\chapterquote{If a man will begin with certainties, he shall end in doubts; but if he will be content to begin with doubts, he shall end in certainties.}{Francis Bacon}

Optimalisatie is een belangrijk onderwerp in de artifici\"ele intelligentie. Allereerst komen bij heel wat algemene problemen in de artifici\"ele intelligentie vormen van optimalisatie kijken. Het bepalen van de optimale parameters, het kortste pad of de kleinste beslissingsboom die een bepaalde fractie van de data juist classificeert zijn enkele voorbeelden.

\paragraph{}
Hoewel elke tak in de artifici\"ele intelligentie met deze problemen wordt geconfronteerd, bestaat er een specifieke tak die zich focust op deze problemen: ``Operationeel Onderzoek'' ofwel ``Operational Research''. Problemen die men tracht op te lossen zijn doorgaans moeilijk van aard: er is sprake van een groot aantal potenti\"ele oplossingen, het evalueren van een mogelijke oplossing is geen sinecure en de regels aan dewelke een oplossing moet voldoen zijn vrij complex.

\section{Operationeel Onderzoek}

Operationeel onderzoek probeert doorgaans optimalisatieproblemen op te lossen in een zeer concrete en praktische setting. Belangrijke voorbeelden zijn bijvoorbeeld het genereren van productieplanningen, het optimaliseren van winstmarges,... Omdat de zoekruimte dermate groot is, is het vinden van de exacte oplossing meestal niet mogelijk, men stelt zich in de meeste gevallen dan ook tevreden met een benaderende oplossing.

\paragraph{}
De laatste jaren is er een trend naar telkens meer ge\"integreerde systemen: algoritmen die onafhankelijk van het concrete probleem toch een oplossing kunnen uitrekenen. Deze evolutie is vooral te wijten aan de kost die de ontwikkeling van benaderingsalgoritmen met zich meebrengen. Omdat ontwikkelingskosten meestal hoger liggen dan de kosten ten gevolge door rekentijd is het financieel interessant om te investeren in systemen die met een minimale inspanning kunnen worden ontwikkeld. Belangrijk hierbij is de bouw van componenten die men kan hergebruiken voor het effectief oplossen van andere problemen. Een belangrijke groep van dit soort systemen zijn \emph{Hyperheuristieken}.

\section{Parallelle algoritmen}

Het oplossen van problemen met probleemonafhankelijke algoritmen komt met een kostprijs: we verwachten dat op maat gemaakte algoritmen effici\"enter zullen werken en dus binnen een gegeven tijd betere oplossingen zullen voorstellen. Door te investeren in betere machines kan men dit verlies enigszins goedmaken. Men kan de kloksnelheid van een processor echter niet eindeloos opdrijven. Deze fysische beperking betekent bijgevolg een limiet op de resultaten die \'e\'en processor kan afleveren.

\paragraph{}
Parallelle algoritmen worden meestal ingezet wanneer de rekenkracht van \'e\'en processor tekortschiet. Door het programma op te splitsen in verschillende deelprogramma's die elk op \'e\'en processor draaien hoopt men het rekenvermogen te kunnen opdrijven. Deze trend is bovendien ook zichtbaar door de introductie van meerdere \emph{kernen} ofwel \emph{cores} in moderne processoren. Vertaald naar optimalisatieproblemen hopen we dus met behulp van parallelle algoritmen tot betere oplossingen te komen die sneller uitgerekend kunnen worden.

\section{Onderzoeksvraag}

In deze masterthesis onderzoeken we of het mogelijk is om de prestaties van hyperheuristieken te verbeteren door deze parallel of verschillende processoren te laten draaien. Een belangrijk deel in deze onderzoeksvraag is welke componenten hiertoe kunnen bijdragen.

\paragraph{}
Vermits hyperheuristieken probleemonafhankelijk werken, moeten ze door middel van ervaring leren hoe men het probleem kan oplossen. Door verschillende processen ervaring met elkaar te laten uitwisselen kan de tijd waarin men voornamelijk rekenkracht investeert in het opdoen van ervaring mogelijk worden verkort. Dit kan echter ook averechts werken: wanneer men te snel in een exploitatiefase komt, is het mogelijk dat men uitspraken doet op basis van een te kleine hoeveelheid ervaring. Bovendien kan men de de kennis die \'e\'en processor heeft opgedaan niet altijd zomaar overdragen naar de andere processoren.

\section{Gevolgde methodiek}

In een eerste fase werd de beschikbare literatuur geraadpleegd. Er is veel literatuur te vinden rond het parallelliseren van metaheuristieken. Parallelle hyperheuristieken en hyperheuristieken in het algemeen zijn echter een vrij nieuw domein. Er bestaan dan ook slechts enkele concrete implementaties en weinig onderzoeken in verband die het effect van de verschillende paradigma testen. De belangrijkste werken worden dan ook vermeld in \sscref{defparhyhe}.

\paragraph{}
Om meer inzicht te verwerven in de structuur van hyperheuristieken werden 16 verschillende implementaties onderzocht. De resultaten van dit onderzoek worden gerapporteerd in \chpref{chesc} en \appref{chesc}. Op basis van deze studie werden er enkele hypotheses naar voren geschoven die mogelijk verklaren waarom sommige hyperheuristieken beter werken dan anderen.

\paragraph{}
Op basis van de opgedane kennis werd een systeem ge\"implementeerd die het parallelliseren van hyperheuristieken ondersteund. Dit systeem wordt uitvoerig besproken in \chpref{parhyf}. Ook werd een concrete hyperheuristiek die uit de vermelde studie, \emph{AdapHH}, aangepast zodat deze op dit systeem kan werken.

\paragraph{}
Om de invloed van de verschillende componenten te testen, werden twee problemen ge\"implementeerd. Door vervolgens de parameters aan te passen of componenten artificieel uit te schakelen, kunnen we de invloed van deze componenten op de prestaties van de hyperheuristiek nagaan. De resultaten van dit onderzoek staan in \chpref{resul}.

\paragraph{}
Enkele activiteiten met betrekking tot deze thesis werden niet opgenomen in dit werk. Dit komt omdat deze activiteiten een te kleine relevante bijdrage leverden. Het gaat hier in de eerste plaats om de implementatie van een systeem die het implementeren van een hyperheuristiek op een gestructureerde manier toelaat. Een programmeur kan door verschillende modulaire componenten samen te nemen een eigen hyperheuristiek bouwen. De bedoeling van dit systeem is om op termijn meer inzicht te verwerven in de invloed van sommige componenten op de prestaties van een hyperheuristiek.

\paragraph{}
Daarnaast werd een visualisatiesysteem genaamd \emph{ParVis} ge\"implementeerd. Dit systeem toont de toestand waarin de verschillende processoren zich bevinden samen met de berichten die onderling verstuurd worden. De bedoeling van dit softwarepakket is om de werking van een parallel algoritme beter te kunnen begrijpen. De broncode van dit softwaresysteem is te downloaden op \url{http://goo.gl/YPuMr}. Het softwarepakket omvat ook een voorbeeld die de werking van een parallel \algo{Sum-Product}-algoritme en een \emph{asynchrone GatherAll}\footnote{Zie \secref{mpimod}.} illustreert.


\section{Structuur}
In \chpref{defi} defini\"eren we het probleemdomein en de concepten hieromtrent. We bespreken in dit hoofdstuk ook de relevante literatuur samen met de huidige stand van zaken.

\paragraph{}
\chpref{chesc} omvat een onderzoek naar sequenti\"ele hyperheuristieken. We onderzoeken 16 verschillende implementaties die in 2011 werden voorgesteld op een competitie. Op elk van de implementaties wordt kritiek geleverd en op het einde schuiven we enkele hypotheses naar voren waarom sommige hyperheuristieken beter presteren dan anderen. Een deel van deze hypotheses wordt ook verder beargumenteerd met empirisch onderzoek.

\paragraph{}
We bespreken het \emph{ParHyFlex}-systeem in \chpref{parhyf}. Dit systeem ondersteund de bouw van parallelle hyperheuristieken en werd gebouwd op basis van de literatuurstudie in \chpref{defi} en de hypotheses die voortkomen uit \chpref{chesc}. Het systeem omvat drie grote componenten: \emph{uitwisselen van ervaring}, \emph{opdoen van ervaring} en \emph{afbakenen van een zoekruimte}.

\paragraph{}
In \chpref{paradaphh} werken we een concrete hyperheuristiek uit. \emph{ParAdapHH} is een parallelle variant van \emph{AdapHH}, de winnaar van de competitie die we in \chpref{chesc} hebben bestudeerd. We bespreken de werking van \emph{AdapHH}, maken een analyse over de verschillende bronnen van parallellisatie samen met de uiteindelijke implementatie.

\paragraph{}
In \chpref{resul} testen we het systeem met behulp van twee problemen: \prbm{Max-3Sat} en het \prbm{Finite Domain Constraint Optimization Problem}. Er worden verschillende testresultaten voorgesteld die de invloed van bepaalde componenten en parameters onderzoeken.



%%% Local Variables: 
%%% mode: latex
%%% TeX-master: "masterproef"
%%% End: 

%%% ASPELL CHECK 2013-05-20
\chapter{Definities en State-of-the-Art}
\label{hoofdstuk:1}

In de inleiding hebben we de kort de verschillende concepten besproken die in deze thesis een belangrijke rol zullen spelen. In dit hoofdstuk gaan we hier dieper op in: we formaliseren de concepten in definities en geven een kort overzicht van de belangrijkste wetmatigheden rond deze concepten.

\section{Optimalisatieproblemen}

We beginnen deze sectie met een formele definitie van een optimalisatieprobleem:

\begin{definition}[Optimalisatieprobleem]%, harde beperkingen, evaluatiefunctie, fitness-waarde
Een optimalisatieprobleem $\Pi$ is een tuple $\Pi=\tupl{X=A_1\times A_2\times\ldots\times A_n,c,f}$ waarbij $X$ een verzameling is van een set configuraties voor $n$ variabelen, $c:X\rightarrow\BBB$ een afbeelding van zo'n configuratie naar een Booleaanse waarde, die bepaald of de configuratie voldoet aan de ``harde beperkingen''. $f:X\rightarrow\RRR$ stelt een evaluatiefunctie voor die bepaald in welke mate een configuratie wenselijk is. De waarde van de evaluatie van een configuratie \fun{f}{x} wordt ook wel de fitness-waarde genoemd.
\end{definition}
Bij een optimalisatieprobleem gaan we op zoek naar een configuratie $x\in X$ die aan de harde beperkingen voldoet en de evaluatiefunctie optimaliseert. Meestal maakt men het onderscheid tussen een minimalisatie en een maximalisatie. In deze thesis zullen we altijd we een bij een optimalisatieprobleem altijd streven naar een configuratie $x\in X$ met een zo laag mogelijk evaluatie \fun{f}{x}. We kunnen echter eenvoudig elk maximalisatieprobleem $\tupl{X,c,f}$ omzetten in een minimalisatieprobleem $\tupl{X,c,f'}$ met $f':X\rightarrow\RRR:x\mapsto-\fun{f}{x}$. Formeel zoeken we dus naar een configuratie \xstar{} die we het \emph{globaal optimum noemen}.

\begin{definition}[Globaal optimum $\xstar$]
Een globaal optimum voor een zoekprobleem $\Pi=\tupl{X,c,f}$ is een configuratie $\xstar$ waarbij:
\begin{equation}
\xstar=\displaystyle\argmin_{x\in X'}\fun{f}{x}\mbox{ met }X'=\accl{x|\forall x\in X:\fun{c}{x}=\true}
\end{equation}
\end{definition}

Het is niet ongewoon dat er verschillende configuraties zijn met een gelijkaardige fitness-waarde. Dit geldt tevens voor het globaal optimum. Daarom defini\"eren we ook een optimum-set $\Xop$: een set met alle configuraties met een minimale fitness-waarde voor het probleem.

\begin{definition}[Optimum-set $\Xop$]
Een optimum-set $\Xop$ voor een zoekprobleem $\Pi=\tupl{X,c,f}$ is een set van geldige configuraties $x\in X'$ waarvoor geldt:
\begin{equation}
\Xop=\accl{x|x\in X'\wedge\fun{f}{x}=\fun{f}{\xstar}}
\end{equation}
\end{definition}

\paragraph{}
In een algemeen geval kunnen de domeinen $A_i$ van de variabelen $x_i$ oneindig groot zijn en bijvoorbeeld $\RRR$ omvatten. Geen enkele machine met een eindig geheugen kan echter alle elementen uit een domein met oneindig veel elementen voorstellen. We zullen daarom altijd de domeinen $A_i$ als eindig beschouwen. In het geval het domein van een variabele in werkelijkheid oneindig is, discretiseren we dus dit domein en beperken we het aantal elementen met een onder- en bovengrens. Indien er door discretisatie fouten worden ge\"introduceerd, kunnen we deze oplossen door het domein fijner te discretiseren.
\paragraph{}
Vermits zowel de harde beperkingen $c$ als de evaluatiefunctie $f$ hier een ``\emph{blackbox}'' zijn, zullen we om $\xstar$ te berekenen, over een significant deel van de verzameling $X$ moeten itereren. We verwachten dus dat de tijdscomplexiteit om een dergelijke oplossing te vinden gelijk is aan:
\begin{equation}
\bigoh{\abs{X}}=\bigoh{\displaystyle\prod_{i=1}^n\abs{A_i}}
\end{equation}
Indien we de assumptie maken dat alle domeinen dezelfde zijn dan bekomen we:
\begin{equation}
\bigoh{\abs{X}}=\bigoh{\abs{A_1}^n}\mbox{ indien }\forall A_i,A_j: A_i=A_j
\end{equation}
We zien dus dat deze tijdscomplexiteit exponentieel stijgt met het aantal variabelen~$n$. Optimalisatieproblemen in het algemeen liggen dan ook in \comp{NP-hard}.

\subsection{Complexiteit van optimalisatieproblemen}

Sommige optimalisatieproblemen liggen in \comp{P}. \algo{Karmarkar's algoritme}\cite{linearProgrammingInP} bijvoorbeeld lost het \prbm{lineaire optimalisatie} probleem op in \bigoh{n^{3.5}L} met $n$ het aantal variabelen en $L$ de diepte van de discretisatie in bits. Dit komt omdat we beperkingen plaatsen op de vorm van de evaluatiefunctie $f$ en de harde beperkingen $c$. Bij lineair programmeren betekent dit dat de evaluatiefunctie kan geschreven worden als het inwendig product tussen de vector van de variabelen en een vector met constanten. Het harde beperkingen moeten voor te stellen zijn zodat wanneer we de vector met de variabelen vermenigvuldigen met een matrix met constante elementen, alle elementen in de resulterende vector kleiner zijn dan een andere vector met constante elementen. Ook andere optimalisatieproblemen zoals bijvoorbeeld \prbm{Maximum Flow} en \prbm{Minimum Spanning Tree} zijn problemen die met polynomiale algoritmen kunnen worden opgelost.

\paragraph{}
Toch is er weinig ruimte voor optimisme. Een logische veralgemening van \prbm{Lineaire optimalisatie} is immers \prbm{Kwadratische optimalisatie}. Onder sommige omstandigheden kunnen we dit probleem reduceren naar een geval van \prbm{Lineaire Optimalisatie}\cite{Kozlov1980223}, maar een algemeen \prbm{Kwadratisch optimalisatie} probleem ligt in \comp{NP-hard}\cite{qpInNP}. Ook andere bekende optimalisatieproblemen zoals \prbm{Travelling Salesman Problem (TSP)} en \prbm{Integer Programming (IP)} liggen in \comp{NP-hard}.

\paragraph{}
Tot slot dient men in de context van optimalisatieproblemen een kanttekening maken dat een polynomiaal algoritme meestal niet meteen impliceert dat dit ook op kleine gevallen sneller werkt dan zijn exponenti\"ele tegenhangers. Een populaire methode bij het oplossen van \prbm{Lineaire optimalisatie} is bijvoorbeeld het \algo{Simplex}-algoritme. Klee en Minty\cite{klee:1972} construeerden echter een een geval waarbij het algoritme exponentieel veel tijd vraagt. Toch is \algo{Simplex} in de meeste gevallen sneller dan \algo{Karmarkar's algoritme}.

\section{Heuristieken}

Hoewel de meeste optimalisatieproblemen \comp{NP-hard} zijn, is in een praktische context de configuratie met een optimale evaluatiefunctie net van cruciaal belang. Voor de meeste toepassingen is een configuratie die aan de harde beperkingen voldoet en de fitness-waarde van de echte oplossing benadert voldoende. In dat geval wordt meestal een heuristiek ge\"implementeerd:

\begin{definition}[Heuristiek]
Een heuristiek is een programma die gegeven een optimalisatieprobleem $\Pi=\tupl{X,c,f}$ een oplossing berekent $\xdot$ in een redelijke tijd. Doorgaans voldoet deze oplossing aan de harde beperkingen ($\fun{c}{\xdot}=\true$) en ligt de voorgestelde oplossing $\xdot$ niet ver van de werkelijke oplossing $\xstar$.
\end{definition}

Deze definitie blijft redelijk vaag en geeft dan ook veel ruimte voor interpretatie. Doorgaans verwachten we dat het algoritme stop in polynomiale tijd en in de meeste gevallen worden er ook beperkingen gezet op hoe ver de fitness-waarde \fun{f}{\xdot} mag afwijken van de optimale fitness-waarde \fun{f}{\xstar}, al zijn beide voorwaarden niet strikt noodzakelijk. Minsky\cite{minskyHeuristic} schrijf hierover:
\begin{quote}
``Hints'', ``suggestions'', or ``rules of thumb'', which only usually work are called heuristics. A program which works on such a basis is called a heuristic program. It is difficult to give a more precise definition of heuristic program - this is to be expected in the light of Turing's demonstration that there is no systematic procedure which can distinguish between algorithms (programs that always work) and programs that do not always work.
\end{quote}

% Een belangrijke theorema stelt dat fout onmogelijk constant kan zijn:
% \begin{theorem}
% Voor geen enkele heuristiek bestaat geen getal $E$ waarvoor geldt:
% \begin{quote}
% dat voor elke probleem-instantie van het probleem, $\fun{f}{\xdot}-\fun{f}{L}\leq L$.
% \end{quote}
% \end{theorem}


\section{Metaheuristieken}

\subsection{Problemen met heuristieken}

Heuristieken bieden een antwoord door een algoritme uit te voeren die in polynomiale tijd een oplossing zal uitrekenen. In het geval het algoritme snel genoeg is, en we kunnen leven met de garanties die dit algoritme biedt, is dit een acceptabele methode. In de meeste gevallen is dit echter niet zo. Praktische problemen zijn groot en complex en doorgaans kan een heuristiek weinig garanties bieden. Een ander probleem met heuristieken is dat ze weinig aanpasbaar zijn: stel dat we een bepaalde tijd ter beschikking stellen, dan zal het algoritme zich doorgaans niet aan deze beperking houden: ofwel loopt het algoritme eerder af en maakt het dus geen gebruik van alle beschikbaargestelde rekentijd, indien het algoritme niet afloopt binnen de gespecificeerde tijd is er ook geen sprake van een parti\"ele oplossing.

\subsection{Formele definitie}

Metaheuristieken proberen deze problemen op te lossen. We geven eerst een formele definitie van een metaheuristiek waarna we relevante terminologie invoeren.

\begin{definition}[Metaheuristiek]
\deflab{metaheuristic}
Een metaheuristiek is een algoritme die een oplossingsruimte $S\subseteq\accl{x|\forall x\in X:\fun{f}{x}}\subseteq X$ beschouwd met $\xstar\in S$. Verder beschouwd het \'e\'en of meer overgangsfunctiies $h_i:S^{k_i}\times\RRR^{l_i}\rightarrow S$. De metaheuristiek werkt door \'e\'en of meerdere instanties $s_1,s_2,\ldots,s_j$ uit $S$ te genereren. En vervolgens herhaaldelijk deze overgangsfuncties toe te passen op deze instanties. De resultaten van deze functietoepassingen kun gebruikt worden in andere toepassingen van de overgangsfuncties. Het algoritme stopt wanneer aan een bepaalde stopconditie voldaan is (bijvoorbeeld: het algoritme draait een bepaalde tijd op de machine). Hierna wordt de oplossing met de beste fitness-waarde als uitvoer teruggegeven.
\end{definition}

Op basis van deze definitie kunnen we ook een algoritme op hoog niveau opstellen zoals beschreven in \algref{metaheuristicGeneral}.

\begin{algorithm}[H]
 \SetAlgoLined
 %\KwData{}
 %\KwResult{how to write algorithm with \LaTeX2e }
 Bereken een initi\"ele set van oplossingen $P_1$ en hun fitness-waarde\;
 $B_1\leftarrow\displaystyle\argmin_{x\in P_1}{\fun{f}{x}}$\;
 \Repeat{het stopcriterium is bereikt}{
  Genereer op basis van $P_t$ en $t$ stochastisch een nieuwe set oplossingen $P_{t+1}$ samen met hun fitness-waarde\;
  $B_{t+1}\leftarrow\displaystyle\argmin_{x\in P_1\cup\accl{B_t}}{\fun{f}{x}}$\;
  $t\leftarrow t+1$\;
 }
 \KwRet{$B_t$}
 \caption{Hoog niveau beschrijving van een metaheuristiek\cite{DBLP:journals/jc/ShonkwilerV94}.}
 \alglab{metaheuristicGeneral}
\end{algorithm}

Typisch aan metaheuristieken is een (sterke) aanwezigheid van toevalsgetallen. Zo hebben we bij de signatuur van de overgangsfuncties ook een set re\"ele getallen opgenomen. Deze getallen worden door het toeval gegenereerd en bepalen mee het resultaat van de functie. Een belangrijk concept die we hiermee kunnen introduceren is de omgeving ofwel ``\emph{neighborhood}'' van een overgangsfunctie:

\begin{definition}[Omgeving van een overgangsfunctie]
De omgeving van een overgangsfunctie is de verzameling van alle oplossingen die we kunnen genereren vanuit een gegeven set van oplossingen ongeacht de waarde van de functie. De omgeving van een functie $h_i$ is dus:
\begin{equation}
\fun{\mathcal{N}_{h_i}}{s_1,s_2,\ldots,s_{k_i}}=\accl{s|s=\fun{h_i}{s_1,s_2,\ldots,s_{k_i},\xi_1,\xi_2,\ldots,\xi_{l_i}}}
\end{equation}
\end{definition}

\subsubsection{Local Search}
Het concept van een omgeving is belangrijke omdat het meteen ook een populaire zoekstrategie voor metaheuristieken introduceert: \emph{lokaal zoeken} ofwel ``\emph{local search (LS)}''. Deze zoekstrategie vertrekt van een gegeven oplossing en zoekt - volgens een bepaalde omgeving - alle oplossingen in de buurt af op zoek naar een betere oplossing. In het geval we zo'n oplossing vinden wordt deze oplossing de nieuwe oplossing en beginnen we vervolgens een zoektocht rond de nieuwe oplossing. Local Search komt voor in twee smaken: ``\emph{first-improvement}'' en ``\emph{best-improvement}''. In het geval van \emph{first-improvement} wordt de eerste oplossing die beter is dan de huidige oplossing de nieuwe actieve oplossing. In het geval van \emph{best-improvement} doorzoeken we de volledige omgeving en migreren we naar de beste oplossing in de omgeving.
\paragraph{}
Local Search is een krachtige optimalisatietechniek die doorgaans tot acceptabele resultaten kan leiden. Het probleem zit hem in het idempotente karakter van local search: \'e\'enmaal we local search op een oplossing hebben toegepast heeft een tweede maal local search toepassen met dezelfde omgevings-definitie geen zin: we weten immers dat er in de omgeving geen betere oplossingen gevonden zullen worden want anders was het algoritme de vorige keer niet bij deze oplossing gestopt. Omwille van deze reden kan local search zelf geen volwaardige metaheuristiek worden genoemd. We verwachten immers dat als we meer tijd investeren, we op termijn altijd tot betere resultaten $\xdot$ ofwel de echte oplossing $\xstar$ zullen komen. Local search vormt echter de basis voor een groot aantal metaheuristieken die doorgaans een mechanisme implementeren om de oplossing uit het lokaal optimum te laten migreren.

\subsection{De ``Zoo van de Metaheuristieken''}

\defref{metaheuristic} is vrij abstract. Daarom zullen we in deze sectie de belangrijkste families van metaheuristieken kort beschouwen.

\subsubsection{Genetische Algoritmes}

\emph{Genetische algoritmen} ofwel ``\emph{Genetic Algorithms (GA)}'' zijn de eerste familie van metaheuristieken en werken op basis van een populatie: een set van oplossingen. Uit deze populatie worden twee of meer oplossingen geselecteerd. De selectie is meestal gerelateerd aan de fitness-waarde van deze oplossingen. Vervolgens past men een recombinatie-operator toe: een overgangsfunctie die twee of meer oplossingen als invoer neemt en op basis hiervan een nieuwe oplossing genereert die eigenschappen van alle ouders deelt. Doorgaans past men op deze oplossing ook een mutatie toe: men zal de oplossing lichtjes aanpassen door middel van een andere overgangsfunctie. Vervolgens de oplossing onder bepaalde voorwaarden in de populatie ge\"injecteerd. In ruil voor de nieuwe oplossing zal de populatie ook \'e\'en of meer oplossingen uit de populatie verwijderen.

\subsubsection{Simulated Annealing}

\emph{Simulated Annealing (SA)} is de eerste gepubliceerde metaheuristiek. Het is een schema gebaseerd op het \algo{Algoritme van Metropolis}. Het werkt aan de hand van \'e\'en overgangsfunctie en \'e\'en oplossing die soms de actieve oplossing wordt genoemd. Het algoritme voert telkens de overgangsfunctie uit op de actieve oplossing. De resulterende oplossing beschouwen we als de nieuwe oplossing met een kans:
\begin{equation}
\fun{p}{\mbox{accept $s_1\rightarrow s_2$}}=\fun{\min}{1,e^{\brak{\fun{f}{s_2}-\fun{f}{s_1}}/T}}
\end{equation}

Deze formule stelt dat indien de nieuwe oplossing beter is dan de oude oplossing, we altijd accepteren. Indien de nieuwe oplossing minder gunstig is accepteren we met een bepaalde kans die kleiner wordt naarmate het verschil in fitness-waarde groeit. In de formule staat ook een onbekende parameter $T$ ofwel \emph{temperatuur}. De temperatuur bepaalt hoe sterk de kans daalt naarmate de kloof groeit. Het is een variabele die initieel op een positief getal wordt gezet en gedurende het zoekproces langzaam naar 0 zakt. Dit betekent dat we in het begin sterk geneigd zijn om een slechtere oplossing te accepteren. Op het einde accepteren we bijna uitsluitend oplossing die beter zijn dan hun ouder. Hoe de temperatuur concreet evolueert kan vrij ingesteld worden en introduceert dan ook een groot aantal varianten.

\subsubsection{Iterated Local Search}

In de vorige sectie bespraken we \emph{Local Search}. Het probleem met \emph{Local Search} is echter dat het convergeert naar een lokaal minimum en vanaf dat moment geen vooruitgang meer kan maken. \emph{Iterated Local Search (ILS)} is een zoekproces dat afwisselt tussen twee fases: in de \emph{local search}-fase optimaliseert het programma de oplossing tot het in een lokaal optimum terecht komt; in de \emph{perturbatie}-fase voert men een overgangsfunctie uit die met een zekere kans in staat moet zijn om een oplossing te genereren die uit het lokale optimum ontsnapt.

\subsubsection{Variable Neighbourhood Search}

\emph{Variable Neighbourhood Search (VNS)} is een metaheuristiek die verder bouwt op \emph{Iterated Local Search}. In het geval van \emph{Variable Neighbourhood Search} is er echter sprake van verschillende definities voor een omgeving $\calN_i$ in een iteratie zullen in de \emph{shake}-fase migreren naar een random oplossing in de omgeving $\calN_i$ en vervolgens passen we \emph{local search} toe op basis van deze omgeving. In het geval we tot een betere oplossing komen, wordt de eerste omgeving $\calN_1$ opnieuw de actieve omgeving. In het andere geval kiezen we de volgende omgeving $\calN_{i+1}$. Op het moment dat alle omgevingen zijn uitgeput beginnen we ook opnieuw bij $\calN_1$.%in het geval we een lokaal optimum volgens omgeving $\calN_i$ bereiken, zullen we \emph{local search} toepassen op basis van de volgende omgeving $\calN_i$ in de hoop dat deze omgeving ons naar een betere oplossing zal leiden. In het geval alle 

\subsubsection{Tabu Search}

\emph{Tabu Search}

\subsection{Classificatie in de ``Zoo van de Metaheuristieken''}

\subsection{Modelleren van Metaheuristieken}

\paragraph{}
Traditioneel modelleert men een metaheuristiek aan de hand van een Markov-model. Een toestand stelt een oplossing voor en er vertrekken bogen uit de oplossing naar andere oplossingen indien een overgangsfunctie de ene oplossing in de andere kan omzetten. Elke boog kunnen we ook een label toekennen met de kans dat deze transitie zal plaatsvinden. In het geval de kansen niet veranderen naarmate de tijd vordert, speken we van een stationaire Markov-keten. In het ander geval is de keten niet-stationair.

\paragraph{}
Elke metaheuristiek die we hierboven beschreven hebben dient telkens drie belangrijke vragen in het achterhoofd te houden\cite{DBLP:journals/jc/ShonkwilerV94}:
\begin{enumerate}
 \item kan het globale optimum altijd gevonden worden door de metaheuristiek,
 \item hoe kunnen we identificeren dat we het globale optimum gevonden hebben, en
 \item hoe lang zal het duren alvorens we dit optimum gevonden hebben
\end{enumerate}

Wanneer we geen details kennen in verband met de vorm van de evaluatiefunctie $f$ spreekt het voor zich dat we enkel zeker kunnen zijn dat we een globaal optimum hebben gevonden door middel van \emph{exhaustive search}. Ook wat betreft de derde vraag verwachten we een \emph{exhaustive search} proces alvorens we zeker zijn dat we het optimum hebben gevonden. De verwachtte tijd alvorens dit gebeurt kan echter sterk verschillen van de tijd die we dienen te besteden aan het doorzoeken van een significant deel van de zoekruimte. Berekenen wanneer we gemiddeld dit optimum zullen bereiken staat bekend als het \prbm{Hitting Time Problem}. Om dit probleem te formaliseren dienen we eerst de notie van de raaktijd van een Markov-keten te defini\"eren:

\begin{definition}[Raaktijd $\theta$]
We defini\"eren de raaktijd $\theta$ van een Markov-keten $\calP$ als:
\begin{equation}
\fun{\theta}{\calP}=\min\accl{t|P_t\in\calP\wedge\Xop\cap P_t\neq\emptyset}
\end{equation}
\end{definition}

Het \prbm{Hitting Time Problem} gaat vervolgens op zoek naar de raaktijd van 

\paragraph{}
Op basis van \algref{metaheuristicGeneral} 

\section{Besluit van dit hoofdstuk}
Als je in dit hoofdstuk tot belangrijke resultaten of besluiten gekomen
bent, dan is het ook logisch om het hoofdstuk af te ronden met een
overzicht ervan. Voor hoofdstukken zoals de inleiding en het
literatuuroverzicht is dit niet strikt nodig.

%%% Local Variables: 
%%% mode: latex
%%% TeX-master: "masterproef"
%%% End: 

\chapter{Studie naar Sequenti\"ele Hyperheuristieken (CHeSC2011)}
\label{hoofdstuk:2}

Alvorens zelf hyperheuristieken te implementeren, is het interessant om te analyseren aan welke eigenschappen een goede hyperheuristiek moet voldoen. We verwachten immers dat bij het effici\"ent paralleliseren van deze algoritmen sommige wetmatigheden kunnen worden overgenomen. Vandaar deze studie naar \abseqe{} \abhhn{}.

\section{Implementatie-set}

\subsection{\abhf{}}

In 2010 publiceren \auth[5586064]{Burke et al.} een paper waarin ze \abhf{} voorstellen. \abhf{} is een klassenbibliotheek geschreven in \abjava{}. Het laat toe dat de gebruiker hyperheuristieken implementeert zonder dat het algoritme de details kent van het onderliggende probleem en de \abllhn{}. Het systeem werkt op basis van geheugen: een lijst waarin tussentijdse oplossingen worden opgeslagen en uitgelezen. Het systeem biedt vervolgens de mogelijkheid om een \abllh{} met een specifieke index toe te passen op de oplossing op een specifieke plaats in het geheugen en het resultaat op een specifieke plaats op te slaan. Daarnaast kunnen gebruikers ook de objectiviteitswaarde van een bepaalde oplossing in het geheugen opvragen en vragen of twee oplossingen in het geheugen equivalent zijn.

\paragraph{}
Het programma heeft dan wel geen idee wat de onderliggende \abllhn{} precies doen, ze worden wel gecategoriseerd in 4 verschillende types:
\begin{enumerate}
 \item \emph{\abmt{}}: hierbij wordt de oplossing op een toevallig manier aangepast. Dit soort heuristieken heeft geen componenten die inschatten of de verandering onmiddellijk of op termijn tot betere resultaten zal leiden.
 \item \emph{\abco{}}: dit zijn de enige heuristieken die twee oplossingen recombineren in een nieuwe oplossing. Het is uiteraard de bedoeling dat de nieuwe oplossing karakteristieken gemeen heeft met beide ``ouders''.
 \item \emph{\abrr{}}: deze heuristieken breken een deel van de oplossing af, om ze dan vervolgens met behulp van bijvoorbeeld een gretig algoritme terug op te bouwen.
 \item \emph{\abls{}}: dit is een familie van algoritmen die herhaaldelijk mutaties uitvoeren indien deze mutaties ook per stap winst opleveren. Indien geen enkele mutatie meer tot een beter resultaat leidt stopt het algoritme.
\end{enumerate}
Het softwaresysteem houdt ook de mogelijkheid open om een \abllh{} te classificeren onder ``other'', maar voor zover ons bekend zijn er nog geen heuristieken in \abhf{} geschreven die onder deze categorie vallen. Verder kan er verwarring optreden over het feitelijke verschil tussen \abrr{} en \abls{}. We kunnen immers bij beide families verwachten dat ze minstens een oplossing opleveren die beter is dan het origineel (\abrr{} zal immers tot in dat geval het afgebroken gedeelte reconstrueren zoals het origineel). Een verschil die men doorgaans maakt is dat \abls{} een operator is die idempotent is.

\subsection{\abchescy{}}

Om meer aandacht te vestigen op \abhf{} werd in 2011 een wedstrijd georganiseerd door de universiteit van Nottingham: de ``\emph{Cross-domain Heuristic Search Challenge (\abchescy)}''\cite{Burke:2011:CHS:2177360.2177415}. De verschillende programma's krijgen een set van verschillende problemen en worden gequoteerd op basis van de kwaliteit van de oplossingen die ze na 10 minuten uitvoer afleveren.%TODO(check)
De wedstrijd omvatte problemen uit zes verschillende domeinen: \prob{Maximal Satisfiability}, \prob{Bin Packing}, \prob{Personnel Scheduling}, \prob{Flow Shop}, \prob{Travelling Salesman Problem} en het \prob{Vehicle Routing Problem}.

\paragraph{}
In totaal telde de competitie 20 teams. We hebben met onze studie de zestien implementaties die gedocumenteerd werden bestudeerd. Tabel \ref{tbl:chescParticipants} bevat een lijst met de verschillende implementaties en geeft aan welke systemen in de studie opgenomen werden.

\begin{table}[hbt]
  \centering
  \begin{tabular}{rllrc} \toprule
    \#&Naam&Auteur/Team&Score&Bestudeerd\\\midrule
    1&	\emph{AdapHH}\cite{chesc-adaphh,chesc-adaphh2,348072}	&	Mustafa M\i{}s\i{}r&	181.00&	$\checkmark$\\
    2&	\emph{VNS-TW}\cite{chesc-vns-tw}&				Mathieu Larose&		134.00&	$\checkmark$\\
    3&	\emph{ML}\cite{chesc-ml,chesc-ml2}&				Mustafa M\i{}s\i{}r&	131.50&	$\checkmark$\\
    4&	\emph{PHUNTER}\cite{chesc-phunter}&				Fan Xue&		93.25&	$\checkmark$\\
    5&	\emph{EPH}\cite{chesc-eph}&					David Meignan&		89.75&	$\checkmark$\\
    6&	\emph{HAHA}&							Andreas Lehrbaum&	75.75&	\\
    7&	\emph{NAHH}&							MFranco Mascia&		75.00&	\\
    8&	\emph{ISEA}\cite{chesc-isea}&					Jiri Kubalik&		71.00&	$\checkmark$\\
    9&	\emph{KSATS-HH}\cite{chesc-ksats-hh}&				Kevin Sim&		66.50&	$\checkmark$\\
    10&	\emph{HAEA}\cite{chesc-haea}&					Jonatan Gomez&		53.50&	$\checkmark$\\
    11&	\emph{ACO-HH}\cite{chesc-aco-hh}&				Jos\'e Luis N\'u\~nez&	39.00&	$\checkmark$\\
    12&	\emph{GenHive}\cite{chesc-genhive}&				CS-PUT&			36.50&	$\checkmark$\\
    13&	\emph{DynILS}\cite{chesc-dynils}&				Mark Johnston&		27.00&	$\checkmark$\\
    14&	\emph{SA-ILS}&							He Jiang&		24.25&	\\
    15&	\emph{XCJ}&							Kamran Shafi&		22.50&	\\
    16&	\emph{AVEG-Nep}\cite{chesc-aveg-nep}&				Thommaso Urli&		21.00&	$\checkmark$\\
    17&	\emph{GISS}\cite{chesc-giss}&					Alberto Acu\~na&	16.75&	$\checkmark$\\
    18&	\emph{SelfSearch}\cite{chesc-selfsearch}&			Jawad Elomari&		7.00&	$\checkmark$\\
    19&	\emph{MCHH-S}\cite{chesc-mchh-s,conf/gecco/McClymontK11}&	Kent McClymont&		4.75&	$\checkmark$\\
    20&	\emph{Ant-Q}\cite{chesc-ant-q,sis/ant-q}&			Imen Khamassi&		0.00&	$\checkmark$\\
    \bottomrule
  \end{tabular}
  \caption{Deelnemers van de \abchescy{} competitie\cite{chesc-results}.}
  \label{tbl:chescParticipants}
\end{table}

we zullen de bestudeerde implementaties kort bespreken en vervolgens enkele hypotheses aanbrengen waaraan goed presterende hyperheuristieken waarschijnlijk dienen te voldoen.

\section{Implementaties}

\subsection{\emph{Ant-Q} (\#20)}
\label{sss:ant-q}
\subsubsection{Implementatie}
\emph{Ant-Q}\cite{chesc-ant-q,sis/ant-q} combineert ``\emph{ant-computing}''\cite{Michael:2009:AC:1596832.1596835} met ``\emph{Q-learning}''\cite{citeulike:5925674}. Dit doet men door een graaf te beschouwen waar de metaheuristieken de knopen voorstellen. De bogen bevatten een niet genormaliseerde kans om deze boog te nemen. Vervolgens voert men op een populatie oplossingen metaheuristieken uit door van knoop naar knoop te bewegen. De volgende knoop wordt gekozen op basis van een kansverdeling volgens de bogen die verbonden zijn met de oorspronkelijke knoop. Nadat de heuristiek is toegepast, wordt de kansverdeling van alle bogen die de oplossing tot dusver heeft gevolgd aangepast. De oplossing die op dat moment de beste is beloont alle bogen die hij gepasseerd heeft. Door de waarde van de bogen aan te passen zullen andere oplossingen meer geneigd zijn om een gelijkaardig pad te kiezen.
\subsubsection{Kritiek}
\begin{itemize}
 \item Het volledige pad van de winnende oplossing krijgt een bonus (ook bogen die hier helemaal niet te hebben bijgedragen)
 \item Wanneer de beste oplossing lange tijd hetzelfde is, krijgen de relevante bogen een grote bonus, hierdoor zit er na verloop van tijd nog weinig creativiteit in het systeem.
 \item Het type metaheuristiek speelt geen rol in het algoritme. Hierdoor is er een grote kans dat \abls{} heuristieken na verloop van tijd vaak worden toegepast. Dit leidt bovendien tot een stabiel systeem: \abls{} heuristieken zijn immers idempotent waardoor de beste oplossing dezelfde zal blijven.
\end{itemize}
\subsection{\emph{MCHH-S: Markov Chain Hyper-Heuristic} (\#19)}
\label{sss:mchh-s}
\subsubsection{Implementatie}
\emph{MCHH-S}\cite{chesc-mchh-s,conf/gecco/McClymontK11} werkt op een gelijkaardig manier aan Ant-Q (zie \ref{sss:ant-q}): men ontwerpt een graaf waar de knopen de metaheuristieken voorstellen en de bogen overgangen die men met een kans labelt. Het algoritme verschilt echter omdat er slechts \'e\'en oplossing in het netwerk rondwandelt. Daarnaast worden de kansen ook op een andere manier berekend: enkel de laatste boog verandert op basis van de onmiddellijke verandering van de fitness-waarde van de oplossing. Het algoritme verschilt ook omdat het niet telkens de nieuwe oplossing accepteert: alleen indien de oplossing beter is, of probabilistisch volgens het aantal opeenvolgende iteraties dat er nog geen betere oplossing werd gevonden.
\subsubsection{Kritiek}
\begin{itemize}
 \item Dit algoritme kan in een lokaal optimum terecht komen vermits \abls{} doorgaans voor de grootste winst zorgen en dus vaker beloond zullen worden. De idempotentie van \abls{} heuristieken zorgt er echter voor dat we veel rekenkracht verliezen met het herhaaldelijk toepassen van \abls{} heuristieken.
 \item Men maakt geen onderscheid tussen de verschillende types heuristieken: mutatie zal meestal tot een tijdelijk slechtere oplossing leiden. De bogen naar mutaties bestraffen is echter waarschijnlijk niet wenselijk. (net als bij \emph{Ant-Q}, zie \ref{sss:ant-q}).
\end{itemize}
\subsection{\emph{SelfSearch} (\#18)}
\label{sss:selfsearch}
\subsubsection{Implementatie}
\emph{SelfSearch}\cite{chesc-selfsearch} werkt met een populatie van oplossingen. Tijdens elke iteratie kiest men een \abllh{} die men vervolgens op alle oplossingen toepast. Hierdoor verdubbelt de populatiegrootte. Om terug op de originele populatiegrootte uit te komen selecteert men de beste unieke oplossingen. De keuze van de heuristiek gebeurt probabilistisch en op basis van twee strategie\"en: exploratie en exploitatie. Bij de exploratie krijgen heuristieken die een resultaat opleveren die verschilt van het origineel meer gewicht. Bij de exploitatie vooral heuristieken die in het verleden tot verbetering leidden.
\subsubsection{Kritiek}
\begin{itemize}
 \item Heeft de neiging in een lokaal optimum vast te raken: indien de populatie in een lokaal optimum zit, kan geen enkele mutatie die een tijdelijk slechtere oplossing levert de populatie terug uit dit lokaal optimum halen. De kans dat een volledige populatie in een lokaal optimum zit is uiteraard klein, maar desalniettemin kan dit algoritme resulteren in het toepassen van veel zinloze \abllh{}.
 \item Metaheuristieken die in het begin slecht presteren hebben meestal weinig kans op herintroductie: vermits ze na de initi\"ele fase minder gekozen worden, kunnen frequenter gekozen heuristieken een buffer van probabilistisch gewicht opbouwen.
 \item Er is slecht \'e\'en migratie van de exploitatie-fase naar de exploratie-fase. Indien deze fase op een fout moment gekozen wordt, is er geen weg terug.
\end{itemize}
\subsection{\emph{GISS: Generic Iterative Simulated Annealing Search} (\#17)}
\label{sss:giss}
\subsubsection{Implementatie}
\emph{GISS}\cite{chesc-giss} gebruikt \emph{Simulated Annealing}\cite{citeulike:1612433} als hyperheuristiek: bij elke iteratie kiest men uniform een beschikbare metaheuristiek die men toepast op het probleem. Vervolgens accepteert men deze oplossing volgens de procedure van simuated annealing. Crossover metaheuristieken worden toegepast op de laatste en voorlaatste oplossing. Indien er lange tijd geen verbetering zichtbaar is, wordt het systeem herstart vanaf een toevalsoplossing.
\subsubsection{Kritiek}
\begin{itemize}
 \item Uniforme selectie van \abllhn{} is waarschijnlijk niet interessant. Sommige \abllhn{} zijn immers nagenoeg overal beter dan anderen.
 \item \abco{} heuristieken toepassen tussen de laatste en de voorlaatste oplossing levert meestal weinig op, vermits de laatste gegenereerd is door een \abllh{} toe te passen op de voorlaatste.
 \item Er is geen overdracht van zoekervaring bij een (mogelijke) herstart. We kunnen echter verwachten dat we ook uit vorige rondes nuttige informatie kunnen leren.
 \item Er wordt opnieuw geen onderscheid gemaakt tussen het type van de \abllhn{}. Hierdoor verliest men mogelijk veel rekenkracht aan nutteloze operaties.
\end{itemize}
\subsection{\emph{AVEG-Nep: Reinforcement Learning Approach} (\#16)}
\label{sss:aveg-nep}
\subsubsection{Implementatie}
\emph{AVEG-Nep}\cite{chesc-aveg-nep} is gebaseerd op ``\emph{Reinforcement learning}''\cite{rlaiacaml}. Bij een reinforcement learning algoritme hebben we vijf componenten nodig:
\begin{enumerate}
 \item Een \emph{toestandsvoorstelling}:
\end{enumerate}
een toestandsvoorstelling, een set acties, een \emph{reward}-functie, een \emph{policy} en een \emph{learning} functie. Een simpele \emph{reward} functie is het verschil in fitness. In deze paper deelt men dit verschil ook door de tijd die het bekomen van dit verschil in beslag neemt. Een actie wordt voorgesteld door een tuple: enerzijds de familie waartoe de metaheuristiek behoort, anderzijds de waarde van de relevante parameters gekwantiseerd per 0.2. Indien een actie gekozen wordt kiezen we uniform een metaheuristiek die tot deze familie behoort en zetten we de parameters op de overeenkomstige waardes. Een toestand stelt het gewogen gemiddelde van het verschil in fitness-waarde over de tijd voor (dit gewogen gemiddelde wordt bereknt via \emph{exponential smoothing}). Als \emph{learning policy} wordt $\epsilon$-greedy 
gebruikt. Als \emph{learning} functie ten slotte gebruikt men ook een \emph{exponential smoothing} functie. Een laatste aspect is dat men 4 verschillende agenten tegelijk laat werken op de het probleem. De rede is dat men bij het toepassen van een crossover heuristiek een andere oplossing nodig heeft. In dat geval kiest men de oplossing van een andere agent om een crossover mee te realiseren.
\subsubsection{Kritiek}
\begin{itemize}
 \item Weinig semantiek in de keuze van metaheuristieken: tweemaal dezelfde \emph{local search} heuristiek met dezelfde parameters toepassen op eenzelfde oplossing levert niets op en leidt enkel tot tijdverlies.
 \item Geen onderscheid tussen metaheuristieken van dezelfde familie: terwijl de parameters een andere semantische betekenis kunnen hebben, en de \'en\'e heuristiek soms significant beter kan werken dan de andere.
 \item Semantiek van de toestand niet duidelijk: naarmate het algoritme vordert verwachten we een minder sterke groei van de fitness-functie. Hierdoor komen we in onverkende toestanden waardoor de exploitatie waarschijnlijk eerder laag is.
\end{itemize}
\subsection{\emph{DynILS: Dynamic Iterated Local Search} (\#13)}
\label{sss:dyn-ils}
\subsubsection{Implementatie}
Dynamic Iterated Local Search\cite{chesc-dynils,journals/orsnz/ksosils} is een implementatie met twee kleine wijzigingen: vermits de metaheuristieken parameters hebben, probeert het algoritme deze parameters te optimaliseren. Hiertoe houdt het een vector voor een aantal waardes van deze parameters. Deze vector wordt gebruikt om de kans uit te rekenen dat de overeenkomstige parameter-waarde geselecteerd wordt. Indien een metaheuristiek de oplossing verder verbeterd wordt de geprobeerde parameter-waarde beloond. Anders wordt deze bestraft. Een tweede aanpassing is de non-improvement bias: we verhogen de parameter evenredig met het aantal opeenvolgende iteraties waarin we de oplossing niet konden verbeteren. Door de parameter te verhogen zoeken we een groter gebied af met local search en muteren we de oplossing ook sterker. Hierdoor hoopt men uit een lokaal optimum te ontsnappen.
\subsubsection{Kritiek}
\begin{itemize}
 \item Selectie van de pertubatie nogal eenvoudig
 \item Vaste volgorde van het toepassen van local-search: hierdoor kan \'e\'en \emph{local-search} de andere blokkeren.
\end{itemize}
\subsection{\emph{GenHive: Genetic Hive Hyperheuristic} (\#12)}
\label{sss:genhive}
\subsubsection{Implementatie}
GenHive\cite{chesc-genhive} is een hyperheuristiek die werkt op basis van een genetisch algoritme. Een individu in dit genetische algoritme is een sequentie van metaheuristieken die op het probleem worden toegepast. Enkele van deze individuen zijn actief: ze worden toegekend aan een oplossing in de zoekruimte en worden er bij een iteratie op toegepast. Nadien worden de resultaten ge\"evalueerd. De beste strategie\"en blijven behouden. De overige worden passieve individuen. Men dient echter aan elke oplossing die beschouwd wordt een strategie toe te kennen. Hiertoe worden individuen die in de vorige iteratie passief waren gerecombineerd met de beste individuen.
\subsubsection{Kritiek}
\begin{itemize}
 \item Algoritme bevat veel parameters: populatiegrootte, aantal individuen actief, aantal individuen die na de iteratie behouden blijft,...
 \item Ineffici\"ent om altijd alle oplossingen verder te ontwikkelen in een iteratie,
 \item De beste individuen blijven aan dezelfde oplossingen gelinkt, terwijl men deze zou kunnen gebruiken om andere oplossingen ook significant te verbeteren.
 \item Redundante aspecten in een oplossing (een sequentie oproepen die nooit een beter resultaat kunnen genereren) wordt niet ge\"elimineerd: niet effici\"ent met tijdsgebruik
\end{itemize}
\subsection{\emph{ACO-HH: Ant Colony Optimization} (\#11)}
\label{sss:aco-hh}
\subsubsection{Implementatie}
ACO-HH\cite{chesc-aco-hh} maakt gebruik van de Ant-Colony Optimization\cite{hom/aco} techniek. Net als bij Ant-Q (zie \ref{sss:ant-q}) beschouwd men een grafe waarbij knopen metaheuristieken voorstellen. Een fundamenteel verschil is echter dat de grafe voorgesteld wordt als een tabel met $n$ kolommen en $H$ rijen (met $H$ het aantal metaheuristieken en $n$ een parameter genaamd de \emph{padlengte}). Elke metaheuristiek komt in deze grafe $n$ keer voor. Verder beschouwen we enkel bogen tussen twee verschillende kolommen. Het is de bedoeling dat de mieren een pad afleggen waarbij men $n$ keer een heuristiek kiest. Elke mier start met dezelfde initi\"ele oplossing. Wanneer deze in een knoop aankomt past hij de overeenkomstige metaheuristiek toe en kiest een nieuwe knoop op basis van de feromonen van de paden die naar de volgende kolom leiden. Wanneer alle mieren de laatste kolom bereikt hebben worden de resultaten ge\"evalueerd. De feromonen worden aangepast naargelang mieren die over 
dit pad hebben gewandeld tot betere/slechtere oplossingen komen. Ook dient men een nieuwe init\"ele oplossing te kiezen voor de mieren van de volgende fases. Doorgaans neemt men de beste oplossing over alle fases heen, tenzij de huidige fase deze oplossing heeft gegenereerd.
\subsubsection{Kritiek}
\begin{itemize}
 \item Neiging om in een lokaal optimum vast te zitten: indien een sequentie initieel veel vooruitgang boekt, zullen meer mieren dit pad kiezen. Het wordt echter moeilijk indien deze sequentie stagneert om een nieuw pad te bewandelen.
 \item Redudante aspecten in een sequentie worden niet noodzakelijk ge\"elimineerd: niet effici\"ent met tijdsgebruik. Men zou in de grafe \emph{nul-operaties} kunnen voorstellen. Een mier die in zo'n operatie terecht komt voert nadien geen operaties meer uit tot het einde van de iteratie.
\end{itemize}
\subsection{\emph{HAEA: Hybrid Adaptive Evolutionary Algorithm} (\#10)}
\label{sss:haea}
\subsubsection{Implementatie}
Dit algoritme\cite{chesc-haea,Gomez04selfadaptation} houdt telkens drie oplossingen bij: de ouder, het kind (een oplossing gegenereerd uit de ouder) en de beste oplossing tot nu toe. Daarnaast houdt men twee verzamelingen bij van heuristieken: \texttt{heu} houdt vier heuristieken bij uit de lijst van alle heuristieken. \texttt{loc} houdt vier heuristieken bij uit de lijst van hillclimbers. De heuristieken die in deze verzamelingen zitten zijn ``in use''. Dat betekent dat bij iedere iteratie we een algemene heuristiek kiezen uit \texttt{heu} en een hillclimber uit \texttt{loc}. Vervolgens passen we deze heuristieken na elkaar toe op  de ouder. Bij een crossover heuristiek combineren we de ouder met de beste oplossing. Indien dit kind beter is dan de ouder accepteren we het kind als nieuwe ouder en belonen we beide heuristieken. Dit doen we door de kans te verhogen dat ze de volgende maal opnieuw gekozen worden. Indien het kind niet beter presteert verlagen we de kans dat de 
heuristieken nogmaals gekozen worden. Indien na enkele iteraties er nog steeds geen verbetering is, kiezen we nieuwe elementen voor \texttt{heu} en \texttt{loc}.
\subsubsection{Kritiek}
\begin{itemize}
 \item Crossover wordt niet optimaal gebruikt: de kans is groot dat de huidige oplossing al dicht bij de beste oplossing zit.
 \item Beide metaheuristieken worden beloond terwijl de oorzaak van de verbetering eerder te wijten kan zijn door het combineren van de twee configuraties.
 \item Bij het kiezen van nieuwe sets wordt de opgedane kennis over metaheuristieken weggegooid.
\end{itemize}
\subsection{\emph{KSATS-HH: Simulated Annealing with Tabu Search} (\#9)}
\label{sss:ksats-hh}
\subsubsection{Implementatie}
KSATS-HH\cite{chesc-ksats-hh} is een implementatie die gebaseerd is op Si\-mu\-la\-ted-An\-nea\-ling: men houdt telkens een actieve oplossing bij. Na het toe\-passen van een metaheuristiek accepteert men het resultaat van de oplossing indien de oplossing beter is, of met een bepaalde kans (die exponentieel daalt naarmate het resultaat veel slechter is) een slechtere oplossing accepteert. Een intelligent aspect hierbij is dat men op basis van ervaring uit het verleden gebruikt om het verschil in fitness-waarde eerst te normaliseren (het verschil kan immers afhankelijk zijn van het probleem domein of de instantie). Het koelingsschema werkt ook exponentieel maar de factor waarmee men vermenigvuldigt verschilt in iedere tijdstap en hangt af van de het aantal iteraties die men in \'e\'en tijdseenheid weet te realiseren. De keuze van de heuristiek die wordt toegepast werkt op basis van Tabu-Search: het algoritme houdt een lijst bij van heuristieken. Heuristieken die erin slagen om de 
oplossing te verbeteren stijgen in de lijst. Heuristieken die daar niet in slagen dalen een plaats en worden tabu voor de volgende 7 iteraties. De uiteindelijke selectie van de heuristiek gebeurt door twee heuristieken uit de lijst te selecteren die niet tabu zijn. De heuristiek met die het hoogst in de lijst staat wordt dan gekozen.
\subsubsection{Kritiek}
\begin{itemize}
 \item Het systeem bestraft heuristieken voordat het algoritme begint: er zit een inherente orde in de lijst. Het element die als laatste geclassificeerd staat kan per toeval net de best presterende heuristiek zijn. Het duurt vrij lang voor deze een acceptabele selectie-kans krijgt.
\end{itemize}
\subsection{\emph{ISEA: Iterated Search by Evolutionary Algorithm} (\#8)}
\label{sss:isea}
\subsubsection{Implementatie}
Iterated Search by Evolutionary Algorithm\cite{chesc-isea} is een algoritme die gebaseerd is op het eerder gepubliceerde \emph{POEMS}\cite{eurogp06:KubalikFaigl}, een single candidate algoritme. Men probeert het op te lossen door het proces onder te verdelen en een sequentie van \emph{epochs}. In zo'n \emph{epoch} voert men een evolutief algoritme uit op prototypes. Een prototype is een sequentie van een variabel aantal metaheuristieken. Bij elk van deze metaheuristieken zijn ook de parameters reeds vastgezet. Tijdens een epoch wordt in een vast aantal iteraties een populatie van prototypes door een genetisch algoritme geoptimaliseerd. Nadien wordt het beste resultaat die met deze prototypes bereikt werd als nieuwe single candidate gebruikt in de volgende epoch. 
\subsubsection{Kritiek}
\begin{itemize}
 \item Redudante aspecten in een sequentie worden niet noodzakelijk ge\"elimineerd: niet effici\"ent met tijdsgebruik.
 \item Na een tijdstap wordt de opgedane ervaring weggegooid.
\end{itemize}
\subsection{\emph{EPH: Evolutionary Programming Hyper-heuristic} (\#5)}
\label{sss:eph}
\subsubsection{Implementatie}
EPH\cite{chesc-eph} werkt op basis van twee populaties: een populatie oplossingen en een populatie van sequenties van metaheuristieken. Beide populaties evolueren tegelijk. De populatie oplossing bestaat uit $N$ oplossingen die ``random'' ge\"initialiseerd worden. Telkens bij de evaluatie van de populatie van de sequenties worden er nieuwe oplossingen gegenereerd. Telkens nadat zo'n sequentie is toegepast op een oplossing. Zal men proberen dit resultaat in de populatie proberen in te brengen. Een oplossing wordt in een populatie ingebracht indien het een betere fitness waarde heeft dan minstens \'e\'en oplossing in de populatie en de fitness-waarde nog niet voorkomt. Ter compensatie wordt de slechtste oplossing uit de populatie gehaald. Een sequentie metaheuristieken bestaat uit twee delen: een pertubatie-gedeelte met een maximale lengte van twee, en een local-search gedeelte van variabele lengte. Tot de pertubatie behoren de mutatie, crossover en ruin-recreate. Naast de 
metaheuristiek die we toepassen bevat een sequentie ook informatie over de parameters. Het algoritme zet nog enkele extra beperkingen op sequenties: indien we twee pertubatie-heuristieken beschouwen dienen ze verschillend te zijn; een crossover heuristiek kan enkel op de eerste plaats staan. Bij het local search gedeelte wordt een heuristiek ofwel \'e\'enmaal ofwel volgens een Variable Neighborhood Descent schema\cite{hom/vns} uitgevoerd. De populatie van sequenties wordt initieel random bevolkt en evolueert doormiddel van mutatie en selectie. Hiervoor worden vier types mutaties gebruikt die met uniforme kans worden gekozen: modificeren van de pertubatie-parameters, modificeren van de local-search-parameters, verwijderen/toevoegen van een pertubatie, permutatie van de local-search heuristieken. Op elke sequentie wordt een mutatie toegepast. Daarna wordt via 2-tournament de populatie opnieuw gehalveerd: twee toevallig gekozen sequenties nemen het in enkele rondes tegen elkaar op. In een ronde worden ze op 
eenzelfde individu in de oplossingsverzameling toegepast. De sequentie die na de rondes het vaakst met de beste nieuwe oplossing komt, wordt geselecteerd in de nieuwe generatie van sequenties.
\subsubsection{Kritiek}
\begin{itemize}
 \item Redudante aspecten in een sequentie worden niet noodzakelijk ge\"elimineerd: niet effici\"ent met tijdsgebruik.
\end{itemize}
\subsection{\emph{PHUNTER: Pearl Hunter} (\#4)}
\label{sss:phunter}
\subsubsection{Implementatie}
Pearl Hunter\cite{chesc-phunter} is zoals de naam uitlegt gebaseerd op de jacht op parels en zeedieren. Dit proces kan het best uitgelegd worden als herhaalde diversificatie. Dit doen men door in de eerste plaats de metaheuristieken op te delen in twee categorie\"en: \emph{dives} die overeenkomen met de local-search heuristieken en \emph{surface moves} die de andere heuristieken omvat. Daarnaast deelt men \emph{dives} verder op in \emph{snorkling} en \emph{deep dives}. \emph{Snorkling} houdt in dat men een local-search algoritme met lage \emph{depth of search} uitvoert en stopt op het moment dat men een betere oplossing ontdekt. \emph{Deep dives} daarentegen zoeken met een hoge \emph{depth of search} en stoppen enkel wanneer er geen verbetering meer waargenomen wordt. PHunter combineert \emph{surface moves} en \emph{dives} in zogenaamde \emph{move-dive} iteraties. Bij zo'n \emph{move-dive} vertrekt men van enkele initi\"ele oplossingen. Vervolgens past men \emph{surface moves} toe en 
laat men de populatie groeien. Nadien ordent men de oplossingen. Enkel op interessante oplossingen wordt \emph{snorkling} toegepast. De beloftevolle oplossigen die uit deze \emph{snorkling}-fase komen, worden dan gebruikt als basis voor een \emph{deep dive}. Een \emph{deep dive} bestaat uit een sequentie van verschillende local-search metaheuristieken. Omdat de volgorde waarin deze heuristieken worden toegepast een belangrijke rol kan spelen worden verschillende volgordes parallel ge\"evalueerd. Na het \'e\'enmalig toepassen van zo'n sequentie worden de sequenties vergeleken. Enkel de beste sequentie wordt dan nog herhaaldelijk toegepast. Een laatste aspect die uniek is aan PHunter is \emph{surface learning}: men evalueert omgevingen op basis van het aantal local-search iteraties het kost in de \emph{snorkling}-fase om de oplossing te verbeteren. Afhankelijk van het aantal iteraties wordt een omgeving gecategoriseerd als \emph{shallow water}, \emph{sea trench} of \emph{buoy in the water}. Op basis van de 
omgeving wordt een strategie bepaald voor de \emph{surface moves} in de volgende iteratie. De vier mogelijke strategie\"en zijn: \emph{average calls}, \emph{crossover emphasized}, \emph{crossover only} en \emph{online pruning}. De strategie wordt gekozen op basis van een \emph{decision tree} die werd opgesteld door \emph{WEKA} voor een bepaalde testcase. Een laatste aspect is het \emph{mission restart} principe: indien voor een lange tijd geen betere oplossing wordt gevonden of de oplossingen in de populatie zijn door \emph{surface moves} te gelijklopend, begint men weer men een nieuwe initi\"ele oplossing.
\subsubsection{Kritiek}
\begin{itemize}
 \item \emph{Decision tree} lost het probleem met de parameters niet op: parameters zitten in de \emph{decision tree} (opgesteld op basis van een aantal testen op historische data).
 \item \emph{Mission restart} gooit alle opgedane ervaring weg.
\end{itemize}
\subsection{\emph{ML: Mathieu Larose} (\#3)}
\label{sss:ml}
\subsubsection{Implementatie}
ML\cite{chesc-ml} is gebaseerd op de meta-heuristiek \emph{Coalition Based Metaheuristic (CBM)}\cite{chesc-ml2}. In het \emph{CBM} algoritme, worden verschillende \emph{agents} gegroepeerd in een ``\emph{coalition}''. Agenten die tot dezelfde \emph{coalition} behoren, zullen vervolgens parallel de zoekruimte onderzoeken en onderweg ervaring uitwisselen over welke metaheuristieken moeten worden geselecteerd. Dit leerproces gebeurt aan de hand van re\"inforcement learning. Elke agent werkt verder met een \emph{Diversification-Intensification cycle}. In de diversification stap passen we een mutatie of ruin-recreate heuristiek toe. In de intensificatie cyclus passen we meerdere local-search heuristieken toe todat geen enkele local-search metaheuristiek de oplossing nog verder verbeterd. Daarna wordt beslist of de gevonden oplossing de nieuwe actieve oplossing van de agent wordt. De reinforcement learning probeert om een zo goed mogelijke zoekstrategie te ontwikkelen voor iedere agent. Hiertoe dienen we toestanden op te stellen (we proberen immers een metaheuristiek te kiezen in een bepaalde toestand). Als toestand worden voorwaardes gebruikt over welke metaheuristieken al zijn toegepast. Vervolgens probeert het leeralgoritme een matrix op te stellen die gewichten toekent die aangeven in welke situatie welke metaheuristiek waarschijnlijk het voordeligste is. De keuze van de metaheuristiek wordt dan uiteindelijk gemaakt via een roulettewheel-procedure op basis van de toestand waarin we ons op een gegeven moment bevinden. Naast het leren op basis van eigen ervaring wordt ook ervaring opgedaan door de matrices van andere agenten. Hervoor wordt aan mimetism learning gedaan: een agent probeert de matrix van een andere agent te imiteren door een lineare interpolatie tussen de eigen matrix en de vreemde matrix te nemen. Een agent zal enkel een nieuwe gewichtentabel leren wanneer hij tot een beter resultaat is gekomen in de laaste diversificatie-intensificatie cyclus. Verder zal een agent zal enkel zijn eigen matrix met anderen delen (die dus van deze matrix leren), als hij zelf tot een globaal beter resultaat komt.
\subsubsection{Kritiek}
\begin{itemize}
 \item Reinforcement learning introduceert nieuwe parameters, die natuurlijk intelligent moeten gekozen worden,
 \item Een goede metaheuristiek kan afhangen van de huidige oplossing. Deze wordt niet in rekening gebracht.
\end{itemize}
\subsection{\emph{VNS-TW: Variable Neighborhood Search-based} (\#2)}
\label{sss:vns-tw}
\subsubsection{Implementatie}
VNS-TW\cite{chesc-vns-tw} is zoals de naam doet vermoeden gebaseerd op Variable Neighborhood Search. Het is een populatie-gebaseerd algoritme en werkt op basis van vier stappen: shaking, local search, tabu en vervangen-selectie. Elke iteratie passen we toe op slechts \'e\'en oplossing. Aan het einde van de iteratie kunnen we eventueel van actieve oplossing veranderen. Bij stap 1 -- shaking -- gebruikt men een toevallig gekozen heuristiek uit de \emph{mutation} of \emph{ruin-recreate} heuristieken en past men deze toe op de actieve oplossing. Vervolgens zullen we in de local search een local search heuristiek kiezen en deze toepassen op het resultaat van de shaking. De heuristiek wordt gekozen op basis van rang: eerst kiezen uit een set heuristieken die nog niet gekozen zijn of een beter resultaat opleverden (rang 1), daarna uit een reeks heuristieken die de vorige keer een gelijkwaardige (maar niet gelijke) oplossing opleverden (rang 0). Indien geen enkele heuristiek meer aan deze 
voorwaarden voldoet, of we stellen na $c$ opeenvolgende pogingen geen verbetering vast, stoppen we de local search. In de volgende stap (tabu) zullen we op basis van de afgeleverde oplossing door de local search, de \emph{shaking} heuristieken aanpassen. Dit doen we via de Tabu-methode\cite{journals/heuristics/BurkeKS03}: indien het eindresultaat slechter is dan het orgineel komt de heuristiek in de tabu-lijst terecht. Indien we tot een gelijkaardig resultaat komen doen we dit in 20\% van de gevallen. Tot slot passen we de populatie aan: indien het resultaat beter is dan het orgineel passen vervangen we het orgineel. Indien het resultaat slechter is, vervangen we de slechtste oplossing in de populatie door het resultaat. We kiezen de nieuwe actieve oplossing op basis van een 2-tournament selectie uit de populatie.
\subsubsection{Kritiek}
\begin{itemize}
 \item Kleine kans dat het algoritme vastloopt op een lokaal optimum: vermits enkel de beste oplossingen worden geaccepteerd in de populatie, kan ontsnappen uit een lokaal optimum enkel gebeuren in de \emph{deversify-intesify-cycle}. Indien het systeem in een breed lokaal optimum terecht komt, is de kans klein dat het algoritme hier in 1 stap uit geraakt. De populatie laat echter niet toe dat we dit in 2 stappen doen.
\end{itemize}
\subsection{\emph{AdapHH: Adaptive Hyper-heuristic} (\#1)}
\label{sss:adaphh}
\subsubsection{Implementatie}
AdapHH\cite{chesc-adaphh,conf/lion/MisirVCB12} werkt met een populatie van metaheuristieken die de \emph{Adaptive Dynamic Heuristic Set (ADHS)} wordt genoemd. Het is de bedoeling dat deze set heuristieken zich aanpast zodat telkens de op dat moment interessantste metaheuristieken in de set zitten. Hiervoor deelt men de tijd op in fases. Heurstieken worden op het einde van een fase ge\"evalueerd met vier metrieken: het aantal maal dat men een tot dan toe beste oplossing aanreikte, de totale fitness verbetering, de totale fitness verslechtering en de gespendeerde tijd. Men combineert deze metrieken in een gewogen som. Op het einde van een fase berekent men voor iedere metriek de score. De slechtste helft van de metrieken wordt tijdelijk uit de set verwijdert volgens het principe van Tabu Search. Indien een metaheuristiek bij herintroductie onmiddellijk weer verwijdert wordt, wordt het aantal tabu-fases met \'e\'en verhoogt voor deze heuristiek. Indien deze teller een maximum bereikt 
wordt de heuristiek definitief geschrapt. Ook metaheuristieken die tijdens de fase geen betere oplossing vonden, maar wel in verhouding een lange tijd lopen worden tijdelijk op tabu geplaatst. Op het einde van een fase wordt ook een probabiliteitsvector berekent die de kansen op selectie in de volgende fase voorstelt. Hierbij is vooral het aantal maal een tot dan toe beste oplossing per tijdseenheid de belangrijkste factor. Elke metaheuristiek houdt ook een lijstje van 10 andere metaheuristieken bij die na de huidige metaheuristiek kunnen worden uitgevoerd. Afhankelijk van het resultaat van een heuristiek na een andere toe te passen, worden kans-waardes aangepast. Telkens wanneer een eerste metaheuristiek is geselecteerd, kiest het algoritme op basis van de kansen in de overeenkomstige lijst een tweede metaheuristiek. Indien de tweede metaheuristiek niet in de tabu-lijst voorkomt, passen we de tweede metaheuristiek toe na de eerste. Anders voeren enkel de eerste metaheuristiek uit. Telkens nadat we deze 
combinatie van metaheuristieken hebben toegepast dienen we het move acceptance systeem uit te voeren. Dit systeem wordt het \emph{Adaptive Iteration Limited List-based Threshold Accepting (AILLA) system} genoemd. Dit systeem maakt gebruik van de fitness-functie van de vorige beste oplossingen. Normaal worden enkel oplossingen geaccepteerd die de oplossing verbeteren. Indien dit niet het geval is kijken we naar het aantal iteraties dat we nog geen verbetering zien. Afhankelijk van de waarde van de teller accepteren we oplossingen met een fitness-waarde van de beste-oplossing van $t$ fases geleden. Parameters van metaheuristieken worden opgeslagen in het element van de populatie. Op basis van een reward-penalty systeem wordt naar de meest optimale combinatie gezocht.


\subsection{Conclusies}


%De ``\emph{Cross-domain Heuristic Search Challenge (CHeSC)}'' was een wedstrijd georganiseerd in 2011. De wedstrijd spitste zich toe op het ontwikkelen van hyperheuristieken in ``\abhyfl{}''\cite{hyflex2012,5586064}. \abhyfl{} is een klassenbibliotheek geschreven in Java 

\subsection{Een item}
Een tekst staat nooit alleen. Dit wil zeggen dat er zeker ook referenties
nodig zijn. Dit kan zowel naar on-line documenten\cite{wiki} als naar
boeken\cite{pratchett06:_good_omens}.

\section{Tabellen}
Tabellen kunnen gebruikt worden om informatie op een overzichtelijke te
groeperen. Een tabel is echter geen rekenblad! Vergelijk maar eens
tabel~\ref{tab:verkeerd} en tabel~\ref{tab:juist}. Welke tabel vind jij het
duidelijkst?

\begin{table}
  \centering
  \begin{tabular}{||l|lr||} \hline
    gnats     & gram      & \$13.65 \\ \cline{2-3}
              & each      & .01 \\ \hline
    gnu       & stuffed   & 92.50 \\ \cline{1-1} \cline{3-3}
    emu       &           & 33.33 \\ \hline
    armadillo & frozen    & 8.99 \\ \hline
  \end{tabular}
  \caption{Een tabel zoals het niet moet.}
  \label{tab:verkeerd}
\end{table}

\begin{table}
  \centering
  \begin{tabular}{@{}llr@{}} \toprule
    \multicolumn{2}{c}{Item} \\ \cmidrule(r){1-2}
    Animal    & Description & Price (\$)\\ \midrule
    Gnat      & per gram    & 13.65 \\
              & each        & 0.01 \\
    Gnu       & stuffed     & 92.50 \\
    Emu       & stuffed     & 33.33 \\
    Armadillo & frozen      & 8.99 \\ \bottomrule
  \end{tabular}
  \caption{Een tabel zoals het beter is.}
  \label{tab:juist}
\end{table}

\section{Lorem ipsum}
Tenslotte gaan we hier nog wat tekst voorzien zodat er minstens een
bijkomende bladzijde aangemaakt wordt. Dat geeft de gelegenheid om eens te
zien hoe de koptekst en de voettekst zich gedragen.

\section{Besluit van dit hoofdstuk}
Als je in dit hoofdstuk tot belangrijke resultaten of besluiten gekomen
bent, dan is het ook logisch om het hoofdstuk af te ronden met een
overzicht ervan. Voor hoofdstukken zoals de inleiding en het
literatuuroverzicht is dit niet strikt nodig.

%%% Local Variables: 
%%% mode: latex
%%% TeX-master: "masterproef"
%%% End: 

\chapter{\emph{ParHyFlex}: Parallel \emph{HyFlex}}
\label{hoofdstuk:3}

\chapterquote{A skilled transition team leader will set the general goals for a transition and then confer on the other team leaders working with him the power to implement those goals.}{Richard V. Allen}

Op basis van de analyse in hoofdstuk \ref{hoofdstuk:2}, werd een systeem genaamd \emph{ParHyFlex} ge\"implementeerd die het mogelijk maakt om hyperheuristieken te implementeren in een parallelle context. De broncode van dit systeem is te vinden onder \url{https://www.github.com/KommuSoft/ParallelHyFlex??} en wordt nog verder actief ontwikkeld. Daar het beschrijven van parallelle algoritmes complex is, geven we eerst een algemeen overzicht. Daarna bespreken we welke probleemafhankelijke componenten we hebben toegevoegd om dit systeem beter te doen werken. Vervolgens beschrijven we de principes achter het parallel uitvoeren van de hyperheuristieken. We eindigen met een bondig overzicht en bespreken welke inzichten uit het vorige hoofdstuk relevant waren voor de implementatie van dit systeem.

\section{Systeem}

\paragraph{}
Zoals al eerder aangehaald maakt een hyperheuristiek een keuze uit verschillende transitie-functies. Deze transitiefuncties kunnen parallel ge\"implementeerd worden. Zo kunnen we bijvoorbeeld \emph{Local Search} heuristieken over verschillende processoren verdelen. Een nadeel van deze %TODO

\paragraph{}
Zowel \emph{HyFlex} als \emph{ParHyFlex} maken een duidelijk onderscheid tussen enerzijds het probleemafhankelijke gedeelte (de kennis die een de gebruiker zelf dient te injecteren om het systeem te laten werken) en het probleemonafhankelijke gedeelte (de strategie die bepaalt welke heuristieken we uitvoeren).

\subsection{Probleemafhankelijk gedeelte}

We werken echter in een parallelle context. Daarom is het interessant dat het probleemafhankelijke gedeelte meer functionaliteiten ter beschikking stelt die uitgebuit kunnen worden door het bovenliggende systeem. Concreet denken we hierbij aan vier zaken: \emph{afstandmetrieken}, \emph{ervaring-generatoren}, \emph{zoekruimte beperkers} en \emph{multi-objectieven}

\subsubsection{Afstandmetrieken}
\abhf{} laat problemen een functie aanbieden die twee oplossingen met elkaar kan vergelijken. Deze functie kunnen we theoretisch omvormen tot een afstandsmetriek (we stellen de afstand tussen twee dezelfde oplossingen gelijk aan 0, en tussen twee verschillende aan een arbitraire constante groter dan 0). Deze metriek levert echter weinig informatie op. In een sequenti\"ele context gebruikt men soms het aantal mutaties die tussen een oplossing en \'e\'en van zijn voorouders om de afstand af te schatten. Dit is natuurlijk slechts een benadering. In het geval van parallelle uitvoer zullen we bovendien meestal niet over deze informatie beschikken. Daarom is het nuttig om de afstand tussen twee oplossingen te kunnen inschatten. In het geval de afstand geen triviaal gegeven is, kan men verschillende afstandsmetrieken defini\"eren en beslist de bovenliggende hyperheuristiek over de waarde van de metrieken. Een afstandmetriek is dus gedefinieerd als:
 \begin{equation}
  \delta_i:\SolSet^2\rightarrow\RealSet^+
 \end{equation}
 
\subsubsection{Ervaring-generatoren}
Elk proces draait een eigen hyperheuristiek en komt een sequentie heuristieken tegen. Uitwisselen van de sequentie kan potentieel een voordeel opleveren omdat de hyperheuristieken met meer kennis van zaken kunnen beslissen. Het doorsturen van alle heuristieken is doorgaans niet mogelijk omdat dit een te grote druk op het netwerk zet en bovendien de overige processoren te veel rekenkracht zouden investeren in het analyseren van de ontvangen oplossingen. Door het uitwisselen van ervaring, een compacte voorstelling van beschouwde oplossingen, zouden we dit probleem kunnen oplossen.

\subsubsection{Zoekruimte-beperkers}
Wanneer processoren oplossingen met elkaar uitwisselen lopen we de kans dat de verschillende processoren op termijn vergelijkbare populaties onderhouden. Dit laatste is nuttig wanneer sterke oplossingen in de buurt liggen van de oplossingen in de populatie. Indien de populaties echter rond eenzelfde lokaal optimum liggen, is dit nefast. In dat geval proberen alle processoren het lokale optimum te zoeken in een eenzelfde gebied, en wordt migratie naar mogelijk betere oplossingen in een ander gebied minder evident. Het introduceren van een component die diversificatie afdwingt kan helpen te voorkomen dat we op ijle populaties stuiten.

\subsubsection{Multi-objectieven}
Alle processoren proberen hetzelfde optimalisatieprobleem op te lossen. Door extra objectieven te introduceren, kunnen we echter een meer divers zoekproces aanbieden. Deze extra objectieven zijn eerder virtueel en dienen meer als een \emph{tie-breaker} in bijvoorbeeld gevallen waarbij twee oplossingen dezelfde fitness-waarde hebben.

\subsubsection{Afdwingbare beperkingen als probleemonafhankelijke ervaring}

Een probleem bij het genereren van \emph{ervaring} en het beperken van de \emph{zoekruimte} is dat dit op een probleemonafhankelijke manier dient te gebeuren: de bovenliggende hyperheuristiek heeft geen details over de structuur van de configuraties en kan bijgevolg niet zelf de zoekruimte beperken of conclusies genereren. We kunnen ervaring voorstellen als een object waar de hyperheuristiek de specificaties niet van kent, maar in dat geval moet ervaring wel enkele algemene functionaliteiten kunnen aanbieden die nuttig zijn. Om dit probleem op te lossen voeren we het concept van een \emph{afdwingbare beperking} in.

\begin{definition}[Afdwingbare beperking]
Een \emph{afdwingbare beperking} is een 3-tuple: $\tupl{c,c^+,c^-}$. $c:\SolSet\rightarrow\BoolSet$ is hierbij een functie die controleert of een gegeven oplossing aan een bepaalde voorwaarde voldoet. $c^+:\SolSet\rightarrow\SolSet$ is een functie die een gegeven oplossing minimaal kan aanpassen zodat deze aan de voorwaarde voldoet. $c^-:\SolSet\rightarrow\SolSet$ past oplossingen minimaal aan zodat ze niet aan de voorwaarde voldoen. De set van alle afdwingbare beperkingen die we op een probleem kunnen toepassen noteren we als $\HypSet$.
\end{definition}

We kunnen een afdwingbare beperking als een vorm van ervaring zijn. In de loop der tijd kunnen we immers een hypothese ontwikkelen dat sterke oplossingen aan een bepaalde voorwaarde voldoen (bijvoorbeeld een variabele in het probleem krijgt een vaste waarde). We kunnen dan oplossingen aantrekken naar de hypothese door $c^+$ op willekeurige oplossingen toe te passen. Anderzijds kunnen we ook oplossingen afstoten van de hypothese met $c^-$. De bovenliggende hyperheuristiek dient echter niet op de hoogte te zijn welke voorwaarden een concrete afdwingbare beperking stelt, zolang deze maar de oplossingen kan manipuleren.

\paragraph{}
Afdwingbare beperkingen kunnen we eveneens gebruiken om de zoekruimtes te beperken. Elk proces kan immers een aantal afdwingbare beperkingen gebruiken om een bepaalde zoekruimte te beschouwen, terwijl het de afdwingbare beperkingen van de andere processoren gebruikt om uit de buurt van de andere zoekruimtes te blijven.

\paragraph{}
De hyperheuristieken zelf kunnen geen ervaring genereren, ze hebben immers geen weet van de structuur van een oplossing. Daarom zal het specifieke probleem dus een set functies defini\"eren die we \emph{hypothese-generatoren (hypogen)} noemen:
\begin{definition}
Een \emph{hypothese-generator (hypogen)} $g_i:\SolSet^{n_{g_i}}\rightarrow\HypSet$ is een functie die op basis van een set oplossingen een afdwingbare beperking kan genereren.	
\end{definition}

\subsection{Probleemonafhankelijk gedeelte}

Het probleemonafhankelijke gedeelte wordt ge\"implementeerd door de hyperheuristiek en kan dus los gezien worden van \emph{ParHyFlex}. Het probleemafhankelijke gedeelte biedt echter functionaliteiten aan waarvoor we ondersteuning kunnen bieden in het probleemonafhankelijke gedeelte.

\paragraph{}
In \emph{ParHyFlex} werden daarom de volgende componenten ge\"implementeerd: \emph{uitwisselen van oplossingen}, \emph{afbakenen van zoekruimtes}, \emph{genereren van ervaring} en \emph{onderhandelen over een nieuwe zoekruimte}. In de volgende subsubsecties zullen we deze taken verder bespreken.

\subsubsection{Uitwisselen van oplossingen}

Elke processor werkt met een eigen lokaal geheugen, maar reserveert ook plaats voor de geheugens van de andere processoren. Op het moment dat een nieuwe oplossing naar een lokale geheugencel geschreven wordt, zal op basis van een \emph{uitwisselingsstrategie} beslist worden met welke processoren deze oplossing zal worden gedeeld. De taak van het verzenden en ontvangen van een oplossing samen met een reeks uitwisselingsstrategie\"en wordt ondersteund door \emph{ParHyFlex}.

\subsubsection{Afbakenen van de zoekruimte}
 
De zoekruimte bewaken is ook een verantwoordelijkheid van \emph{ParHyFlex}. Hiervoor voorziet men twee sets van afdwingbare beperkingen: positieve en negatieve. Telkens wanneer er een nieuwe oplossing wordt gegenereerd\footnote{Of via uitwisseling in het geheugen wordt ingeladen.} zal \emph{ParHyFlex} alle beperkingen in de positieve set afdwingen en \'e\'en beperking uit de negatieve set. Het afdwingen gebeurt in een willekeurige volgorde. Dit komt omdat de beperkingen met elkaar kunnen interfereren: een eerste beperking kan een variabele op \'e\'en waarde zetten waarna de volgende beperkingen deze wijziging weer ongedaan maakt. Men kan dit probleem proberen op te lossen door alle permutaties uit te proberen in de hoop dat \'e\'en mutatie toch tot het correcte resultaat leidt. Dit is echter niet noodzakelijk zo, en bovendien vereist een dergelijke oplossing exponenti\"ele tijd. We nemen aan dat de beperkingen meestal minimaal met elkaar interfereren en dat een zoekruimte niet strikt moet worden bewaakt. De hierboven vernoemde strategie is niet verplicht. Men kan door een interface te implementeren een andere strategie hanteren.

\subsubsection{Genereren van Ervaring}
 
Telkens wanneer \'e\'en van de processoren een nieuwe oplossing voortbrengt, kan hij deze oplossing -- samen met andere oplossingen -- omzetten in een afdwingbare beperking. Een processor kan echter niet alle beperkingen blijven bewaren: het uitwisselen van ervaring dient snel te gebeuren, we dienen een voldoende grote zoekruimte te behouden en bovendien kunnen we net een beperking genereren die het zoeken de foute kant opstuurt. Daarom maken we gebruik van een \emph{ervaring-set}, een set van vaste grootte waar gegenereerde beperkingen in worden bewaard. De elementen in de set worden telkens ge\"evalueerd: telkens wanneer er een nieuwe oplossing wordt gegenereerd, zal de \emph{ervaring-set} kijken aan welke beperkingen de oplossing voldoet. Op basis van de fitness-waarde van de oplossingen kunnen de beperkingen dan ge\"evalueerd worden. Door de lijst van fitness-waardes op te delen in waardes waarbij de beperking wordt gerespecteerd en waardes waarbij dat niet het geval is, ontstaan twee sets aan punten. Met een online algoritme\cite[p. 232]{citeulike:175026} berekenen we voor beide sets het gemiddelde en de variantie. Door de evaluaties van beide sets als onafhankelijke normale verdelingen te beschouwen, kunnen we de kans uitrekenen dat een fitness-waarde van een oplossing die aan de voorwaarde voldoet kleiner is dan de oplossing die niet aan de voorwaarde voldoet. Naarmate de kans groter wordt maken we de assumptie dat de beperking beter is. Omdat de gegenereerde beperkingen ook fout kunnen zijn, dienen we de set regelmatig van nieuwe hypotheses te voorzien, dit proces heet \emph{amnesie}. \emph{Amnesie} wordt op geregelde tijdstippen toegepast: oude hypotheses worden uit de set gehaald op plaats te maken voor nieuwe hypotheses. We wensen dat sterke hypothese meer kans maken om te overleven maar wel de kans lopen om te verdwijnen. Daarom rangschikken we de beperkingen op basis van hun evaluatie. De kans dat de hypothese vervolgens uit de set wordt gehaald berekenen we vervolgens op basis van de Benford-verdeling\cite{citeulike:748130}.

\subsubsection{Onderhandelen over een nieuwe zoekruimte}

Elke processor houdt een \emph{ervaring-set} bij. Het is de bedoeling dat deze ervaring wordt gebruikt om een nieuwe zoekruimte af te dwingen. Bovendien kan ervaring uitgewisseld worden met andere processoren zodat deze later ook hun zoekruimtes aanpassen. Tegelijk willen we voorkomen dat de zoekruimtes te homogeen worden en dus potentieel sterke oplossingen genegeerd worden. Dit zijn de taken van de \emph{onderhandelaar}. De \emph{onderhandelaar} is een component die af en toe geactiveerd wordt. Een deel van de afdwingbare beperkingen worden uit de \emph{ervaring-set} gehaald om opgenomen te worden in het positieve component van de \emph{zoekruimte}. Deze beperkingen worden via groepscommunicatie doorgestuurd naar de andere processoren. Een deel van de ontvangen beperkingen komen terecht in de \emph{ervaring-set} een andere deel vormt de basis van het negatieve gedeelte van de \emph{zoekruimte}. Omdat een deel van de afdwingbare beperkingen vanaf dan in de \emph{ervaring-set} van de andere processoren wordt ge\"evalueerd (met een andere zoekruimte), is men in staat om zo'n beperking op een objectievere manier te evalueren\footnote{Sommige afdwingbare beperkingen leiden immers enkel tot sterke resultaten in een bepaalde \emph{zoekruimte}.}.

\subsubsection{Invloed van hoofdstuk \ref{hoofdstuk:2}}

\subsection{Overzicht}

\importtikz[1.4]{parhyflexstructure}{parhyflexstructure}{Structuur van \emph{ParHyFlex}.}
Op \imgref{parhyflexstructure} geven we schematisch de structuur van \emph{ParHyFlex} weer.	De componenten die gemarkeerd worden met een asterisk, zijn component die niet aanwezig zijn in \emph{HyFlex}. Het blok in het grijs %TODO

\paragraph{}
Een deel van de geheugencellen is gemarkeerd met een schuine streep. Deze geheugencellen stellen vreemd geheugen voor waarvan er lokaal een kopie wordt bijgehouden. De geheugencellen kunnen uitgelezen worden, maar er kan geen oplossing naar geschreven worden.

\paragraph{}
\importtikz[1.4]{parhyflexwerking}{parhyflexwerking}{Schematische voorstelling van de kern van \emph{ParHyFlex}.}
Op \imgref{parhyflexwerking} beschrijven we kort het proces die een berekende of ontvangen oplossing doormaakt. Deze oplossing -- op de figuur $s_1^{(0)}$ -- wordt eerst aangepast door de zoekruimte: alle positieve hypotheses en \'e\'en negatieve hypothese worden toegepast op de oplossing en wordt aangepast tot $s_1^{(E)}$ die binnen de zoekruimte valt. De fitness-waarde wordt berekend en de evaluaties van de reeds aanwezige hypotheses in de ervaring-set worden aangepast (de data wordt voor elke hypothese opgenomen in \'e\'en van de twee normale verdelingen). Verder wordt met behulp van \'e\'en van de hypothesegeneratoren  een hypothese gegenereerd die met een bepaalde kans opgenomen wordt in de ervaring-set. De oplossing wordt vervolgens in het geheugen opgenomen en eventueel doorgestuurd naar andere processoren.

\paragraph{}
Op geregelde tijdstippen treed er amnesie op in de \emph{ervaring-set}: een deel van de hypothese worden uit de set verwijdert. Dit gebeurt op basis van de twee normale verdelingen per hypothese. Op die manier kan men zich ontdoen van foute hypothese, en maakt men ruimte voor nieuwe hypotheses.

\paragraph{}
Op vaste tijdsintervallen zal de \emph{onderhandelaar} een deel van de hypotheses uit de \emph{ervaring-set} halen. Een deel van deze hypotheses vormen de nieuwe positieve set van de \emph{zoekruimte}. De overige worden doorgestuurd in de \emph{ervaring-set} van de andere processoren ge\"injecteerd. Een deel van de doorgestuurde hypotheses vormt ook een basis van de negatieve set van de \emph{zoekruimte}.

\section{\emph{ParAdapHH}}

Naast het ontwikkelen van een systeem om hyperheuristieken op verschillende processoren te kunnen laten werken, vereist het testen van het systeem sowieso dat we een concrete hyperheuristiek ontwikkelen. Een logische keuze is om \emph{AdapHH}, de hyperheuristiek voorgesteld door Mustafa M\i{}s\i{}r te implementeren. We geven eerst een motivatie voor deze hyperheuristiek. Daarna bespreken we in meer detail de werking van de hyperheuristiek samen met de wijzigingen om het systeem op verschillende processoren te laten werken.

\subsection{Motiviatie}


\chapter{Resultaten en Speed-up}
\label{hoofdstuk:4}



\section{Probleemset}

\subsection{\prbm{Max3Sat}}

\prbm{Max3Sat} is een algemeen gekend probleem. Het is afgeleid van \prbm{Sat} waar gegeven een set Booleaanse variabelen en een set Booleaanse expressies met uitsluitend deze variabelen, we op zoek gaan naar een configuratie waardoor alle Booleaanse expressies waar zijn. \prbm{3Sat} beperkt de expressiviteit van de expressies tot een disjunctie van drie atomen\footnote{Een atoom stelt de waarde van een variabele zelf voor, of zijn negatie}.

\paragraph{}
We kunnen \prbm{3Sat} omvormen tot het optimalisatieprobleem \prbm{Max3Sat} door het aantal falende expressies als de fitness-waarde voor een configuratie te defini\"eren.

\subsection{Finite Domain Constraint Optimization Problem (\prbm{FDCOP})}
\include{hfdst-n}
\chapter{Besluit}
\label{besluit}

\chapterquote{Man sage nicht, das schwerste sei die Tat; I da hilft der Mut, der Augenblick, die Regung; I das schwerste dieser Welt ist der Entschluss.}{Franz Grillparzer}

In \secref{conclusions} worden de verschillende besluiten die doorheen deze thesis werden getrokken opgesomd. In \secref{potentials} overlopen we de verdere potenti\"ele ontwikkelingen van dit systeem.

\section{Besluiten}
\seclab{conclusions}

\subsection{Hyperheuristieken}

Hyperheuristieken zijn een familie van benaderingsalgoritmen die voor een grote aantal optimalisatieprobleem een acceptabel antwoord proberen te genereren. De algoritmen werken op basis van een gegeven set transitiefuncties en een iteratieve Monte-Carlo simulatie.

\paragraph{}
Het onderzoek naar hyperheuristieken is vrij recent en mist momenteel nog een theoretische basis: de prestaties van een concrete hyperheuristiek zijn erg afhankelijk van de de onderliggende set transitiefuncties.

\paragraph{}
Er bestaan enkele implementaties van \emph{parallelle hyperheuristieken}. De meeste implementaties werken volgens het \emph{master-slave} paradigma. De prestaties van de \emph{hyperheuristieken} inzake \emph{speed-up} zijn eerder wisselvallig.

\subsection{\emph{ParHyFlex}}

\emph{ParHyFlex} is een systeem die de implementatie van hyperheuristieken ondersteunt. Het probeert een platform aan te bieden waarbij zowel de probleemafhankelijke heuristieken als de hyperheuristiek in kwestie zich minimaal bewust zijn van het feit dat het algoritme parallel wordt uitgevoerd.

\paragraph{}
Het systeem is georganiseerd volgens het \emph{peer-to-peer} communicatie paradigma: er bestaat geen hi\"erarchische relaties tussen de verschillende processoren. Het \emph{Island Model}, een model bij genetische algoritmen vormt een inspiratiebron voor het systeem: oplossingen worden uitgewisseld met behulp van asynchrone communicatie via een \emph{uitwisselingsstrategie}.

\paragraph{}
Om meer processoren te richten op interessante gebieden wordt een concept genaamd \emph{afdwingbare beperkingen} gebruikt: een set voorwaarden die bij iedere oplossing kunnen worden afgedwongen. Op geregelde tijdstippen onderhandelen processoren over de actieve \emph{afdwingbare beperkingen}. Hierdoor kan men rekenkracht focussen op interessante gebieden en vermijden dat processoren te gelijkaardige oplossingen onderzoeken.

\paragraph{}
\emph{Afdwingbare beperkingen} laten ook toe om ervaring te genereren. Op basis van gegenereerde oplossingen kan met deze \emph{beperkingen} genereren en evaluatie gebeurt door een analyse te maken naar de kwaliteit van de oplossingen die wel of niet aan deze \emph{beperking} voldoen.

\subsection{\emph{ParAdapHH}}

\emph{ParAdapHH} is een parallelle variant van \emph{AdapHH}, een hyperheuristiek ge\"implementeerd door Mustafa M\i{}s\i{}r.

\paragraph{}
Het systeem probeert heuristieken beter te evalueren doordat elke processor de waarde van de bijbehorende metrieken doorstuurt. De samengestelde van al deze metrieken wordt vervolgens meegenomen in de evaluatie.

\paragraph{}
De \emph{learning automaton} in het systeem probeert ook te leren op basis van meer data. De verschillende gewichten van de \emph{learning automata} van de processoren worden uitgewisseld en elke \emph{learning automaton} beschouwt een interpolatie tussen de eigen gewichten en de vreemde gewichten.

\paragraph{}
Om een nieuwe oplossing te accepteren wordt deze vergeleken met een lijst van historisch beste fitness-waarden. Door een gedistribueerde lijst uit te wisselen, is men in staat om sterkere convergentie naar een optimale oplossing af te dwingen. Om te voorkomen dat een hyperheuristiek te lang geen oplossing accepteert, bestaat een vast deel van de historische waarden uit fitness-waarden van oplossingen die lokaal gegenereerd werden.

\subsection{Resultaten}

\section{Potenti\"ele ontwikkelingen}
\seclab{potentials}

Vooral op het gebied van het \prbm{Finite Domain Costraint Optimization Problem (FDCOP)} zien we potentieel interessante ontwikkelingen. Er bestaan reeds verschillende pakketten die toelaten op optimalisatieproblemen in logische taal uit te drukken. We denken dan bijvoorbeeld aan \emph{ECLiPSe}. De meeste van deze bibliotheken werken met behulp van een \emph{branch-and-bound}-mechanisme.

\paragraph{}
\emph{Branch-and-bound} garandeert dat de optimale oplossing op termijn gevonden wordt, maar kan voor de meeste complexe problemen geen sterke oplossing in aanvaardbare tijd vinden. Door hyperheuristieken in het proces te betrekken zien we enkele voordelen.

\paragraph{}
Hyperheuristieken leveren altijd een oplossing af binnen een zekere termijn. Dit is ook mogelijk met het \emph{branch-and-bound} mechanisme. Dit laatste mechanisme zoekt in een beperkte tijd echter enkel een set van gelijkaardige oplossingen af. De kans is groot dat hyperheuristieken in deze tijd tot betere oplossingen komen omdat het zoekterrein meer divers is.

\paragraph{}
Een hyperheuristiek kan ook gebruikt worden als een vorm van \emph{preprocessing}. Door een hyperheuristiek eerst een benaderende oplossing te laten uitrekenen wordt een strenge \emph{bound} bepaald. Het aantal \emph{backtracking}-stappen in het effectieve zoekproces kan hierdoor gereduceerd worden.

\paragraph{}
Het parallel uitrekenen van hyperheuristieken kan resulteren in een softwarepakket die optimalisatieproblemen gespecificeerd in een logische taal kan uitrekenen. Door dit in een parallelle context uit te voeren kan men een oplossing met potentieel een arbitraire fout in een arbitraire tijd uitrekenen.

\paragraph{}
Omgekeerd denken we dat logisch programmeren ook een bijdrage kan leveren bij het ontwikkelen van metaheuristieken en hyperheuristieken. Door het uitdrukken van optimalisatieproblemen in logische taal, specificeert men enkel het probleem en is het de verantwoordelijkheid van het algoritme in kwestie om met zichzelf om te vormen tot een effici\"ent optimalisatiesysteem voor het specifieke probleem.

\paragraph{}
Tot slot kan men in een later stadium ook een brug maken met \emph{energy complexity}\cite{Roy:2013:ECM:2422436.2422470,conf/icpp/KorthikantiAG11}: het berekenen met hoeveel processoren men het effici\"entst een kwalitatieve oplossing kan berekenen.

%%% Local Variables: 
%%% mode: latex
%%% TeX-master: "masterproef"
%%% End: 


% Indien er bijlagen zijn:
\appendixpage*          % indien gewenst
\appendix
\chapter{Communicatiemodel van \emph{ParHyFlex}}
\applab{a}

\chapterquote{The most important thing in communication is to hear what isn't being said.}{Peter F. Drucker}

Bij de uitwisseling van oplossingen en ervaring, het onderhandelen over een nieuwe zoekruimte en het uitwisselen van toestanden is communicatie vereist. \emph{ParHyFlex} gebruikt hiervoor twee vormen van communicatie: \emph{Message Passing Interface (MPI)} en \emph{User Datagram Protocol (UDP)}.

\subsection{Motivatie}

\subsubsection{\emph{Message Passing Interface (MPI)}}

\emph{MPI} is een standaard communicatieprotocol speciaal ontwikkeld voor parallelle algoritmen. Het omvat zowel directieven voor \emph{point-to-point} communicatie en \emph{collectieve} communicatie. Zowel de zender en de ontvanger kunnen kiezen om deze communicatie op een synchrone of asynchrone manier af te handelen.

\paragraph{}
Een groot voordeel van \emph{MPI} is dat er voor de meeste configuraties een implementatie beschikbaar is. Bovendien heeft men veel onderzoek ge\"investeerd in effici\"ente implementaties voor groepscommunicatie op verschillende netwerkstructuren (\emph{hypercube}, \emph{cycle},...). Er zijn verschillende netwerkkaarten ge\"implementeerd waar de \emph{MPI} commando's rechtstreeks in de hardware worden ge\"implementeerd en op die manier de processor ontlasten van een groot deel van de communicatie-aspecten.

\paragraph{}
\emph{MPI} legt weinig voorwaarden op inzake hoe de commando's ge\"implementeerd worden. De meeste implementaties werken op een manier die vergelijkbaar is met \emph{TCP} waarbij men meestal enkele optimalisaties inzake groepscommunicatie.

\paragraph{}
\emph{MPI} kent verschillende versies. %TODO

\subsubsection{\emph{User Datagraph Protocol (UDP)}}

\emph{UDP} is een protocol in de transportlaag die op een onbetrouwbare manier boodschappen doorstuurt. Boodschappen worden in pakketten doorgestuurd: sequenties aan bytes die op zichzelf staan.

\paragraph{}
Onbetrouwbare communicatie is vrij ongewoon in een parallelle context. De meeste parallelle algoritmen falen immers wanneer de resultaten die door andere processoren niet worden doorgestuurd. In het geval van metaheuristieken en hyperheuristieken is het echter niet noodzakelijk om informatie uit te wisselen: stel dat een oplossing niet wordt doorgestuurd, kan de potenti\"ele ontvanger nog altijd de oude oplossing gebruiken om verder te rekenen.

\paragraph{}
Werken met onbetrouwbare communicatie levert bovendien ook enkele voordelen op. Wanneer men moet verzekeren dat informatie wel degelijk de ontvanger bereikt, moet men een systeem implementeren waarbij de ontvanger een bericht terugstuurt dat de boodschap is aangekomen\footnote{De zogenaamde \emph{ACK}-pakketten.}. Heel wat implementaties zullen deze boodschappen effici\"ent proberen te implementeren. Toch zal men een deel van de beschikbare bandbreedte altijd gebruikt worden om de communicatie betrouwbaar te maken. Zeker in de context van een lokaal netwerk -- een configuratie waarbij een groot deel van de pakketten sowieso toekomt -- is dit een niet onbelangrijke kostprijs.

\paragraph{}
\emph{UDP} maakt ook het gebruik van \emph{multicast} pakketten eenvoudiger. Een \emph{multicast} pakket wordt naar meerdere ontvangers tegelijk gestuurd om zo de bandbreedte te sparen. Omdat betrouwbaarheid geen vereiste is, zal een pakket een constante kost teweegbrengen in het netwerk. In het geval we dit op een betrouwbare manier doen (bijvoorbeeld over het \emph{Transmission Control Protocol (TCP)}) moeten er bevestigingspakketten worden teruggestuurd die in totaal een kost teweegbrengen die schaalt met het aantal ontvangers. In het geval van \emph{TCP} werken we bovendien met een \emph{sliding window protocol}: slechts een deel van de fragmenten van een boodschap zijn tegelijk in omloop zijn. Indien \'e\'en ontvanger dus niet antwoordt -- bijvoorbeeld omdat deze op dat moment andere pakketten ontvangt -- kan dit ertoe leiden dat andere ontvangers geen verdere boodschappen meer ontvangen. \emph{Multicast} onder \emph{TCP} is bovendien geen sinecure\cite{dshp}: ontvangers moeten zichzelf eerst toevoegen aan een \emph{multicast group} om pakketten te ontvangen.

\paragraph{}
Een beperking aan \emph{UDP} is de pakketgrootte. Elk pakket heeft een maximale grote van 65'527 bytes aan data. Deze beperking is ingevoerd om te voorkomen dat een pakket lange tijd een communicatielijn kan opeisen waardoor andere entiteiten in het netwerk niet meer aan bod komen\cite{Tanen2003}. Vermits het niet zeker is dat een \emph{UDP} pakket wel degelijk op de bestemming toekomt, moet alle informatie dus in \'e\'en pakket worden opgeslagen. Het gevolg is dat sommige data niet uitwisselbaar is met behulp van \emph{UDP}. Door enkele eigenschappen van \emph{TCP} over te nemen in een nieuw protocol kan men dit probleem oplossen.

\section{Overzicht van de Communicatie}

In het algemeen beschouwen we volgende vormen van communicatie in \emph{ParHyFlex}:
\begin{enumerate}
 \item de probleeminstantie;
 \item geheugenconfiguratie;
 \item oplossingen;
 \item onderhandelingen over een nieuwe zoekruimte; en
 \item een deel van de toestand van de hyperheuristiek.
\end{enumerate}
in de volgende subsecties zullen we de verschillende vormen verder bespreken.

\subsubsection{De probleeminstantie en geheugenconfiguratie}

Wanneer \emph{ParHyFlex} wordt opgestart, zal \'e\'en van de processoren het optimalisatieprobleem inlezen. Het is de bedoeling dat het probleem vervolgens ook door de andere processoren wordt ingeladen. Hiervoor maken we gebruik van synchrone communicatie over \emph{MPI}. De probleeminstantie is immers een noodzakelijk deel van de data en als processoren reeds andere communicatie zouden aangaan met de processoren zijn de effecten oncontroleerbaar. Bovendien betreft het een eenmalige uitwisseling in het begin van het proces. Hierdoor zijn de communicatiekosten van minder belang.

\paragraph{}
Ook de geheugenconfiguratie wordt uitgewisseld. Hieronder verstaan we het aantal oplossingen die een processor tegelijk opslaat samen met enkele instellingen hoe oplossingen zullen worden gecommuniceerd. \emph{ParHyFlex} laat enkel toe dat de hyperheuristiek bij aanvang het geheugen juist afstelt. Daarom verloopt ook deze communicatie over een synchrone \emph{MPI} verbinding. De argumentatie is dezelfde als bij het uitwisselen van de probleeminstantie.

\subsubsection{Uitwisselen van oplossingen}

Het uitwisselen van oplossingen

\subsubsection{Onderhandelingen over een nieuwe zoekruimte}

Op geregelde tijdstippen vat \emph{ParHyFlex} een proces aan waarbij men over een nieuwe zoekruimte onderhandelt. Hiervoor dienen de verschillende processoren een voorstel voor hun eigen zoekruimte aan de andere processen mee te delen. Dit is een typisch voorbeeld van een \texttt{GatherAll}-operatie: een vorm van groepscommunicatie waarbij elke processor een deel van de data bezit. Op het einde van de operatie bezitten alle processoren alle delen. De meeste van \emph{MPI} bevatten geen implementatie voor een asynchrone \texttt{GatherAll}-operatie\footnote{De versies die dit niet ondersteunen zijn 1.0\cite{mpi10}, 1.3\cite{mpi13}, 2.0\cite{conf/europar/GeistGHLLSSS96,mpi20}, 2.1\cite{mpi21} en 2.2\cite{mpi22}. Versie 3.0\cite{mpi30} ondersteund dit commando wel.}, daarom hebben we deze zelf ge\"implementeerd. De details van deze implementatie staan in \secref{mpimod}.

\subsubsection{De toestand van de hyperheuristiek}

Een hyperheuristiek houdt meestal een toestand bij waarin hij bijvoorbeeld de prestaties van de heuristieken in opslaat. Daar de waarde van deze parameters sterk afhangt van het aantal gevallen waaruit deze data is opgebouwd, kan het interessant zijn deze data uit te wisselen: we doen dan immers een uitspraak op basis van meer gegevens. Voor deze taak werd een component ge\"implementeerd ter ondersteuning. De hyperheuristiek beschrijft in het begin welke data uitgewisseld moet worden en schuift de lokale data vervolgens in het systeem. Het systeem zal stuurt regelmatig de data rond waardoor de data van de andere systemen ook uitgelezen kan worden in het lokaal systeem. Ook hiervoor maken we gebruik van een asynchrone \texttt{gather all} implementatie (zie \secref{mpimod}).

\section{Aangepast \emph{MPI}-model: \emph{Non-blocking Gather All}}
\seclab{mpimod}

Het \texttt{GatherAll} commando zorgt voor de uitwisseling van gegevens tussen meerdere processoren. Voordat het commando wordt opgeroepen beschikt elke processor over een deel van de data. Na de operatie zitten bij alle processoren alle delen in het geheugen. Een \texttt{GatherAll} instructie kan men dus bekijken als een collectieve \texttt{Broadcast}: elke processor stuurt zijn deel van data naar alle andere processoren.

\paragraph{}De na\"ieve implementatie waarbij men inderdaad elke processor een \texttt{Broadcast} operatie laat ondernemen, vereist dat we $\bigoh{p\cdot\brak{p-1}/2}=\bigoh{p^2}$ berichten over het netwerkt sturen. Men kan stellen dat een voordeel van deze implementatie is dat de berichten in \bigoh{1} over het netwerk worden verstuurd. Als we echter de assumptie maken dat elke machine slechts \'e\'en bericht tegelijk kan ontvangen of de communicatielijnen de berichten sequentieel doorsturen, vereist deze operatie dus \bigoh{p} tijd.

\subsection{\texttt{GatherAll} met \bigoh{p\log p} berichten}
Een implementatie die minder berichten oplevert ordent de processoren in een \emph{hypercube}\cite[algoritme 4.7]{books/bc/KumarGGK94}. In het geval van $p$ processoren. Stel $d=\ceil{\log_2p}$, dan kunnen we de processoren ordenen in een $d$-dimensionale kubus. Elke processor heeft hierbij ofwel $d$ ofwel $d-1$ buren: processoren die slechts in \'e\'en dimensie van elkaar verschillen.

\paragraph{}
Processoren kunnen informatie uitwisselen met de buur van een bepaalde dimensie: zelf zend de processor alle data door waarover men op dat moment beschikt naar deze buur. Vermits de buur-relatie voor een specifieke dimensie symmetrisch is, zal deze buur ook alle data waarover hij beschikt doorsturen.

\paragraph{}


\subsection{Algoritme}

\importtikz[1]{asynchronegatherall}{asynchronegatherall}{Werking van een \texttt{GatherAll} operatie op een \emph{HyperCube}.}
\paragraph{}
\imgref{asynchronegatherall} toont dat door incrementeel informatie uit te wisselen met de buur van een telkens hogere dimensie, na $d$ stappen alle processoren over alle informatie beschikken. Een formele beschrijving van dit algoritme staat in \algref{gatherallsequential}.

\begin{algorithm}[hbt]
 $\rslt\leftarrow\mbox{eigen deel van de data}$\;
 \For{$i=0\mbox{ \textbf{\emph{to}} }d-1$}{
  $\mbox{\textbf{partner}}\leftarrow\idet\mbox{ XOR }2^i$\;
  $\funm{send}{\textbf{partner},\rslt}$\;
  $\msgt\leftarrow\funm{receive}{\textbf{partner}}$\;
  $\rslt\leftarrow\rslt\cup\mbox{\textbf{bericht}}$\;
 }
 \caption{\texttt{GatherAll}\cite{books/bc/KumarGGK94}.}
 \alglab{gatherallsequential}
\end{algorithm}

\subsection{Asynchrone aspecten}

De implementatie in \algref{gatherallsequential} werkt met synchrone communicatie: processen worden geblokkeerd tot een succesvolle uitwisseling plaatsvind. Het algoritme die we willen implementeren zal met asynchrone communicatie werken. Men kan dit implementeren door dit proces bijvoorbeeld op een aparte \emph{thread} te laten werken: een proces brengt de data onder in de context van de \emph{thread} en werkt verder. Nadien wordt regelmatig gecontroleerd over de \emph{thread} al de nodige informatie heeft verzameld.

\paragraph{}
\emph{ParHyFlex} werkt met \'e\'en \emph{thread}. Ontvangen berichten worden verwerkt telkens wanneer een heuristiek is uitgevoerd. \algref{gatherallasync} is een aangepaste versie van het synchrone algoritme. Het algoritme werkt met twee lijsten: $\rslt$ en $\impt$. $\rslt$ houdt de tot dusver ontvangen gegevens bij, deze kunnen dan later uitgelezen worden en verder uitgestuurd worden. $\impt$ slaat per dimensie op of de bijbehorende buur reeds zijn deel van data heeft opgestuurd. $z$ bevat de kleinste dimensie waar we nog geen data naar hebben gestuurd. Wanneer we een bericht ontvangen van een buur, berekenen we waar deze data moet worden opgeslagen en markeren we het relevante item in de $\impt$-lijst. Dit doen we ook wanneer we de lokale data aangeboden krijgen\footnote{Vermits het algoritme asynchroon verloopt kunnen er berichten vanuit de andere processoren worden gestuurd alvorens de lokale processor de \texttt{GatherAll}-instructie oproept.}. Bij beide gebeurtenissen controleren we of we op dat moment zelf data kunnen uitsturen: de \mbox{zendData}-functie. Deze functie controleert per dimensie of we over voldoende data beschikken: dit wil zeggen dat alle buren met een lagere dimensie de data al hebben doorgestuurd. Vervolgens stellen we het pakket met de relevante data samen en wordt dit asynchroon verstuurd. De operatie is uitgevoerd wanneer we naar alle buren data hebben verstuurd en ontvangen hebben.

\begin{algorithm}[hbt]
 $\rslt\leftarrow\funm{array}{p,\nult}$\;
 $\impt\leftarrow\funm{array}{d+1,\fals}$\;
 $z\leftarrow0$\;
 \WhnRcv($\tupl{\sndr,\msgt}$){
   $k\leftarrow\idet\mbox{ XOR }\sndr$\;
   $\bset\leftarrow\sndr\mbox{ AND }\brak{\mbox{NEG } k-1}$\;
   $\impt\fbrk{\log_2k}\leftarrow\true$\;
   $\rslt\fbrk{\bset:\bset+k-1}\leftarrow\msgt\fbrk{0:k-1}$\;
   $\funm{zendData}{}$\;
 }
 \Func($\funm{GatherAll}{\owdt}$){
   $\rslt\fbrk{\idet}\leftarrow\owdt$\;
   $\impt\fbrk{0}\leftarrow\true$\;
   \funm{zendData}{}
 }{}
 \Func($\funm{gereed}{}$){
   \Return{$z\geq d\wedge\impt\fbrak{d}$}\;
 }{}
 \Func($\funm{reset}{}$){
   $\impt\leftarrow\funm{array}{d-1,\fals}$\;
   $z\leftarrow 0$\;
 }{}
 \Func($\funm{zendData}{}$){
   \While{$z< d\wedge\impt\fbrk{z}$}{
	 $l\leftarrow 2^z$\;
	 $\bset\leftarrow\idet\mbox{ AND }\brak{\mbox{NEG } l-1}$\;
	 $\msgt\leftarrow\funm{array}{l}$\;
	 $\msgt\fbrk{0:l-1}\leftarrow\rslt\fbrk{\bset:\bset+l-1}$\;
	 $\funm{isend}{\idet\mbox{ XOR }l,\msgt}$\;
	 $z\leftarrow z+1$\;
   }
 }{}
 \caption{Asynchrone \texttt{GatherAll}.}
 \alglab{gatherallasync}
\end{algorithm}


\subsection{Onbetrouwbare communicatie}



%%% Local Variables: 
%%% mode: latex
%%% TeX-master: "masterproef"
%%% End: 

% ... en zo verder tot
\chapter{Poster}
\label{app:poster}
Op \imgref{poster} vindt men de poster die ontwikkeld werd met betrekking tot deze masterthesis. De poster werd voorgesteld op 8 mei 2013. De originele poster kan men downloaden op \url{http://willemvanonsem.ulyssis.be/poster.pdf}.
\begin{figure}[hbt]
\centering
\includegraphics[height=\textwidth,angle=-90]{poster-gray}
\caption{Poster met betrekking tot deze masterthesis}
\imglab{poster}
\end{figure}

%%% Local Variables: 
%%% mode: latex
%%% TeX-master: "masterproef"
%%% End: 


\backmatter
% Na de bijlagen plaatst men nog de bibliografie.
% Je kan de  standaard "abbrv" bibliografiestijl vervangen door een andere.
\bibliographystyle{abbrv}
\bibliography{referenties}

\end{document}

%%% Local Variables:
%%% mode: latex
%%% TeX-master: t
%%% End: