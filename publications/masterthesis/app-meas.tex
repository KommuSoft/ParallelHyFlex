\chapter{Meetresultaten}

In \chpref{resul} werden de resultaten grafisch voorgesteld. In deze bijlage geven we meer gedetailleerde cijfers.

\paragraph{}
In de verschillende tabellen worden de evoluties van de sterkste oplossingen doorheen de tijd beschreven. De om de tabel overzichtelijk te houden werden de tijdstippen gediscretiseerd per $500~\mbox{ms}$. Per tijdstip worden volgende parameters beschreven: minimum, maximum, eerste kwartiel, tweede kwartiel, derde kwartiel en standaardafwijking.

\section{Eenvoudige problemen}

\subsection{Per aantal processoren}

\inputresult{small-p1}{Evolutie van oplossingen bij $1$~processor}
\inputresult{small-p2}{Evolutie van oplossingen bij $2$~processoren}
\inputresult{small-p3}{Evolutie van oplossingen bij $3$~processoren}
\inputresult{small-p4}{Evolutie van oplossingen bij $4$~processoren}

\tblrefs{small-p1,small-p2,small-p3,small-p4} beschrijven we de evolutie van de verschillende oplossingen op verschillende processoren.

\section{Moeilijke problemen}

\subsection{Per aantal processoren}