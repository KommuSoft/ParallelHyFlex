\chapter{Studie naar Sequenti\"ele Hyperheuristieken (CHeSC2011)}
\label{hoofdstuk:2}

Alvorens zelf hyperheuristieken te implementeren, is het interessant om te analyseren aan welke eigenschappen een goede hyperheuristiek moet voldoen. We verwachten immers dat bij het effici\"ent paralleliseren van deze algoritmen sommige wetmatigheden kunnen worden overgenomen. Vandaar deze studie naar \abseqe{} \abhhn{}.

\section{Implementatie-set}

\subsection{\abhf{}}

In 2010 publiceren \auth[5586064]{Burke et al.} een paper waarin ze \abhf{} voorstellen. \abhf{} is een klassenbibliotheek geschreven in \abjava{}. Het laat toe dat de gebruiker hyperheuristieken implementeert zonder dat het algoritme de details kent van het onderliggende probleem en de \abllhn{}. Het systeem werkt op basis van geheugen: een lijst waarin tussentijdse oplossingen worden opgeslagen en uitgelezen. Het systeem biedt vervolgens de mogelijkheid om een \abllh{} met een specifieke index toe te passen op de oplossing op een specifieke plaats in het geheugen en het resultaat op een specifieke plaats op te slaan. Daarnaast kunnen gebruikers ook de objectiviteitswaarde van een bepaalde oplossing in het geheugen opvragen en vragen of twee oplossingen in het geheugen equivalent zijn.

\paragraph{}
Het programma heeft dan wel geen idee wat de onderliggende \abllhn{} precies doen, ze worden wel gecategoriseerd in 4 verschillende types:
\begin{enumerate}
 \item \abmt{}: hierbij wordt de oplossing op een toevallig manier aangepast. Dit soort heuristieken heeft geen componenten die inschatten of de verandering onmiddellijk of op termijn tot betere resultaten zal leiden.
 \item \abco{}: dit zijn de enige heuristieken die twee oplossingen recombineren in een nieuwe oplossing. Het is uiteraard de bedoeling dat de nieuwe oplossing karakteristieken gemeen heeft met beide ``ouders''.
 \item \abrr{}: deze heuristieken breken een deel van de oplossing af, om ze dan vervolgens met behulp van bijvoorbeeld een gretig algoritme terug op te bouwen.
 \item \abls{}: dit is een familie van algoritmen die herhaaldelijk mutaties uitvoeren indien deze mutaties ook per stap winst opleveren. Indien geen enkele mutatie meer tot een beter resultaat leidt stopt het algoritme.
\end{enumerate}
Het softwaresysteem houdt ook de mogelijkheid open om een \abllh{} te classificeren onder ``other'', maar voor zover ons bekend zijn er nog geen heuristieken in \abhf{} geschreven die onder deze categorie vallen. Verder kan er verwarring optreden over het feitelijke verschil tussen \abrr{} en \abls{}. We kunnen immers bij beide families verwachten dat ze minstens een oplossing opleveren die beter is dan het origineel (\abrr{} zal immers tot in dat geval het afgebroken gedeelte reconstrueren zoals het origineel). Een verschil die men doorgaans maakt is dat \abls{} een operator is die idempotent is.

\subsection{\abchescy{}}

Om meer aandacht te vestigen op \abhf{} werd in 2011 een wedstrijd georganiseerd door de universiteit van Nottingham: de ``\emph{Cross-domain Heuristic Search Challenge (\abchescy)}''. De verschillende programma's krijgen een set van verschillende problemen en worden gequoteerd op basis van de kwaliteit van de oplossingen die ze na 10 minuten uitvoer afleveren.%TODO(check)
De wedstrijd omvatte problemen uit zes verschillende domeinen: \prob{Maximal Satisfiability}, \prob{Bin Packing}, \prob{Personnel Scheduling}, \prob{Flow Shop}, \prob{Travelling Salesman Problem} en het \prob{Vehicle Routing Problem}.

\paragraph{}
In totaal telde de competitie 20 teams. We hebben met onze studie de zestien implementaties die gedocumenteerd werden bestudeerd. Tabel \ref{tbl:chescParticipants} bevat een lijst met de verschillende implementaties en geeft aan welke systemen in de studie opgenomen werden.

\begin{table}[hbt]
  \centering
  \begin{tabular}{rllrc} \toprule
    \#&Naam&Auteur/Team&Score&Bestudeerd\\\midrule
    1&	AdapHH\cite{chesc-adaphh,chesc-adaphh2}&		Mustafa M\i{}s\i{}r&	181.00&	$\checkmark$\\
    2&	VNS-TW\cite{chesc-vns-tw}&				Mathieu Larose&		134.00&	$\checkmark$\\
    3&	ML\cite{chesc-ml,chesc-ml2}&				Mustafa M\i{}s\i{}r&	131.50&	$\checkmark$\\
    4&	PHUNTER\cite{chesc-phunter}&				Fan Xue&		93.25&	$\checkmark$\\
    5&	EPH\cite{chesc-eph}&					David Meignan&		89.75&	$\checkmark$\\
    6&	HAHA&							Andreas Lehrbaum&	75.75&	\\
    7&	NAHH&							MFranco Mascia&		75.00&	\\
    8&	ISEA\cite{chesc-isea}&					Jiri Kubalik&		71.00&	$\checkmark$\\
    9&	KSATS-HH\cite{chesc-ksats-hh}&				Kevin Sim&		66.50&	$\checkmark$\\
    10&	HAEA\cite{chesc-haea}&					Jonatan Gomez&		53.50&	$\checkmark$\\
    11&	ACO-HH\cite{chesc-aco-hh}&				Jos\'e Luis N\'u\~nez&	39.00&	$\checkmark$\\
    12&	GenHive\cite{chesc-genhive}&				CS-PUT&			36.50&	$\checkmark$\\
    13&	DynILS\cite{chesc-dynils}&				Mark Johnston&		27.00&	$\checkmark$\\
    14&	SA-ILS&							He Jiang&		24.25&	\\
    15&	XCJ&							Kamran Shafi&		22.50&	\\
    16&	AVEG-Nep\cite{chesc-aveg-nep}&				Thommaso Urli&		21.00&	$\checkmark$\\
    17&	GISS\cite{chesc-giss}&					Alberto Acu\~na&	16.75&	$\checkmark$\\
    18&	SelfSearch\cite{chesc-selfsearch}&			Jawad Elomari&		7.00&	$\checkmark$\\
    19&	MCHH-S\cite{chesc-mchh-s}&				Kent McClymont&		4.75&	$\checkmark$\\
    20&	Ant-Q\cite{chesc-ant-q}&				Imen Khamassi&		0.00&	$\checkmark$\\
    \bottomrule
  \end{tabular}
  \caption{Deelnemers van de \abchescy{} competitie\cite{chesc-results}.}
  \label{tbl:chescParticipants}
\end{table}


%De ``\emph{Cross-domain Heuristic Search Challenge (CHeSC)}'' was een wedstrijd georganiseerd in 2011. De wedstrijd spitste zich toe op het ontwikkelen van hyperheuristieken in ``\abhyfl{}''\cite{hyflex2012,5586064}. \abhyfl{} is een klassenbibliotheek geschreven in Java 

\subsection{Een item}
Een tekst staat nooit alleen. Dit wil zeggen dat er zeker ook referenties
nodig zijn. Dit kan zowel naar on-line documenten\cite{wiki} als naar
boeken\cite{pratchett06:_good_omens}.

\section{Tabellen}
Tabellen kunnen gebruikt worden om informatie op een overzichtelijke te
groeperen. Een tabel is echter geen rekenblad! Vergelijk maar eens
tabel~\ref{tab:verkeerd} en tabel~\ref{tab:juist}. Welke tabel vind jij het
duidelijkst?

\begin{table}
  \centering
  \begin{tabular}{||l|lr||} \hline
    gnats     & gram      & \$13.65 \\ \cline{2-3}
              & each      & .01 \\ \hline
    gnu       & stuffed   & 92.50 \\ \cline{1-1} \cline{3-3}
    emu       &           & 33.33 \\ \hline
    armadillo & frozen    & 8.99 \\ \hline
  \end{tabular}
  \caption{Een tabel zoals het niet moet.}
  \label{tab:verkeerd}
\end{table}

\begin{table}
  \centering
  \begin{tabular}{@{}llr@{}} \toprule
    \multicolumn{2}{c}{Item} \\ \cmidrule(r){1-2}
    Animal    & Description & Price (\$)\\ \midrule
    Gnat      & per gram    & 13.65 \\
              & each        & 0.01 \\
    Gnu       & stuffed     & 92.50 \\
    Emu       & stuffed     & 33.33 \\
    Armadillo & frozen      & 8.99 \\ \bottomrule
  \end{tabular}
  \caption{Een tabel zoals het beter is.}
  \label{tab:juist}
\end{table}

\section{Lorem ipsum}
Tenslotte gaan we hier nog wat tekst voorzien zodat er minstens een
bijkomende bladzijde aangemaakt wordt. Dat geeft de gelegenheid om eens te
zien hoe de koptekst en de voettekst zich gedragen.

\section{Besluit van dit hoofdstuk}
Als je in dit hoofdstuk tot belangrijke resultaten of besluiten gekomen
bent, dan is het ook logisch om het hoofdstuk af te ronden met een
overzicht ervan. Voor hoofdstukken zoals de inleiding en het
literatuuroverzicht is dit niet strikt nodig.

%%% Local Variables: 
%%% mode: latex
%%% TeX-master: "masterproef"
%%% End: 
