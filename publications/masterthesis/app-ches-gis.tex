\section{\emph{GISS: Generic Iterative Simulated Annealing Search} (\#17)}
\seclab{giss}
\subsection{Implementatie}
\emph{GISS}\cite{chesc-giss} maakt gebruik van \emph{Simulated Annealing}\cite{citeulike:1612433} als basisprincipe: bij elke iteratie kiest men uniform een beschikbare metaheuristiek die men toepast op het probleem. Vervolgens accepteert men deze oplossing volgens het principe van \emph{simuated annealing}. Crossover metaheuristieken worden toegepast op de laatste en voorlaatste oplossing. Indien er lange tijd geen verbetering zichtbaar is, wordt het systeem herstart vanaf een toevallige oplossing.

\subsection{Kritiek}
\begin{itemize}
 \item Uniforme selectie van \abllhn{} is waarschijnlijk niet interessant. Sommige \abllhn{} zijn immers nagenoeg overal beter dan anderen.
 \item \abco{} heuristieken toepassen tussen de laatste en de voorlaatste oplossing levert meestal weinig op, vermits de laatste gegenereerd is door een \abllh{} toe te passen op de voorlaatste.
 \item Er is geen overdracht van zoekervaring bij een (mogelijke) herstart. We kunnen echter verwachten dat we ook uit vorige rondes nuttige informatie kunnen leren.
 \item Er wordt opnieuw geen onderscheid gemaakt tussen het type van de \abllhn{}. Hierdoor verliest men mogelijk veel rekenkracht aan nutteloze operaties.
\end{itemize}