\section{\emph{MCHH-S: Markov Chain Hyper-Heuristic} (\#19)}
\seclab{mchh-s}

\subsection{Implementatie}
\emph{MCHH-S}\cite{chesc-mchh-s,conf/gecco/McClymontK11} werkt gelijkaardig aan \emph{Ant-Q} (zie \secref{ant-q}): men ontwerpt een graaf waar de knopen de metaheuristieken voorstellen en de bogen overgangen die men met een kans labelt. Het belangrijkste verschil is dat slechts \'e\'en oplossing in het netwerk rondwandelt. De kansen ook op een andere manier berekend: enkel de laatste boog verandert op basis van de onmiddellijke verandering van de fitness-waarde van de oplossing. Een nieuwe oplossing wordt ook niet meteen geaccepteerd: alleen indien de oplossing beter is, of probabilistisch volgens het aantal opeenvolgende iteraties dat er nog geen betere oplossing werd gevonden.

\subsection{Kritiek}
\begin{itemize}
 \item Dit algoritme kan in een stabiel systeem terecht komen: vermits \abls{} meestal voor een grote winst zorgen en dus vaak beloond zullen worden. Hierdoor zullen bogen naar \abls{} \abhn{} gelabeld worden met hoge kansen. De idempotentie van \abls{} heuristieken zorgt er echter voor dat het na elkaar toepassen van de heuristieken zinloos is. Men kan veel rekentijd verliezen met het herhaaldelijk toepassen van \abls{} heuristieken.
 \item Men maakt geen onderscheid tussen de verschillende types heuristieken: mutatie zal meestal tot een tijdelijk slechtere oplossing leiden. De bogen naar mutaties bestraffen is echter waarschijnlijk niet wenselijk. (net als bij \emph{Ant-Q}, zie \secref{ant-q}).
\end{itemize}