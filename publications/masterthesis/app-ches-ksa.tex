\section{\emph{KSATS-HH: Simulated Annealing with Tabu Search} (\#9)}
\seclab{ksats-hh}

\subsection{Implementatie}
\emph{KSATS-HH}\cite{chesc-ksats-hh} werkt volgens het principe van ``\emph{Simulated Annealing}''\cite{citeulike:1612433}: er is sprake van \'e\'en actieve oplossing. Na het toepassen van een \abllh{} accepteert men het resultaat van de oplossing indien de oplossing beter is, of met een bepaalde kans (die exponentieel daalt naarmate het resultaat veel slechter is) een slechtere oplossing accepteert. In plaats van dit principe toe te passen op het werkelijke verschil in fitness-waarde wordt met behulp van ervaring uit het verleden de relatieve verbetering uitgerekend\footnote{Het verschil kan immers afhankelijk zijn van het probleem domein of de instantie.}. Het koelingsschema werkt ook exponentieel maar de factor waarmee men vermenigvuldigt verschilt in iedere tijdstap en hangt af van de het aantal \abhn{} die men per tijdseenheid kan toepassen.
\paragraph{}
De keuze van de heuristiek die wordt toegepast werkt op basis van ``\emph{Tabu Search}''\cite{DBLP:journals/informs/Glover89}: het algoritme onderhoudt een lijst bij van actieve heuristieken. Heuristieken die erin slagen om de oplossing te verbeteren stijgen in de lijst. Heuristieken die daar niet in slagen dalen een plaats en worden ``\emph{tabu}'' voor de volgende 7 iteraties. De selectie van de heuristiek die wordt toegepast, gebeurt door twee heuristieken uit de lijst te selecteren die niet ``\emph{tabu}'' zijn. De heuristiek met die het hoogst in de gegenereerde lijst staat wordt dan gekozen.

\subsection{Kritiek}
\begin{itemize}
 \item Het systeem bestraft heuristieken voordat het algoritme begint: er zit een inherente orde in de lijst. Het element die als laatste geclassificeerd staat kan per toeval net de best presterende heuristiek zijn. Het kan lang duren voor deze heuristieken toch worden geselecteerd, in het begin van de uitvoer is het immers eenvoudiger om de oplossing te verbeteren.
\end{itemize}