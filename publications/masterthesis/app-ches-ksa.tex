\section{\emph{KSATS-HH: Simulated Annealing with Tabu Search} (\#9)}
\label{sss:ksats-hh}
\subsection{Implementatie}
\emph{KSATS-HH}\cite{chesc-ksats-hh} is een implementatie die gebaseerd is op ``\emph{Simulated Annealing}''\cite{citeulike:1612433}: er is sprake van \'e\'en actieve oplossing. Na het toepassen van een \abllh{} accepteert men het resultaat van de oplossing indien de oplossing beter is, of met een bepaalde kans (die exponentieel daalt naarmate het resultaat veel slechter is) een slechtere oplossing accepteert. Een intelligent aspect hierbij is dat men op basis van ervaring uit het verleden gebruikt om het verschil in fitness-waarde van de oplossingen eerst te normaliseren (het verschil kan immers afhankelijk zijn van het probleem domein of de instantie). Het koelingsschema werkt ook exponentieel maar de factor waarmee men vermenigvuldigt verschilt in iedere tijdstap en hangt af van de het aantal iteraties die men in \'e\'en tijdseenheid weet te realiseren. De keuze van de heuristiek die wordt toegepast werkt op basis van ``\emph{Tabu Search}''\cite{DBLP:journals/informs/Glover89}: het algoritme houdt een lijst bij van heuristieken. Heuristieken die erin slagen om de oplossing te verbeteren stijgen in de lijst. Heuristieken die daar niet in slagen dalen een plaats en worden ``\emph{tabu}'' voor de volgende 7 iteraties. De uiteindelijke selectie van de heuristiek gebeurt door twee heuristieken uit de lijst te selecteren die niet ``\emph{tabu}'' zijn. De heuristiek met die het hoogst in de lijst gegenereerde lijst staat wordt dan gekozen.
\subsection{Kritiek}
\begin{itemize}
 \item Het systeem bestraft heuristieken voordat het algoritme begint: er zit een inherente orde in de lijst. Het element die als laatste geclassificeerd staat kan per toeval net de best presterende heuristiek zijn. Het kan lang duren voor deze heuristieken toch worden geselecteerd (vermits in het begin tot een betere oplossing komen vrij eenvoudig is).
\end{itemize}