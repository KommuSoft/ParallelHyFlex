\section{RH: Relay Hybridisation}
Naast \emph{ADHS} is \emph{RH} ook een mechanisme om heuristieken te selecteren. Per iteratie in de fase zal met stijgende kans dit mechanisme geactiveerd worden. Op basis van een \emph{learning automaton} wordt een heuristiek gesecteerd die wordt toegepast op de actieve oplossing. Elke heuristiek onderhoud een lijst met heuristieken die effectief bleken als tweede transitiefunctie. Met een bepaalde kans wordt een heuristiek uit deze lijst geselecteerd. In het andere geval wordt er een toevallige heuristiek geselecteerd. Die tweede heuristiek wordt dan toegepast op het resultaat van de eerste heuristiek. De \emph{learning automaton} gebruik een \emph{lineair reward-interaction update schema} waardoor combinaties die globaal betere oplossingen vinden meer kans maken in de volgende iteraties.

\paragraph{}
In het geval van een \emph{learning automaton}, is \emph{mimetism} een populaire oplossing: het nabootsen van de toestanden van de andere processen. Na elke fases stuurt het proces de verschillen in de kansvector door naar de andere processoren. Deze verschillen worden gedeeltelijk doorgerekend. Door de wijzigingen slechts gedeeltelijk door te rekenen, hopen we voorkomen dat de kansvectoren uit de hand kunnen lopen.