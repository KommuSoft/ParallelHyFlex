\section{Overzicht van de Communicatie}

We beschouwen volgende vormen van communicatie in \emph{ParHyFlex}:
\begin{enumerate}
 \item de probleeminstantie;
 \item geheugenconfiguratie;
 \item oplossingen;
 \item onderhandelingen over een nieuwe zoekruimte; en
 \item een deel van de toestand van de hyperheuristiek.
\end{enumerate}
in de volgende subsecties zullen we de verschillende vormen verder bespreken.

\subsection{De probleeminstantie en geheugenconfiguratie}

Wanneer \emph{ParHyFlex} wordt opgestart, zal \'e\'en van de processoren het optimalisatieprobleem inlezen. Het is de bedoeling dat het probleem vervolgens ook door de andere processoren wordt ingeladen. Hiervoor maken we gebruik van synchrone communicatie over \emph{MPI}. Het inlezen van de instantie is een noodzakelijke eerste stap. We verwachten dat de synchrone communicatie het proces zal vertragen. Het is echter een stap in het opzetten van het systeem die slechts \'e\'enmalig wordt uitgevoerd. We verwachten dat deze stap bijgevolg niet significant doorweegt in het totale algoritme.

\paragraph{}
Ook de geheugenconfiguratie wordt uitgewisseld. Hieronder verstaan we het aantal oplossingen die een processor tegelijk opslaat samen met enkele instellingen hoe oplossingen zullen worden gecommuniceerd. \emph{ParHyFlex} laat enkel toe dat de hyperheuristiek bij aanvang het geheugen juist afstelt. Daarom verloopt ook deze communicatie over een synchrone \emph{MPI} verbinding. De argumentatie is dezelfde als bij het uitwisselen van de probleeminstantie.

\subsection{Uitwisselen van oplossingen}

Het uitwisselen van oplossingen is een essenti\"ele taak in \emph{ParHyFlex}. Dit principe is gesteund op het principe van het \emph{Island Model}\cite{parallelgeneticalgorithms}. Oplossingen worden uitgewisseld met behulp van onbetrouwbare communicatie via het \emph{UDP} protocol.

\paragraph{}
Een typisch argument in het gebruik van onbetrouwbare communicatie is dat de data na verloop van tijd niet meer relevant is. Ook in het geval van hyperheuristieken is dit argument van toepassing: wanneer het communiceren van een oplossing teveel tijd in beslag neemt is de kans groot dat wanneer de oplossing uiteindelijk aankomt, de kwaliteit inferieur is. Processoren zullen bovendien regelmatig oplossingen uitwisselen, het missen van enkele oplossingen hoeft dus niet te betekenen dat dit op termijn niet kan worden goedgemaakt met andere oplossingen.

\subsection{Onderhandelingen over een nieuwe zoekruimte}

Op geregelde tijdstippen vat \emph{ParHyFlex} een proces aan waarbij men over een nieuwe zoekruimte onderhandelt. Verschillende processoren formuleren een voorstel voor een eigen zoekruimte. Dit gebeurd typisch aan de hand van een \texttt{GatherAll}-operatie: een vorm van groepscommunicatie waarbij elke processor een deel van de data bezit. Op het einde van de operatie bezitten alle processoren alle delen. De meeste van \emph{MPI} bevatten geen implementatie voor een asynchrone \texttt{GatherAll}-operatie\footnote{De versies die dit niet ondersteunen zijn 1.0\cite{mpi10}, 1.3\cite{mpi13}, 2.0\cite{conf/europar/GeistGHLLSSS96,mpi20}, 2.1\cite{mpi21} en 2.2\cite{mpi22}. Versie 3.0\cite{mpi30} ondersteund dit commando wel.}, daarom werd deze zelf ge\"implementeerd. De details van deze implementatie staan in \secref{mpimod}.

\subsection{De toestand van de hyperheuristiek}

Een hyperheuristiek houdt een toestand bij waarin bijvoorbeeld de prestaties van de heuristieken wordt opgeslagen. De betrouwbaarheid van deze parameters is onderhevig aan het aantal onderzochte gevallen. Daarom kan het interessant zijn deze parameters uit te wisselen. Voor deze taak werd een component ge\"implementeerd ter ondersteuning. Elke processor houdt een verzameling van serialiseerbare objecten die regelmatig worden uitgewisseld. Ook hiervoor maken we gebruik van een asynchrone \texttt{gather all} implementatie (zie \secref{mpimod}).