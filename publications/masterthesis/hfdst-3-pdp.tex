\section{Probleemafhankelijk gedeelte}

In een parallelle context kan het interessant zijn om het probleemafhankelijke gedeelte meer functionaliteiten ter beschikking te laten stellen. Concreet denken we hierbij aan vier zaken: \emph{afstandmetrieken}, \emph{ervaringsgeneratoren}, \emph{beperken van de zoekruimte} en \emph{multi-objectieven}

\subsection{Afstandsmetrieken}
\abhf{} laat problemen een functie aanbieden die twee oplossingen met elkaar kan vergelijken. Deze functie kunnen we theoretisch omvormen tot een afstandsmetriek\footnote{We stellen de afstand tussen twee dezelfde oplossingen gelijk aan 0, en tussen twee verschillende oplossingen aan een arbitraire constante groter dan 0.}, maar levert weinig informatie op. In een sequenti\"ele context kan men het aantal heuristieken die tussen een oplossing en \'e\'en van zijn voorouders liggen gebruiken om de afstand af te schatten. Dit is natuurlijk slechts een benadering. In het geval van parallelle uitvoer zullen we bovendien meestal niet over deze informatie beschikken. Daarom is het nuttig om de afstand tussen twee oplossingen te kunnen inschatten. In het geval de afstand geen triviaal gegeven is, kan men verschillende afstandsmetrieken defini\"eren.
 
\subsection{Ervaringsgeneratoren}
Elk proces draait een eigen hyperheuristiek en komt een grote hoeveelheid oplossing tegen. Uitwisselen van de sequentie kan potentieel een voordeel opleveren omdat de hyperheuristieken met meer kennis van zaken kunnen beslissen. Het doorsturen van alle oplossingen is doorgaans niet mogelijk omdat dit een te grote druk op het netwerk zet en bovendien de overige processoren te veel rekenkracht zouden investeren in het analyseren van de ontvangen berichten. Door het uitwisselen van ervaring, een compacte voorstelling van beschouwde oplossingen, zouden we dit probleem kunnen oplossen.

\subsection{Beperken van de zoekruimte}
Door oplossingen uit te wisselen lopen we de kans dat de verschillende processoren op termijn vergelijkbare populaties onderhouden. Wanneer populaties echter rond eenzelfde lokaal optimum liggen, is dit meestal nefast\cite{parallelgeneticalgorithms}. In dat geval proberen alle processoren het lokale optimum te zoeken in een eenzelfde gebied, en wordt migratie naar mogelijk betere oplossingen in een ander gebied minder evident. Het introduceren van een component die diversificatie afdwingt kan helpen te voorkomen dat we op ijle populaties stuiten.

\subsection{Multi-objectieven}
Alle processoren proberen hetzelfde optimalisatieprobleem op te lossen. Door extra objectieven te introduceren, kunnen we echter een meer divers zoekproces aanbieden. Deze extra objectieven zijn eerder virtueel en dienen meer als een \emph{tie-breaker} in bijvoorbeeld gevallen waarbij twee oplossingen dezelfde fitness-waarde hebben.

\subsection{Afdwingbare beperkingen als probleemonafhankelijke ervaring}

Een probleem bij het genereren van \emph{ervaring} en het beperken van de \emph{zoekruimte} is dat dit op een probleemonafhankelijke manier dient te gebeuren: de bovenliggende hyperheuristiek kent de details in verband met de structuur van de configuraties niet. Bijgevolg kan bijgevolg niet zelf de zoekruimte beperken of conclusies genereren. We kunnen ervaring voorstellen als een object waar de hyperheuristiek de specificaties niet van kent, maar in dat geval moet ervaring wel enkele algemene functionaliteiten kunnen aanbieden. Om dit probleem op te lossen voeren we het concept van een \emph{afdwingbare beperking} in.

\begin{definition}[Afdwingbare beperking]
Een \emph{afdwingbare beperking} is een 3-tuple: $\tupl{c,c^+,c^-}$. Met $c:\SolSet\rightarrow\BoolSet$ een functie die controleert of een gegeven oplossing aan een bepaalde voorwaarde voldoet. $c^+:\SolSet\rightarrow\SolSet$ een functie die een gegeven oplossing minimaal aanpast zodat deze aan de voorwaarde voldoet. $c^-:\SolSet\rightarrow\SolSet$ tenslotte past oplossingen minimaal aan zodat ze niet aan de voorwaarde voldoen. De set van alle afdwingbare beperkingen die we op een probleem kunnen toepassen noteren we als $\HypSet$.
\end{definition}

We kunnen een afdwingbare beperking als een vorm van ervaring zijn. In de loop der tijd kunnen we immers een hypothese ontwikkelen dat sterke oplossingen aan een bepaalde voorwaarde voldoen (bijvoorbeeld een variabele in het probleem krijgt een vaste waarde). We kunnen dan oplossingen aantrekken naar de hypothese door $c^+$ op deze oplossingen toe te passen. Anderzijds kunnen we ook oplossingen afstoten van de hypothese met $c^-$. De bovenliggende hyperheuristiek kent de voorwaarde die wordt voorgesteld niet, maar kan wel oplossingen naderhand manipuleren.

\paragraph{}
Afdwingbare beperkingen kunnen we eveneens gebruiken om de zoekruimtes te beperken. Elk proces kan immers een aantal afdwingbare beperkingen gebruiken om een bepaalde zoekruimte te beschouwen, terwijl het de afdwingbare beperkingen van de andere processoren gebruikt om uit de buurt van de andere zoekruimtes te blijven.

\paragraph{}
De hyperheuristieken zelf kunnen geen ervaring genereren, ze hebben immers geen weet van de structuur van een oplossing noch de afdwingbare beperkingen. Daarom zal het specifieke probleem dus een set functies defini\"eren die we \emph{hypothese-generatoren (hypogen)} noemen:
\begin{definition}[Hypothese-generator]
Een \emph{hypothese-generator (hypogen)} $g_i:\SolSet^{n_{g_i}}\rightarrow\HypSet$ is een functie die op basis van een set oplossingen een afdwingbare beperking kan genereren.	
\end{definition}